\section{拾伍}

{\centering\subsection{1}}

乔文君跟六公子说,我不想跟你吵架。但实际上,那天晚上的吵架就是从这句话开始的。说完“我不想跟你吵架”,他们就开启了吵架模式。吵架是没有逻辑的,一个话题说不过你,就换个话题,挑个新毛病继续吵。他们从东吵到西,从傍晚吵到入夜。说来说去,核心问题还是,那包哑药是怎么回事?

那是正月之后的事,小五子逃跑后的半个月,沉狮谷虽不至于春暖花开,但总算是冰雪融化,可以出去转转。乔姑娘终于坐上马车,一路往上,出了沉狮谷,上集市买点布料首饰。其实这些都是让小玉去买的,她直接去见六公子,集市尽头的来祥客栈。大堂的店小二没有多嘴,问她“打尖还是住店”什么的。她直奔二楼,呼吸急促,一路走到拐角的房门前,六公子已经打开房门等着她。一进门,她就扑到六公子怀里。也许是思念之情,她在他怀里放声哭出来。六公子左臂抱着她,伸出右手,从里面把房门关上。

一直到傍晚,夕阳西下,乔文君从床上坐起来,拨开窗子往下看,只见小玉已经替她买好了东西,马车停在客栈门口的雪地上,等她出来一起回沉狮谷。

“我得回去了。”她说。乔文君放下窗户,背对着六公子穿好衣服,之后等了好一会儿,转回身看着他,长叹一口气:“太晚回去,我爹又要疑神疑鬼了。”

六公子没说话,就那么深情地望着她。乔文君舍不得,又必须要离开,她尽量把衣服穿慢点,再慢一点,最后连大衣都已经穿好了,又解下来重系里面的扣子。

“不然,你跟我走吧。”六公子在身后说。

乔文君苦笑,摇头道:“我爹什么样子,你又不是不知道,他要是不管我,我早就跟你走了。”

“那就别让他管你了,我带你走。”六公子递给她一包药,说,“你回去就收拾行李,今夜子时就动手。入睡之前,你把这包药放到茶里,给你爹喝了,让他安心睡到天明,我夜里去沉狮谷接你,带你离开这里。”

乔文君不接,问他:“这是什么药?”

“这是哑药,”六公子说,“但你放心,药效只有八个时辰,只要你爹到时候发不出声。等他恢复功力时,我们已经逃得远了,这事就成了。”

“这不可能,那是我爹。”

乔文君拒绝了他,弯腰把鞋子穿好,准备出门。六公子显然不高兴,忽然来了一句:“那我先杀昆仑公子好了。”

乔文君站在门口,皱眉看着他,奇怪他为什么这么说。

“你还是不想出来,毕竟是嫁了昆仑公子,只想在沉狮谷厮守,等着他哪天回来。”

“你怎么会这么想,我是跟你有了闹闹的,这婚姻跟小五子一点关系都没有,他完全是为了你,背了这黑锅。”

六公子冷笑,说:“真是一日夫妻百日恩啊!你俩才睡了几天啊,就已经叫他小五子了,就开始帮着他说话了。”

乔文君眼神坚定地告诉六公子:“我俩没事,你别多想。”

六公子只是笑,满脸的讥讽表情,语气尖刻地说:“你们俩有没有事,是你们俩的事,和我没任何关系。”

乔文君急了,还嘴道:“跟你没关系?这孩子是不是你的?当初是不是你找了一大堆理由,说娶不了我?当初是不是你让我说,这孩子是昆仑公子的?你说昆仑公子消失了一段时间,可能是死了,现在人家出现了,你倒是吃起醋来了?你让我怎么办?你当我想嫁给他吗?我想嫁的人是你啊!”

连发一通火,她眼泪都掉下来了。她抹掉眼泪,走到窗前,推开窗户往下看。小玉等得无聊,已经从马车里出来了,站在雪地里直跺脚。

六公子也心软了,安慰她几句,说我当时是这样说的,时机一到,我肯定会娶你。“可是现在要娶你时,你却推三阻四,我才会多想。”

乔文君想了想,是啊,等了好几年,不就是在等这一天,可为什么这一天来了,反而会有点不舒服呢?她到六公子身前,接过他手里的药包,说:“你要答应我,别杀昆仑公子。我跟他没什么事,你放心吧。”

她说着话,一路走到门口,关上门之前,六公子听见她说:“你二更时分过来接我。”
\newline

{\centering\subsection{2}}

小五子和吴思若趴在泥地里,听着远处乔文君和六公子的吵架,片言只语逐渐让两个人清楚,乔帮主哑掉的来龙去脉。显然吴思若更震惊,原来乔文君不是小五子的老婆,原来他只是替别人养孩子。吴思若凑到小五子耳边,低声问:“是谁把你绑过来的?”

小五子不敢出声,只是冲百尺之外的六公子努了努嘴。

“为什么要绑你呢?”吴思若问。

还是不能说话,自己满嘴的泥巴,要吐出来,才能把话讲清楚,但这“呸”的一声,别说是六公子,武功弱点的乔文君都能听得清清楚楚。他用手抹了一下脖子,那意思是,他绑我,是要杀了我。

吴思若看明白了,接着往下问:“他干嘛\footnote{原文“干吗”}要杀你啊?”

天啊,这让我怎么不出声就跟你讲清楚?可能想杀我灭口吧。比如那哑药,明明就是六公子的,假借乔文君之手,非要嫁祸给我,让乔帮主这个老糊涂对我恨得咬牙切齿。到底怎么弄的呢,以后有机会还是要问个明白。

但现在还不是听故事看热闹的时候,趁他们吵得凶,咱们先想办法逃命。他伸手指指左边,又指指右边,示意从哪条路上跑出去。吴思若这才意识到,是哦,我们得想办法逃出去。她挺起身,左右看着,两边都不太好出去,无论上山还是下山,都要惊动那对狗男女。吴思若想了想,建议他们往后挪,换个地方,至少别在原地待着等死。

可又不能站起来走过去,两个人匍匐在泥浆里,一点一点往后蹭。小五子无所谓,可惜了吴思若一身的白衣服白鞋。往后蹭了几十米,一棵砍倒的大树横在后面,挡住去路。但也差不多了,已经听不到那边的吵架声,意味着这边小声说话没问题。风吹过树叶沙沙作响,小五子借着风声把泥巴轻轻吐出来,大口喘着气。吴思若让小五子帮忙,弄些泥巴糊到她后背的白衣上。小五子也不客气,双手捧着泥浆,把她后面抹了个遍。然后他趴下去,让吴思若给他也在后背上抹一抹。

“你本来就是泥人了,还抹什么?”

也是,看看袖子前襟就知道。他叹口气,看着前方。吴思若还是想不通,胳膊肘怼了怼小五子,问道:“他们俩弄出来的孩子,为什么说是你的呢?”

“当年以为我死了吧,说是昆仑公子的,死无对证。”

“可你早知道,是吗?”

“拜过堂之后知道的。”

“那你早讲啊。”

“人家的事情,我讲出来干嘛\footnote{原文“干吗”}?”

“那是人家的事情吗?你跟人家拜堂,那是我和你的事情。”

还好声音不大,不然喊出来,像是这边也要吵一架。六公子似乎听到点动静,转身往这边看一眼,一片漆黑,也不见人经过,转回去继续跟乔文君解释着。这边的两个人不说话了,撑起下巴并排看着前方,看着那两人头顶上的月亮,仿佛他们不是趴在泥浆里,而是坐在屋顶
上荡着腿相互依偎着赏月。那些风也变得暖了,泥土也变得芬芳了,情不自禁地要把手从泥浆草根里穿出去,去握对方的手。

这边如此美好,那边却越吵越厉害,最后六公子撂了句狠话:“既然他都是太子了,你快早早跟他进宫,日后做你的皇后罢!”

乔文君“你你你”地答不上来,负气跑下了山。六公子要追下去哄她,下山之前,他还是要过来这边看看,嘴上说“文君别走”,脚上几个大步跨过来,拔剑便朝小五子刚才躺倒的地方扎下去。上来就下死手,对小五子杀之而后快,吴思若脸色都变了,小五子把她手握得更紧一点,仿佛六公子要杀的不是他,而是她吴思若。

两剑下去,六公子也知道这泥巴里没有人,他剑尖冲地,在周围十步开外平趟了一圈。他停下来,四处张望,目光扫过这边时,并没有看到月光下的两个泥人。乔文君已然下山,渐渐消失不见,六公子只好放弃这一片泥地,提剑追了下去。

小五子和吴思若听着他脚步越来越远,看样子已不会再回少林寺。他拉着吴思若从泥浆里站起来,双腿早就发麻,一下子站立不住,小五子一个趔趄,用手撑住地。吴思若在身后问道:“他为什么要杀你?绝不至于蠢到怀疑你,嫉妒你,要把你杀死的程度。你和他到底有什么过节?”

“我不知道。”小五子朝东边望过去,天已泛白,朝阳之下,露珠化成一层层的水气往上升,他抹了抹被泥糊住的脸,说道:“叫上李准驸,我想早点回皇宫看看。”
\newline

{\centering\subsection{3}}

到了京城,小五子要给吴思若两套好衣服,作为少林泥沼相救,把白衣弄脏的补偿。千挑万选,吴思若在集市选中了两件,再往下就不知道哪件好了。小五子说,两件都要了,这件你见五公主,这件你见父皇。

吴思若吓了一跳,原来在这儿等着我呢,她坚决不跟小五子进宫。小五子说,那不行,文思清都没跟我去,你吴思若再不跟我去,显得我小五子出来这三年,一个女人都没捞到,这就不是面子的问题了,这是有损国威啊,这事要是传到天竺、东瀛、高丽,会被诸国王子取笑的。

明白他在开玩笑,吴思若也没法跟他较真,原则性地直说不去。小五子悄声问她,是不是顾虑守宫砂的事情?

“我小五子从来就没往心里去。”

“你真从来都没往心里去吗?”吴思若反问.

“那又能怎么样?”小五子承认有,“我恨自己没早点碰上你,这怨不了你。”

吴思若道:“总有一天,你的后宫佳丽越来越多,你就犯不上在我这里遭这份心罪了。”

小五子坚持要带她,一激动还说出了,直接封你为太子妃的这种话。吴思若急了,吼道:“我吴思若配不上你行不行?”

小五子撂下狠话,哪天我要是看到你配上谁,我就杀了那个人,我把你能配上的男人全杀光,我让你今生今世,能找的就是我小五子一个男人。吴思若还是摇头,跟他讲,你要是要的话,我现在就给你,但是别让我跟你回去了,不然哪天真有太多的事传到你耳朵里,那可真是有损国威了。小五子眨巴着眼睛看她,他当然想不到,吴思若在讲自己的出身。他只说:“你要是不跟我进宫,咱们就在这儿耗着,我也不去做太子了。”

说到做到,小五子果然不提进宫的事,叫李准驸备了些家伙事,每天跟个纨绔子弟一样的遛鸟、斗蛐蛐。吴思若开始以为,他在赌,赌谁先服软。后来发现,他真的不在乎要不要当太子。她想到权宜之计,先跟他进去看看.等到小五子跟五公主,跟父皇相认了,自己再找机会溜出来就是了。

改主意的事告诉他,小五子就要把鸟笼蛐蛐笼砸了。当时正好赶上五公主知道李准驸回来,想在宫中密请他。李准驸荣幸之至,觉得这是驸马的待遇。小五子觉得这样有意思,他要吴思若和一只手跟他一块儿化妆成李准驸的跟班,混入皇宫,找好时机,再给五妹一个惊喜。

几个人照小五子在丐帮的方法化,脸上涂上黑泥、黑黢黢一片,你要是不动,都不知道,那几个是人。

进了皇宫,开始好奇张望。门口一个太监不让李准驸带随从进去。小五子不忿,也是有恃无恐,跳起来跟太监打了起来。公主在里面传唤道:“既然是李大人的亲信,就将他放进来吧。”

打从进门,小五子就盯着五公主,自己的亲妹妹,果然好看。就是不在皇宫里,不说她是五公主,放诸四海都是一等一的漂亮女人。李准驸也意识到,小五子可能失礼了。他忙对五公主介绍,说:“此人跟我南征北战,东征西讨,立下汗马功劳。”

公主冷笑道:“你个九门提督,有什么南征北战,东征西讨,最南是前门,最北是安定门,最东是东直门,最西
也就是西便门。”

说话间,小五子还不时地偷看公主,觉得李准驸说的不错,就那四个字,貌若天仙。五公主问李准驸:“昆仑公子这回可带回来了吗?”

李准驸答道:“小人不才,让昆仑公子从少林寺跑掉了。”

“让你去押一个人,出去一晃,小半年才回来,人还让他跑了。”五公主大怒,“我看你这九门提督,是不想干了吧?”

李准驸低头,不敢吭声,时不时偷看小五子,心想你倒是帮一下啊,我这都要被拖出去斩了,你还在旁边看热闹?

五公主又问:“可否有太子的消息?”

李准驸又看看小五子。他冲李准驸眨眼摇头。也是,太子当然比公主大,他让你演,你就放心大胆地发挥吧。

李准驸鼓足勇气,卖了关子,他说:“江湖上传言,太子已为百花谷的少谷主昆仑公子所杀。”

“你这个人真是糊涂至极,跑出去大半年,还没有看出其中蹊跷。”五公主说,“我今天告诉你这个秘密,太子和昆仑公子本来就是一个人。几年前,太子卧底到百花谷,就是为了查清楚前朝余孽,我听说那个百花谷,还在惦记着他们的沈家天下,谷中依然养着大量的宫女和太监,我们的太子就是化名为昆仑公子,进入的武林。”

李准驸问道:“那么,百花谷的人到底是敌是友?”

公主把他面前的盘子推到地上,怒道:“你这个饭桶,别吃了!”

小五子这时接话:“李大人,公主的话我都听明白了,我早跟你说过,进宫之前,先填填肚子,你以为朝廷的饭碗就那么容易端啊?”

公主瞪大眼睛,问道:“什么人,这么大胆!我和李大人说话,轮不到你这个下人插嘴!”

公主继续问李准驸:“既然,昆仑公子和太子是一个人,你跟我说一个死了,一个跑了,你到底还想不想要这个九门提督的位子?”

李准驸不敢回答,偷看小五子,想要不想要,那不是你太子说了算的吗?小五子拍拍胸膛,意思是我来。他往前走两步,故意插科打诨,言语放肆,后来甚至还说自己饿了,伸手过来抓肉吃。

五公主被激到大怒,让人把这个人拉出去斩了。小五子觉得这个时候,可以与公主相认了,这时候把脸上的泥抹掉,说:“五妹,你这是想杀昆仑公子啊,还是想杀
太子?”

五公主愣了一下,仔细看着小五子,忽然眼泪就掉下来了,摇头道:“我想你三年,你还这么戏弄我?”

这不是闹着玩吗?之后小五子给她介绍了一只手和吴思若。吴思若把长发落下,露出女容。见五公主情绪平静下来,小五子又开始胡编乱造,他说:“我三年前,就已经和吴思若拜堂成亲,只等着见过父皇,便可以册封她为正室——太子妃。”

五公主听得直皱眉,故作笑容,上前和吴思若寒暄几句,忽然之间就变脸,喊道:“来人哪,把这些冒充太子的反贼给我拖出去!”

小五子以为她也在开玩笑,没当回事,说:“五妹,你别闹了,叫人加菜,是吧?这菜已经够了,不用再加了。”

五公主干笑一声,说:“那你就多吃点,以后可就吃不着这么好的饭菜了。”

小五子这时才有点懵。公主接着下命令,说:“这几人妖言惑众,全部押入大牢!”

太子原来不行,管事的还是五公主,李准驸这墙头草左右看看,虽然不明就里,但是立即转换态度。他对五公主说:“不用再叫人上来了,这事我九门提督最在行!”

“一会有你更在行的事呢。”五公主冲他笑笑,点点头,“你也一起去地牢,陪陪他们吧!”

顷刻之间,到底是怎么了?任凭小五子自以为聪明绝顶,此时也摸不清,五公主这是什么情况。他低声问:“李大人,我这个太子是真的吗?”

“是真的。”

“那她这个五公主是真的吗?”

“是真的。”

“那押我们去大牢这事也是真的吗?”

“我看不假。”

没等小五子想明白,几人已经被五公主的侍卫拿下。她问身边太监小顺子:“李大人这次过来,都有谁知道?”

太监小顺子答:“只有李大人的随从,和门口的几个太监宫女。”

公主对小顺子说:“半个时辰内,把这些人全找到,杀掉。”

接着她问小顺子:“什么该说,什么不该说,你都知道吧?”

“出了这门,我就是个哑巴。”

“也不用,”五公主说,“有人问起,你就说,李大人接
回一个假冒太子的人,已经被我处决了。”
\newline

{\centering\subsection{4}}

小五子在牢里想不明白,五公主已经在外面,忙着给他这次的冒失擦屁股。关进大牢的第二天,三王爷就来了一趟宫里,就如一只野兽,闻到了猎物的气息。三王爷带人在宫里转了一圈,没看到小五子。

他找到五公主,打着哈哈,刚泡好的茶,还没喝下第一口,就忙不迭地问:“听说太子回京了,你们兄妹团聚,我这做皇叔的很是高兴,给你们来送份贺礼。”

五公主倒是不着急,慢慢喝两口茶,告诉三王爷:“三皇叔的消息果然灵通,的确来了一个冒充太子的小贼,我估计他就是奔着三皇叔的这份重礼来的,此人被我当场戳穿,就地处决,三皇叔有空也帮我查查,这些人什么来头,谁在给他们撑腰?”

三王爷脸色大变,说:“我哪有时间去查五公主要找的人啊?”

“我听说,这几个小贼是从西北方向过来的,那不正是三皇叔您的地盘?”

“西北大了,要是各个为非作歹的人,都拿我来是问,怕是皇兄醒来,也不肯啊。”

茶果然一口没喝,三王爷带人离开。他碰了一鼻子的灰,出宫后就让亲信查明,当天出了什么状况,这几人身在何处。

查也是白查,亲信到晚上回报说:“宫中的太监和宫女全部换掉了。”
\newline

小五子这几天一直在地牢,出来这一年,早就习惯被关起来了。三间联排的大牢,小五子在最中央,左边是一只手,右边是李准驸。彼此看不到,但说话能听见。小五子吩咐他们在地上掘洞,挖到够他们钻过来时,两个人再往中间汇合。

他一边命令他们快点挖,一边催问道:“吴思若在哪间牢房?”

他知道问了也白问,一起关进来的。小五子不知道,他们俩当然也不知道。小五子喊了几声吴思若,不见她应答。此时小顺子带领一帮侍卫进入地牢,一路走到小五子牢房前。挖洞的两个人也停了手。

虽然叫小顺子,他年纪也不小了,起码在宫中待了个十年八年的,自己是真是假,他早该知道。小五子上前
走几步,问道:“你可认得我?”

“自然认得,”小顺子倒是没隐瞒,“但我劝太子啊,最好不要再跟奴才多说话了,要不然这些人全得死,我小顺子也性命不保。”

说完小顺子毕恭毕敬地站在牢门口,貌似等一个大人物。过了一炷香的工夫,地牢铁门打开,五公主走了进来,她让狱卒打开牢门,进入小五子的牢房。

五公主看看地上的残羹剩饭,问:“这是你一天的伙食?”

“你还有脸问我?”小五子哭笑不得,“我们丐帮的伙食都比这强。”

“那你就别吃了。”

五公主一脚把地上的残羹剩饭踢翻。她把门口的两名狱卒叫到跟前,问其中一个:“这饭菜都是你送的?”

狱卒点点头。

五公主给他左右脸各一个耳光,转身又问另一个狱卒:“他这身囚服,是你给他换上去的?”

另一个狱卒也点点头。

公主再来两个耳光。

两个狱卒不解,嗫嚅道:“启禀公主,地牢就是这样的饭菜,关进地牢的也都得穿这套囚服。您这是说,伙食好还是不好啊?”

“你们知道这里关的是谁吗?出了地牢,我都得给他叩首下跪!从现在开始,他一天的饭菜,由御膳房供给,把这身囚服给我换了,要是这人在牢里出一点毛病,你们别想再活着见我!”

小五子问道:“你想把我伺候到什么时候?两年前我在钱记肉店的时候,也是这样对待将要上砧板的猪的。”

“那哥哥就在这儿一直待到过年吧,”五公主似乎被逗乐了,忍住笑,一本正经地说,“真要是杀猪的时候,我再请哥哥出去帮忙。”

“既然你肯叫我哥哥,我也就大胆问一句,”小五子提议道,“你把我的夫人吴思若,也跟我关一块儿吧。”

“吴姑娘我有更好的安排,”五公主笑道,“哥哥就不要操心了。”

她转身对小顺子说:“把吴姑娘推到午门问斩,这点绝无半点虚假,立即斩首!”说完她对小五子笑笑,那表情似乎还有些许妩媚,她说:“你确实是太子,以后要做皇帝的,我杀了你的女人,那时你尽可以报复我,但现在,还是我五公主说了算。”
\newline

{\centering\subsection{5}}

李准驸和一只手连夜挖通地道,把三间牢房打通之后,三个人一起进了西面一只手的牢房。一只手说:“我早已察看好地形,把两侧的墙通开,我们就可以逃出地牢。”

小五子问李准驸:“一般要犯在午门是几时问斩?”

“午时居多。”

小五子让他们抓紧往西边挖,说一定要在午时之前逃出去,拼了命也要救吴思若。他一直惦念着这件事,充满愧疚,要不是我逼她跟我一起进宫,她也不会遭此大难。他只能抠着手指数数,盼望早一点出去。直到一只手喊着“挖通了”。小五子急着钻过去,发现又是一间牢房。小五子让他们继续往西挖,李准驸提出一个问题:你说,我们午时之前要赶到午门,但问题是,我们在里面也看不到天色,现在是什么时辰都不知道,没准吴姑娘早死了。小五子瘫坐在地上,大概几秒钟之后,吼着叫他们快点挖。

“活要见人,死要见尸。”
\newline

而那边,小顺子已经在跟两名刽子手交代,把吴思若的头蒙上,带到午门按期发落,一会儿把人头给我提回来。

一路带到午门,说是秘密处决,不希望有什么变故。等到午时,两位刽子手看看沙漏,其中一个摘下她的头套,说:“时辰差不多了。”

另一个刽子手拿出纸笔,问她还有什么要说的。

“我们哥俩干了十年刽子手,杀人无数,我们得让每一个死在我们刀下的人,明白你的死跟我们哥俩可没关系,以后做鬼,也别找我俩麻烦。”

吴思若没什么想说的,要说也是心里话,也许跟小五子真的是相见恨晚,没能让他喜欢上一个干干净净的自己,要是还能有下辈子,我肯定在人海茫茫里早早地把他找出来,一辈子跟着他。头一个刽子手又问一遍,有没有什么要说的。吴思若摇头道:“无话可说。”

另一名刽子手却接过纸笔,在纸上写了几个字,说:“这是你讲的,按个手印吧。”

吴思若看到上面四个大字,无话可说。将死之人,却忍不住笑了。吴思若手指伸嘴里,使劲一咬,借着血按了个指印,眼泪也滴在了纸上。拿着头套的刽子手问,那就让我们给你戴上头套吧。吴姑娘说:“不必了,能否劳烦大哥拿个铜镜,摆在我面前,我想看着自己死。”

奇怪的要求,但还挺特别。铜镜递过来,放到地上,吴思若低头看着镜子里的自己,等刽子手举刀下刀。忽然镜子里出现一个蒙面人,对着持刀的刽子手拍了一掌,刽子手手起刀落。吴姑娘一闭眼睛,再睁眼时,一个人头滚到了她的身前。

头一个刽子手慌了,扔刀就跑,蒙面人一跃而上,身后一掌将他击毙。吴思若明白此人要救她,她刚要说话,蒙面人冲她摇了摇头,示意她快走,走得越远越好。吴思若原地站了一会儿,扭头离开了午门。
\newline

连续挖了几个牢房,每个牢房里面都关着一个或是神志不清,或是早已绝望的重犯。一只手后来提醒他:“我师姐可能早就死了,而且这个牢房没个头,我们还是想别的办法吧!”

小五子问:“你告诉我什么办法?你告诉我,有什么办法!”

一只手无奈不语。小五子趴在地上,徒手往前挖。外面有狱卒进门的声响。李准驸建议:“咱们还是快点回去吧,他们要是看你不在牢房,不定出什么事了。”

小五子不愿前功尽弃。剩下两个人对了下眼神,明白进来的狱卒是送饭的,太子的饭菜比他们的要好得多,挖了这么久,早已饥肠辘辘,不然等吃饱了,再回来继续帮他干吧!

想想而已,太子在上,李准驸怎么敢抗旨?三个人一句话都不说。李准驸忍不住了:“好吧,我去挖。”

小五子回头看着一只手,最后他也顶不住了,趴下来帮忙。一只手和李准驸一边挖,一边低声抱怨:“这一夜加一天,我们挖了六十多间牢房了,整个地牢被我们打成了一个大通铺。”

“我怀疑它是个圆形,”李准驸说,“再挖几天,估计咱们就能回到最初的那个起点,吃上好的饭菜了。”

那为什么不直接回去等饭菜呢?这问题像咒怨一样,一直缠着他。忍不了的时候,李准驸一推手,说:“我不干了,你们爱谁干谁干!”

“我命你继续挖,你在违命抗旨?”小五子反问。

“你算什么啊,命令得了我?我跟你说,我早就在外面找人托关系了,没几天就放出去,用不着跟你在这里下苦力!”李准驸喊道,“像你们这样,打通里面没用,打通外面才是王道!”说完,李准驸又撅着屁股一间一间地爬回去。

一只手看得瞠目结舌,指了指李准驸离开的那个
洞,说:“五帮主,那你也别可着我一个人一只手使唤了,我房里有牌九,我去拿回来,以后谁输了谁去挖。”

说完他也钻过去了,一时半会儿没再回来。小五子知道只剩下自己了,开始一点点地挖,为了解闷,他就不断地自言自语,然后忽然站起来,这些话都是以前对吴思若讲的,可是,世界上再也没有她了。后来挖不动了,他就坐在原地,像个孩子一样哭了出来,他的记忆只有三年,那么他是不是应该像个三岁的孩子一样,一下子承受不了这么多的劫难。

放声大哭以后,他感觉好多了。更加拼命地往前挖,挖到最后一间,牢里坐着一个活死人,披头散发双目紧闭,地上的饭菜早已发霉,看来已多日没有进食。小五子急着出去,也没有搭理他,继续挖了两个时辰,土已经挖光了,露出来坚硬无比的花岗岩,这应该就是牢房的尽头了,可是他却一点办法都没有,挑着地上发霉的饭菜吃个精光。吃完探探老头的鼻息,自语道,有吃有喝的,你练什么闭气大法啊。

没有任何回答。过了一会儿,小五子觉得此人面熟,想了半天才记起,此人是何员外家的老管家。小五子对他说:“你不是死了吗?怎么又躲这来了?不管你为什么在这,我正好想跟你打听个人,你们何帮主的师父向问和长什么样?现在在哪儿?”

小五子\footnote{原文“小五连续用了几招”}连续用了几招,揪揪他耳朵,对着他耳朵眼嚎叫:“饭菜来了!”

此人还是一动不动。小五子就开始把他当沙袋弄拳脚,但他就是坚如磐石,雷打不动。到后来,小五子都折腾困了,躺他旁边睡了一觉。

睡到半夜,小五子忽然想通了,拍着脑门说:“我笨死了,你就是向问和,四处找你都找不着,躲地牢可真是好地方。”

之后几天,他出奇地兴奋,对着狱卒每天送来的饭菜查日子。到腊月初八那天,小五子开始觉得,向问和身体有异象,浑身在抖,到后面越来越厉害,整个牢房都开始跟着颤动。小五子记起谷主告诉他的两个穴位,按着次序点下去,大概一炷香的工夫,向问和忽然倒地,小五子以为记错顺序了,不小心错杀了向问和,随即跪地,磕了几个头,我小五子既然犯下弥天大错,就在此跟你守灵三日。

守到第二天,他挺不住了,竟然睡着了,醒来看见地上几十碗饭菜,全都变成了空碗。向问和早已醒来,自言自语说自己没吃饱。小五子激动了半天,说:“前辈,我真
担心您好不容易活过来之后,再撑死了,再说这些饭菜都馊了,就算没撑死,也得丢半条命。”

向问和打了两个响嗝,盯着这个年轻人,问这胸前的两个穴位是不是他点的。

小五子讲了何府灭门的事情,讲了百花谷谷主教他的点穴顺序。向问和沉默许久,连叹几口气,询问小五子,是何人灭了何员外一家。

“那人蒙着面,”小五子说,“其实你三位师哥我都见过,跟他们说过话,但是仔细想想,我还是无法判断出是哪一位,因为当时蒙面人说话的时候压着嗓子。”

向问和问他是怎么讲话,小五子学了几句,向问和说,此乃气声。说话时声带未动,连男女都无法分辨清楚,更别说是哪位师哥了。说完这些,他不想再提何府了,他说这段时间最担心的是,大弟子何振生太粗心,忘了自己心肺异于常人,反着长的,还是师姐最了解他。

向问和说,他这神功练成之后,还得用一个月才能恢复元气,现在体力与常人无异。等一个月以后,他就能把小五子从这深牢大狱中救出。
\newline

{\centering\subsection{6}}

小五子在牢里面呆着,陪着向老前辈。三王爷当然是不放心,都说在宫里见到了太子,总不至于是捕风捉影。没隔几日,他又带人去宫里转了一圈,后来找到五公主,跟她商量道,倘若太子在宫中,就让他见一面,叔侄二人聊聊登基大事,倘若太子不在宫中,下落不明,那么三日后,即为正月初一,当初的三年之约也已到期,天下人可说不得他三王爷是夺权篡位之人。

“太子武艺微末,怕是早被三叔借机杀掉了吧,何必,到这里找我要人?”公主讥讽道,“至于登基的事情,我五公主说话算话,父皇三年未醒,太子不见踪影,自然该由三叔料理朝政。只是离登基之日还有三日,我劝三叔不必心急气躁,一副胜券在握的样子。”

听起来话里有话,那就更加让人不放心了。三王爷派人打探,得来密报,皇宫地牢中某间班房,关押着一个穿华服的男子,饭菜异常丰盛,是从御膳房每日端来伙食,料定此人必是太子。

三王爷眉毛一挑,连夜派人把此人秘密押回王府。抓捕持续了一夜,直至次日,见到此人,三王爷愣在原地,面前被抓来的是九门提督李准驸。

一只手是看着李准驸被带走的。本来他先占住这
里,太子的牢房,每日有御膳吃,每天有新衣穿。李准驸过来把他赶走了,让他回自己的牢房,他说你要是听话呢,等我出去后,自然会把你也带走,不然小心我以后出去弄死你。一只手没办法,钻回到自己牢房,每天再有御膳,也就是闻闻味儿,从洞里看看,今天又是什么好吃的。李准驸被带走那晚,一只手是在隔壁听见的,只听一阵骚动,牢门打开,几个狱卒把李准驸提走了。一只手那时还自言自语道:“果然把外面打通,要比把里面打通好使。”

五公主是第二天听说的,三王爷从牢房押走了一个人。她赶紧派人去看,回报说,太子依然在牢中,只是他们误抓走了李大人。其实这个人也不是太子,换成了一只手,小五子正在向问和的牢房里,跟他谈天说地呢。但五公主不知道,她惦记着离登基之日只剩两天了,她命人把太子带回宫中,沐浴更衣,准备登基。

一只手是第二个被带走的,他想李准驸果然讲信用,派人救他来了。只不过救他来的是一帮太监宫女,他们把他抬出牢房,好吃好喝伺候着,直到换上龙袍的时候,一只手吓坏了。再傻他也知道,这是皇帝的衣服,龙袍加身,难道他才是太子?

一只手仔细回想了一下自己的过去,不像小五子,他所有的事情,都还记得清清楚楚,后来他认定,他以前的爹妈一定是养父母,老皇帝定是有什么难言之隐,把他寄养在那里,现在是回宫登基的时候了!他双臂从龙袍里抖出来,看看自己仅存的一只手说:“原来这才叫一手遮天啊!”

有太监过来禀报,说五公主在外面候着,准备向您请安、请罪。一只手想到五公主的阴晴不定,心狠手辣,连忙让太监回复说,太子累了,五公主也早点休息吧。

可是他却睡不着,可能是这辈子也没睡过这么好的床,他把伺候他的小太监叫过来,说:“咱俩换床睡呗,太软的我没法睡。”

小太监诚惶诚恐,跟他换了房。当晚一只手听到一阵窸窸窣窣的声音,有人密报,小太监死在了太子房里,凶手已不知去向。一只手知道是冲他来的,他让知情的人先瞒着,谁也不许说出去。

这些小太监吓死了,他们早听前辈太监讲过,凡是听了不该听的,见了不该见的事情,必死无疑。其中一个争宠的小太监,主动过来对一只手说:“太子,您应该把知情的人全杀掉灭口,这事儿由我来办,我出了这个门,保证就是个哑巴。”

一只手看看他,又看看诸位宫女太监,对众人道:“你们把他杀了灭口吧。”

一帮宫女太监的火早就憋大了,一起扑上来,捂死了这个小太监。

经历这一番风波,一只手冷静下来了,他传密旨,说要见见他的五帮主,他知道说他一只手是太子,无非是自我欺骗,他只不过就是住在五帮主的牢房,吃着他的御膳,穿着他的华服,被人错认的太子,大难不死,他明白了,太子这差事,也不好当。

天亮之前,一只手带侍卫进入地牢,走到最深处的牢房。小五子看见他的服饰吓了一跳:“你别说你这太子之位,是玩牌九赢回来的。”

一只手叹了口气,让侍卫打开牢房,跟他商议这两天遇到的蹊跷之事,想跟小五子换衣服,一只手这边脱下来,小五子还没来得及穿上,又一帮侍卫进了大牢,把打个半死的李准驸送回到牢中。
\newline

五公主这边却找疯了,第二天不见太子的踪迹,登基大典马上又要开始了。五公主令太监总管小顺子,就算把京城翻了个遍,也得把太子给我找出来。太监总管提醒公主:“不管怎么说,您得上朝了。”

五公主是硬着头皮进了大殿。文武百官到现在都不知道,今天谁当皇帝。没看见太子,自然是三王爷坐定了皇帝的位置。其中一个大臣,洋洋洒洒宣读了一个多时辰老皇帝的丰功伟绩,而老皇帝却一直睡着,不省人事呢。

有人喊着:“时辰已到!”

那个大臣也真是厉害,文章还剩那么长,说收就收,一句简短总结,说臣子们拥护当今圣上为太上皇,万岁万岁万万岁!众人也不能反驳,跟着一起三呼万岁。五公主不干了,说:“我父皇鞠躬尽瘁,岂能三言两语所概括?不行,把文章读完!”

大臣愣了一下,继续读稿子,又读了一个时辰,百官皆已困倦,哈欠连天。令官又一次喊:“时辰已到!”

那位大臣的文章再次卡在嗓子眼,重复总结道:“臣子们拥护当今圣上为太上皇,万岁万岁万万岁!”然后众人再次三呼,声音却比之前微弱了许多。

五公主再次反驳:“我父皇执政二十余年,岂是三两个时辰所能概括,不行,必须把这些全部宣读完!”

大臣继续宣读:“嘉和八年,二月初五,圣上赏菊,赞菊花之美,乃为天下花卉所难及。嘉和八年,二月初六,圣
上探望赵贵妃,说,希望你身体能快些好。嘉和八年,二月初六下午,圣上赞御膳房厨师创建的一道新菜,命名为鸡跳墙。嘉和八年,二月初六傍晚,圣上二次探望赵贵妃并送去一朵菊花,及鸡跳墙,赵贵妃病情仍未有好转,一刻钟后,遂去王贵妃寝宫过夜。午夜过半,圣上兴之所至,把王贵妃、李贵妃、杨贵妃都招入房中,共议国家大事。嘉和八年······”

“差不多够了。”三王爷起身打断大臣,说,“早上天没亮就开始大典,现在天都黑了,皇兄的丰功伟绩还没讲到一半。”

公主接道:“那就让文武百官早些休息,明日继续。”

“说好今日登基大典的,五公主为何迟迟拖延?”

两派党羽争执不下,五公主知道今天顶不过去了,只好宣布大典开始。传令官喊道:“恭请皇上登基!”

三王爷整整衣衫,缓步走到宝座前,转身对文武百官朗声说道:“今天大典辛苦各位爱卿,这皇位我受之有愧,实乃我皇兄膝下无子,顺位于我。”

众人皆下跪,三呼:“万岁万岁万万岁!”

三王爷头一次被如此欢呼,有意拖延了几秒,整整衣衫再喊了句:“列位大臣平身!”

可是百官仍长跪不起。

三王爷整整衣衫又说一次:“众位爱卿平身!”百官还是不起。

三王爷脑后冒出一个声音:“平身!”百官齐声答:“谢万岁。”

然后他们纷纷起身。三王爷以为劳累一日,出了幻听幻视。他回身一看,只见小五子身穿龙袍,早已坐在九龙宝座上。
\newline

{\centering\subsection{7}}

小五子登基后,立即办理两件事情,第一件事是,命人寻查当年将文宰相满门抄斩的罪魁祸首,第二件事,筹备带兵攻打海南岛,平复逆贼南海真人。这两件事头一件事是为文思清所办,后一件是为苏子瑶所办,唯有吴思若被问斩之事,他无法立即复仇,思量着如何进展。

向问和老前辈已被接入宫中,小五子封他为御前大将军,这官职具体干什么也不知道,但御前两个字他明白,就是在皇帝身边呆着。小五子有时候想和向老前辈学武功。向问和跟他说,当年师父教他无为神掌的时候,第一件事就是自废武功,心无旁骛,方可继续往下练。

“那么,陛下之前练的是什么功呢?”

是啊,什么功呢?小五子随口一说,说是神掌神力。向问和好奇,那是什么?问题是小五子也不知道。他说今日太晚了,明天展示给你看。

晚上小五子命人将桌子锯掉一角,再稍许粘合。第二天拉来向老前辈,要跟他比比,是他的神掌神力厉害,还是向老前辈的无为神掌厉害。小五子先一掌劈下那个桌角。然后轮到向老前辈使用无为神掌,一掌劈下去,桌面晃动,地上都震得起了尘土,感觉都要山崩地裂了,抬手时桌子却纹丝不动。两个人屏息等待,一般不都是这样的吗,看起来没变化,等上个一分半分,整个桌面会突然崩塌。

可是这次没变化,一顿饭都吃完了,桌子还完好无损地立在那儿。向问和自己都不敢相信,自己花费二十年练成的掌法却毫无威力,他一再摇头,又反复说,师父一定是另有深意。

小五子也够讨人嫌的,地上抓只活蚂蚁放到桌面,说:“我知道,这劈桌子也实在是难为你,咱们先拍死只蚂蚁,怎么样?”

奇耻大辱,向问和苦笑一下,一掌拍下去,一样的山崩地裂,手掌一开,那只蚂蚁在桌面上一动不动,没一会儿,竟毫发无损地爬走了。向问和看着自己手心,半天说不出话。

小五子哈哈大笑,道:“你这无为神掌果真是无所作为呢。”

向问和苦思冥想,一夜之间,竟然满头白发。他把自己关在小黑屋里苦苦练习,练到深处,不止是桌子,甚至连一张白纸他都无法击碎。他觉得自己废了,主动找到小五子,向他辞去职务,说:“别说是御前保护皇上了,自己连个手无缚鸡之力的书生都不如,何况还封我个御前大将军。老夫身上有张羊皮,献于陛下,也算是感谢救我出关之恩。”

小五子接过羊皮后,说:“向老前辈,你为人忠厚,出去后,别把失掉武功之事告知旁人,武林人士忌惮无为神掌之名,量他们不敢与你为难,这样也可保住性命。我小五子行走江湖数十年,你见我会武功吗?不会,我今天跟你说实话,我之所以能慢慢坐到了皇帝这个位置,也是有自己的处事原则的,就是嘴上凶一点,能吓走对手最好,吓不走对手,就赶紧跑路。”

把向老前辈送走,小五子又要找点新乐子。虽然李准驸和一只手都不怎么样,但总算是同甘苦一起过来
的。他想给他们加官进爵,李准驸原来是九门提督,升什么好呢,有天忽然有灵感了,让人在京城又开凿四扇城门,命李准驸为十三门提督。李准驸似乎也没那么高兴,九门,十三门,不都是北京城吗?

可一只手,实在不知道给他什么好,小五子想到可以成立一个反赌协会,让他做会长。小五子命他以后在膊,要将你断了手的手臂挥舞给大家看。荣升会长后,一为官执政的思路很清晰,先用三个月,把那些太监宫女培养成赌徒,再大刀阔斧地去反赌。

五公主还是隔三岔五地向小五子请安,这次行君臣之礼之后,小五子迟迟不喊平身,就让五公主在地上跪着。小顺子在旁边劝道:“公主最近身子不大好,陛下就让公主起来吧。”

小五子笑道:“你这小顺子倒是够忠心的,但是我听说啊,我不在的这三年,京城百官送你的银子也有十万八万了吧?”

然后他递给小顺子一张名单,小顺子跪着爬过来接住,看也不看,直喊冤枉,小五子对公主道:“他说冤枉,难道是我查错了?要不五皇妹,你来查查?”

公主依然跪着回答:“小顺子跟我几年,一我信得过,二就算他真的拿了点碎银子,也不是什么大事,还请圣上放他一条生路,我让他把银子全退回来就是了。”

“皇妹快快请起,一时间跟下人动气,竟然忘了你还在跪着。”小五子说完吩咐人过来,同时做了个手掌下劈的手势,说:“斩了,就在这给我斩了,让我和公主都看着。”

侍卫抽刀而出,刀起刀落,小顺子人头落地。小五子让侍卫别把尸体拖走,先留在这儿,我和公主要聊两句话。侍卫退下,小五子和公主之间隔着尸体。公主不敢直视小顺子的人头。小五子问道:“处斩吴思若的事,是他办的吧?听说办得还不错。”

“命令是我下的,跟小顺子没有关系。”

“皇妹不是生气了吧?你杀我夫人,我动你一个下人都不行吗?我在牢里面天天想,穿着这身龙袍也在想,吴思若怎么着你了?二话不说你就问斩?”

公主回答:“过去的事,你都不记得了。”

“记不记得,关吴思若什么事?”

“刚才你跟小顺子拉了一笔账,我也跟你聊聊吴思若。嘉和十三年七月,杭州的紫竹院招进来一个16岁的小姑娘,不出一年,嘉和十四年,她已经是紫竹院的头牌,
甚至是整个杭州城赫赫有名的吴思若。浙江省巡抚张峰瑞,杭州知府李金,这两个人都曾经点过她。你要是想要知道这个姑娘是谁,也不用问我五公主,要是圣上有机会,下一趟江南,找这两个人打听一下,就知道我为什么要杀吴思若了,为什么觉得她不配你了。”

小五子瞪大眼睛看着,好半天没说话。他喊人进来,让他们把小顺子尸体给收了,他看着下人在清理尸体,转身对公主说:“我们俩配不配的事,你说了不算,但是你和谁配不配的事,却是我这个做天子的说了算。你也不小了,我不在这三年,你代父皇治理朝政有功,现在我回来了,登基了,你也该找个人嫁了吧?”

那么嫁给谁呢,得给五公主找个最“般配”的人,最好是恶心她一辈子。早朝结束后,小五子调侃李准驸说:“李大人,我第一次见你的时候,你身边的跟疯子似的一妻一妾哪去了?”

李准驸愣了一下,义正词严道:“那是朝廷的要犯!罗刹国派来的女间谍!腐蚀我们朝廷的官员!密谋里应外合颠覆政权!”

小五子笑问:“罗刹国人不应该金发碧眼的吗,怎么长得和我们一样啊?”

李准驸说:“她们自幼学习中原文化,潜入我国已久,所谓近朱者赤,耳濡目染,所谓文化对人的影响,不但说话口音全改了过来,长得也和我们越来越像。”

“你觉得我会信吗?”小五子说,“我就是告诉你,以后要注意,因为你不再是十三门提督了,你的这些问题,以后整个朝廷文武百官都要盯着,你就要当驸马了,我准备把五公主许配与你。”

李准驸“扑通”一跪,谢主隆恩。

翌日,早朝小五子宣布两件事,一是不顾百官的劝谏,追封吴思若为皇后。二是择吉日,将五公主嫁给十三门提督李准驸。五公主得知消息后,在宫中闹了一通,堵住小五子道:“我等你三年,你这么对我?”

小五子让太医开些镇定的药方,让公主服了之后早些休息。公主在夜里几次哭醒,却没有力气起来。礼官过来问公主的出嫁日期。小五子问他该是什么日子。礼官拿起黄历,挑了几个良辰吉日。

小五子打断他:“良辰吉日,不应该我来定吗?我觉得哪天好,难道有错吗?”礼官低头,连说:“陛下说的是。”

“那就通知五公主,明天出嫁李准驸。”

\newpage