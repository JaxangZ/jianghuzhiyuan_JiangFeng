\section{拾叁}

{\centering\subsection{1}}

那就别打了,小五子一停手,二十几个侍卫跟着一起停,之前摔出去的也都从地上爬起来,原来都是陪太子殿下玩的。他左右看看,街对面有一家文相居,看起来不错,两亩地大小,三层楼高,估计是设宴请客的地方。小五子指指文相居,按他的意思是在这儿设个宴,摆上一桌菜,两壶酒,互相先盘盘道,看看自己这太子是真是假,这个李准驸是什么来头,大家可以边吃边聊,边喝边观察。哪曾想自己话都出口了,李准驸却摇头,他说:“不用了吧,太子殿下有所不知,属下走南闯北,找了你快三年,好不容易找到了,咱们就赶紧回京城吧。”

小五子看着他,不说话,张了两次嘴,继续看着他。他在想,太子对臣子应该什么样,尤其是被下面人拒绝的时候,应该说点什么好。他还没想好怎么说呢,李准驸马上又改口了,他说:“太子殿下,不然咱们边赶路,边设宴?”

小五子上身向后靠,眯着眼睛看他,问:“怎么边赶路,边设宴?”

李准驸拍了两下手,对面文相居发出轰隆隆的下楼声,大门推开,一时间上千个手持兵器的侍卫从里面出来。这么小的门,这么多的人,全出来还得一炷香的功夫。文相居四个面,一千多个侍卫从门口出来后,分向左右,让出正门,将另外的三个面围住。

小五子看看李准驸,不知道李准驸弄这么多人,到底要干什么。又回头看看文思清、吴思若、常公公和八光,他们都跟他一样疑惑,都是震惊于这闻所未闻的铺张。

“准备吧。”李准驸说。只见他伸出手掌,冲着近前的几个侍卫,手心向上抬了抬。刚才在赌场扛镐的农夫,也是跟小五子搏斗时,摔得最狠最远的那个,竟然是他们队长。他看到李准驸的手势,点点头,向前走几步,站在李准驸和侍卫们之间,高声喊了一句:“起驾!”

全场“唰”的一声,所有人齐整整地跺了下右脚,在
文相居的三面站直,队长喊:“一!”侍卫们统一蹲下来,左腿朝前,右腿在后面下弯撑着地面。队长走过去,将文相居的大门关上,退后几步,冲准备好的侍卫们喊道:“二!”小五子看见每个侍卫好像都从地上捡起什么东西,之后双手使力抓着。队长喊:“三!”这声“三”长一些,似乎更有力,侍卫们一声嘶吼,一个个努力地站起来。紧接着让小五子瞠目结舌的画面出现了,那幢三层楼高的文相居,居然被这一千多个人,硬生生地拔了起来。

队长做了个往这边来的手势,一千个多人扛着文相居,朝他们走过来。凑近一点,小五子看到每个侍卫肩上都扛着一根铁杆,而这根铁杆,正是从楼底板伸出来,一千多根铁杆在楼底板阡陌交错,刚好能撑住底层的负重,也能令这些将整幢楼扛走。但还是太沉了,那可是三层楼啊,小五子感觉他们每走一步,后脚跟都能在地上砸个坑出来。离小五子只有两三尺远时,农夫队长喊了声:“下轿!”这些人慢慢蹲下来,弯腰将铁杆卸下来,
一寸一寸地下移,直到将文相居稳稳地放置在地面上。

李准驸将文相居的大门拉开,右手向前,请小五子进去,当然,现在不是小五子了,是太子殿下。他回头看看,示意文思清、吴思若她们一起来。文思清表情有点不对,手抓着门环,满脸通红,气都有点喘不匀,似乎随时要晕倒的样子。

小五子向前扶住她,旁边的吴思若将她搀起来。文思清好像是哭了,脸上有泪水。“是不是不接受我做太子?”小五子想,“可我又不是故意的,况且我现在是真是假都不知道。”

但也来不及多想,马上李准驸高声喊道:“有请太子殿下入座!”

这么有仪式感,这三年做梦,哪怕是白日梦,也不敢想自己会是太子。所有礼仪他都不懂,站在门口他不知道,太子进殿是先迈左脚,还是先迈右脚。反正不会是两脚一起跳进去,对吧?常公公走过来,挽住他,几乎是两脚腾空一般,大步走进去。双脚下落时,他双腿有些
发软。文相居哪里只是一个饭堂,里面雕梁画柱,长桌御宴,这简直就是宫殿,一个可以移动行走的宫殿。现在想一想,李准驸把太师椅搬到赌场,还能算什么呀?人家可是带着一幢三层的文相居走来走去呢。

吴思若和文思清在后面进来,吴思若是左顾右盼,和小五子一样,一时无法明白,这李大人是什么路数。文思清那股难受劲儿,还没缓过来,虽然也是左右地看,却越看越深情,越看哭得越厉害。最后进来的一只手可就完蛋了,一副没见过世面的样子。八光和尚也挺惊讶的,但没一只手那么没出息,毕竟前几年走南闯北,大家闺秀,小家碧玉的闺房宅子他都去过了,可是比那些要大气,这么大的房子,盖起来都得三五年,更别说是没轱辘,全靠人扛着走,可不是一件小工程。一只手还在惊叹,甩着他那空袖子指手画脚,一惊一乍地说:“五帮主,我跟你混,就对了!我早就知道,我没跟错人。说实话,能有本事把我降服的,肯定不是一般人,不是天子,就是太子!”

一只手大惊小怪的,小五子也不让他闭嘴。有这样一个人挺好,自己不知道说什么了,心里一百个问号解答不了,还需左右观察的时候,一只手可以说个不停,避免冷场。

李准驸请每个人入座,小五子自然坐主位,李大人坐他左边,常公公坐在他右侧,至于文思清和吴思若,都坐在他对面。李准驸招呼人给大家倒茶,之后拍了两下手,侍卫队长小跑着出去,很快传来他一声“起驾”的口令,接着是一,二,三。小五子坐在位子上,感觉整个房间都被拔起来了,但又那么稳,根本就没有一颤一颤的摇晃感,茶杯里的水都看不出晃动。他起身走到窗前,窗外的景色一点点往后退,整个房子确实在移动。李准驸走到他身后,躬着身子陪他看窗外。小五子看着窗外的暮色,头也不回地问道:“李大人,整幢文相居,都是你从京城带来的?”

“正是。”

“够稳的,运这幢楼,花了你不少功夫吧?”

“臣下应该的,”李准驸说,“这一千多人都是精挑细选,不只是力气大,还要耐力好,有长性,更重要的是······”

“身高还要一样,丝毫不差,是吗?”小五子拉下窗户,转回身打断了他的话。

“太子殿下所言极是,只是······有些许的······”

李准驸支支吾吾的,就是不说出来。刚做太子还不
到半个时辰,小五子就明白了朝廷里的生存法则,从下往上论,说你“所言极是”,其实就是说,你说的不对,至于什么是对的,下面的人也不敢轻易说,就给你留一口子,“只是”

“有些许的”什么的,你要是追问,我就讲出来,你要是不问,起码我做臣子的没犯错。换平常,以小五子的性格,他肯定不问,憋死你。只是这次他真挺好奇的,头两条是力气大,耐力好,这第三条倘若不是身高一样,又是怎能做到这么稳的?他让李准驸说出来,别支支吾吾的。李准驸低着头只说“是,是,是”,就是不往下说。小五子让他快点,后来提醒他:“你平常做官,怎么说是你的事,起码别跟我来这一套,以后要是在我这儿,再这么含糊不清的,小心你掉脑袋!”

“是,是,”李准驸低着头,舌头都打结了,抢在小五子发作前,抢在小五子喊“拉出去斩了”之前,赶快说出真相,“太子殿下,您其实是对的,就是身高一样,选出来的一千多人,他们的身高是一模一样的。”

小五子背手看着他,摇着头,不对,肯定不对。傻子都能从你的表情看出来,我刚才说错了,你现在又反口说我讲的对,只要是太子讲的,就都对,是不是?他不想聊了,转身问房间里的几个侍卫:“这里除了李准驸,谁还能管事?把他给我替下来!”

李准驸赶快在小五子身后跪下来,求太子饶他一命,他讲出来就是。小五子已经不指望他再说什么了,继续问侍卫:“这里谁管事?”

李准驸跪地上连珠炮似地回答:“其实开始选的标准是,身高都一样,可是您坐在里面有些颠,后来您提拔了微臣,微臣苦想了三天三夜,终于想明白,光是身高一样,还不行,主要是肩膀要一样高,毕竟这些个铁杆,都要扛在肩上行进。”

早说不就完了吗,弄得这么费劲。侍卫群里站出来一个小伙子,毛遂自荐说:“太子殿下,我行,我能替李大人把事情做好。”

小五子面带笑意地看着他,话确实是自己问的,他一时半会儿还不知道怎么回复。小伙子以为自己说错话了,指着李准驸补充道:“我说错了,不是李大人,是这个姓李的蠢货!我肯定能干得比他好。”

这回小五子知道怎么讲了,笑着重复他的话,问道:“你肯定能干得比他好?”

“肯定比他好。”他拍着胸脯保证,“说实话,这姓李的啥都不干,说是找太子,找殿下您,可是这三年,每天就是游山玩水,扛着这房子跑东跑西的,穷折腾我们。”

小五子点点头,说:“你很好,还能做李大人的事,那一会儿就看看,李准驸肯不肯让你替他做事吧。”

小伙子没明白,李准驸听出来,小五子这是要放他了,跪地上“咚咚咚”地磕头谢恩。小五子转身,让他起来吧,说:“这事其实干得不错,只是有点铺张了。”

“是,是,”李准驸忐忑地回答,“所以只带了一栋楼上路,文相家里的花园和池塘,就没有抬到江南来。”

“什么文相?”

“就是京城,朝廷里的······”

李准驸这次没想遮掩,张嘴就要回答,不过被一只手的尖叫声打断了。厨房陆续上菜,一盘盘菜由侍女端上来,在桌上摆盘,一只手惊呼道:“这里还有厨子!”他拿起筷子,站起来夹了一口蜜汁蹄髈,嘴里嚼得直冒油,看着摆盘的宫女才反应过来,又尖叫道:“还有宫女!”之后又看了看常公公,深吸一口气,说:“还有太监!全了!宫里有的,这儿都有!”

常公公想让他闭嘴,可是一张嘴,又是尖声尖气的,貌似证明有太监这事,他说得对。他叹口气,拾起筷子夹了一片藕夹。一只手凑过来,笑眯眯地问:“常公公,这菜我都没吃过,你之前在宫里,应该常吃吧?”

常公公憋着一股气,右手拿筷子,左手握着一只手椅子的扶手,感觉他把气全撒在椅子上面,弄得一只手的椅子“咔嚓咔嚓”地响。他强颜欢笑,示意一只手坐下来说:“想吃什么,够不到,常公公给你夹就是。”

一只手笑嘻嘻的,自言自语说:“果然是太监出身,就是会照顾人,回头我······”话说一半,他又尖叫起来,想往上蹦,屁股却感觉粘在上面,拔不起来。小五子走过去,看到椅子上没东西,也不知道一只手到底唱的是哪出。他问一只手:“怎么了?”

一只手额头上直冒汗,跟要断气了似的,磕磕巴巴地说:“烫······烫!”

小五子注意到,常公公的手还在握着一只手椅子的扶手,那就是发内力,把椅子逼热逼烫,甚至还要逼出吸力,毕竟一只手几次想逃,都不得不贴在椅子上,喘着粗气呻吟。

小五子问常公公:“会死人吗?”

“会。”

小五子愣了一下,问道:“所以,你不是跟他弄着玩呢?”

“没闹着玩,我想弄死他。”

这么搞有点大,小五子挠挠头,他替一只手求情,他
说:“挺好的孩子,就是嘴欠了点儿,差不多就放了吧。”

一只手也跟着哀求:“对对对,常公公,不不不,钱老板,我对你也没恶意,就是开玩笑,您就把我当个屁,放了吧?”

哀求声中,小五子看着他,好半天,常公公点了点头。小五子从桌上拿起一杯茶水,倒在一只手的椅子上,水刚落上去,椅面上就“嗞嗞”地冒着白气,温度降下来了,可是一只手还是起不来,直到常公公把手拿开,一只手“噌”的一下,从椅子上跳起来,离开饭桌,推开文相居的大门,跳下去,和扛房子的侍卫们一起步行去了。

大门打开,大家才看到,外面天色已晚,再不吃晚饭,可就要入夜了。很快就有宫女将一只手湿漉漉的椅子抽走,在原位换了把新椅子。小五子也不想回他的太子主位了,索性在这张椅子上坐下来。

刚才一只手被烫的时候,李准驸已经将那名积极踊跃的小伙子处理掉。他把农夫队长叫过来,低声吩咐他,把二五仔带到厨房,处理掉后,直接从窗口扔出去。农夫队长多问了一句:“怎么处理掉?”

李准驸盯着他,嫌他脑子太笨,他让队长凑过来,对着他耳朵,轻声说:“这种事情,你要让我讲那么明白吗?我让你把他带到厨房,我还说了,处理完从窗口扔出去,你还问我怎么处理?怎么处理,你去问厨子。”

队长恍然大悟,表示他明白,刚才是一下子想多了。他冲李准驸眨了眨眼睛,一脸坏笑地说:“简单直接最好。”

“对,简单直接最好。”

队长揪着小伙子的头发就往厨房走。小伙子求李准驸饶过一命,看姓李的不说话,他又求太子殿下救命。可是太子那时忙着救一只手呢,根本不知道,身后发生了什么。农夫的力气是大,揪着头发,能把小伙子提到半空中,三步两步就离开了大堂。之后是厨房里的惨叫,当然,全都被大堂一只手的叫声给盖过去了。

一只手逃出去的时候,农夫队长也回来了,他俯身到李准驸耳边,对他报告:“全都处理好了,他现在留在厨房,给厨子改刀、摘菜。”

李准驸笑笑,拍拍农夫队长的肩膀,夸道:“很好,非常好。”

队长貌似很得意,颇为自诩地说:“就按照您的意思,简单直接地把他处理掉。”

“对,简单直接。”李准驸笑笑,忽然变脸,质问道,“按照我意思?我他妈是这意思吗?”

队长蒙了,回想着说:“您说的,怎么处理,让我去问厨子。”

李准驸无奈叹了口气,起身指了指他,说:“把你这身皮扒下来,你也去厨房摘菜、改刀去吧。”

队长想多问两句,真心想知道,自己到底哪里错了。可是李准驸不给他机会,他走到饭桌前,到小五子面前,又一次变脸,满面春风的样子,从桌下拿出两坛酒,说自己这三年找太子的过程中,也收藏了一些好酒。两个坛子一模一样,都是陶罐红盖,他仔细辨认了一番,指着左手边的那坛,说:“这两坛都是绍兴名酒,这坛是女儿红,女儿刚生下来的那天,当父亲的,就酿下了这坛酒,起码要等一十六年,到女儿出嫁那天,才能打开品尝。”

李准驸介绍完第一坛,准备介绍第二坛。谁知被之前不吭声的八光抢了白,他靠在椅背上,扯着嗓子问:“那第二坛一定是状元红了,是不是?”

话到嘴边,被人堵住了。李准驸摸着坛子上的红布,卡在原地。八光接着抢话:“那一定是,儿子刚生下来的那天,当父亲的,就酿下了这坛酒,本来要等到儿子考上状元,再打开来庆祝。可是这一等,可不只是十六年,没准等上二十六年,三十六年,要是他子子孙孙都不争气,不学无术,等上六百年,这坛酒都不一定喝得上。哈哈哈哈。”

也不知道好笑在哪儿,八光说完,还自带音效一般,“哈哈哈哈”个不停。李准驸脸都绿了,迅速调整状态,还是面带笑意地请示小五子:“太子殿下,那咱们就打开喝吧。”

“李大人,先别急着开酒!”八光说道,“我刚才说的,对不对啊?”

李准驸皱眉看着八光和尚,不知道自己哪里得罪他了,这明显是故意找茬。能看出来,这假和尚武功不错,应该比刚刚把椅子烧烫的常公公还要好。真要是打起来,他李准驸倒也不怕,首先,他做了几年的九门提督,可不是白练的,再就是这千人侍卫,起码有几十名好手,真要围攻这和尚,可不是刚才对太子殿下那种闹着玩的把式了。可眼下不宜发作,这么多人看着,何况还是太子带来的人。他“哈哈哈”地干笑几声,吹捧道:“这位高僧果然见多识广,让李某人着实佩服。”客套话起了个头,他也讲不下去了,干脆硬起茬结束,他说:“对,你说的对。”

“虽然我说得对,可酒是你的,打开之前,你总得给
五帮主介绍一下啊。”

他还没完,这是要逼我动手吗?李准驸愣了愣,再次调整状态,摸着第二坛酒说:“这第二坛呢,叫状元红,如果生下来的是儿子,当父亲的就酿下这坛酒,等儿子日

他说了一半,实在不想重复说完。小五子把酒拽过来,拉下红盖子,示意大家,就这么喝吧。他拎起酒坛,给每个人倒上。别人都已经习惯了,酒在面前的杯中倒满,“谢谢”都不用说,他不就是小五子吗,田独肉铺的伙计出身,唯有李准驸诚惶诚恐,受宠若惊的样子,恨不得跪下来接酒。

菜也陆续上齐了,小五子看着一桌子饭菜,还愣了一下,居然全是自己爱吃的。就算有些这三年没吃过,但一看菜品的成色,也一定合自己的胃口。李准驸献媚说:“太子殿下,这都是按您的口味准备的,厨房里的四位厨子,也都是您之前的御用厨师,尽管您消失了三年,可我把他们都留住了,就等着您回来呢。”

“这都是我之前爱吃的?”小五子指着这些饭菜问。

李准驸点了点头,似乎很得意。其实不问,小五子也知道,不管吃过没吃过,都是他的菜。那么他和李准驸一定很熟,而这个李准驸自然是朝廷里的钦差,他说自己是太子,十有八九,是没跑了。小五子瞪了一眼常公公,倘若我是太子,失忆之前在宫里,他就认识我,这三年,又是装哑巴的钱老板,又是尖声尖气的太监,摇身一变,又是百花谷的沈总管,讲了那么多,没一句实话。

那就先吃饭喝酒吧,小五子端起酒杯,谁知道里面盛的是女儿红,还是状元红。举杯的时候他开了句玩笑,他说:“那这房子呢,也是我喜欢的,给我留的?”

这有点像抬杠,也没指望他回答,可是李准驸这次居然又点头,他说:“当时您说喜欢文府,责令章武水章大人,连根把文相居拔起来的,之后几次出京南下,住的都是这幢房子。”

小五子的酒杯已经在嘴边了,他停下来,盯着李准驸,问:“谁是章武水?”

“以前的九门提督,过去一直跟着您干来着。”

“现在呢,人在哪里?”

李准驸犹豫不说,小五子呵斥他,快讲出来。他说:“章大人以前官运很好,顺风顺水,一路高升,后来因为办事不力,出了点差错,被您治罪斩首了。”

小五子追问:“他出了什么差错?”

这下不隐瞒了吧,索性全讲出来,但还是要跪下来
说。李准驸说:“太子殿下明察,所谓差错,就是这幢房子,就是您刚才问我的问题,是身高一样,还是肩膀一般高?章大人老糊涂了,居然找了一千多个身高一样的侍卫,可是肩不一样高啊,行进起来自然摇晃,有一次下江南,还是这幢房子,颠得实在厉害,桌子上的茶都洒出来了。您当时龙颜大怒,问我们这些做下人侍卫的,谁能替章大人做事?你当时问了三遍,微臣那时还只是章大人的副手,可是太子殿下有困难,自然要赴汤蹈火。于是微臣站出来,表明我可以胜任章大人,保证您在里面,可以四平八稳地到达目的地。”

“然后呢?我怎么处置的章大人?”

“跟平常一样,找个理由把章大人问罪斩首了。”

小五子深吸一口气,他是个什么样的太子啊?尽管这几年,他并不觉得善良是美德,可他也从来没想过,自已会是个坏人,一个万恶不赦的恶人。那就把酒喝了吧,希望这一切都是假的。然而就算他不是太子,他也是昆仑公子,这也是个人人喊打的过街老鼠。

他仰头将手中的酒一口气喝掉,猛地一下,把酒杯摔在地上。然后他一一望着与他同行的这几个人。常公公面无表情,看起来这些他早就知道,见怪不怪。八光和尚是刮目相看的表情,没想到你小子比我狠多了,去少林寺关二十年的,应该是你才对。小五子看看吴思若,她则和他一样恍惚疑惑,只是比他多了一丝心疼,小五子没法心疼他自己,吴思若却比他还要难受。他又看看文思清,她还沉浸在悲伤之中,不是为小五子,莫名其妙地自我感伤,看着桌上的碗筷掉眼泪,似乎是睹物思人的样子。

小五子忽然想起了些什么,走出几步,推开大门,手抓着门环,上身后仰在外面,在行进的房子里看着房外的牌匾。李准驸还在喊着:“殿下,小心!”小五子已经走回到大堂,他问李准驸:“这幢楼叫文相居?”

“对,文相的府上。”

小五子走到桌前,拿起文思清面前的碗,看了看碗底,上面写着“文相家府”。他浑身发抖,坐到文思清旁边,双唇打颤地咽了两口唾沫,眼看着眼泪就要溢出眼眶,他哑着嗓子问道:“思清,整个这幢房子,是不是你家?”
\newline

{\centering\subsection{2}}

文思清是在天快亮的时候离开的,他们喝了一整夜
的酒,女儿红、状元红,两坛酒干掉,又换上汾酒、米酒。原来不只是绍兴名酒,三年寻找太子,李准驸走到哪里,都会搜罗当地最好的酒带上。房子在夜里持续行进,众人一路上推杯换盏,最后是房子不晃,人却开始晃起来了。

以前在文相府不喝酒,府上没有藏酒,无论是车夫、伙夫,还是管家,只要进了大门,都是滴酒不沾,她祖父和父亲几代为官,家族的规矩如此,不可以贪杯误事。连根拔起,抬出京城,变成了文相居,反倒是杯觥交错,一醉方休了。

是啊,文相府怎么改成文相居了呢,可能是不想太惹眼吧。以至于在赌场门口,文相居就立在街边,她都没看出来,这三层楼竟是自己的家。直到她走进来,迈进大门的第一步,看到房间里的陈设格局,一切都没有变,仿佛能看到自己童年少年时的所有痕迹,在这张桌子吃饭,在那扇屏风后面玩耍,从楼梯上跑下来,一路欢笑着跑出大门。而那时祖父还在,父亲还在,更重要的是,母亲还活着。而今却换成了这样一帮酒足饭饱之徒,在里面摆宴席。

她呼吸急促,险些晕倒,吴思若扶着她,找张椅子坐下来。桌前已经摆好的碗筷,还是熟悉的花纹图案。她拿起来,看着碗底,绝对没有错,是自己的家,是她长大的文相府,连碗筷桌椅,他们都保持原样的留着,那碗底上写着“文相府上”。

倘若小五子没发现,文思清永远不会说,她不知道自己家族的灭门,跟他有什么关系。眼前的这个失忆的、没有身份的男人,以前是小五子,后来成了昆仑公子,这次摇身一变,又成了太子,一人之下,万人之上,他到底对文家干了些什么,这些她都不确定。

坐在长桌一角,她尽量克制,可眼泪止不住地往外流,想得越多,心里越悲伤。后来终于被小五子看出来,拿起她手上的碗,问她:“思清,整个这幢房子,是不是你家?”

她大概愣了几秒钟,硬挤出一丝笑容说:“哪有,我家哪有这样气派?”

小五子还在看着她,显然他不信文思清的说法。她把酒端起来,扬声建议大家,先吃饭喝酒。“折腾了一整天,我都快饿死了。”她说。

小五子不动,其他人也不好起身举杯。文思清俯身贴着小五子耳朵,低声讲:“家里被满门抄斩的时候,我那时还小,还是个女流之辈,我什么都不知道,所以不要
再问我了。何况,你也什么都不记得,趁我们现在还一无所知,把这场酒喝完吧,他日再见,你我还不知道如何面对呢?”

小五子没吭声,不敢看她。文思清再次端杯站起来,说:“在座的都是我这半年新认识的朋友,大家为了小五子,东奔西跑地漂泊了这么久。现下终于知道,他是太子,皇宫里的人,有御前侍卫保护,以后大家再不必躲躲藏藏,就算诸位心性清高,不愿去讨荣华富贵,可至少能过上太平日子。我先替小五子把这杯干了。”

文思清说完,将杯中酒一饮而尽。所有人看着小五子。八光等不及了,他可不管这些,怎么着文思清也算是他师姐,他站起来第一个捧场喝掉。之后是常公公,表示田独承蒙你的照顾,虽为长辈,也要敬你这一杯。再后来是吴思若,也没多余的话,说喜欢不喜欢,总还是跟你一起在昆仑山庄面对过生死,在姑苏茶馆亲历了苏子瑶被杀,无论如何,也要和你把这杯酒喝掉。

只剩下小五子了,他没有起身,但把酒倒满,坐在原地一口喝掉,紧接着又倒第二杯,第三杯。太子喝酒,李准驸哪敢干瞅着,你一杯,我三杯,你三杯,我九杯。九杯下肚,他呼呼一肚子的酒气,招呼厨房,吩咐道:“把这一桌子饭菜撤掉,换上新烧的菜。”

小五子让他等下,问他:“这些菜不是刚上来的吗?怎么要换掉?”

“上来小半个时辰了,太子殿下。”

“可还没有凉啊。”

“是,可是您摸一摸,已经温了。”

小五子手指碰下盘子边,基本还算是温热,他想了想,问李准驸:“所以,我过去是这样的?这么多菜,稍微一放,我就让人重做?”

李准驸点了点头。 “以后不必了。”他说。

“本来就该换的,您是太子,又不是寻常百姓,怎么能吃剩菜剩饭?”

“我说,以后不必了!”小五子打断他。他又喝掉一杯,李准驸这回不敢跟着喝。小五子放下酒杯说,“不用重做,以后连做都不用做了,叫厨子们先回京城吧。”

“那您以后吃饭怎么办?”

“走到哪里,吃哪里!”小五子走到窗前,推开窗户,冲李准驸吼道,“你往外看看,江湖上天天有赶路的,有饿死在路上的吗?”他指着外面扛着房子的侍卫,说:“我跟他们吃一样的。”

李准驸明白了,但这着实为难,太子要跟侍卫吃一样的,怎能让他吃糠咽菜?所以反过来理解,这一千多名侍卫,吃的要和太子一样,大鱼大肉,山珍海味。这么一来,伙食费肯定不够了。李准驸说:“我着手去办。”

他走进厨房,又看见了那个要顶他的小伙子,在菜板前一边抱怨,一边切着芦笋。本来就一肚子气,眼前一个现世报正好给他撒气。他走过去,左手抓着他头发,右手抄起他手里的刀,在小伙子大腿上扎了一刀。然后他伸手要,一个看懂了的厨子又递过来第二把。李准驸瞄准后,在他左腿上又扎了第二刀。右腿那一刀,好像触及了动脉血管,血从腿上喷出来,溅到李准驸的脸上。两刀一左一右插在小伙子腿上,他要死不死地躺在地上哼哼唧唧。李准驸起身伸手,始终没表现的农夫队长,从后面递给他一条白手帕。李准驸接过来,擦了擦脸上的血,命令所有人,道:“刀就就这么插着,谁也别给他拔下来,谁要是想拔,那就插自己身上!”

没人敢出气,李准驸要求厨子把衣服脱下来,换上侍卫的衣服,同时将厨子的白衣服交给农夫队长,让他挑几个不中用的侍卫。

“穿上这些衣服,一会儿到太子面前晃一圈,说自己是厨房的厨子,给太子跪下来谢恩,”他指着脱下来的衣服说,“领赏之后,就早点儿滚回京城吧。”

队长接令要走,李准驸叫住他:“等会儿!我话还没讲完呢,你走什么走!”

农夫队长愣在原地,搓着双手,也许真是农民出身,此时要是能给他一把镐,他还能自在点儿。李准驸交待第二件事。

“现在是盘缠不够用了,”他说,“需要派人快马加急,到朝廷取些银票回来。可是朝廷也不好去。”话说一半,他又把自己否定了,找到了太子,挺大的一份功劳,人还没见着,就伸手要银子,到最后立多大功都被抹平了。他说:“这样,你带几个侍卫,扮作强盗,看沿途哪家宅子大,富得流油,去抢他几票回来!”

农夫队长问:“抢多少?”

李准驸笑了:“这话说的,你得看他们有多少啊。要是家底就一万两,你能抢出三万两?”

“全抢光?一文钱都不给人剩?”

“当然抢光!”李准驸又乐了,不是有多好笑,而是这农夫出身的队长,好用是好用,可老是问一些不可理喻的问题,还剩不剩钱给人家,为什么要剩呢,奇怪了,钱这么好的东西,没有理由剩下来一些啊。

虽然不认同,但他懂了。这次他接令离开,李准驸再次喊住他:“等会儿!我跟你说了多少遍,我话说完,你再走!”

农夫队长一脸无辜,还是搓着手,低声解释:“我以为你说完了。”

不解释还好,解释两句,把李准驸的怒火勾起来了。他起身朝队长走过去,经过地上的小伙子时,弯腰将他腿上的刀拔下来,对着队长比划。小伙子双手掐着大腿根在地上惨叫,似乎这叫声让李准驸冷静下来,真把队长杀了,他就无人可用了。他慢慢压下怒火,说:“我这回说完了,你去办吧。记得,每次都要听到,我告诉你,我说完了,才算是我讲完。”

队长俯首致意,面朝着李大人,一步步退出厨房。李准驸走回来,经过惨叫的小伙子时,弯腰将刀又插回到他的腿上。这次他已叫不出,嗓子发出来的声,光是气息,一点声音都没有。趁自己还没晕过去,他得想想为什么,可能是位置不对,好狗不挡道,死在这一来一往都能经过的地方。他双臂扒地面,拖着全是血的双腿往墙边移。

李准驸懒得管他了,转回身对那几个已换上侍卫服装的厨子交待:“以后就穿这身衣服,该下厨就下厨,如果有人来打听,你们就说,是驻扎在厨房的侍卫。”这次不等厨子走,李准驸说完自己要离开厨房,走到门口想起来,补充道:“对了,从明天开始,御厨里的饭菜做一千份,不单我吃,太子要吃,咱们这些兄弟也要吃到你们做的鲍鱼龙虾。”

“但我们是御厨,我们只给宫里做宴,”一个貌似有风骨的御厨站出来,反驳道,“我们不伺候那些下人。”

这事是生气,但怎么着,也不能杀厨子,一路上饭菜还得有人做,就算做出来,菜里面又擤鼻涕,又吐痰的,你也防不住。能怎么办呢,衣食者为大,李准驸挠头看着他,好一阵才找到一个说辞:“他们也是宫里的人,他们都是皇上身边的左膀右臂!”

他可不能再等反驳了,说完就关门走出去。进到大厅,却是另一番景象,刚刚进厨房,也就两炷香的功夫,外面已经瞬间干掉了五六坛酒,有人已经喝倒,伏在桌前,有人拿着酒坛,只要碰到还活着的,就一定要碰坛喝掉。

“五帮主带头喝的,他喝得最多,”一只手提着酒坛,站在他旁边说,“他一个人就干了三坛酒。”

这时李准驸才意识到,一只手上来了,不跟部队行军,有好酒好肉,也知道别饿着自己。他朝太子走过去,
此时小五子已经喝多,靠在座位上,不省人事,他要服侍太子就寝,转身问吴思若和文思清:“你们两个,谁是太子妃?”

两个女人瞪大眼睛,回避了那么久,那么敏感的一个话题,就被李准驸这么不经意地带了出来。李准驸又催问一遍:“快点儿!谁是太子妃,准备侍寝啊?”

吴思若看着文思清问:“侍寝,是洞房的意思吧?”

文思清摇着头,并非不是,而是不知道。李准驸问了两遍,也不见有人应声,自言自语道:“真是的,机会来了,也不知道把握。”

他蹲下来,后背对着太子,双手从身后抱住小五子的腰,一挺身把他背起来,踏上楼梯,往卧室里扛,有两个侍卫赶过来,说是要搭把手,被他一个眼神,就给撵回去了。

说实话,把握机会这一点,没人能比得上他李准驸,他自己都清楚,还顶替他的位置,开玩笑,过去十年,净是他顶别人,可不见谁有本事把他替下来。李准驸这名字,可不是盖的,花尽心思取的名字,状元中榜的文章,都不比他这三个字来的巧妙。他研究了很久,飞黄腾达的最高级别在哪儿,皇上天子,就不用说了,这要看命,别说是平民百姓,有些皇子,就算他生在皇宫,都不一定能坐上龙椅的。宰相、钦差呢,这要看运,运势好时,倾权朝野,但要是走背运,站错队,一个不小心就满门抄斩,全家九族都跟着遭殃,就像是文相,两朝元老,还是被抄家杀头。权高且牢固,思前想后,最稳的就是做驸马爷了,跟朝政党政无关,谁来做皇帝,我都是享我的荣华富贵,只要公主活着,驸马爷可是没法罢免的,哪怕公主死了,我儿子还是皇上的外孙,再说了,真做宰相,他也没本事坐上去,没本事坐得牢,驸马似乎不需要真才实干,但其实,这里面要学要练的可多了。

楼下那两个小妞,根本不行,机会都到眼前了,不知道该干什么,该怎么样把太子的心拢住。他李准驸什么样呢,一个公主我都不认识,长得肥瘦美丑,都不知道,照样敢叫这个名!慢慢来呗,老皇帝二十七个公主呢,早晚有一个是我的。可惜自己不是女的,不然此时对太子下手,一保一个准,现在是太子妃,过两年就是皇后、皇太后了。

他把小五子扶上床,看着他熟睡,犹豫要不要帮他更衣沐浴。他去撸鞋子,扒衣服,衣服脱到一半,想想还是算了,反正再殷勤,也当不上太子妃,被人当场抓包,别说是驸马爷、九门提督,怕是脖子上这脑袋瓜都保
不住。

他给小五子把被子盖好,吹灭蜡烛,黑灯瞎火的,还不打算离开,万一太子做噩梦唤人,或是忽然醒来要喝水呢,做奴才的,要心思细一点才是。摸着黑,他靠到一张太师椅上,双臂撑在倚着头静坐,脑袋不断下沉,竟睡着了。

他是被呼噜声吵醒的,刚一睁眼,呼噜声就不见了,闭眼刚睡一会儿,轰隆隆的呼噜声吵得要死。试验几次,他确定那是自己的呼噜声。那就是犯了大错,呼噜声那么大,声音洪亮不说,还时不时地转调,呼噜噜变成呜呜鸣,转眼又变成呵哈哈。以前做九门提督,干活累身体疲还好,还能睡得实一点,这三年出来找太子,不是赌场坐一天,就是躺在这行走的房子里,跟姑娘们学讨女孩喜欢的话术,居然把打呼噜这臭毛病给惯出来了。

还好太子没醒,毕竟喝了不少酒。他喝一口茶,掏出手绢,擦擦额头上的汗,黑暗中闻到一股血腥味。手绢放在鼻子下面,深吸两口,是血的味道,估计是在厨房给小伙子捅刀时,擦脸的手绢又揣回衣服里,那额头上应该是红的,手上也是血。这时他愣住了,不是为这血,是想到一件更重要的事情,关乎一生的大问题,这么大的呼噜声,哪个公主敢嫁给他啊?

他又静坐了一会儿,这次没睡着,下来之前他想明白了,真要是娶不到公主,没人敢嫁他,那就再改一次名字,李准富,反正发音都一样,改名这事,越改越有前途,李准驸之前,他爹妈给的名字,还叫李准福呢。再就是他想明白,当不上驸马爷,他就跟太子混,鞍前马后做奴才,以后他登基了,也不用封自己什么官,免得给了乌纱帽,找机会又把乌纱帽摘下来,给钱,给田,给女人就行。不当驸马,我反而可以活得更风流。

他下来的时候,酒都喝完了,一个喝得比一个大,各个瘫在地上、椅子旁,不省人事。他招呼侍卫过来,把他们扶上楼,照顾好他们,不然明天太子醒来,看到这样子,成什么体统?

果然是豪宅,一楼吃饭,二楼做菜,三楼全是客房。侍卫们两人一组,将醉酒宾客一个个往上背。李准驸看着他们忙乎,忽然问道:“少个人,那个姓文的小姑娘呢?”
\newline

{\centering\subsection{3}}

小五子再醒来时,发现文相居少的已不是文思清一
个人了,八光和尚和常公公也在夜里离开了。他猜测,应该是文思清先走。等到众人酒醉,她推开门,从行进的房子里跳出去,逆着人流往南走。他不知道她要去哪儿,也许她自己都不知道,但肯定不是田独,那已是所有憧憬承诺梦碎的地方。这文相居,就是她儿时的房子,她竟然要从这里离开,变得无处可去,无家可归。

可能走出去两三里路,八光追出来了。他和她一起离开少林,一起来的南京,当然要跟她一起走。常公公,小五子在琢磨这个人,他为什么要走,他们是去京城,皇宫啊,而那里不是他呆了二三十年的地方吗?或许是怕小五子追问吧,没别的理由了,常公公知道太多,却一句话都不讲,同去京城的话,架不住小五子一路的盘问。还不只是问话这么简单呢,他现在已经是太子了,倘若他还是满嘴胡话,没半点诚意,小五子当然要把他关入地牢,反复折磨的。

“如果是我,”小五子想,“也会跑得远远的。”

还好,吴思若还在,一只手出出进进,蹿上蹿下,也没有不告而别的意思。在中午,他到吴思若的房间里坐了一会儿,也没多说话,这时候多说什么,哪怕是一点点欢喜的情绪,都显得他俩是那么狗男女。他坐在桌前,看她读书,若不是窗外的景色不断移动,怎么看,都不像是一幢移动的房子。坐了一下午,看了一下午灰尘浮在阳光里,到最后他说出五个字:“你不要走了。”

吴思若放下书,望着他,没有答应,但也不是拒绝。小五子继续说:“起码等我查到,我到底是一个什么样的人。”

小五子说完下了楼。楼下的李准驸正躲在厨房里,一个上午都在忙着点银票,昨天夜里农夫队长带人,打劫了一家盐商,掠来三万多两的银票。李准驸数了一遍又一遍,就仿佛那不是抢来的钱,是刚从钱庄把自己的钱取回来一般。农夫队长站在他面前,足足等到他数第三遍,问道:“李大人,这些钱够了吧,兄弟们做没本钱的买卖,把人家洗劫一空,还想着送回去一些。”

李准驸没应声,他还要数第一遍,这次确切数字是三万五千六百七十七两,他眼珠转了两圈,盘算了一下,又数第五遍,将钱分成两堆,一堆是一万七千八百三十八两,另一堆也是一七八三八。多年的老规矩了,一半留作公用,一半留给自己。可多出来这一两怎么办,规矩就是规矩,放在自己那堆,超过一半,这就是贪污了,可不是好官,那是昏官,贪官,可是放在公用那一堆,他又觉得吃了大亏。思前想后,面前的农夫队长让他灵光
一现。他把一两银子扔给他,说道:“拿去,跟你的兄弟们分一分。”

队长接过这一两银子,在手里捏了捏,为难道:“李大人,您赏的太多了。”

李准驸了解他的人,知道他是真觉得多,决然不是讽刺。这种农民出身的武将好养活,肥都不用施,浇浇水,晒晒阳光,都能给你长出沉甸甸的麦穗。李准驸头也不抬,他打算再数一遍,别出什么差错,一世英名,就毁在这点银子上了。没问题,都是一七八三八。农夫队长还捏着那一两银子,说:“以前都是多出几文给我们,这次是一两,实在是有点多。”

“那就带你兄弟们,去吃点好的,李大人赏你们的,哪有要回去的道理?”

李准驸说着把钱收起来,一半交给账房先生,做这几天的开销,一半揣进自己的兜,多出来那一两,还卖了个天大的人情。这么聪明的头脑,说实话,他只想做驸马,不做宰相,主要是为了让那些大臣们有路可走。

房子继续前行,昼夜不停,跑了快十个时辰,转眼就要出山东了。下午时分,有官兵堵住了行进的道路,为首的王姓武官骑在马上,挥舞着大刀喊着:“捉拿歹人。”其他士兵跟着帮腔,话音不齐地问道:“昨晚临净县内的盐商张老爷家洗劫,是不是你们干的?”

李准驸吩咐众人:“保护太子!”

他推门出去看看。三两句话就搞清楚,大家是一起的,太子被找到的事,先不用说,重要的是,李准驸让他明白,站在他面前的可是九门提督李大人。武官反应也可以,马上换了副嘴脸,说:“昨晚临净出现了歹人、强盗,下官心系李大人,是前来保护李大人的。”

那就敲笔竹杠吧,李准驸表示,昨晚不只是那个姓张的盐商被劫,自己的行军队也被抢了不小的数目,还不知道你们把案子办得怎么样,钱有没有追讨回来。武官愣了一下,很快明白,这是在要钱,就看他要多少了。武官向他请示:“请问李大人,歹人抢走了多少?”

就还是三万多吧,搞太大也不合适。李准驸这次说双数,二四六八十一类的,免得一会儿拿到手,又是一笔烂账。武官想了想,回复道:“下官今天中午抓了一伙强盗,刚好是抢了这些数目,应该就是从李大人这里抢的。”

“人,我就不要了,给他们一次改过自新的机会。”李准驸说,“可是这钱呢,还要麻烦王统领多出力了。”

武官明白了,说:“李大人,您先赶着路,下官这就给您取,一会儿给您送过来。”等到李准驸点头,武官喊了
声;“收队!”两千人的部队,整齐划一地把路面腾出来。

下午钱果然送来了,二一添作五,一人一半,谁也不要占谁便宜。之后他便一直陪太子,他也能看出,小五子情绪不对,为官这么多年,这点察言观色的本事当然要有。于是他扯闲篇,什么离谱,什么好听,他就扯什么。他说,太子殿下,您失忆的原因主要是白龙马下凡,教了你十二字神功,神拳神腿,神枪神棍,神掌神力,这项神功练得越深,天上能记起的事越多,地上能想起的事越少,才会变成现在这样。

这马屁拍得真可以,小五子听着都想笑。呀,这不就是所谓“龙心大悦”吗?他得多问两句,说:“我都这么大本事了,以后天下都是我的了,为什么还要吃苦练功啊?你这话不可信啊。”

李准驸被问得满脸涨红,换别人,就可以说,我又不是你肚子里蛔虫,我哪知道你怎么想。跟太子可不行,他先打太极,挠着头说:“您讲过来着,下官记性不好,一下子给忘了。”

“记性不好就不要做官了,回去种田吧。”小五子轻描淡写地说,“种田再种不好,那活着都是累赘。”

李准驸搓着脸,忽然激动起来,叫道:“你苦练十二字神功,是为了一件大事,而武林中,没人有本事干这件事,你只能自己修炼!”

“什么大事?”

“飞到月亮上,把嫦娥接回来,把那只兔子也带回来。”

小五子看着他,也是够蠢的,刚才差点信了,真以为有件大事要办。他不想再跟他贫了,叹口气说:“我这太挤了,让我透口气。”

李准驸看着这块两亩大的大堂,皱眉想着,这里还挤?

“就这么一栋房子,我那些朋友都没得住,看到你能够占一间房,我很是欣慰啊。”

这回他明白了,这是要撵他走。李大人连滚带爬地跳出去,说:“属下去给太子殿下抬楼!”

小五子坐到窗口,看到李准驸果然替下了窗下的一名侍卫,跟着一起抬。小五子说:“李大人,等到了京城,就把文相居迁回原址,改为文相府,物归原主吧。”

李准驸肩上压着铁杆,浑身吃力,好半天回复一句:“遵命!”

\newpage