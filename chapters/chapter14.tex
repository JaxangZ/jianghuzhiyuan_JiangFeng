\section{拾肆}

{\centering\subsection{1}}

进京城那天,小五子在那间移动的房子里,二楼卧室的大床上,摇摇晃晃地做了个梦。他梦见文思清来找他,回到文相居,上了二楼,看见小五子在熟睡,不着急叫醒他,一声不响地等他醒过来。见他睁开眼睛,文思清说,我还是舍不得你,回来看看你。小五子连忙下床,过来拉住她,说我们一起回京城,跟我去皇宫里吧。文思清摇头,对他微笑,手指冰凉任他攥着,她说,你还是先带吴姐姐回去吧,我爹爹、爷爷,都是宫里的奸人所杀,我还不知道是谁,要是让他看见了我还活着,而且跟你在一起,我怕他背后给你使坏。小五子说,怕什么,我现在是太子了,又不是人人喊打的昆仑公子,谁要是敢给我使坏,我把他满门抄斩。最后四个字让文思清顿了一下,她把手从小五子双手抽出来。满门抄斩,那不就是文家的遭遇吗?小五子也知道,自己说错话了,他马上转话题,他说这样吧,我进宫第一件事,就是养二十条狼狗,天天不喂食,就让它们饿着,等我把这个人给查出来,逮着他,直接
扔进狗屋,把这二十条狼狗喂饱了为止,我让他骨头都不剩!文思清听着头皮发麻,劝道,你现在是太子了,不用那么残忍吧,真查出来是谁,把他赐死就好了。小五子摇头反驳道,他杀了你一家几十口人,有时我想想你,十几岁就没爹没娘,他把你卖了,换着主子伺候,听说这些,我比你还难受。小五子噙着泪,继续说,我过去可能是有点轻浮,说话做事不过脑子,有时候伤了你的心,但我今天告诉你,你那骨灰盒不用成天抱着了,母亲不在,你有我呢,你可以信任我,我会永远永远地把你当做我亲人来对待,我替你把母亲厚葬了吧!文思清后退一步,抱紧骨灰盒,说,你不要碰我妈妈,我会把母亲葬了的,我也不会一直依赖她,像个长不大的小姑娘,我就不跟你去皇宫了,我这次去少林寺,把骨灰交给我师父,葬在嵩山上。文思清说完,面对着小五子往后退,到门口时,转身下了楼。小五子光着脚,一路追下去,看着文思清从大门跳出去,小五子打开窗户冲文思清喊,你也怀疑是我,对不对?你害怕,杀死你全家的那个恶人是我。文思清回头望着他,冲他摇头,声音哽咽地表示,不会的,不会是你,老天爷不
会对我这么残忍。说完她离开,小五子看着她背影消失在千人的行军队伍之外,自己身体开始发烫,摇晃,眼看就要从房子里摔出去。这时他醒了。

真讨厌,还没死,继续活着,还要面对这些死结一般的问题。醒来后,他一直在回想,好像不仅仅是梦,以前在田独,傍晚时分,和文思清在山坡看着日落烤肉的时候,似乎也说过这番话,找到你仇家,我一定帮你报仇,杀他全家什么的。可那时还没有吴思若啊,两个人都不知道,小五子有天会成为太子啊。

“哪是有天会成为太子?”他自言自语,“一直是太子,我是有天知道了,自己过去是太子。”

正午时分,小五子他们终于到了京城。街上的人也多起来了,这么大的房子在闹市区闪转腾挪,竟然可以毫无障碍。小五子从二楼窗前看了一会儿,明白有人在前面清路,前方道路的行人及摆摊的商贩,早就被这些侍卫给赶走了。小五子皱皱眉,疑惑自己过去是不是也这般蛮不讲理。

楼梯上传来脚步声,李准驸上来请示:“太子殿下,要不要入宫之前,先准备一下?”

“准备什么?”小五子问道。

李准驸没敢说,一如既往地支支吾吾,吴思若在旁边打量着小五子,出来这么久,一身破烂衣裳,补丁都不知道打了几个了。

“可能是要你换身体面点的衣服吧。”她说。

说到李准驸心里去了,他连连点头。小五子也抬起
双臂,看看自己这身衣服,说道:“既然这样,那就先不进皇宫了,去文相府看看吧。”

李准驸愣住了,问道:“我们不是一直在文相府吗?”

“我说的真文相府,去这房子应该在的地方!”小五子喊了出来,又冷静一下,说,“我要去那儿看看。”

李准驸喊遵命,命令农夫队长朝文相府开拔。差不多半个时辰,小五子感觉房子下沉,文相居停在街右侧,对面就是以前的文相府。李准驸陪着小五子和吴思若下来,站在街道中央,他看着曾经的文相府。

本来是想找到文思清的痕迹,文相府盛世时期的样子,真到了原址,他发现也没什么好看的,几年没人打理,宅子又被拔没了。他往里走去,园林里都是老树上的枯枝和疯长的野草,林子深处的凉亭断了一根柱子,整个亭子盖斜扣在石桌石椅上,穿过园林是一个池塘,很意外,里面居然还有水,可是那水面漂满着泛黄的落叶和一层层翻着白肚的死鱼,时不时有野猫站在岸边,探出前爪,想捞两条死鱼吃。

目之所及,皆是一片破败景象。小五子在里面转了一圈,出来之后对着文府看了半分钟,回身指着那栋房子,吩咐李准驸:“把这房子移回去,原封不动地摆回原处。”

李准驸弓着身子说是,随后低声让人备一辆马车,要四匹马,车上面的轿子要够大、够宽敞,再备一身华服。受命的侍卫小跑着去准备。小五子眯眼想了想,让李准驸把房子放回去之后,将里面的园林、池塘,都打扫
干净,那些死树、死鱼都换掉,再重新盖一个亭子。

李准驸重复了两遍,说盖最好的,一定要挡雨遮阳的。最后他忐忑地问了一句:“把文相府置办好,太子殿下,您是要住进来,还是给谁留着?”

“谁也不进来,”小五子说,“在里面养二十条狼狗,把他们训练得凶狠一点,要到咬人吃人的程度。”

李准驸寻思片刻,脸上挂着笑,做出一副我懂你的表情。小五子没看明白,但实在懒得多问,免得他又“这个”

“那个”地支支吾吾。

之前筹备的侍卫把马车赶了过来,双手奉上华服。李准驸请小五子上车更衣,并宣布太子要起驾回宫。

小五子没接衣服,看着马车上的棚顶,说:“就用这架马车,带我先去趟少林寺。”

李准驸吓了一跳,赶快提醒他:“太子殿下,咱们少林寺在河南。”

“我知道。”

“往返一趟,要一两个月呢。”

“我知道,”小五子看着他,问,“怎么了?”

李准驸吭哧瘪肚的,磕巴了半天,说:“那咱们先回宫,见过父皇和五公主,让他们知道,太子找到了,您回来了,歇上几天,再南下少林寺,怎么样?”

小五子斜眼看他,质问道:“我要怎么样,需要你教吗?”

李准驸吓得低头,不敢吭声。小五子又补了一句:“我就是不想这么快进宫,我还没想好当上太子该怎么办,所以才要去少林寺!”
\newline

{\centering\subsection{2}}

李准驸十四岁进侍卫队,没读过什么书,从站岗放哨的扛刀小兵做起,一路做到九门提督,摸爬滚打快十五年,做官的道理早就烂熟于心,知道想往上爬,光是谄媚肯定不够,龙心大悦只是一时的,有时候还不一定,哪天心情不好,正反话一想,断定你是个马屁精,你的为官生涯可能也就到头了,没准命都到头了。那还得靠什么本事呢?具体方法说不清楚,简单点说就是,你得让主子觉得你好用,李准驸是个好使的家伙,说话办事,都能做到人家心里去。常言道“察言观色,揣度圣意”,这东西最不可信,察言观色,就是看眼神啊,“揣度”这两个字更不靠谱,完全靠猜的,靠看眼神瞎猜,靠运气,猜中了,飞黄腾达,猜错了,脑袋都不够赔。

李准驸才不信这个,他要暗中做调查,比如这一次,太子说要去少林寺,那他是寻仇还是报恩,杀敌还是会友呢?他可不敢察言观色地猜,他派几个人先跑一趟,四处打听,打听到少林寺的方丈一年前去过昆仑山庄,参加了屠龙大会,会后还把文思清给掳到少林寺,关了大半年。文思清他见过,当晚不告而别的那女孩,据说文相居就是他们家的。看太子那副魂不守舍的样子,应该很爱她,为她着迷。

“那方丈掳走了她,关起来,”李准驸自言自语,他还没捋清楚,为官的心思用得太多,智商这一块就成了他的短板,他捻着手指算,文思清是太子的朋友,方丈是文思清的敌人,那方丈就是太子的·····他忽然惊呼起来,“太子这是要去少林杀敌寻仇了!”

明白这一层,就可以做些万全准备了,他让农夫队长就近找三千精兵进来。队长望着他,三千精兵,到哪里去找?真是的,打劫派他去,调兵力这种事,也要他来干。李准驸说,去找当地的知府县令,就说太子从京城南下,让他们派送兵力,负责护卫。队长都接令出去了,李准驸又叫住他,让他别提太子,就说是朝廷大员出巡。

“光是朝廷大员还不够,”队长为难道,“知府县令都不听的。”

“那你就暗示他,”李准驸说,“你暗示他,来的是太子,是五公主,但别说是他们。”

“怎么暗示?”

“笨死了,你就说,”李准驸有点急,智商虽然不高,武功虽然不好,但是说话技巧上,他可是信手拈来,“你就说,具体是谁要我保密,不方便明说,但一定是太子、五公主这个级别的人物经过此地。”

原来还可以这样讲,农夫队长反应片刻,恍然大悟地笑出来,真是跟着李提督,每天都能学到新东西呢。这么说话,调兵也便利,太子、五公主这个级别的主子,几千兵马凑足,那些知府县令恨不得自己带兵,前来支援。

“兵马银票送过来就好了,”农夫队长学会了,举一反三,对着本地官员打官腔,“到时候我自会多替你美言几句。”

事情办得顺利,去少林寺的队伍一天比一天壮大,每天都有新兵加进来。有一天吴思若骑在马上回头看,浩荡大军行在崎岖山路上,从山头一直连到山尾。是看错了吗?吴思若揉揉眼睛,瞪大双眼看得再真切一点儿,我们从京城过来时只有二百人啊,怎么这么一看,两万人都不止呢?李准驸跟她解释:“太子是什么人,皇上的儿
子,日后的天子,那自然是皇恩浩荡,天下人人都想加入我们了。”

原来小五子这般厉害,她看看小五子,以后还真不能老欺负他了。

人凑齐了,接下来就是探口风了。看看太子对少林寺到底怎么想。他找人做好文章,历数少林寺这几年犯下的八宗大罪,找机会读给小五子听。当然都是编的,只不过编得还不够过分,他添油加醋,全换成杀人放火,强抢民女的罪。扣这一顶帽子还不够,行至嵩山脚下,他又安排人哭丧挡路,说是要管事的下来,听她讲一讲冤情。小五子从轿子上下来,示意她讲出来。为首的女人满腹委屈,她说,少林寺最近扩建修庙,要把他们家的房子拆掉,把种小麦的一亩二分地给占掉,当家的男人不让拆,集结村民帮忙,来守家护院,结果昨晚少林寺十八罗汉下山,用大力金刚指把他们当家的给打死了。

事情不算复杂,欠债还钱,杀人偿命,哪个和尚下的手,把他拎出来,还你条命就是了。主要是他们哭得让小五子头皮发麻,毛骨悚然。讲话的是个女人,穿着白孝服,看起来是当家的老婆,每说半句话要先顿一顿,哭号一通再往下说。后面的是两个七八岁的孩子,一儿一女,没准是两个儿子,或是两个女儿,谁知道呢?孝帽做得太大,完全看不到脸,就一大大的白帽子套在头上,母亲在前面哭诉冤情时,他们在身后挥着小手,一把一把地撒纸钱,有几张还飘到了小五子头发上。真是六月沉冤,飞雪连天。

小五子劝他们别着急,他虽不是少林的,但肯定帮他们讨个说法。他回到轿子上,把李准驸叫到窗边,问他伸冤的事情应该怎么办。李准驸拍着胸脯说:“太子殿下,您尽管放心交给我来办下。”

小五子想了想,去年昆仑山庄方丈也出席了,西北六公子,丐帮的长老,在他面前都是小角色。好像他一出手,就把那两个道长也摁住了,现在想想,还心有余悸。而且,听说方丈在少林寺还不算一流高手,罗汉堂的十八罗汉各个都要比他厉害。

他只是在回想,李准驸以为太子还在犹豫,再次力荐自己,小五子说:“不是我信不过你,只是少林寺高手如云,我怕咱们不能全身而退啊。”

“太子殿下,您多虑了。那方丈就是吃了熊胆豹子胆,也不敢和朝廷作对啊。”

李准驸这段颇有表演性质,一边说着,一边转身,朝队尾眺望。小五子随着他看过去,吴思若跟着一块儿看,浩浩荡荡的队伍一直连到河水的那一边。她睁大眼睛,
这时才反应过来,呀,李准驸,原来你早都准备好了!

李准驸的意思是,别看少林寺都是和尚,其实皆为巧言令色之徒。无论犯下何种罪行,都有可能被他们搪塞过去,所以自然要先给他们一个下马威,之后再慢慢地审。小五子表示可以,就按你说的办。

李准驸拍了拍胸膛,抿着嘴“嗯”了一声,接着像一只骄傲的公鸡,往山上走几步,冲队伍大喊着:“三军将士听令,包围少林寺!”

哪来的三军将士,就是叫起来有气势吧。真要是三军将士,也轮不到他一个九门提督来指挥。

不到半个时辰,两万名乌合之众把少林寺围了个水泄不通,可是奇怪了,一个露面的和尚都没有。不都是高手吗,内力深厚,一只蚊子飞上山都能听得见,这怎么两万多人爬上来了,里面也没个动静。李准驸看着少林寺死气沉沉的大门,说:“太子殿下,让我来!”

他走到门前,闭着眼睛推开门,进去就大喊道:“把你们这儿所有喘气的,都给我叫出来!”

等他睁开眼睛,却吓了一大跳,上千名和尚都聚在大堂内,每个人掌心向上,拇指掐着食指,观音坐莲一般地盘坐在蒲团上,都睁着眼睛目视前方,一动不动,也不看他。李准驸深吸一口气,别害怕,双腿不能抖,太子还在外面看着他呢。他握紧双拳,又叫了一通:“方丈呢,给我滚出来!”

一个个都在装死,李准驸都这么找打了,少林寺的一票武僧还在容忍他。这时,人群后面有人微微动了一下,方丈从墙边站了起来。神情困倦,跟刚醒一样,问题是脸色还蜡黄,看起来体力不支,晃晃悠悠地走过来。

那就不怕了,骄傲的公鸡又挺起来胸膛,冲方丈怒斥道:“整个少林,就你一个会喘气的吗?”

方丈承认道:“确实就我一个喘气的。”

“那这些都是死人不成?”李准驸走过去推了几个和尚,可是个个定住了一般,一动不动,试了几个人的鼻息,果然都不带喘气的,但是脖子上有温度。这就有点疹得慌了,他指着那些和尚问方丈:“这是活人,还是死人?”

“活人。”

李准驸皱眉想想,有些结巴地问道:“那怎么都不喘气?”

方丈解释:“这两月,本寺的弟子们正在修炼闭息大法,所谓闭息,自然是不呼,也无息了。”

李准驸手指点了半天,一句话都问不出来,回头看眼轿子里的太子,终于想到可以问什么。他说:“你可知
道我是谁?”

方丈走近来看,他不认识李准驸,就是看一百年也没用,但看这架势,明白是朝廷来的钦差。李准驸身后的农夫队长这几天开窍了,知道倘若方丈直愣愣地说不认识,李准驸的脸面肯定挂不住,在后面用口型提示他:“李,大,人。”

方丈看懂了,双手合十,喜笑颜开,说:“原来是李达仁大人。”

“算你识相,也还记得我,不过,你说一次大人就行了,不用说两次。”

方丈愣了一下,又看看李准驸后面扛镐的那个人,这次他没给口型提示,不知道接点啥,可他是和尚头子啊,打从进寺修行那天,还是小和尚时就知道,聊天尬住的时候,这四个字最管用了:“阿弥陀佛。”

“大概三四年前,我和你在皇宫里有过一面之缘,”李准驸说,“你当时被五公主召见进宫,就在五公主面前,你立下军令状,应下来的事,可还记得?”

“阿弥陀佛。”

“我问你,是否还记得!”

“善哉善哉。”

“你当时说了什么,请回答我!”

方丈咽了口唾沫,不是健忘,是真的不知道,自从去年中了南海真人的断魂掌,谈未来还好,一旦聊到过去,就是神情恍惚。他双手合十鞠躬,请李达仁大人稍作歇息,他去去就来。

李准驸以为他要出去,结果哪也不去,就穿梭在闭息的和尚里寻找。这种情况方丈早就准备,想不起来的地方,就找慧根问。倒不是慧根跟他年头久,而是他记性好,虽然不识字,看不懂书,写不了信,但是过耳不忘。只要是以前有谁无意中提起的,哪怕是最无聊的家长里短,他都能记得明明白白的。可是,慧根长什么样,方丈此时却忘记了。他一个个贴近了看,脸都快贴上了,也想不起来慧根的样子。

实在没办法,方丈摇醒一个小和尚,问他:“慧根呢?”

小和尚揉着眼睛,看到这么多官兵进来,吓了一大跳,起身穿过几个和尚,指着一个已经长出头发的和尚说:“这就是慧根。”

李准驸不耐烦了,他在等,太子也在等啊,他说:“我问你话,你老找什么慧根?”

“阿弥陀佛。”

方丈争取时间,赶快把这个长了头发的和尚叫醒,
低声问道:“慧根,三四年前,我在皇宫立的军令状是什么?”

有求必应,最喜欢别人跟他打听事了,慧根说:“方丈,这个军令状我再熟不过,前前后后你一共提起过四次,第一次是刚从京城回来那次,给沈老前辈过生日,你从京城带来定福居的点心,最上面那几块还被压碎了。”

“你就直接说,是什么吧。”方丈打断他。

慧根眨巴着眼睛,难怪方丈记性不好,细节都不关心,怎么可能记得住。他叹了一口气,直截了当回答他:“玫瑰糕。”

“这是什么暗语?”

“你从京城带回来的点心。”

“你弄错了,我问的是,我从京城领的军令状是什么?”

慧根说:“你答应五公主,三年之内把昆仑公子抓到。”

原来跟昆仑公子有关,这个人耳濡目染,方丈不用多问。他转身,看着李准驸,朗声说:“李达仁大人,我立的军令状是,三年之内把昆仑公子抓到。”

李准驸也蒙了,刚一进来就一副活死人墓的样子,说是闭关闭息,问两句话,又要叫醒一个人,让这个人把慧根找着,再跟他打听,这都是怎么了?李准驸挠着头,说两句话,情绪都接不上,刚才那点气势都没了。

他要重来,朗声提气说道:“请再说一遍,你答应五公主什么了?”

“我答应五公主,三年之内,抓到昆仑公子。”

“现在已经快四年了,”

方丈低声问慧根:“昆仑公子他人呢?”

“你根本没抓着,”慧根说,“他在昆仑山庄被大漠仙人和蓬莱阁老给抢走了。”

方丈额头冒汗,知道不能照慧根的原话回答,记性虽然不好,但这点人情世故,总还是有的,说了那些话,就是承认少林寺武艺不精,矮人一头了。

李准驸摇头道:“你们这帮和尚啊,一个个好吃懒做,干什么都不积极,你知道本大人这三年来日夜兼程,办得什么大事吗?我把太子给找回来了!”

把太子找回来?方丈一头雾水,太子是走丢了,还是离家出走,为什么要你找回来?慧根拉拉他袖子,提醒他:“太子是被昆仑公子给劫走了。”

脑子转三圈,这里面的因果他想明白了,怪不得抓昆仑公子。接下来就是表演了,方丈先演大惊,随后又大
喜,感慨道:“善哉善哉,太子还活着,是吗?我以为太子早被昆仑公子杀害了,没想到李达仁大人真是神通广大,令老衲刮目相看。”

“太子不但活着,此刻还到了少林寺,就在门外等候。”

“太子来了?”方丈惊道,“老衲现在就去迎驾太子莅临本寺!”

他说完,转身对着和尚们一声大吼:“出关!”

所有闭息的和尚全都站了起来,大口呼吸,仿佛要把这几个月闭息丢掉的气,全给补回来。

李大人点头道:“你们少林寺,真不把我李准驸当回事,原来说醒就醒的,我李准驸来了那么久,也不见你们醒来迎接。”

方丈连忙解释:“我们少林寺人口众多,每天的伙食开销就是一大笔钱。五十年前,前辈高僧为了缓解生计问题,创立了这一套闭息大法。每年我们香火钱紧张的时候,就让众弟子修炼这闭息大法,挺过这青黄不接的两个月。”

这倒是新鲜,以静制动,一睡几个月,钱都省下来了。方丈说话时,李准驸朝出关出息的和尚们望过去,只见起来的弟子们都乱作一团,把找到的干粮,使劲往嘴里塞,比丐帮的吃相还难看。方丈也觉着丢人,一声怒吼,让他们把干粮放下,接着命令他们列仗。

虽是三月未进食,总还有点名门正派的样子,大家强撑着站起来,在门前排成两列,方丈跨出门,冲小五子的轿子喊道:“少林寺恭迎太子莅临本寺!”

说完他带头叩首。小五子从轿子里出来,带着吴思若往寺里走,跨过门槛时,方丈抬头看了一眼,让自己牢牢记住,太子长什么样。慧根却冲小五子看痴了,他拉着方丈说:“不好了,太子被掉包了,这是昆仑公子。”

方丈警告他,不要开玩笑,这是要祸从口出的。旁边的小和尚跟着慧根帮腔说:“这就是昆仑公子,我跟您去年在昆仑山庄,亲眼见到的。”

方丈看看别人,还有几个和尚也在点头。那就是出大事了。他忽然跳起来,命令所有和尚摆开罗汉阵。

“将昆仑公子拿下!保护太子!”

可是,和尚们手持棍棒,将为首的小五子和吴思若围住,他们互相看着,昆仑公子拿下了,太子又在哪儿呢?
\newline

{\centering\subsection{3}}

达摩堂外一片混乱,人群中李准驸先喊了声:“住
手!”

之后他做了个手势,农夫队长一声令下,外面轰隆隆的脚步声,地面都在震,两万散兵大踏步向寺庙靠拢,把少林寺包得更紧了。有十几个冲到前面的都被挤到了门里,摔在蒲团上。

方丈不记得李准驸,但确定他是朝廷命官,先抓昆仑公子,还是先保太子,他自有轻重,听他的就对了。他命和尚们停手,看李大人怎么说。

可李大人没好话,上来就是痛骂,他呵斥每一个和尚,骂道:“你们这一帮社会闲散人员,朝廷把你们组织到这个庙里,每年拨三千两银子,供你们吃,供你们喝,怕你们在庙里呆得闷了,不开心,还教会了你们闭息大法,可你们不知道感恩,居然还敢对太子动粗?”

“我们是保护太子。”方丈辩解道。

“怎么保护的?我问你怎么保护的!”李准驸说着小跑过去,毕恭毕敬地把小五子扶起来,替他掸去身上的灰。

眼前的这个就是太子?方丈费力回想着,他低声问慧根:“是不是搞错了?”

慧根也糊涂了,自己最引以为傲的记性,竟然出了错。他自言自语:“明明是他啊。”

“你们少林寺,数百年来,一直是武林第一大门派。”李准驸指着每个和尚的光头说,“我跟你们说,朝廷早就看你不顺眼,一直想着把你们这武林第一门派的招牌扯下来,再召集各门各派,重新竞标。你看看人家丐帮,比你们少林寺的人还多,但人家不花朝廷一分钱,每年反而向朝廷上缴五千两税银!”

方丈连忙辩解道:“丐帮的人都是衣不蔽体,浑身恶臭,他们那副样子走街串巷,实在有损国威,圣上颜面也无光啊。”

李准驸叹了口气,说:“我又何尝不知,这就是朝廷上这两年一直在争论的观点啊!”

这样就热闹了,小五子左右看着,开始火药味十足,但很明显,李准驸后几句演不下去,没力气了,发火也是很耗体力的。那就我来吧,小五子清清嗓子,指着方丈叫嚣:“少废话,赶紧给我把昆仑公子交出来,三年前我假装不会武功,跟他出皇宫,就是为了探寻他的底细。昆仑公子这小贼武功是强,可跟我没法比,三下两下,就被我打得落花流水,仓皇而逃,我听说,他就是躲到你们少林寺来了。快交出来!”

方丈说:“昆仑公子不就是······”

“你还敢顶撞我?我问你什么,你就给我说什么。”小五子说,“半年以前,你是否亲赴昆仑山庄,见过昆仑公子?”

“话是这么说,我可不是为了见他,才去的昆仑山庄。”

“我问你什么,你答什么!是,还是不是?”

“是。”

“你承认就好,我问你,昆仑公子有一个老婆,叫文思清,是,还是不是?”

小五子问完这句,没看方丈,先回头看吴思若,生怕她不高兴。还好,吴思若心眼没那么小,起哄架秧子一样,双手上扬,笑着让他问下去。那就好,我先玩一会儿再说。他转回身看方丈。这问题没坑,方丈这次没犹豫,说:“他是有个老婆叫文思清。”

“说是和不是就行,大家都赶时间,不用讲那么多。”

方丈点点头。小五子接着问:“离开昆仑山庄,你是不是把她请到少林寺来了,安置在菜园子里,你还让两个和尚好吃好喝地供着她?”

“是。”

“你是不是跟她说过,要等昆仑公子过来接她?”

“是。”这次回答的是慧根,答过后,他低声对方丈解释,“这件事您不知道,但我都记得。”

方丈白他一眼,对小五子点头道:“是。”

“昆仑公子乃是本朝第一逆贼,你少林寺实为本朝第一大派,这窝藏昆仑公子,不正是与朝廷为难,跟我太子孙天奇作对?”

方丈说不上来,慧根对他咬耳朵。他听完学话讲给小五子:“我拿下昆仑公子的老婆后,已第一时间通报五公主,请她派人埋伏在周围,活捉昆仑公子。”

“我只问你,是,还是不是,不用讲道理教育我!我再问你,你以前是否见过我?认识我?”

“见过,认识。”

“在哪里见过?”

“在昆仑山庄。”

小五子又问:“在昆仑山庄,你见到了我,也见到了昆仑公子,抓到了昆仑公子的老婆。当今圣上,常年昏迷,此等要事,你该通报太子我,为何舍近求远?偏要请命于远在京城的五公主?”

方丈回答道:“我那时不知你为太子。只以为你是······”

小五子又把他打断,追问道:“你见到我了,不知道我
是太子,你以为我是谁?”

“你是······”

“不要狡辩了。李大人!”

李准驸赶紧过来,躬身道:“微臣在。”

“李大人,你是什么时候认出我是太子的?”

李准驸回答:“属下见太子第一眼,便已认出。”

“那你我以前素不相识,你是怎么认出我的?”

素不相识,这李准驸就不会答了。他为难几秒,说道:“太子一身真命天子之相,身上所散发的光芒,非寻常所见,自然只能是太子,天下无第二人有此黄龙之相。”

李准驸说完长吁一口气,在心里夸自己一百句“好样的”。谁知那口气吐出来一半,小五子接着问一句:“还有呢?”

“还有,还有,”李准驸把那口气憋回去,翻眼皮沉思着,灵光一现,补充道,“还有属下三年来,日日夜夜,心系太子,但凡见到太子,自然是喜出望外,哪有照面不识的道理?”

“说得好!”

这回总算放心了,那口气吐出来吧。

“哪怕是普通百姓,一眼即知我是太子,可这这少林寺高僧,与我长谈数日,却假装不认识我,不知我身份,李大人如何看?”

隔山打牛,隔我李准驸打方丈,这他最拿手了。李准驸谄媚道:“太子殿下,有所不知,少林寺除了闭息大法这一套神功外,还有一套绝学,那就是装聋作哑,我看方丈已然修炼到深不可测的程度。”

小五子摇头道:“何止是深不可测,简直是深不见底,高山仰止。我看方丈这练到最高的境界就是指鹿为马,不信你们现在就听,方丈这装聋作哑功练到何种程度。我说完了,方丈你请讲吧!”

小五子说完向后退一步,还真是,把舞台中央留给方丈。方丈双手合十,满腹委屈,憋了好半天,终于说了一句:“阿弥陀佛,善哉善哉。”

那该怎么办呢,少林寺这么多人,连同方丈该怎么弄呢?说实话,小五子也不知道。这时外面一声吼叫,大喊着:“昆仑小贼,你害老夫几个月说不出话,今天老夫要拿你开开嗓!”

这是谁啊,哪来的葱姜蒜?李准驸守护到小五子面前,冲外面喊着:“你要找的昆仑公子,他早都跑啦。再说,你又是什么人,敢来我们这里撒野!”

他不说少林寺,说我们这里。方丈也听出,李准驸
言语里已经把少林看扁了。乍一听,他也没听出对方是谁,犹豫要不要把这场子找回来。李准驸站在小五子前面,跟个大雕似的张开双臂,挡着小五子左右看不着。小五子把他踹开,走到门口,外面还是那两万散兵,也没见谁在门口。他回想着,弄得人家几个月说不出话,自己也没这本事啊。有两个词倒要想一想,老夫,那说明年纪不小了,老头子了;开开嗓,这个很奇怪,吟词唱戏吗,还要吼两嗓子开一下,未见其人,先闻其声,对啊,这是狮吼帮的乔帮主啊。

那自然说到就到,小五子走到方丈面前,大声问道:“你们少林寺里通外敌,该当何罪?”

方丈擦擦脑门上的汗,低头道:“是。”

“我没问你,是还不是,我问你该当何罪?”

“全凭太子处置。”

小五子指了指门外,说:“我现在给你一个将功补过的机会,你该怎么办?”

“明白。”方丈对众弟子命令道,“保护太子!”同时不忘加一句,“我知道你们刚刚闭息大法出关,体力不支,但也要拼尽全力,只要能保太子平安,朝廷当然不会慢待我们的!”

这些和尚听明白了,方丈话里有话,他这是见机要挟朝廷,逼上面多发些好处。他们各自有气无力地站起来,更夸张的是,有十几个站都站了,双腿一软又倒下去了。小五子没那么有经验,李准驸自然一听就懂。他看着这些出工不出力的和尚,又看看一脸无辜的方丈,质问道:“你这是要挟朝廷吗?”

方丈瑟瑟发抖,直摆双手,上下牙打战地回答道:“阿弥陀佛,我们少林寺哪敢跟朝廷要条件?只是大家连饭都吃不上了,所谓闭息大法,也只是苟活一口气,不知道能不能保护得了太子。”

“这还不是要挟?赤裸裸的要挟!这简直就是拿太子做人质的要挟!”李准驸怒不可遏,前倾着身子咆哮,这要不是太子,按照他的性格,哪怕是自己的老婆孩子在方丈手上,他都不惯着方丈这毛病。

小五子奇怪了,他问李准驸:“就算方丈不肯帮咱们,山上山下门口不还有两万精兵吗?”

精什么兵,李准驸有苦说不出,那都是跟当地知府县令要的,领盒饭过来凑数的。他当然不能说,欺君之罪,只能寄望于给方丈施压。一直没说话的吴思若笑了出来,她对方丈说:“方丈大师,您听我说一句,我是个外人,还是个女施主,朝廷能不能拨款的事,我当然说了不
算。只是我觉得呢,倘若太子真的是在你少林寺出了事,别说以后能不能跟朝廷要银两,只怕是五公主啊,老皇帝啊,在京城挥兵南下,铲平了这嵩山少林,你们就是连闭息大法,怕是也练不成了呢。”

一番话令方丈听进去了,确实不能死在我这儿。他命令众弟子摆罗汉阵,等候强敌。小五子提醒方丈,可能是狮吼帮的乔帮主,大家用棉花塞住耳朵。

大概等了一炷香的功夫,乔帮主终于上得山来,进到达摩堂。他一个人进来的,见到方丈,先是寒暄两句,一日不见,如隔三秋之类的,然后又问满山那些生火烧饭、打盹儿睡觉的官兵是怎么回事,不是朝廷上的狗官又来找麻烦吧?李准驸脸色不好看,但忍着没发作。方丈装糊涂,他说:“我不知道啊,我今天醒来之后,就一直没出门。”

乔帮主愣了一下,说:“好几万人,把山都围死了,你不知道?”

“不知道,确实没出门。”

乔帮主想了想,观察了一下大堂,看到这些和尚摆的罗汉阵,看到他们耳朵里都塞着棉花,再看看小五子被奉为上宾的样子,估计他们要联合,一起对付他狮吼帮了。

可狮吼帮就来了他一个,一对这二三百,下面还有两万,一对这两万零二三百,那就先讲道理吧。他巡视一圈,朝小五子怒视过去,质问道:“昆仑小贼,老夫今日来跟你讨个说法。我花费重金,好心请你回沉狮谷,你若是天性放浪,难以管教,那就跑则跑矣,我也不去强捉你回来。可你为何去而复返,还对我下毒,令我失声?”

说话前,小五子就在对他微笑,这番话讲完,小五子还在对他笑。乔帮主想起来了,他带着耳塞,啥也听不见。他冲小五子做手势,两个食指贴在耳边向外扩,那意思是,你把耳塞拿出来,我跟你说两句话。

小五子其实全听见了,耳塞一点儿不管用,一个字都没落下。他脸虽然在僵笑,心里在想乔帮主的问话,自已搭了半条命从沉狮谷跑出来,就是给他俩胆儿,也不敢再回去,再就是哪来的毒药,好像还是哑药,给别人下就算了,可这是狮吼帮当家的啊,人家是靠嗓子吃饭的,确实有点过分了。

但我能怎么办呢,小五子想,摘下耳塞说,这事不是我干的。这肯定没用,没准人家还觉得我故意狡辩。那就别摘了,既然能装听不着,我也能装看不着。他心里直摇头,脸上保持笑容,双目无神地看着乔帮主。

见小五子不配合,乔帮主又做了一次拔耳塞的手势。小五子无动于衷。那就跟方丈谈谈,他能听得到。他冲方丈合十说:“方丈大师,请昆仑公子把耳塞摘掉。”

“阿弥陀佛,有什么话,你跟我说就行,”方丈说,“等你下山,我自会转达。”

那就是不肯摘,你不摘我摘。乔帮主大步朝小五子走过去,到他耳边去摘耳塞。这时两个和尚跳过来,挡到他和小五子之间。乔帮主伸出左右手,和两个和尚各对一掌。本来他狮吼帮也不是以掌力见长,而这两位僧人,也远不是少林的一等高手,双掌对双人,竟然打了个平手。又不是跟少林寺有过节,纠缠下去没意思,他脚下腾挪,想从左侧绕过去,这时发现身后还有三个和尚扯住了他的衣摆,令他转不过去。

“想以多欺少吗?”

“阿弥陀佛,只想请乔帮主收手,咱们有话可以从长计议,细细道来。”

“我是想跟你们好好说!可你们全装听不到,看不着,把我当傻子!”

乔帮主吼了两嗓子,但还远不是狮吼功,只是嗓门比较大,完全是心里太憋屈,他双拳对四手,一时间跟这一帮和尚打得手忙脚乱,怕是今日命丧于此,乔帮主心里想着,这是要逼他发狮吼功了。

他向后跳一大步,运气发力,全身被热气笼罩,周遭的几个和尚已无法近身。唯有方丈等几位高手可以与之抗衡,但这刹那之间,已来不及赶过去,只得反向去保护小五子。方丈站在小五子身前,屏息相抗,这时看见乔帮主张大嘴要往外吼,他们等了一会儿,他喉咙里只发出沙哑的滋滋声。持续了半分钟,乔帮主也没发出力来,反而全身的热气散去,瘫软在地上。

这就是威震两江的狮吼功?李准驸半张着嘴巴,看着瘫倒的乔帮主。今天他算是长了见识,原来武林的门槛这么低,谁都能搞点威震江湖的东西来。以前老说,朝廷武林是相互忌惮的,武林忌惮朝廷七八分,其实朝廷也忌惮武林二三分。现在看起来,武林里要都是乔帮主这种奇人异士,朝廷这种忌惮,也就是自己吓自己。

那就不劳方丈大驾了,他九门提督李准驸得空手擒拿狮吼帮帮主。他跟农夫队长要根绳子,搓搓手,走过去拎起乔帮主的头发,将他双手捆在后面,缠上绳子,打了个猪蹄扣。

猪蹄扣也叫双环结,小五子再熟悉不过了,那是杀猪的标准打结方式,一只手腕套一个环,中间伸出一根长
绳,可以把人像待宰的猪一样吊起来。慧根还在跟方丈补着课,他说:“以前狮吼功不是这样的,很厉害的。”

方丈狠狠地瞪了一眼,冷冷说道:“我知道,我只是记性不好,我不是傻瓜!”

按照慧根的理解,记性不好和傻子是一回事,但他没有争辩,吸一口气,沉默抗议。方丈此时全是谜团,但起码有一件事能确定,这个“初次见面”的乔帮主,算是彻底废掉了,而且估计就是冒充太子的昆仑小贼所陷害。

昆仑小贼在干嘛\footnote{原文“干吗”}呢,他也有点难受,眼前这个无力老人,怎么说自己对他也是拜过高堂,喊过爹的。他不能看他如此境地。他让李准驸给乔帮主松绑,解开猪蹄扣。

“太子殿下,这个乔帮主什么来头,”李准驸问小五子,“是活捉回京城,还是就地处决?”

“也不用捉,也不用杀,请他走吧。”见李准驸还不明白,小五子补充道,“他就是要抓我回去做女婿。”

李准驸眼珠转三圈,颇为诡异地笑了。他说:“这事好办,他之前抓你回去做女婿,咱们这回以牙还牙,把他女儿抓回宫里做太子妃。”

吴思若看看小五子,揶揄他:“你可以啊,太子还没当上呢,倒是预定了好几个太子妃。”

小五子想反击,看乔帮主的样子,也不想在他面前太轻佻。这时候发出一个清脆的声音,一个孩子喊着“爹爹”,跑进达摩堂。看起来蹒跚学步的样子,摇摇晃晃地走进来,抱住小五子的腿。

李准驸见小五子没抗拒,反而把孩子抱了起来,赶紧过来拍马屁,直接给两岁多的闹闹,双膝叩拜,口中念道:“属下李准驸叩见,叩见······”他转身问身旁的亲信:“我叩见谁?太子的儿子,我应该叫什么?”

亲信哪里会知道,连连摇头,跟着李大人做就是了。这几个亲信,加上农夫队长,和李准驸一起跪下来,李准驸带着他们,叫不出闹闹的称谓,干脆连磕三个头,默默起身。

“真是笑话!到底谁是太子?”

外面传来男人的声音,一男一女走了进来,男的是西北六公子,而他身旁的女人则是乔文君。她看见地上的乔帮主,赶快扶他起来,跟方丈讨了碗水,喂给乔帮主,有些气力的时候,乔帮主盯着小五子问:“昆仑公子,你为何如此阴毒,废我狮吼功?”

这回小五子不能装听不见了,他看着乔帮主,想真诚点解释给他,这时他发现乔文君在对他微微摇头,又轻轻点点头。小五子没明白,左右看看,这边是西北六公
子,那边是乔文君,自己怀里抱的是他们的儿子闹闹,那么这位父亲中的哑毒。他大概明白了,你们这一对狗男女。
\newline

{\centering\subsection{4}}

小五子能猜个大差不差,乔帮主的功力丧失,肯定跟他们两个有关系。眼看众人都在场,他也不便戳穿,还好乔帮主力不从心之后,也没有深究,一场争论暂时平缓。李准驸令人准备伙食,晚上大家在寺里吃斋饭,十几个人相安无事地吃着青菜豆腐,然后就跟组团旅行似的一团和气,假模假样地寒暄两句,相互说着:“那就早点休息吧。”

可谁也睡不着,各揣心事,暗流涌动,怕对方加害自己,加上自己还有那么多烦心事想不明白。先出来的是吴思若,不是睡前散心,她想离开少林寺,离开小五子。一下子那么多变故,都不知道往下该怎么办了。起初不就是计划来金陵,拉着一只手跟小五子告别的吗?可是刚到长江口,苏子瑶就被杀了,要走的话说不出口,跟着进了百花谷,本想陪同几日,等小五子缓过来就不辞而别的,碰到了李准驸,小五子转眼又成了太子,文思清都走到她前面去了,她也不好把他一个人扔下来,又跟着来了少林寺,这下好了,小五子的老婆孩子都碰到了,人家一家子团聚,她在旁边算什么?真是的,本来可以走得洒脱一点,一拖再拖,现在却要灰溜溜地离开了。

她记得来时的路,想着不要去告别了,再见到小五子的样子,谁知道会不会哪根筋搭错,又迈不开步子了。少林寺南大门紧锁,周围一片蛙声,有两个小和尚盘坐在地上守在大门口,彼此也不说话,睁大眼睛,一动不动,估计是晚上没吃饱,又在这里练闭息大法。

那不管他们了,想办法开门出去吧。吴思若走到门前,研究了好半天,门是有点高,翻过去不难,但难免弄出动静,惊动寺里的和尚。自己又不是歹徒,不告而别而已,何苦把众人都吸引过来,让自己难堪。她左右看看,想找找还有什么豁口能出去。这时旁边一直没动静的小和尚说话了:“北面的门开着,有条小路能下山。”

吴思若愣了一下,蹲下来打量小和尚,问道:“你们闭息大法,可以自己醒来的?”

“我们哪会闭息啊,”另一个小和尚说话了,“练到现在,还练不明白,被师父罚到大门口反省的。”

“但你俩刚才看起来挺像的。”

“那只是装作一动不动而已,蚊子咬了,还是会痒,肚子饿了,还是会叫。”

说到这里,吴思若明白了,原来不是青蛙叫,真是两个孩子饿坏了。她摸摸身上,也没什么干粮,索性留点银子给他们,说了声“谢谢”,向北门走去。

到了北面她发现,不单是没有门,连墙都被拆了一大半,不会是这些和尚饿得真吃土,把墙吃没了吧?月光下能看见,果然有条小道,她踩上去张望,原来是向上走到山头,那边就是下山的路了。

几天没下雨了,地面还是有些湿,不少的泥,她找有草的地方走,别踩得太实,施展一点轻功,脚尖点在地上。走出二里地,她听见有个女人在说话:“有什么事,就直接在少林寺光明正大地说,鬼鬼祟祟的,把我约到这里干什么?”

那是乔文君的声音,那就是小五子约她喽。尽管算情敌,她可不想趴他们墙根,打算从右侧神不知鬼不觉地绕过去。右侧就是泥路了,鞋子肯定要脏,前几步还小心翼翼,抓着树枝,尽量别粘泥,树枝被她抓得直摇晃,树影晃在月光下。后来想想算了,别为了一双鞋子,打扰了小五子和乔文君的深夜幽会。

不管不顾,她双手放开树枝,踩在泥里行走。不到十步的样子,脚下有什么绊了一下,她一个趔趄,轻叫了一声,险些摔倒。站稳后,她朝乔文君那边望过去,已经绕出去很远了,她听不到乔文君说话,估计以她的内力,也听不到自己的声音。看她样子还在抱怨,这么叫她出来干吗。是啊,他们是两口子啊,出来干吗?直接敲门进房就好了嘛!

她回头望一眼,想看看绊倒她的是什么鬼东西。好像是个活物,马或者狗,全身裹在泥浆里翻腾。她走近看一眼,居然是个人,双手绑在背后。又是哪个小和尚,什么功没练好,被师父罚到这里?虽然浑身是泥,可是还能看出来头顶有头发,那就是不是和尚,我救你一命,可不是坏你师门规矩。她提剑过去,想让他翻过去,把背后的绳子一剑削断,这时才依稀看到了他的脸,满脸都是泥,嘴上还塞着布条,发出呜呜的求救声,即使这样,吴思若也能一眼认出这是小五子,可那个人是谁,那个大半夜把乔文君约出来,把小五子绑在这里的人又是谁?

\newpage