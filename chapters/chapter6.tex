\section{陆}

{\centering\subsection{1}}

方丈说要找文思清谈谈,但她在少林寺呆了一个月,也没见着方丈的面。她晚上住在寺外菜园,白天跟着和尚一起进寺。头一个星期她就摸清了寺里的日常,卯时敲钟起床,天都是黑的,和尚在千佛殿集合打罗汉拳。跟晨起早操似的,每天打一套,然后还不开早饭,要去诵经堂做早课。大家敲着木鱼,根据自身修为,念什么的都有,《金刚经》、《易筋经》、般若波罗蜜的《大般若经》。混杂在朗朗经文中,文思清清楚地听到,有两个小和尚嘴里含糊地反复念叨六个字,好饿啊,开饭啊。问题是节奏还不对,人家木鱼敲两下,他俩咚咚咚能敲五六下。

这两个小和尚就是照看文思清的,照字在前,看字为后,好好看住她,别让她跑了,也别让她饿死,如果可能的话,也让她听听经文,学学佛法,别白来少林一趟。两个和尚是兄弟俩,哥哥十九,弟弟十六,净字辈的,一个叫净空,一个叫净虚,文思清到现在也没分清,净空净虚到
底谁是谁,他们总是哥哥弟弟地叫。有一次弟弟叫了哥哥的法号,回头被他哥哥好一顿打,我是你哥你知道不,直呼其名,目无尊长。

刚开始文思清还挺担心,两个男孩都不小了,挤在一间房,怕他们晚上摸上床来。接触几天文思清明白少林寺为什么安排他们来照顾了。两个小和尚傻乎乎的,小时候更傻,弟弟七八岁时淘来一本《葵花宝典》,从一个乞丐那儿花二十文买来的。两个孩子按照书上指引,一人一刀把自己切了,弟弟跟着哥哥一心一意地练神功。大概练了五年,他们发现这本书是假的,别说上天入地,爬树掏鸟都费劲,两个孩子傻眼了,跟父母讲了这本书的来龙去脉。文思清想他们父母也够可怜的,上辈子造什么孽,生了这么两个缺心眼的。

神功没学会,大侠梦还在,反正都这样了,不当和尚也是当太监,哥俩两年前跑少林寺来了。他们想学大力金刚指,可是入寺三年多,师父只叫他们到菜园里种菜,清晨练练罗汉拳,每天打一套,九个小节,每节一八二八共八个节
拍。这是学功夫吗,弟弟闹了好几回情绪,每次都是哥哥给他讲道理,少林寺不同于其他门派,讲究打好根基,头三十年你打不过别人,后三十年别人打不过你。

“可是,锄地施肥做早操是什么根基呢?”

哥哥说不上来,他也同样疑惑,找个机会跟师父请教,我们打一套罗汉拳,其实就是练基本功吧?师父摇摇头,没听明白,什么基本功,一日之际在于晨,早上打一套,是让你们一整天都有力气挑肥种菜。

难道是来错地方了,可是天下武功出少林啊。哥哥叮嘱弟弟,别气馁,师父和方丈在考验咱俩呢,咱们好好种菜,侍奉佛祖,师父看在眼里,吃在嘴里,总会把十八罗汉的看家本领全教给咱们。弟弟不相信,每回这时候都要说,藏经阁扫地的八光快五十了,不还是在扫地?

八光也是和尚,十多年前出家,前一任方丈不给他法号,就让他叫原来的名字,说一姓一名都是浮云,倘若名姓不改而品行转善,才是真正的修成正果。修得可好呢,十多年没出过藏经阁的院儿,念经敲钟打罗汉拳,他统统不参与,吃饭都是兄弟俩轮流送。

“像他那样可不行,”哥哥说,“我们准备好了,机会自然就来了。”

终于在前几天,师父从武林大会上回来,把一个姐姐带到菜园,说是昆仑公子的女人,可一定要看好了。哥哥双手合十,一百二十个保证。昆仑公子啊,说武林第一高手也不为过,以前在老家,满大街都是他的通缉令。虽然后来上山种菜,远离江湖,但听说这次的武林大会,上百个门派愣是没拦住昆仑公子,让他带着大漠仙人和蓬莱阁老给逃出去了。

哥哥老成持重,深知求人办事得先把人伺候好,每天换着花样地给文思清做斋饭。菜园里没有鸡,但他会做素鸡,没有鱼,但他会做浆水鱼鱼,天黑后还要给她磨碗豆浆,说是安神补脑。忙活一礼拜,漂亮姐姐人都吃胖了,哥俩觉得可以跟她拜师了。这天晚上一如往常,弟弟把第二天她要穿的干净衣服放在床头,哥哥端一碗豆浆过去,看着文思清咕咚咕咚地喝完。

“是不是觉得跟前几天不一样?”哥哥给她讲,“里面我加了黑芝麻糊,又磨了些五谷掺在豆浆里。”

刚才喝太快了,文思清咂巴嘴回味着,好像是香了一点,再来一碗吧。哥哥犹豫了一下,他想趁热打铁,先说说拜师学艺的事。他问这几天你对我们哥俩还满意吧?文思清还认真想了想,说你俩挺好的,要是能放我走就更好了。哥哥陪笑说,其实以你的本事,想走就走,
我们哪能拦得住你?

“怎么拦不住?”

哥哥没回答,跟在床头叠衣服的弟弟使了个眼色。弟弟拿了两个烛台放在桌上,每个上面插四根蜡烛,他一一点亮。哥哥面对桌子,向后退几步,扎马步运气,挥出一掌,八根蜡烛上的火焰摇摇晃晃,最终中间的两个灭掉了。收掌吐气,哥哥对文思清行了个礼,说:“这掌劈空掌我们哥俩练了一年多,可惜不得其法,如何发力还请前辈指教。”

文思清完全看蒙了,这都是什么呀,她皱眉问你这么费劲干什么呢,把蜡烛吹灭就好了呀。哥哥点头称是,说你太高看我们了,外功还没有练到,说到吐息之法,更是无从谈起。

吹个蜡烛有这么难吗?文思清她过去查看,挺正常的蜡烛,生日许愿似的,她闭上眼睛吹一口气,把剩下六根蜡烛全都灭掉。哥哥抢过去说可不是这样的,他掏出火石把八根蜡烛一根根再点起来,拉着文思清后退几步到床头,说咱们要用排山倒海之势把蜡烛吹灭。

文思清摇头想笑,那神仙都吹不灭啊。她不去管蜡烛了,但还是想教育一下两个孩子,走俩步都不行,人活着不能那么懒。

都怪他哥太虔诚,弟弟早就不信这个女人了,他走过来说你就承认吧,你一点武功都不会,对不对?文思清点点头,当然不会,会我早跑了。

“你就是个欺世盗名的骗子!”弟弟好大的反应,情绪都要崩溃了,早不信这个女人,这回连少林寺都不相信了。

“到底是怎么了,我骗谁了?”

“你骗了全天下,你说你是昆仑公子的女人,可你什么都不会!”

弟弟不玩了,把僧袍脱下来摔地上,甩手出门了。他哥哥追出去,两人先是争吵,后是商量,一会又回到房间里,忽然扑向文思清,上前锁住文思清的两个肩膀,把她从头顶摔出去。文思清都吓傻了,趴在地上疼得直掉眼泪,声音一颤一颤地问,你们为什么打我?兄弟俩也慌了神,跪地上给她赔不是,说没想摔你,就是想试试你功夫。

“可我告诉你,我不会武功啊。”

“因为江湖人心叵测,有些高手深藏不露,功夫是试出来的,不是说出来的。”

文思清瞪了他们一眼,两个孩子不敢说话,想扶她起来,帮她揉揉肩膀。文思清警告他们,别碰我。哥俩就把手臂放下来,跪坐在地上等文思清哭。大家就那么耗着,文思清越哭越厉害,浑身疼得不行,还一肚子委屈,她想小五子,又想当时台上另外三个姑娘,想到她们的样子又放声哭起来。

叠好的衣服里有手绢,哥哥去床头找出来,递给她时又解释一次,真不知道你一点武功都不会,你可是昆仑公子的女人。

“我也是最近才知道他是昆仑公子。”

“那他看上你哪儿了?”弟弟问。

感觉问题怪怪的,文思清不搭理他,接过哥哥的手绢,擦完眼泪擤鼻涕。可弟弟还在追问,你一点儿武功也不会,长得又不好看,昆仑公子到底看上你哪儿了?文思清停下来,把手绢折好放进脏衣篓,瞪着他问:“你是认真问我,还是故意嘲讽我?”

“我认真问的。”

真是的,没有比他再认真的表情了。文思清倒吸一口气,起身看眼镜子里的自己,过去那种铜镜,即使那么朦胧,依然看不到自己有多美。哥哥说了,高手都是深藏不露的,她心头一酸,小五子,能看上她的昆仑公子,到底是个什么样的男人啊?
\newline

{\centering\subsection{2}}

文思清是被方丈请到少林寺的,我请你来,你不答应我弄死你。本来方丈救了她,武林大会那天小五子被两个老头掠走,临走时把房子弄塌了,屋顶砸下来的一刻,她以为自己完了,赶紧蹲下来,找个个高的挡一挡,混乱中有个和尚揪住了她头发,拖着她,赶在灰飞烟灭之前把她拽出房子。文思清那次疼哭了,和尚双手合十,说自己是少林寺的方丈,男女授受不亲,更何况是出家人,还好抓的是头发,没有辱没了女施主。

她皱眉看着方丈,没头发也没胡子,感觉眉毛也被人刮掉了,一下子看不出多大年纪,合起来的手掌上还残留着几十根刚拽下来的头发。文思清捂着头皮也不好怨他,就说多谢方丈,后会有期。可方丈不让她走,说她受伤不轻,建议先去少林寺养伤,再慢慢商议江山社稷。

文思清又摸摸头皮,看手上也没血,说自己没受伤,而且我能商议什么江山社稷?方丈望着摇头,说施主不要妄自菲薄,你是昆仑公子的妻子吧?她说算是吧,反正小五子答应会回来娶我。

“他一定会来找你了?”

文思清点点头。看起来方丈对她的回答很满意,冲她微笑,突然伸手给了她一掌。文思清一口血喷了出来,方丈低着头阿弥陀佛,说这位女施主受伤不轻,还是随我去少林寺养伤吧。

汴梁离少林寺不远,连拉带拽两天就到了,等了半个月也不见方丈来找她商议。小和尚弟弟说方丈忙,可忙呢,每天都有人带着家伙来打听昆仑公子的女人,方丈要忙着招待他们,阿弥陀佛,有失远迎,顺便再露几手功夫,把茶杯捏碎啦,把桌角剔下来一块啦,争取不吃饭就把他们吓走。

“他们打听我做什么呢?”

“挖你的肉啊,今天卸你一只胳膊挂在城楼上,告诉昆仑公子,再不出现明天把另一只胳膊也卸下来。”

“那要是明天还不来呢?”

“那就再卸一只呗,笨死了。”

但是大多数都走了,知道不是少林寺的对手,留点香火钱就作揖告辞。有几个门派不自量力,非要留下来吃饭,要见识一下少林十八铜人。方丈到哪凑十八个人去,就跟人攀交情,边吃边聊,发现贵派师爷和我们师叔祖五十年前是朋友,大家能不打就不打,真要打少林寺当然不怕你。

月底的时候三王爷带几个人来了,先礼后兵,方丈招呼他们留下来吃饭聊聊。可能是一点荤腥都没有,三王爷吃两口急了,说天下武艺出少林,咱们切磋一下。方丈心里发毛,忙跟师弟交待,凑十八个武功还行的,抹上铜粉摆摆阵仗。

可没那么多人,小和尚哥哥都被拽过去了。将近二十个和尚抹了半斤的铜粉,三王爷寻思一下,说咱们一对一吧,我这边四个高手,你那边也出四个,咱们切磋为主,杀人为辅。三王爷说完转身问你们谁先来。马长老跃跃欲试,从田独到罗刹,再到昆仑山庄,熬了那么久,终于有表现的机会了。

少林寺这边是方丈出战,他不敢怠慢,丐帮除了向问天,就是关长老和他两大高手。一上来他就下杀手,同时出右脚和左拳攻击马长老的两肋。马长老没见过这招式,本来应该是无影脚,左右腿扫过去夹攻,十几年前方丈左腿坏掉了,自创了这右腿加左手的功夫。马长老一个踉跄,方丈一跃到了他身后,拍了他后心一掌。那天小和尚哥哥在场,弟弟留在菜园除虫。风波过去之后,哥哥给弟弟比划了一夜,一招一式,方丈是怎么给马长老留情面,陪他多打几十招,一脚一拳,就是不把他打死,直到马长老躺地上起不来了,方丈才向后一跃,说承让承让,老衲也只是侥幸得胜。

“这些都是江湖上的规矩,”哥哥教育弟弟,“你是把人打个半死,还得讲人家手下留情。”

弟弟猛点头,真的是,江湖处处有门道。他问第二场呢,谁和谁对打。

第二场是迎客道长,他上前一步说马长老识大体,不堕少林寺的百年威风,先让了一场。说话的空隙还冲地上的马长老轻蔑一笑。然后他抽出那把弯弯曲曲的剑,说在下不才,哪位领教几招。

除了方丈就是十八铜人了,小和尚哥哥低下头,尽量往别人身后藏。方丈沉吟了一下,说这些弟子年纪尚轻,下手没有轻重,还是老衲陪道长再过几招。迎客道长脸都吓白了,说这可不行,说好你们四个我们四个,如果第二场还是方丈,那我们就继续派马长老迎战。

可是马长老腿都快被打折了,扶墙都站不起来。三王爷失望得直摇头,他的人输了就算了,居然还能临阵脱逃。六公子是硬骨头,不能给三王爷丢面子,站出来接招,接下来一直到晚饭前,六公子生生被方丈殴打了一个多时辰。

眼看日落,方丈停手不打了,说四场比武我勉强赢了两局,大家打个平手,留下来一起吃个斋饭吧。不说斋饭还好,青菜豆腐的三王爷更生气了,说先不忙着吃,我这还有一位高手,方丈若是能接得住他三招,我三王爷转身就下去,决不再来叨扰少林寺。

应该不是说大话,方丈打量着三王爷的长者,他胡子全白了,头发却是纯黑色。方丈请教他尊姓大名。三王爷抢过话说,方丈若是知道他是谁,怕要做缩头龟了。说话时他还特意瞪了一眼迎客道长。

那就不问吧,不知道对方来头,说是接三招,他也不敢硬撑五招,而且要偷换名目,把接三招变成打三招。他算准了第一招虚打推山掌,顺势弯腰去扫他下盘,对方定会跳起来,这时一掌般若禅掌迎过去,三招任务就算完成。先试试他虚实,要是高手他就收手说承让,要是不行他就压着他猛揍,把那一头黑发都给他揪光喽。

按照计划方丈推山掌过去,对方身子后仰,他扫堂腿踢下盘,对方跳起来,方丈施展般若禅掌,对方人在半空躲不过,只好出掌来接。起初方丈没想发全力,震慑一下对方就好,然而对面的老人双脚落了地还不收掌。对方的力道不大,方丈也摸不清他武功路数,他提醒再不松手会震碎他的肝。老人脸憋得通红,让他尽管来。方丈摇头惋惜,掌心加力顶上去,同时看着对方的脸色,但凡不对就收力放手。

一炷香时分,对方快撑不住了。方丈自己没伤着什么,除了头有点疼,也没感觉哪儿有不对。可能对方的路数就是防守,跟你耗的那种。武林功夫大体分来无非是进攻和防守,九成的门派从小就是练各种攻击招数,防守很少见,武学上就是抵消你的力,一场打斗耗到你没力气,再一招制胜。这一掌没多长时间,方丈力气有得是,只是头越来越疼,太阳穴青筋暴起,疼得都要爆炸了。他眼前一黑,捂着脑袋倒在了地上。

那些和尚呆住了,打进少林寺没见过方丈输给谁。有两个把方丈扶起来,前胸后背发掌续力。小半个时辰方丈醒过来,看看四周,让寺里的和尚快去,把文思清带到藏经阁,要跟藏《大悲经》一样地把她藏好。说完看到三王爷又自言自语补了一句,我该小点声说的。接着他宣布自己退出方丈的位子,传给他师弟。众僧问是哪一个师弟。方丈想了想,说名字忘记了,反正不是十六师弟就是二十一师弟。然后他站起来,冲对面的老头说了一句久仰又顿住了,凝视了他好半天才想起自己要说什么:“原来阁下是南海真人。”
\newline

{\centering\subsection{3}}

方丈中了断魂掌,少林寺就垮掉了,他带着三王爷、南海真人、六公子在寺里乱转。他说我知道藏经阁,你们不要瞎找,我十二岁在那边扫地,读过一些书,没一本读完的,每本书读上那么几页,就已经超过了我师父。他带着人穿过千佛殿,走出达摩堂,经过一片鱼塘时停步不走了。他转身喝斥,都是些什么人,擅闯少林寺,看我去禀报师父!仿佛时光倒退五十年,方丈还在十三四岁的年纪,表面上气势不输,不过心里怕极了,找个由头拔腿就跑。三王爷看着方丈在寺里乱撞,迟迟想不起来师父当年住哪一间房。三王爷看着他背影摇了摇头,责怪真人下手有些重了。

“三王爷,我可不是朝廷请来的,只是碰巧大家都要找昆仑公子。”

三王爷不说话,背过去看池塘里的红鲤鱼。真人说罢了罢了,我接下来不滥杀无辜就是。

也杀不着什么无辜,庙里和尚跑了大半,剩下几十个都躲在藏经阁门口。八光不让他们进阁,这些人围着文思清商量把她交出去吧,反正方丈也不行了,那个老头本事够大,还有三王爷,以后朝廷上罩着咱们少林寺。就这么愉快地决定吧,可小和尚哥俩不干,弟弟不吭声,死命抱住文思清,不让她被这些人拖走。哥哥去捶藏经阁大门,哭着说救人一命胜造七级浮屠,我们死在外面也就算了,起码让这位女施主进去躲一躲。那边的八光发火了,说小和尚什么谎都敢撒,自己怕死还造谣少林寺有女人。说是这么说,八光还是忍不住好奇,推门出来,众多光头里一眼就能看到文思清。十几年未近女色,八光一下子看痴了,退回到阁里,红着脸自言自语:“女人要是都能来少林寺,我就不来这儿出家了。”

老田合上门的一刻,方丈领着三王爷和南海真人几个人进来了。方丈双手合十,让他们等一下,他去禀报师父。然后他转身问院子里的和尚,师父是否在阁中清修。有人提醒他,你师父三十年前就圆寂了。方丈愣住了,皱眉摇头说不可能,师父早上还让我背《金刚经》的。看方丈已经这样了,之前有点犹豫的和尚也都想通了,想活下来就得把文思清交出去。弟弟抱着她大哭,和尚们拉不开他手,索性把他俩一起推过去。要一个给俩,南海真人对三王爷笑着说,我还怕不够分,真好,小和尚是你的,这个女人归我。三王爷有点为难,看看手下几个熊包,一世王爷居然被这个真人欺侮。西北六公子站出来,说这个女人我们先借用一下,等请来了昆仑公子,我们王爷连带着她,再多送你几个女人。真人冷笑,王爷当真以为我是好色之徒,也不看看这个昆仑的断魂掌,可是为我所击。

双方推来让去,几句话把八光惹毛了,一个个都是什么玩意儿,跑少林寺来分女人?他从藏经阁跳出来,从墙角抄根扫把说,本来这女人是老子田扒光的,老子这十多年转性了没碰她,但也不能给你们。然后他用扫把杆指着南海真人和三王爷,让他们都滚蛋。田扒光,这名字好熟,但一时想不起来这人是干嘛的。迎客道长哈哈笑起来,说田兄,十几年没见,原来跑到少林寺睡尼姑来了。

“你妈在这儿做尼姑,睡出你这个狗崽子。”

话说完了才认出来者是谁,说迎客,原来是你这个人渣,这么多年还没被你师哥清理门户。迎客道长一副节哀之情,说我师兄几年前不幸仙逝。八光愣了一下,自言自语,应该先弄死你,再来出家的。三王爷低声打听,扫地的这个是什么人。迎客道长说,田兄以前是武林第一淫魔,上至八十老妪,下至五岁孩童,反正是个女的就扒光,久而久之就叫田扒光,倒没人记得他真名叫什么了。说完还不忘补一句,故意很大声,估计作恶的家伙被人切掉了,居然在这儿当和尚。

“切你奶奶个熊!”八光左手拉着裤带,让迎客道长过来看看闻闻,“老子只是转了性,不干那些事了。”

南海真人一直不说话,冷眼看着他。八光被瞧得不舒服,又举起扫把杆指着真人,说你呢,快滚吧。真人还是笑笑不说话,八光将扫把倒个个儿,用扫把穗朝他脸上扇过去。真人上身后仰,出手去接扫把杆,手臂一震,发现这是百十斤玄铁打造的。当下有些狼狈地向后一个踉跄。八光借势上前,招招冲他面门,扫把穗子抖得漫天都是,却全都被真人用衣袖弹开。

迎客知道,田扒光以前的绝技就是剑术,出剑极快,电光火石之间可以在你身上刺十几个窟窿,扫了十几年地,这百斤铁扫把也能被他使得如长剑一般轻盈。即便高手如真人,开始也只能出掌防御,双方斗上几十回合,扫把力道减弱,真人的出手一掌强似一掌,掌掌生风,八光的扫把却连穗子都抖不下来了,光秃秃的枝子别有一番威力,仿佛一根铁爪插在扫把上,真人也不敢贸然出击。

小和尚兄弟俩左右摇头地看着,双方换招实在太快,弟弟看得一阵眩晕,哥哥捂住了他眼睛。听声音会更清晰,他听到出掌的风声,扫把枝在空中抽动的声音,脚落在地面的尘土声,众人时不时的惊呼声,还有一声咳嗽,好像是从藏经阁传来。出掌声慢了,扫把声缓下来,一时没人跃起,再听不到尘土声,众人的头转向藏经阁,发出疑问声。里面是一位老人,声音低沉,他说:“八光,你进来,你打不过他的。”

八光满脸通红,出招更快了,喘着粗气说那就死在他手里,岂能打不过就跑。小和尚哥哥提醒他,对方是南海真人,每一掌都是断魂掌。八光手上没停,眼睛凝视着他,好像打了这么久才刮目相看。真人后退半步,有示好罢手的意思。八光摇摇头,断魂掌最好,往日余罪剪不断,刚好借你之掌了却尘事。他开始乱打,右手扫把进击,左手伸出迎掌,你给我一掌,我捅你个窟窿,大家同归于尽。

真人早不想打了,可是对方疯狗一样搏命,出掌更快,脚下步步紧逼。八光露出空门,扫把横扫过去,真人左肋一阵凉风,衣服被抓烂,他低头看一眼,左肋下被刺穿,露出巴掌大的一片肋骨,他伸手朝八光的天灵盖击去。似乎是二次皈依,八光面带笑容,大吼一声:“师父!弟子不孝,无力再侍奉师父!”

真人皱了皱眉,手上停下来,但手掌依然罩着他头顶。阁中老人轻叹一声说:“我说一百遍了,你不是我徒弟。”

比死还要悲伤,八光深吸口气,闭上眼睛点了点头。真人反倒很高兴,冲八光一声冷笑,手掌离开他脑袋,面朝藏经阁,扑通一声跪下来,带着哭腔喊道:“弟子南海真人叩见师父!”

跟在场所有人一样,小和尚兄弟倒抽一口气,张大嘴巴看藏经阁的大门,不只是断魂掌,仙人掌、蓬莱掌都是里面这位百岁所创。功夫练得好,天下无敌,充其量就是高手,而沈老前辈这般能自创武功、开山立派的才是三百年一遇的大师。真人长跪不起,左肋喷出的血顺着衣角滴到膝盖上。大门紧闭,等了好半天沈老前辈才说出一句话:“你更不是我徒弟,快些走吧。”
\newline

{\centering\subsection{4}}

每天不到寅时八光就会起床,一片漆黑,溪水在屋外汩汩作响,还要再过一个时辰,到卯时少林寺的群僧才会陆续醒来。每次刚醒他都坐在床头一动不动,对着黑暗发一会儿呆,仿佛黑暗深处有什么东西在和他对视。当然是他赢,因为没东西,但他会带着胜利的笑容穿好衣服,洗一把脸往山上爬。

人生苦短,每天还要睡丢几个时辰,藏经阁的沈老前辈已经十年没睡过觉了。为了这一点睡眠,他在山谷的小溪旁盖了间小屋,他怕人看见,和尚们休息了他才下山,和尚们没起床他就要回到藏经阁。这么多年他都没跟别人讲实话,他不是少林弟子,和尚都算不上,虽然他也剃了光头,找人在头顶点了戒疤,但方丈不收他。那是十几年前,还是现在方丈的师父,俗家弟子都不准入,就说他坏事干太多,我佛是慈悲,但你这个淫贼太坏了。

软磨硬泡不成,田扒光夜潜少林寺把方丈给绑了,脱掉袜子堵住他嘴,让方丈别激动,你听我讲,别老淫贼淫贼的,也不换个词儿骂。他先磕三个头,说自己年初绑了一个姑娘,可是这次没扒光她,他发现他喜欢她,放了之后朝思暮想的,他又去找她,按他田扒光的行事方式,把姑娘扒光就好了。这个他偏偏不敢,一见她心砰砰跳,双腿软得走不动道。他跟姑娘商量,我这次还不扒你,你看怎么着能自己把衣服脱了。姑娘就告诉他去少林寺,当五年和尚,把你那些孽根修干净了,我自然会嫁给你。他看方丈听进去了,点头了,把袜子从他嘴里拽出来,说你看看怎么办吧,就五年,多一天都不麻烦你。方丈还是点头,自我认同一般地说:“嗯嗯,对,确实不行。”

田扒光能怎么办呢,简单直接就是揍,他擅长剑术,又不能把方丈捅死,两指掐着剑尖儿,用剑柄捅他。连捅了三天三夜,小和尚送饭都得放门口,别打扰方丈清修。也不用加餐,田扒光饭量没那么大,一人吃刚刚好.右手拿筷子吃饭,左手拿剑柄怼他。第三天夜里方丈终于摇头了,田扒光把袜子拿掉,三天没穿袜子,感觉凉着肾了。方丈摇头念叨:“不行,这样不好。”

田扒光问怎么不好,说出来我帮你分析分析。方丈说你可以把头发剃了,少林寺雇你扫地,外面我们不说你是临时工,可是你我之间要明白,你不是和尚,你就是给我们扫五年地。田扒光双眼放光,说这么好的办法,你摇什么脑袋啊。

“要分配你去人少的地方,免得人多嘴杂露了馅。”方丈讲,“藏经阁人少,可已经有人在那儿扫好几年了,”

田扒光打听什么人。方丈说和你一样,都不是出家人,我们叫他老沈头,年纪大了点,但其实活儿干得不错。田扒光问多大年纪。方丈说九十多吧,说完就摇头,十年前就九十多,现在应该一百多了,

“老而不死是为贼!”田扒光拍桌子站起来,“一个扫地的也能占着位子,不给年轻人腾地方。”

田扒光那年已经不年轻了,四十出头。他建议自己先跟老沈头扫着,他年纪那么大了,我猜他活不过这个星期了。方丈先摇头,再点头,也不知道行还是不行。过两个月田扒光再回想这些,会明白先摇头意味着,活不过这个星期的不是老沈头,应该是田扒光,再点头是说,让老沈头调教调教你,好像也不赖。

他还要点脸,别让人看出来一百多岁老头是被田扒光弄死的。弄点意外死亡吧,下毒是首选,砒霜、鹤顶红、断肠草,十大剧毒熬成一锅粥,盛一碗端过去让老人喝。老人闻得直皱眉,说什么东西,太难喝了。田扒光拉下脸来,说难喝也得喝,不然我一刀捅死你。

“为什么?”

“因为你辜负了我的一片孝心。”

盛情难却,老人有些感动,咕咚咕咚喝下去一滴都没剩。之后田扒光就望着他,十大毒物,平均毙命时长是七秒钟,田扒光数了七十个数也没见沈老头倒下去,他试探问没事吗。老人问他什么事。还什么事,一张嘴都能闻到剧毒混在一起的味儿。田扒光挥了挥面前的怪味儿,说十全大补,你吞吐一下,有没有翻江倒海的感觉。老人深吸一口气,闭着眼睛慢慢吐出来,再睁眼时田扒光已经昏倒在地上了。

活一百岁有什么用,贱命一条,肯定是吃了一辈子脏东西,百毒不侵。他换个思路,意外杀人还不容易吗?他挖一个深坑,就在回藏经阁的必经之路,下面刀尖朝上插了一百多把刀子,盖一层浮土,脚踩上去相当于凌迟。他等了一天,抢着扫把扫地,说了不下一百遍,你早点进去休息吧,我年轻,多干点应该的。老人不理解,年轻为什么要多干,年轻人应该多享乐,老年人玩不动了,才应该多干活。老沈头把院里结结实实扫了三遍,说你继续歇着,我进去给经书掸灰了。

田扒光可歇不了,他要看看老沈头是怎么死的。夕阳西下,他佝偻着身子,腿都抬不动,蹚着尘土往前走。他踩着边儿了,往前一步就是刀山。田扒光在他身后站了起来,屏住呼吸,半张着嘴看他在陷阱上面平蹚过去,依然佝偻着身子,布鞋底趿拉着地,跨一个门槛进了藏经阁。

哪里不对点哪里,他走过去,脚尖轻探一下陷阱边,下面哗啦啦地往刀尖上掉渣。他可不蠢,一脚踩实了作茧自缚,他弄条咸鱼骗只猫过来,蹲在陷阱另一头咪咪喵喵地叫。波斯猫盯着咸鱼亦步亦趋,前脚踩到陷阱,后脚刚抬起来,地表坍塌,一声惨叫。扬起的尘土扑了田扒光一脸,他拿着咸鱼站起来,不应该啊,猫有九命,那个沈老头,十条命也该下刀山才是。

几乎可以确定老人会武功,远胜于田扒光。九十多岁跑这儿来扫地,难道他也有一个八十多岁的意中人?再给自己一次要脸的机会,不行就真刀真枪地干,别怪我欺负百岁老人。那天下午沈老头把庭院阁中打扫完毕,田扒光问西边崖上的夕照石擦过没有。老人不明白,首先那块石头很干净,时不时有人过去修炼打坐,再就是他负责藏经阁,可是夕照石都不在少林寺,是嵩山派的地界,为什么跑那边打扫?田扒光跟他讲道理,如果大家都是各扫门前雪,那么,他顿了一下,好像各扫门前雪就够了,那么,那些不扫门前雪的人怎么办,就得由我们替他们打扫。

“再说了,”他说,“天下不扫,何以扫一屋?”

一老一少抬着扫把过去,要避开嵩山派的值岗关卡,不然人家以为是过来挑事的。夕照石在少林寺往西三里地,下面是深不见底的悬崖,每天日落石面都会被夕阳照得泛金光。田扒光说你上去擦,你替你放哨,咱们做好事千万别让嵩山派的人看见。他看着沈老头颤颤悠悠往上爬,石面滑不留手,跪在上面两掌贴在石面上,才不至于被风吹下去,哪里还能拿起抹布擦石头。田扒光给加油打气,你行,你可以的,战胜自我就会迎来更精彩的未来。可是人家都一百岁了,未来还想多精彩啊?老人撅着屁股在岩石上不敢动,田扒光想是踢他屁股,还是出掌推他下去。能用脚的尽量不用手,但万一他真是高手,闪转腾挪,一脚踢空能把自己翻下悬崖。

他朝夕照石猛跑,随时准备收力,对方就算后脑生眼躲开,自己也不至于冲下去。双脚跃上石头的一刻他推掌出去,沈老头没躲,可似乎也不吃力,浑身跟棉花似的,一掌下去怼不到头。双掌拔出来击他肩膀,沈老头肩头深陷下去。田扒光一掌到底,一直拍到石面上也没碰到老人的衣衫。他脸色煞白,半张着嘴看着沈老头,二十年来横行江湖,今日要毙命于此。可沈老头不还击,只是身形微动将每一掌化解。借你慈悲,要你性命,他掌掌下死手,沈老头的缩骨之法用到极致。田扒光知道,绝世高手的身体唯有头部不能缩小腾开。他右掌朝面门,左掌朝头顶百会击下去。无处躲闪,沈老头依然不还手,一丝恻隐令田扒光停下来,警告他再不出手就真没命了。沈老头摇摇头,闭上眼睛,夕阳映在他白睫毛上闪着几缕金光。

“罢了罢了!”

田扒光收手不打了,大不了不当和尚,硬着头皮把那姑娘扒光了就是,什么你情我爱至死不渝,衣服脱了姑娘都一样,以后还是做我的田扒光。他向后跳一步,打算下山,此时心中一凛,后面不是平路,双腿还没有着地。不知不觉中,几千斤的巨石已被沈老头转了半个圈,自己身后是万丈深渊。双腿不断下坠,指尖几次搭到石面,都因太滑脱了手,半空中他双手乱抓,拉到一只干瘪的手臂,顺着手臂往上望,是沈老头的白眉白发白胡子。

“你刚才为什么不杀我?”沈老头问。

田扒光愧得要死,红着脸说你那么大本事不还手,我哪还有脸杀你。仿佛刚悟到一个禅理,沈老头点着头说,有因有果,要是你方才杀了我,也就没人救你了。说完他松开手臂,背对着下了夕照石。田扒光以为自己完了,任凭身子下坠,仰头看云彩斜阳,死也要向阳而死,这时感觉身子轻飘飘地往上拔,飞上悬崖,越过夕照石,脸朝下摔在山坡上。他撑起来往前看,已经走到山腰的沈老前辈时不时从转弯处露出来,以前觉得他老不中用,现在简直是张三丰再世,跟他一比,自己连个蚂蚁都算不上。
\newline

{\centering\subsection{5}}

头几年田扒光每天都求沈老前辈收他为徒,他早就不喊他老头了,天天跟人扫把后面,抢着干活儿说,要是能做您的弟子,哪怕只是一天也死而无憾。老人停下手里的扫把,斜眼看他,说你本事也不小啊,江湖上没几个人能打得过你。没几个还是有几个的,尤其是这几个联手的时候,他也只能撒腿就跑。但江湖中人嘛,一般不联手,好事坏事大家各干各的,碰见好人嘴上说声久仰,恶人只要没欺负到自己头上,也犯不上多树一个敌人。武林中没善恶,以暴制暴,胜者为王,本事大的自然朋友就多,他田扒光恶事做尽,也没听说谁组团要干他,近几十年唯一一次联手还是很后来的事,大家搞了个联盟说是绞杀昆仑公子,列了他十条罪状,散开各地去寻访。其实大家都明白,罪孽深重的多了,只是昆仑公子多了几张九宫图,早晚要当武林盟主,好人坏人大家不忿。这些都是后话,那时八光还叫田扒光,天天磨着沈老前辈学艺。老人不明白了,以你的剑法,早该带几个徒弟了,怎么还千方百计找别人拜师。田扒光说以前收过一个女徒弟,合练了几个月玉女心经,结果人家姑娘含恨跳江了。沈老前辈听得起疑,问他哪学的玉女心经。田扒光承认他也不会,就借一名,自己没事瞎琢磨的,怎么爽怎么写,写完了跟弟子换着姿势练,练不到半年就露馅了,女弟子就感觉自己功夫没长进,肚子越来越大了,他骗姑娘说是气息不顺,淤结在丹田,为师今晚再帮你通一下任督二脉,换一般人也就信了,偏偏这姑娘绝顶聪明,孩子还没生,就猜到自己怀孕了。

沈老前辈看看他,估计这些都是假的,就为博他一笑,这孩子骨子不坏,当然奸淫无数算不上好,他说的不坏是,这孩子没有那种令人恐惧的野心,就是习惯性管不住自己。他不理他,低头继续扫地,田扒光恨恨地站在一旁,揪头顶的树叶子,一扯再扯,揪得手指翠绿,就那么一亩三分地,一天扫八遍。

田扒光缺点无数,如果说他只有一个优点,那就是恒心。第二天他寅时就来,拿起扫把就开始划拉院子,一直到中午,烈日当头,前辈都没从阁里走出来。那就得干下去,让前辈一推门就能看到他的勤快。午饭没吃,晚饭没吃,院子被他扫了七十多遍。月上梢头他把扫把放在墙角,冲藏经阁大门行了个礼说,前辈我先走了,犹豫片刻他又加了一句,明天我还来扫地。里面没动静,田扒光数十个数转身离开,院子门打开“吱”的一声长响,他退到门外,将门合上,这时前辈在里面说:“你明天铸把一百斤的铁扫帚来扫地。”

他从未教过他一招一式,每天只是扫地,扫把每两个月加三十斤,到二百多斤时已经很难再扫七十多遍的院子了。他一直想问,这都第二年了,终于讲了出来,不教我功夫是因为不认我做徒弟,铁扫把扫地是因为要练臂力,可是即使哪天我加到一千斤又有何用,你也只是用竹扫帚扫地。

“用不着那么多,”沈老前辈在藏经阁里说,“铁扫帚越用越轻,那不是功夫,等哪天你的竹扫帚越用越重,才算是有了一些底子。”

他隔着门和田扒光说话,细细想一下,他们俩已经一年多没见面了,老前辈一直在阁中足不出户。田扒光跟他打听,到底遇见什么事,让他一把年纪不享天伦之乐,跑到少林寺收拾卫生。

“我是来读书的,我要创立无为掌,借少林寺的典籍一阅。”

田扒光问他什么书,他也想看看。这一点沈老前辈没有藏私,从窗户里扔出五六本书。全是经文,拗口难读,有些直接就是天竺梵语。他请教,你拿这个怎么练功。

“这些都是佛经,当然练不了功,我只是要从这些经文悟些武学上的道理。”

田扒光听懂了,但没有兴趣,比铁扫帚扫地还令人费解。他把书摆齐还回去。沈老前辈提醒他以后别再进来了,最近要闭关冥想,无为掌只剩最后一个环节没打通,以后不再碰面,连话都不会对你说了。他要他再坚持一阵,把卫生搞好,每日一餐放在门口,等到出关之日,绝不会亏了你的。田扒光重重点头,知道他看不见,又大声说了句,弟子一切照办。沈老前辈叹了口气,说你现在必须明白,你还不是我弟子,我还不是你师父。“嗯!弟子明白!”

最后一个环节要想通,这一想又是好几年,秋扫落叶冬扫雪,第三年的时候方丈圆寂了,走得匆忙,没来得及交代谁来继任,第四年头上十几个二代弟子打得不可开交,最终被一个三十多岁的和尚杀出重围,力压群僧。可是师兄师弟都不认识他,有一个和尚想起来了,这不是藏经阁扫地的那个吗,老沈头来以前就是他,好像还给老沈头打过两年的下手。新方丈开始编身世,说我在藏经阁扫过地没错,可哪来的师兄,你们全是我师弟,我是老方丈的秘传大弟子,为了本寺的千年大计,蛰伏藏经阁取经学艺来着。寺里的和尚有一大半不信,没关系,证明给你们看,揪起衣领就是一顿暴打,卸胳膊卸腿脚筋挑断,看看是不是本门的正派武功。有几个骨头硬的,牙被打掉几颗还满嘴漏风不承认。新方丈退后一步,承认是自己的错,出手太快,没让师弟看清楚。说完他突然上前,左手抓衣领右手扇巴掌,要么打死,要么跪拜新方丈。

有一件事冲击到了田扒光,新方丈只来硬的,不来软的,头二十天害死一百多个和尚才顺利继位,八方来贺,百鸟朝凤,也就是半年,新方丈摇身一变,就成了慈眉善目的得道高僧,像之前的每任方丈一样,寺里的和尚真心觉得新方丈是有大智慧、值得信赖、可以如大山一般依靠的一寺之主。活下来的大多数和尚都被他害过,毒打禁闭责罚凌辱,如今拥护他是少林寺百年难遇的好方丈。田扒光不明白,人们怎么会那么快就忘了疼痛。

那一年方丈来了藏经阁,正午时分下了漫山的大雪。田扒光说沈老前辈还在闭关。方丈说没关系,我们出去踏雪赏梅。雪是下了不少,可方丈根本不打算赏梅,走出去停下来,转身对田扒光说:“我不知道你来少林寺是什么目的,但我想让你清楚,方丈这个位子是我的,我屁股下面的椅子可结实了,坐不坏。”

两三句话田扒光就听出来,这方丈跟他一样,都是假和尚,他从藏经阁学了不少本事,自然就提防其他后辈扫地僧。田扒光表示,可能是你悟性高,从阁中经书里读出武学奥义,我是屁也没读出来。

“当然读不出来!”方丈像只母鸡咯咯咯地笑,他扬起下巴,点了点藏经阁的方向说,“你跟他学,偷点皮毛都够你独步天下一辈子。”

独步天下,还能一辈子。田扒光确定他不行,也就是少林寺功夫一代传一代,丢得太厉害。田扒光说你去吧,只要你能独步少林寺,屁股底下那椅子你能坐一辈子,我不管你,我对当和尚头儿一点兴趣都没有。

“我也是为了少林寺,”临走时方丈说,“没人比我更适合带领这些和尚了。”

田扒光信,坐稳位子这半年,方丈也拿寺里的和尚当家人待,将发扬武林第一门派做为己任,平时都不见他睡觉,日夜处理繁杂琐事,但凡武林出点事他都在思考,少林寺能捞着点什么,怎么解决才能看似公允,维持体面,而少林寺才是最大获利者。

谁要当和尚头儿,他在等一个人,五周年的时候那姑娘果然来了,不顾父母家人反对,八抬大轿上山,把门的小和尚看出来她是女的,不让她进寺。田扒光扫把来不及放下就飞奔出寺,远远看到意中人在跟小和尚打听,你们田师兄当真在少林寺扫了五年地,当真吃五年斋念五年佛?小和尚不明白,谁是田师兄,见着田扒光过来才反应过来,是他啊,八光在这儿呆了足五年呢。

几年不见她更丰腴富态了,以前是闺房千金,现在都像豪门少奶奶了。她问他法号是什么,打听半天不知道怎么称呼你。田扒光说扒光以前是外号,现在是法号,只是扒字去掉了提手,师父给他起的,寓意八样罪孽统统消光。见她疑虑,他又补上一句,古人不是说嘛,掏光才能养晦,我这八样都掏光,不知道以后要成多大事呢。姑娘听得泪眼婆娑,你果然对我一片痴情,在这儿当了五年和尚。说完她还是哭,五年相思苦,好像要一时片刻都把它哭出去。

田扒光伸手托住她的脸,抹掉她眼泪,把她安顿在小溪旁的草屋里。他说现下还不能走,师父在闭关,他答应过要等他出关才离开少林寺。话说一半他卡住了,看到她正含情望着他,他咽了口唾沫,挥挥手说算了,你休息一下,我明天一早就跟你下山。

他睡地上,把床留给心爱的女人。两人谁也睡不着,互诉衷肠,讲讲这五年过得怎么样。田扒光是假和尚,一时不知道怎么润色这五年。事实证明他想多了,主要是意中人在讲,掺杂着哭声从床上飘下来。她哭着说你对我真好,无论如何我也想不到,这辈子对我最好的男人竟然是你。她说你出家的下半年我就嫁了,京城的一个官宦子弟,家财万贯,婚后第二年她给他生了个儿子,这男人别的都很好,就是脾气有点大,喜欢打女人,抽着鞭子还能气得声音发颤,终于有一次,他照例把她吊在房梁上,一鞭子抽出去,这口气却怎么也上不来,瞪眼指着她气死过去了;第二个男人是江南才子,诗词歌赋样样精通,虽然卖不出去,但他不打她,算是一种别样的幸福,饥肠辘辘却爱意绵绵,她给他生了个女儿,他也很好,可人都有缺陷对吗,诗人才子都有点骚情,有钱的从窑姐儿那里找灵感,没钱的就只好从别人家媳妇那里找灵感,就在上个月,她第二任丈夫被人当街打死了,裤子都没穿。听说凶手是他姘头的老公,带人进家抓了个现行,追到大街上用乱棍打死。但是我男人知道错了,都光着身子跑出来了,何苦还要杀绝呢?

说完她长叹一口气,黑暗中仿佛一颗拉长线的流星,全讲出来她感觉好多了,她说既然活着还得往前看对吗,幸好有你爱着我,明天我们下山,去绍兴把孩子接上,其实这五年也不算是浪费,起码你看,我们还是儿女双全的。

田扒光好半天没说话,他总觉得哪里不对劲。他从席子上坐起来,摸着黑把事情捋一捋,他问:“当时是你说,我若能去少林寺当五年和尚,你自然会嫁给我?”

“对啊,我这不是来了吗,你做到了,我也会兑现承诺。”

“那也不对。”他想不通,对着黑暗深处冥思,“但是,你在这五年又嫁了两回。”

“你想多了,我说我会嫁给你,但我没说我等你,而且就算我嫁了两个人,生了两个孩子,现在的我跟当年那个我还是一样的,我还是我啊。”

还是不对,他不再问了,起身找支蜡烛点亮,走到床头烛光在她的脸前晃了一圈,怪不得丰腴富态了许多,确实还是她,还是那么好看。他将蜡烛放在桌上,手指将烛光掐灭,黑暗中都能听到自己慌张的心跳声。手上还沾着蜡油,他去撕她的衣服。意中人求他不要这样,拼命挣扎往床里面退,说你若是想要,我们现在就可以拜天地。田扒光手上一拽,黑暗里传来布料撕开的声音。对他来说,这是那么熟悉又舒服的声音。他再扯一件,扯第三件,有个奇怪的念头冒出来,他不是在扒衣服,他是在告别,每撕掉一件,都是向这五年的自己挥手说再见,可能也在向她告别,些许不舍,但要对这五年有个交代。手抓过去只剩下肚兜,他听见她在哭,遇人不淑,第三个男人对我也是这般。手指捏着绸子边,他开始害怕了,他松手下床,回到席子上。他怕这一扯下去,就再也见不到她了。

很奇妙,睡得还挺香,一个噩梦续接一个美梦,来来回回都是美好结局。睡到半夜他意识到有嘴唇在亲自己的脸,她从床上下来了,双手抱着他的头连亲了十几下,用哭哑了的嗓子说,我知道你很苦,我对不起你,原谅我吧。田扒光浑身发颤,使了好半天劲才将抖动的上下牙合起来。他侧身抱住她。

一片漆黑,什么都看不见,可是抚摸她的感觉如此真实。她说挺好的,特别好。然而他放弃了,从她身上站起来上了床,对着天棚仰躺。两人一时都不说话,她留在席子上,侧过身对着床边,说可以了,其实这些足够了。他叹了口气,打断她。她也知道,此时此刻最好什么话都不讲。

后来天亮了,面前模糊的脸渐渐渐渐清晰起来。仿佛自言自语,田扒光说什么东西都一样,越用越有,今天赌明天还想赌,今天喝醉明天还想喝醉,可要是长时间不用呢,可能就永远失去这些东西了。
\newline

{\centering\subsection{6}}

二月初二沈老前辈出关了,满面春风神采奕奕,好像在里面几年还胖了一点。那天他亲自下厨,把锅搬到山下田扒光的草屋里,煮了个大猪头,猪舌炒辣椒,猪耳朵凉拌,剩下猪头肉大块蒸了蘸蒜吃。田扒光一口没吃,我现在有法号了,还是八光,肉啊酒啊不能随便用的。

“少林寺把你收了?”

“少林寺没收,是我把自己收了。”

说话没头没尾,沈老前辈也不多打听,嘴里嚼着猪耳朵在纠结下一口吃什么。今天心情特别好,话也多起来,他说我已经不是原来那个我了,现在的我更厉害,以前我只有三掌,现在我已经是断魂掌、仙人掌、蓬莱掌和无为掌,这四掌的创立人了。八光瞠目结舌,虽然以前隐约猜过,但没想到真的是他。南海真人、大漠仙人和蓬莱阁老在江湖上的名头太响了,都知道他们是师兄弟,可从来没人讨论过,谁教的三掌,他们师父到底是什么来头。方丈说的,偷学一点皮毛就能独步武林,那是他夸张,独步少林寺吧,那这三个人各学了他三分之一,却实实在在的并肩当世三大高手。

八光跪地叩拜,说小僧有眼不识泰山,还望老前辈恕罪。沈老前辈说,你当年要杀我都不怪你,不认识我有什么好恕罪的。酒足饭饱,他拍拍肚皮,狠狠地打了个嗝。八光说今晚就不要上山了,不嫌弃的话,就留这里过一夜吧。沈老前辈本来就没打算上山,至于过夜呢,这几年在藏经阁早都睡够了,以后不睡了,一直到死也用不着睡觉了。八光问他这么晚了要去哪里。沈老前辈卖关子,说要去办件大事,见八光满脸不解,他一步一步跟他分析,我自创了无为掌对不对,我这么大年纪了对不对,我随时可能老死对不对,无为掌不能失传对不对,我得找个传人对不对?

“收徒?”

“对,我要收个关门弟子。”

心脏都要跳出来了,八光要克制,装作不知地问他:“想收个什么样的徒弟?”

沈老前辈摇头说:“不知道,出去看看,随便找一个就好。”

怎么会这样,八光脑袋嗡嗡地响。沈老前辈出门时,他克制不住了,抱怨我伺候你五年,你宁可上街随便找一个,也不收我为徒。好像是不好,沈老前辈停住脚步,捋了半天胡子,想到一个两全的好办法:“你别走,等我回来伺候你十年。”

第二天一早他就回来了,八光以为他改主意了,问他是不是徒弟不好找,现在风气坏了,你往江湖走走,发现师父比徒弟还多。沈老前辈摇头,说随便找有什么不好找的,昨晚刚下山就碰着一帮要饭的,干脆就收领头的那个做关门弟子了。八光脸上酸溜溜的,那表情仿佛说,哼,好吃的喂狗也不给我。

“那他什么时候找你拜师学艺啊?”

“教完了,师父领进门,修行在个人。我教了他大半夜,这道门我起码领着他进进出出了三回。”

“那可全看他个人的修行啦。”

八光面露喜色,美其名曰关门弟子,师父只给你半宿的时间,这种弟子不做也罢。这天八光跑上跑下,格外勤快,有一种喜悦是,你没得到的东西,别人也没得到。到晚上沈老前辈看出了他心思,给他讲故事。他说楚王约庄子画条龙,问他多久能画出来,庄子说十五年,头五年过去了,楚王问他画得如何,庄子说还没动笔,又五年过去了,楚王问他画得如何,庄子说还没动笔,到第十五年该交稿的日子,庄子空着手进殿,楚王问他龙呢,庄子说还没画呢,楚王叫人准备狗头铡,庄子叫人准备纸和笔,画画看吧,庄子伏地挥墨,小半个时辰,一条活生生的龙被他画了出来,文武百官交口称赞,庄子说,之前的十五年我虽没画,但我一直在想,画很容易,想明白才是最耗时的。

“无为掌我想了快二十年,”沈老前辈说,“道理想通了,让他去练,他也要明白,一掌苦练十几年,可打出来的时候,这一掌推出去,胜负成败,是生是死,也只是一眨眼的功夫。”

故事讲完,月亮从乌云里出来照在他脸上,八光发现他一夜之间就老了,虽然之前也不年轻,但这次更像是垮了,整个人瘪在那里。八光起身准备下山,他说时候不早,您也早点休息。沈老前辈没反应,眼神直勾勾地看着前面说,我睡够了。八光想劝两句,看他那样子不是听不进去,而是根本听不到。他行礼告辞,沈老前辈依然看不到,目光呆滞看着某个点。

下山的路上他明白了,沈老前辈是活太久了,人要活多久才会活腻,活到你所有的事情做完,然后发现自己还活着。一百岁之前他创立三掌,到无为掌出来,他便实在找不到事情做了。后来他真的不睡觉了,日夜十二个时辰一直睁眼,本来就没事干,多出来的时间更是煎熬。他在藏经阁找个角落,面墙而坐,有时三五天不吃饭不动身。好几次八光都以为他死了,跟高僧圆寂似的枯坐而亡,一推就倒。他轻手轻脚走到他身后,看见他眼睛瞪得老大盯着墙角的斑点。看着墙能想些什么呢,人真能面壁思过吗,八光想他活了一百多岁,活过两朝四帝,从上一朝的思清帝,到亡国的隆治帝,到新朝凌武帝、嘉和帝,四个时代他都见证过,加上自己经历的,那么多的往事细细回想,三五天面壁哪里够啊?

八光在少林寺呆到第十三年,方丈给沈老前辈做了一百一十岁的大寿。少林寺主持了十年,他越来越像一个有道高僧,性格都变了,谦逊内敛,碰着什么事都不紧不慢不慌张。他一个人过来的,知道沈老前辈的也只有他们两个。方丈到藏经阁已经很晚了,一天一夜从京城赶回来,中秋夜里皇宫里出了大事,昆仑公子行刺皇帝未果,将太子劫走了。沈老前辈问昆仑公子是什么人。方丈说他也没见过,这两年的后起之秀,下手挺狠,被他戕害的门派能有几十个,好在没得罪少林寺。沈老前辈点点头,确定这人和他的三个弟子没有关系,也就不想再打听了。后来方丈说到九宫图,沈老前辈来了兴趣,连问好几句。方丈说九宫图啥,他也不知道,少林寺没有这东西,听说昆仑公子那儿有几张,还给少林发请帖说要中秋赏月,实际上就是请大家看看他的九宫图。方丈觉得时间有点蹊跷,都是中秋夜,这边邀请了好多人看九宫图,那边却去宫里行刺皇帝,让大家去昆仑山庄扑了个空,一个晚上计划两件事,昆仑公子到底要干什么?

不知道方丈在问谁,八光转头看过去,沈老前辈在走神,嘴里念叨着九宫图。方丈岔开话题,说明来意,他说从八光那里听说了你这几年的情况,我知道你时间太多了,每天都在熬,在想阎王爷怎么还不把你带走。方丈建议他入我佛门,佛海无边,到时候恐怕你每天都会觉得时间不够用。

沈老前辈回过神来,让他再说一遍。方丈耐着性子又讲了一遍。沈老前辈寻思片刻,婉拒了他的邀请。他先感谢少林寺收留了他二十年多,感谢方丈替他着想,他说这是个好主意,但是这里面有私心,我苟活了一百多岁没有出家,此时却为我的这一点私心烦恼遁入空门,我是在亵渎佛祖。

方丈点头称是,不再和沈老前辈争辩,他问八光怎么打算,要不要少林寺给他补个收徒仪式。八光说我早就把自己收了,我已经是八光寺的弟子了,没办法再当少林寺的和尚了。说完他自己都笑,惹得方丈一起哄笑。笑着笑着二人停下来,他们看到沈老前辈又去角落面壁了。方丈在后面行个大礼,说师父多保重,弟子先去了。

“你我没有师徒的名分,快快去吧。”

方丈深鞠不起,好半天才转身告辞。没师徒名分,却有师徒的情分,第二天他就安排两个小兄弟和尚给八光送菜送饭,午饭送过来,到晚饭他们又来了。八光问他们要送多久,小和尚弟弟挠头说不知道哎,反正方丈说从今天开始每日三餐往藏经阁送。

两个小朋友挺勤快,就是话有点多,尤其是哥哥,每次过来都要打听,你这儿扫几年地了,你师父教你武功了吗,少林寺的功夫到底行不行啊?八光装糊涂,说哪来的师父,扫地这种事还用教吗?哥哥捅捅弟弟,冲他眨眼睛,在他耳边轻声说,你看,我就说种菜比扫地有前途吧。

春夏秋冬,八光先是不记日子,后来连年份都不查了,不知又在寺里呆几年,只看到两个小和尚越长越高,声音却越来越尖。印象里小伙子不是这样发育,可能在少林寺呆太久,外面世界都变了吧。有一阵两个小和尚有点怪,神神秘秘的又忍不住想嘚瑟,他们让八光别说出去,这个秘密只对他讲,菜园里来了一位高手,具体是谁我们不能告诉你,反正跟昆仑公子有关,她都答应收我们为徒了。

“等我们哥俩学好了,”哥哥说,“就收你做开山大弟子,把一身的武艺交给你。”

“两身,”弟弟说,“咱们俩人呢。”

过个十来天,哥俩又耷拉脑袋了。他们不好意思说,八光也不问,估计人家不收,一个个五大三粗,说话却女里女气的,换他八光也不要。好像又过几天,少林寺出乱子了,南海真人一路打到藏经阁,最终跪拜离开。更糟糕的是后面,那些和尚看到八光的本事,听到了阁中老人居然是南海真人的师父,一茬又一茬地过来拜师。一点规矩没有,带艺投师没问题,带师学艺可是江湖大忌。方丈中了断魂掌,少林寺已经乱套了。八光抡着铁扫把守在藏经阁门口,警告他们各回各的庙,敢跨进一步,我用这扫把在你们脸上刺花。

小和尚兄弟俩这次倒出奇的乖,按时按响送饭,多余的要求不提。出事第三天的饭送得有点晚,月上树梢才把晚饭送过来。弟弟脸上有两条血道,似乎是打架被人挠的。哥哥先拎出一篮子酒肉,说这是孝敬里面老前辈的,然后他使个眼神,弟弟去外面抱进来一个麻袋,哥哥说:“这是孝敬您的。”

他太熟了,闻一下他就知道是什么,先不管那麻袋,把酒肉送进去。沈老前辈还在对墙想事情,出事后的三天他一共说过两句,都是第二天中午说的,第一句是,八光,我担心他还没练成,就被我那三个不肖弟子给杀了。这是在藏经阁说的,八光那时还在院子里扫凉亭,他放下扫把进去,问他练什么。这时沈老前辈说了第二句话,无为掌。

就这两句话,换以前八光早说了,再传我一次吧,多一份保障。现在他不说了,跟沈老前辈相处了那么久,他慢慢明白,有些东西不是因为你多想要,人家才给你的。再说他在少林寺呆惯了,他不想下山了,就算练成天下第一,他还是想在这儿扫地。

今天晚上沈老前辈难得又说了第三句和第四句话。先是八光把酒肉篮子放下,说这是那两个小和尚做给你的。沈老前辈头也不回地问:“那孝敬你的呢?”

八光知道他内力好,百步之外的脚步都能听到,可能这就是活着的烦恼,耳朵太好就像蜂巢,哪怕活到一百多岁,那些乱七八糟的声音还是会一窝蜂地往里钻。八光说,我一会儿原封不动地还回去。

“还是打开吧,看一看,你能不能还回去。”

“我怕看过之后就舍不得还了。”

沈老前辈不说话了,没准是今天两句话定额用完了。八光等了一会儿,走出藏经阁,到院子东头的凉亭,从石凳下把麻袋拽出来,里面还在动,前几天闻过这味道。他解开系口绳,不出所料,是那个文思清,昆仑公子的女人,嘴里塞着东西,呜呜呜地喊不出来。她望着他,不住地摇头。八光扭过去,不敢再看她。沈老前辈说打开,看一看,我能不能还回去,不管是看一看我,还是看一看她,反正都看过了,我能还回去的。

可他还想再看一眼,看看她眼睛,转回来和她对视。和以前的那些姑娘一样,她眼睛里充满着恐惧、求饶、绝望,偶尔还会掺杂一丝不切实际的希望。他忽然意识到不是别的,正是这一类眼神让他兴奋不已,过去犯的那些罪行,似乎都是因为这样的眼睛。他不敢再看了,提起麻袋边儿,他今晚终于明白,骗他出家的那个女人,不一定是爱,他只是没有在她眼睛里看到求饶和恐惧,也许当时他就不行,错把那当成两腿发软的爱。被绑那天,她到底是什么眼神呢?热切?期待?无所谓?他说不上来,可能那就是一双荡妇的眼睛。

文思清还在挣扎,高举手臂不肯被套进去,手腕从袋口露出,死活不让系上麻袋。八光看着她的手,情不自禁摸了一下,这一下就仿佛被吸住了,从手腕一直摸到胳膊肘。然后他撕开麻袋,将她双脚抓过来,扯掉袜子,用拇指中指轻抚她脚踝。文思清一直在哭,嘴里塞着东西含混不清地求他放过。八光松开她双脚,站起来,看着文思清坐地上往后退。

“你别跑了,跑不了,”他向她靠近一步,影子罩住她整张脸,“我绝不会放过你这样的姑娘。”
\newline

{\centering\subsection{7}}

方丈说要找文思清谈谈,从八月一直拖到十月,等见面那天方丈都不记得要跟她说什么了。他们约在达摩堂,两个人面对面盘腿坐在达摩脚下,中间放着一壶茶两个杯子,年轻和尚将左右两扇大门打开,退到几百步之外。北方已是深秋,午后阳光映在每一片红叶上,似乎在催它们早点落下去。

方丈不说话,冷眼看着她,中掌之后这成了他的新习惯,脑袋里是空的,全都是陌生人,他等对方说话,抓紧认识每一个人。有两个自称十六师弟和二十一师弟的老和尚,告诉他这是少林寺,而你是这里的方丈。他俩故意轻描淡写,想看到方丈满脸惊讶,我怎么这么厉害!惊讶确实有,但不是因为位高权重,他摸着自己光头,惊讶自己怎么会是出家人。到现在他都不相信,老怀疑这帮和尚藏着什么阴谋,已经一个多月了,他了解自己,天天做梦都是喝酒吃肉娶媳妇,天底下不可能有这样的方丈。

他看经文,中文的都读不明白,更多是梵文硬转过来的,般若波罗蜜,读都读不利索,怎么可能倒背如流还开坛讲道?他问过好几个人,倘若中了断魂掌,记忆是没有,本领会不会丢掉,比如学识,比如武功?所有人都告诉他,不会,我说少林寺,你知道天下第一门派,我说和尚,你知道吃斋念佛,但这些可不是生来就知道的。

那就对了,《金刚经》《易筋经》一窍不通,绝不是方丈。他想各种可能,最符合逻辑的是武林每三十年有一个下油锅大会,所有掌门人聚集一堂,脱光衣服跳到油锅里,炸酥炸脆方可出锅,他一定是真方丈被拉来顶包的。越想越接近真相,一时还跑不了,山上面的和尚换班盯着他,山底下那些也绝不是知客僧,而是怕他冲破重重关卡,为他设置的最后一道墙。他翻箱倒柜,看有什么办法逃出去,柜子底层有几卷少林寺住持记录。打开翻看,一天一页,十年下来攒了三千多页。随便翻一页,他用毛笔在旁边写几个字。之后他愣住了,一样的字迹,的确是方丈,一个不学无术的方丈,架上那些经书从来没读过。

把这十几年的笔录好好读一读,足不出户他连读了三天三夜,第四天早晨合上最后一页,他倒头就睡。傍晚醒来,两个师弟给他送饭,他看着他们铺席支桌,将每样小菜分碟盛出来。他先不吃饭,走过去拿住持记录,问他们文思清是谁?十六师弟说昆仑公子的女人,已经在寺里呆两个月了。

方丈点头说:“那就对了,不是她呆了两个月,是我一直不放她走吧?”方丈把笔录翻到那一页给他们看,“武林大会那天写的,把文思清带回少林寺,近期要和她谈一谈,我要谈什么?”

两个师弟不说话,他们也不知道。

方丈合上笔录,抬头说:“少林寺不能进女人,我让她住了两个多月,一定是要谈件大事。”

当然没法问文思清,知道我要跟你谈什么吗?那就先让她说,他问她在这儿两个月还习惯吗?文思清不说话。方丈知道问得不对,和一帮和尚住一起,她可能习惯吗?他换个问法,问她在这儿过得好不好。她说有时候好,有时候糟,但总算没死掉。文思清是认真的,面无表情,那种劫后余生看淡生死的语气,听得方丈都想给她道个歉。

他给她斟茶,躲开她怨念的眼神,看大门外的落叶。文思清双手握杯小酌一口,她说没有直接的那种好,好的都是苦尽甘来,两个小和尚把她绑起来,怎么挣扎都没用,挠花了弟弟的脸还是被装进麻袋里,扛过去说要孝敬八光。本来他都要放我了,不知道看上我哪一点,可能是手指长手腕细,他说绝不会放过我的,把我拽进小屋,要我把衣服脱了,换上他给我备好的那一套,他在门外等,不知道他什么嗜好,我想没有刀没有绳子,我用什么办法可以自杀,我试着咬舌头,只能疼,根本不可能流血而死,再进来时见我还没换,他给查五十个数,不然他给我换,我把自己衣服一件件脱下来,穿上他给我的,不知道什么衣服,只是很宽松,袖口腰上都要用带子去系,我双脚拖着地面出了小屋,背靠着门,他对我上下打量,说了一句话,这才是习武之人。

“你能相信吗,他要收我为徒,他说看我骨骼奇特,是习武的好料子,上好的料子,说我这种骨质不管谁教,总之是要超过师父的。我说我不学了,我都二十二了,做你徒弟早晚要给你丢脸。八光一个劲地摇头,说你别想跟别人学,昆仑公子也未必打得过我。我能怎么样呢,我若不拜他为师,不知他会对我干什么。”

文思清停下来,又喝了一口茶,问方丈练过武吗?方丈低头看着手掌。文思清说你当然练过,你还打过我一掌,听说当时你为了夺方丈之位,寺里面杀了几十个和尚。她没留意到方丈一脸震惊,继续说:“我是没练过武,八光师父做什么,我就跟着他做,练了两天说我不行,怪我什么都不会,就从扎马步开始,大太阳底下,他拿小棍盯着我不许动,后来看我哭了,估计是失望,他叹口气,陪我一起扎马步,又扎了两三天,藏经阁里的沈老前辈都听不下去了,一个劲儿地骂他蠢材,接着他讲了一堆武学道理,我听不懂又记不住,就看到八光师父一边点头一边冒汗,最后沈老前辈说,照你这么胡乱教,东剪西裁,再好的料子,恐怕被你剪得连手帕都不够做。说着他从藏经阁走出来,八光后来告诉我,沈老前辈已经六七年没出阁了。他背着手出来,眯眼看着我,说确实不错,转身对八光师父作揖,说沈某想收这个女娃娃为徒,不知八光师兄是否应允。没见过沈老前辈对谁那么恭敬,八光师父说这是江湖规矩,跟人借徒弟总要走一个客气点的过场。可八光师父却跪下了,说师父若收我二人为徒,我二人无以为报。沈老前辈也不跟他争辩,转过来对我说,你是我第五个徒弟,也是我关门弟子。八光师父叫我磕头。其实他不说我也知道,老人家一百多岁了,磕一磕也是我自己的福气呢。我得叫沈老前辈师父了,但不让八光叫他师父,八光师父表面上不叫,私下就喊我师姐,越喊越高兴,我都拦不住。他说更高兴的是因为,沈老前辈有事情做了,活着不是耗神等死,起码要等到我出师。说完他瞅着我笑,说我这么笨,师父得活到二百岁才能把我教出师。我就是笨啊,师父教我一遍,旁边跟着学的八光早练熟了,我却练几百次也练不好。但是奇怪呢,倘若只用师父教的招式,我和他对打,八光竟打不过我。”

把话说完文思清忽然就顿住了,仿佛鱼刺卡在嗓子眼,她睁大眼睛望着一直在倾听的方丈。茶水凉了,文思清双手捧起咕咚咕咚喝光。方丈还在想,这些和他要谈的有没有关系,他问当时我对你还说过别的吗。

“你说,你要和我谈的事情关乎江山社稷,可是小五子和江山社稷能有什么关系呢?”

方丈低头翻笔录,经文一窍不通,半个月已把这三千页日记倒背如流。江山社稷,那是朝廷皇位,方丈一路往前翻,三年前去过一次京城,赶回来为沈老前辈过一百一十岁大寿,曾建议他入我少林,被沈老前辈婉拒。他再往前翻,去京城做什么,上面写着觐见五公主,太子被昆仑公子劫持,三年之约,要求少林寺连同各大门派务必救出太子,倘若完成,朝廷重赏少林,继续奉少林为天下第一门派,倘若太子死于非命,少林必定被夷为平地,后面还有一行字,下面画道横线加重,当心三王爷加害太子。他合上笔录,望着门外的秋色皱眉,问她:“小五子是谁?”

“就是你们说的昆仑公子啊,可我一直认他是小五子。”

“和我一样,中了断魂掌?”

“嗯,跟你一样,什么都不记得了。”

“有没有可能,小五子其实是太子?”

文思清拨浪鼓似的摇头,说你这是中了断魂掌,不然你绝不会这么想,武林大会那天那么多人,见着小五子,一大半人都要冲上去复仇,是不是昆仑公子,他们会不知道吗?

“万一那些人都是三王爷安排好的,把太子当昆仑公子杀了?”

文思清摇头,低声说不可能,昆仑公子已经够不可思议了,她怎么可能会嫁给太子?方丈说过去的事记不起来了,这也是他瞎猜,至于小五子到底是谁,等见到他慢慢查问吧,他问她还打算在少林寺呆多久。

“我不再关着你了,你现在随时可以离开。”

文思清说不知道,在少林呆了两个多月,不好的都在变好,不习惯的都在变习惯,就在这儿等小五子,他若来接她,当天就和他下山,他若不出现,就陪师父呆到二百岁。然后她问方丈,你呢,要在少林寺呆多久。方丈躬身斟茶,说想不明白,过去什么样的野心让自己一步步熬到这个位置,自己天生不该是这里的人。说话间茶水溢出来了,他放下茶壶说:“就在这儿一直呆下去吧,我走了,这些人怎么办?”

忽然起阵微风,两片红叶吹到房间里,落在茶壶边,方丈捡起来一片,夹在指间。文思清将另一片捻在手指上,起身向大门走去,秋日傍晚的阳光延绵而悠长,她回头看到自己斜长的影子映在达摩佛像上,她要嫁给小五子,不管他是谁,不管未来发生什么事,一辈子总要嫁给他一回。

\newpage