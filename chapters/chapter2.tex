\section{贰}

{\centering\subsection{1}}

通常要早上五点钟醒来,要杀猪,刮毛剔骨,趁集市开
张前把一整只猪分割成一块块摆到案板上。小五子总要
早醒来一会,他得花点时间寻思一下,自己是谁,睡在哪张
床,昨天的事还记不记得,好确认新的一天可以继续,不用
从头过。他现在都有点怕了,在田独一年多,每件小事他
都认真记下来,生怕哪天一睁开眼睛又忘了。

想不了多久就会被钱老板进门叫醒,开始一天的工
作。钱记猪肉之所以好吃,是因为他们从来不把猪绑起来
杀。打从小五子进肉铺那天,钱老板就跟他说明白了,让
猪跑起来,一刀致命,猪肉里不会有淤血,全身的活肉。

其实钱老板什么都不会说,他是个哑巴,“咿咿呀呀”
的声音都很少发出来,也不见他打哑语,估计是中年失声,
现学都来不及了。不过也无所谓,杀猪就那几个步骤,把
刀磨好,用竹竿捅捅铁笼里的猪,差不多快激怒时,打开铁
笼,任其在院子里横冲直撞。一年多的时间,他早己熟悉
这工作,左躲右闪,还不至于让猪撞到。他在等钱老板的
手势,等猪活动开了,钱老板手臂一挥,小五子抄起杀猪
刀,迎着发狂的活猪,一刀从脖子上划下去,再顺势一挑,
刚好把猪落到案板上。

钱老板讲不出话,但是听得懂,每次这时候小五子都
会说,自己这使刀的本事绝对是天生的,打进你们店我也
没练过啊,没准以前就是一位武林高手,哈都记不起来,给
你杀猪来了。他打听自己到底是怎么来的,就记得一觉醒
来,在你这儿杀猪卖肉了。他是叫小五子吗,以前叫什么
名字?手臂刻了不少字,瑶、百花、断魂、五,看来看去也就
是这个“五”字和自己有关系,父母是谁不知道,上面兄长
青定有四位,就叫小五子,没准他们以前都叫他五爷呢。
他连自己生辰、多大都搞不清楚,说着说着还挺难过,忍不
住地抱怨几句。钱老板说不了话,也懒得搭理他,背着手
走出肉铺,留他一个人,守着一只死猪,等待第一个客人。

第一个客人迟迟不来,小五子冲着手臂发呆,百花又
断魂,这个“瑶”字一定是某个姑娘,他找了一年多,田独没
有哪个姑娘带“瑶”字。那就走出田独,往南方去,可是他
又不敢,人生地不熟,连个朋友都没有,生怕哪天睁开眼
睛,又把这短短的一年多给忘了。

百花一定不是这里,田独哪有花,极北之地,一年最多
有三个月的夏天,大概有九个月都处于寒冬之中。等进了
腊月天都不怎么亮,每天从下午就开始黑下来,一直要熬
到第二天中午才勉强见到太阳。那时有事抓紧办,卖肉的
客人也陆续上来,也就两个时辰,天色又暗下来了。盛夏
时节刚好相反,根本没夜晚,天就那么一直亮着,直到三更
才能抓紧时间休息,可也没多久,感觉刚睡者,公鸡打鸣,
天又大亮,人们又出来活动了。

客人没上来,倒是来俩官兵,卷一张通缉令让他贴店
门口。他问这次通缉谁。当差的说,还不是昆仑公子,朝
廷的命令,抓不到这个人,就永久通缉。他退后两出,看到
墙上能撕的全都撕掉了。他问上次旧版的呢。小五子说
毕竞是做买实,门口老贴通缉令不好。

“下次再撕,把你家猪牵走。”当差的说,“贴新版的吧,
据说这版和昆仑公子本人更像了。”

小五子多刷点糨糊,把通缉令贴墙上,也没觉得和旧
版有什么变化。画像的师傅本事有限,再加上有偏见,故
意把昆仑公子画得贼眉鼠眼,想靠这个抓人,别说两年,二
十年都抓不着。

到了下午人多了起来,明天是秋分,贴秋膘的时节,仿
佛不吃点猪肉,撑不过冬天似的,个个举着银子让小五子
割肉。每份三五斤他都短个三五两,三百斤猪能克扣三十
斤肉钱。不到下午一只猪就只剩下二十斤,这个不能卖,
他给何员外府上留的。这是田独镇上最大户的人家了,老
爷姓何,大腹便便,成天笑呵呵的,待别落欢吃钱记的猪
肉,有时候路过还会跟小五子闲聊两句,估计是中原犯了
事流放到这里。

小五子每天都会给他们留二十斤臀尖,把这个送过
去,这一天算是收工了。何员外的肉可不能短斤少两,吃
好了是要领赏钱的。这一次他扛着臀尖捶大铁门,老管家
慢悠悠打开一条缝,门缝里露出满脸褶子的半边脸,说把
肉放下吧,老规矩,月底一起算。门都不给开,小五子偏要
见见员外,给他看看这肉新鲜不新鲜。他的心思很简单,
见着何员外,说几句奉承话,赚点赏钱花。每到这时候,他
都觉得这管家也是狗仗人势,门缝里看人,再说七老八十
了,还管什么家啊?

员外当然不在,肉还不错,跟老管家要不来半文赏
钱。送完肉小五子从何府出来,耷拉着脑袋往赌场走。他
连输几天了,希望今天能好点,刚开始也不行,几把色子摇
下来,本钱差点输光,直到有个姑娘出现在赌场、手气才开
始好起来。

一般赌场不会有姑娘,这是个少爷带过来的丫鬟,一
口江南口音,谁知道是私奔还是拐卖,跑到田独来了。小
五子才不管这些,玩两把就知道,这少爷他吃定了,色子都
听不明白,更看不出他手上那点活儿。

小五子是杀猪的,赌场里碰到这种人傻钱多的大户
也叫杀猪。小五子对那些眼熟的赌客使个眼色,有个年经
点的明白了,几钱几钱地逆着少爷押,少爷押大,他就押
小,反正色子在小五子那里,每次都是他赢少爷输。也就
三五十把,少爷把钱输光了,身上从里到外翻了一遍,钱都
堆在小五子桌前呢。小五子挑出两贯钱扔给他,这是规
矩,输光了返点回去的盘缠,不至于出门转弯就跳河。

小五子把钱装好,起身准备告辞,少爷一把拉住他,
问:“我这丫鬟,你看能值多少两?”小五子打量她一遍,只
见她从头到尾都捧着一个盒子,也不知道干嘛\footnote{原文“干吗”}用的。小五
子摇头说,“我又没买过人,再说,人不都是搭钱的吗?”

“我五十两买的。”少爷伸手对他比画着,“养了两年,
各种开销怎么着也一百两,你看值多少银?”

〝那不是还是搭钱吗?”

“你还没娶媳妇吧,”少爷让丫鬟退后一步,站直了给
小五子好好看看:“这是两朝宰相的女儿,绝对是大家闺
秀。你也知道我今天赌运不行,既然把我钱都赢完了,那
就把我人也赢走吧。”

小五子再好好看看她,姑娘二十岁左右,管她是不是
宰相女儿,长得确实好看,不知道他的“瑶”跟她比起来怎
么样。他看着手里的银子,十两不到,就坐下来搁桌上,说
咱就来最后一把,大不了就当今天没赢钱。

少爷这次留了个心眼,他后押,等那个年轻赌徒先下,
既然他一直在赢,他就跟他押一样的。小五子拿起来三个
色子,故弄玄虚的样子,对者色子吹了口气,一并投进骰盅
里,三个色子在里面叮叮当当,最后落成了三个六。小五
子赢了,年轻赌徒和少爷都输了。

少爷倒也不难过,好像卸了个包袱,把丫鬟一推,说愿
赌服输,以后她就是你的了。小五子低头不应,将灌铅的
色子换回来,站起来拍拍屁股,对少爷讲,当你欠我十两,
丫鬟你留着吧,跟你赌这把,也就是赌口气,以后你要再想
跟我赌钱,就去钱记肉铺来找我,随时奉陪,要是你还想赌
人的话,就找个稍微漂亮点儿的。

小五子这是反话,不然眼睛盯着丫鬟,都迈不出赌场
的门。在路上他还在犹豫,回去把那姑娘领走吧,钱老板
要是不答应,他们就搬出来住。回到肉铺已是入夜,钱老
板已经睡了,留盏小灯和半盘冷菜在桌子上。吃到一半,
外面下起雨来,钱老板披着衣服走出来,坐到他对面看他
吃饭。小五子手里黄瓜掰一半递给他,钱老板摆摆手,小
五子把右手那半根塞到嘴里,嚼着黄瓜说:“你看我像多
大,二十多?三十多?我是不是该找个女人成家了?”

钱老板不说话,笑了笑回房继续睡觉了。小五子倒
没了胃口,左手那半根黄瓜还没吃,推门又去了赌场。哪
里还有少爷丫鬟的踪影,他张望了一圈,顶着大雨滚回钱
记肉铺。

雨下一整夜,睁开眼睛就能听见雨水夹着冰雹砸在
屋顶上。那就用不着杀猪了,钱记肉铺停业一天。他在房
间里把衣服洗掉,下楼将炉火引燃煮点白粥。钱老板还惦
记圈里的猪,撑着伞出去看看猪圈有没有淹水。刚出去十
来秒,砸着门让小五子出来一趟。

外面雨太大,落在地上噼里啪啦的,钱老板正在房檐
下,站在一个姑娘面前。她换了一身粉衣服,头上多了簪
子,但手上依然捧着那个盒子。小五子愣了几秒,让她先
进门再说。姑娘低声说了两句,大雨中也没听清,小五子
让她大点声。姑娘冲他望过去,满脸的雨水,使足力气说:
“让我进去,我就不出来了!”

小五子想两秒,不敢再拉她进来了,问她怎么了,难不
成在房檐下呆了一宿。姑娘点点头,说我家少爷讲,既然
把我输给你了,就让我过来跟你,没脸再带我回去了。“你
觉得我丑,我给他丢人了,”她指指身上早已浇透的粉衣
说,“他让我梳妆打扮一番再过来。”

这次是小五子不好意思了,低声说也不是嫌你丑,先
进门吧。屋里果然暖和些,他看见姑娘打从进来就开始浑
身发抖。她自我介绍叫文思清,跟者少爷出关后,一路往
北到了田独。小五子想多问两句,比如问问她多大,跟那
少爷到底是怎么回事,为什么说你是宰相的女儿。想想这
些也不大对,好像真是谈婚论嫁,就低着头找来两件衣服,
让她上楼换了,一会儿出来喝热粥。

文思清进去就没了动静,好半天都不见她出来。小
五子几次想进去催催,但担心万一正在换衣服呢。于是他
就下来看白粥在锅里咕嘟,可能是睡着了,白粥咕嘟咕嘟
都要干锅了,文思清都没有再走出来。
\newline

{\centering\subsection{2}}

钱老板不同意文思清住进来,这是他的肉铺,这个家
他说了算。小五子说没问题,你去把她赶走。为此钱老板
还认真写了封长信,揣了几天送不出去,见文思清还挺勤
快的,把家收拾得也干净,便默认她先住着,跟小五子商
量,明年开春必须让她回中原。

分房是个麻烦,一楼是钱记肉铺,二楼一共就两间房,
小五子的房让给了文思清,他就过来跟钱老板挤一挤。别
看他白天是个哑巴,晚上呼噜声倒是不小,将近十天总算
习惯了,有天夜里被说话声吵醒了。听不清说什么,声音
尖细尖细的,讲的内容又含糊。他在夜里睁开眼睛,明白
是钱老板在讲梦话。他仰躺在他旁边,一动不敢动。直到
钱老板翻个身,一把抱住他,右腿骑在他胯上,又打起呼
噜来。

第二天吃早饭他一直在留意钱老板,文思清出去喂
猪的时候,他盯着钱老板说:“你不是哑巴。”

钱老板皱着眉,表示没明白。小五子接着讲,我昨晚
听见你说梦话了,说什么不知道,但是你能说话。钱老板
把筷子放下,指着自己卧室的门,示意他再也不许踏进
一步。

怎么跟文思清解释呢?晚上他抱着被子来到文思清
房间,告诉她钱老板不让他去住了,咱俩以后就在这儿过
日子吧。文思清坐起来,把自己裹在被子里,问他为什么。

“因为我听见他说梦话了。”

“他不是哑巴吗?”

“对,所以他不让我再进去了。”

别说文思清,连他自己都不信,房间里就一张单人床,
小五子把被子扔上去,让她往里一点,袜子都没脱就上了
床。文思清从床上跳起来,表示虽然我是你丫鬟,但你要
是对我有非分之想,我宁可一头撞死。

小五子眨巴着眼睛,不明白她为什么那么激动,就算
不上床,打地铺都不合适,好像赖着不走一样。他让文思
清把门窗锁好,抱着被子出了门。一楼地面太冰,他试试
案板,像头死猪等着分割卖钱,而且是露天的,冷风从被子
缝一股一股地挤进来。有个地方一定暖和,但得有勇气,
但真的暖和。

两只猪对着他的脸呼热气,数九寒天弄得他一脑门
子汗。他卷点纸把鼻孔塞上,大口呼气,后来总算睡着
了。到后半夜他被一阵阵的雷声惊醒,小五子起来瞧瞧雨
有多大,双手伸出去,没见半个雨滴,转身一看,这几只猪
在打呼噜。

他彻底不敢睡了,担心一觉醒来,自己的脑子被震得
再次失忆。他坐起身在猪圈里发呆,东想西想,想到文思
清这么刚烈也算是好事,起码她和那富家少爷不会有什
么,自己要做点什么取得她信任。做点什么呢,女孩子喜
欢什么呢?直到再次睡者,他都没想明白这个问题。

文思清把他摇醒时,他还在猪圈里。文思清说,你还
是回房睡吧。小五子不干了,正是表现的好机会,可话一
出口又是贱嗖嗖的,他说不行,我回房睡觉,有人又要呼天
抢地、撞墙寻死了。文思清保证不撞墙,小五子才抱着被
子爬出猪圈。

也许好事将近,洞房花烛夜,金榜不题名。可文思清
没有跟出来,留在猪圈里。她自言自语我保证不撞墙,就
在这里睡啦。她铺平身下的干草,拍一拍躺了下来。

那可不行,怎么能把文思清丢在这里。小五子又钻
子进来,俩人在猪圈里一顿瞎聊,文思清说小时候的日子
可好了,花园里种的都是云南的小月季,广西的桂花,洛阳
的牡丹,长大了倒是四处颠簸,睡惯了牛棚马厩。但总还
是活着,上面的哥哥、爹爹、伯伯都死了。小五子问什么事
啊,痘疫横行吗?我爹爹是文公啊,她说,惹怒了朝廷,所
有党羽男人斩首,女人流放为奴,京城一夜之间死了一万
多人啊。说完她看着他,表情似乎奇怪这么大的事,他怎
么会不知道。

小五子假装想起来一般点点头,心里盘算一万多人,
比两个田独的人还多,每天得吃几只猪,居然可以一夜之
间全杀掉。聊着聊者,那几只猪都扛不住了,窝成一团打
起呼噜。文思清也越来越困,意识恍惚,不知不觉睡着了。

小五子把她抱回房中,放在床上,拿着铁锯绳子出门
上了山。傍晚时分他拖着一棵水曲柳回到肉铺。照着图
册边学边做,花了一星期打了一张床,摆在房间的另一
侧。铺被褥的时候,小五子就想,以后睡醒睁眼,再也不用
想昨天发生什么了,只要看看对面睡着的姑娘认不认识,
就知道新的一天还要不要继续了。
\newline

秋天就这么过去了,十一月开始大雪封山,人畜都进
不来。到了腊月,钱记肉铺将关门封店。田独也进入极
夜,有时候一整天都见不到大阳,除了睡觉便是望着漫长
黑夜里的飘雪。来田独两个月,文思清也摸清了田独的地
形,三面环山,一条小河狭长穿过。偶尔会有好天气,拨云
见日,难得的晴朗,他们会抓紧这一两个时辰,上山拢火烤
土豆烤肉。钱老板才不会来,只有小五子和文思清靠近火
堆坐下来。以前都是他一个人上来,看山望雪,想想过去
的事情。也没有多过去,他对文思清讲了自己的病症,就
两年的记性,再往前什么都想不起来。小五子说,头一年
除夕夜,就他和钱老板,连年夜饭都没准备,两人就着花生
喝了两壶酒,可能是喝多了,再加上头一年他难过,他车轱
辘话反复问,自己从哪来,父母在哪,老婆孩子有没有,后
来钱老板被问烦了,摔了酒壶要回房睡觉,他就追进去问,
没准钱老板也喝多了,指着夜壶,阿巴阿巴地示意他,这是
你爹用过的,拿去拜吧。

“我知道不是,”小五子说,“我当然知道不是,可是大
年初一的早晨,我还是端着夜壶上山,跪下来冲它磕三个
头,挖个雪坑给埋了。”

文思清拉过他手臂,看着上面的字,说这不还有一个
瑶姑娘吗,等开春了,我陪你去找她吧。小五子摇头,忘了
就忘了,想不起来也就没感情,再说有“瑶”字的姑娘多了,
天南海北没处找。文思清食指在他手臂上的每个字上摸
一遍,最后落在“百花”两个字上说:“去百花盛开的地方
吧”

小五子撕一块肉塞嘴里,他不想只聊自己,岔开话题
问文思清,为什么老带着这盒子,上山都得抱着爬上来。
文思清有意抱紧一下盒子,说是她娘嘱咐的,只要盒子在,
娘就在她身边。文思清不让别人碰这个,小五子俯下身瞧
了半天,问她里面装的到底是什么。

“我娘,”小五子以为她又要说她娘嘱咐什么了,等了
半天文思清说,“我娘的背灰。”

小五子盯了她有一会儿,抽了一下鼻子说,你我都一
样,都没过去了,我找不到过去,你回不到过去,就往前看
吧。其实田独也挺好,钱老板也不错,人苛刻了一点,还算
待我不薄。他肯定不是哑巴,谁知道碰到什么事,躲这儿
来了,岁数也大了,管他过去好人坏人,既然要在田独终
老,真到走不动那天,咱们就给他养老送终吧。
\newline

{\centering\subsection{3}}

眼看着就要过年了,因为有了女人,年味也开始多了
起来。腊月二十三这天,文思清拉了好长的一个货单,吃
穿饮用一直到正月十五,文思清要小五子和钱老板去集市
采购。晴朗的日子来回也要一个时辰,何况漫天大雪,马
车根本拉不动。他们上午天不亮就出发,下午回来天都黑
了,一黑就要下雪,马车拽不动时,两个人下了马车牵马。
离家不到两里路,钱老板留意到雪地里有一行脚印,积雪
一尺多厚,每一步到底都要到膝盖处,而这行脚印只有一
指多深,好像是身负轻功之人脚尖点着雪赶路。钱老板示
意他先回去,他周围查看是否有埋伏。

果然有人来造访,一个身后佩剑的女子背对着门口
坐在桌前等待。文思清刚煮好一锅肉汤,跟小五子说,那
女的走迷路了,说买碗肉汤喝,雪小点儿就走。小五子接
过汤碗,说他去会会这个女人。

“会什么会,可狐媚了呢,一看就是小狐狸。”

说完她还不放心,手拿着抹布跟了进来。小五子把
肉汤放到桌上,从身后绕过去,坐到她对面,想瞧瞧这姑娘
到底怎么个狐媚法,冰天雪地跑到田独,非奸即诈。话还
没有问,这姑娘先是惊到了,半张者嘴盯着小五子,最后把
小五子都看毛了,问她是不是认识自己。文思清留了个心
眼,连忙过去,挡在他俩之间用抹布擦桌子。小五子晃着
头,借空再看两眼这姑娘,倒真是狐媚,尤其是含情凝望的
时侯,感觉把心都掏出来求你揉揉。小五子继续问她:“我
们过去认识?”

那姑娘皱着眉望着他摇头。小五子明白那不是否
认,而是失望。她又看了看文思清,此时她恨不得整个人
趴到桌子上,把他们俩完全隔开。姑娘穿着蚕丝绸缎,一
看就是南方来的,虽然名贵,然而无法御寒。文思清轻蔑
一笑,说天寒地冻的,让小五子去房间拿件厚棉衣送给姑
娘。小五子问清楚是哪个房间。

“就是咱俩那房间啊,压被子的那件。”

姑娘反应了几秒,明白他俩已经成了亲,忙说不必
了。她扭头看看外面的大雪,说时候不早,她要赶路了。
出门前她又望一眼小五子,忍不往地叮嘱道:“少谷主,以
后没人照顾你,一个人去赌场,别再摇三个六了。”

小五子彻底傻了,记忆可以丢失,但是习惯毛病却永
远不会丢。他跑过去堵住门口,撸起袖子让她看手臂上的
字,问她:“你认识我,这里面你知道哪一个?”

文思清过来拉他,让他去修修猪圈,随后瞪着这姑娘
说:“有只母猪发情,把猪圈给拱坏了。”

小五子让她闭嘴,等这姑娘说话。她看着上面的字
忍不住地掉了两滴眼泪,正要讲话时,钱老板顶着风雪推
门进来了。看到是她,他冲这姑娘挥挥袖,示澺她跟他
上楼。

小五子和文思清看着他俩踩着楼梯上去,将房门插
上。里面在说话,他没有听错,先是这姑娘在说话,随后是
一个尖细的声音,跟他那天听到钱老板的梦话一样。两个
人的声音越来越大,最终是那姑娘高声说一句:“谷主有
令,找到少谷主,不惜性命也要把少谷主带回去!”

随后他们打了起来,房门里面叮叮当当,小五子盯着
房门发呆,自顾自地说着,我是少谷主。他抓者扶手上了
楼,站在紧闭的房门前。里面传来击剑的声音,小五子听
见一剑下去,将花瓶劈碎。他深吸一口气,一脚踹开房
门。姑娘回头朝门口看一眼,钱老板赤手夺下她的剑,扺
住她的喉咙。大概持续十几秒,似乎在等她服输,钱老板
将剑收回,剑柄朝外递还给她,做了个出去的手势,不知道
是针对谁,反正姑娘先掩面下楼了。小五子愣在原地,看
着钱老板,他跟没事似的把家具扶正,将地上的花瓶碎片
扫干净。一阵冷风吹进来,姑娘推门离开了。

他一路追出去,雪下得更大了。果然是她的脚印,每
一步只借一层雪的力,便踏出下一步。小五子一脚深一脚
浅地拼命追,实在没力气就坐在雪地喊她。他说我只向几
句话,说完我就回去。姑娘停下来转身,慢慢朝他走回来。

“你一直在找我,对不对,我是谁?”

姑娘犹豫片刻说:“我既然答应了常公公,就绝不会再
讲半个字了。”

“谁?”姑娘没说话,小五子自己想明白了,那么尖细的
声音,装哑巴就是不想让人听出他是宫里的大监。“总要告
诉我一点儿,我叫什么名字?”

姑娘冲他摇头:“名字不能说,说了你就知道你是谁
了。”

〝你叫我少谷主,是什么谷?”

她含泪望着他,咬了咬嘴唇,作揖告辞,说自己只能讲
这么多了,有缘再见,此生多保重。看着她背影,大片大片
的雪花扑进他嘴里,冲她背影喊:“百花谷!我是百花谷少
谷主!那你是我什么人?”

她在远处停下来,但不打算再过来。远远望去她肩
膀一颤一颤的,应该是迎着风雪在哭。小五子也不敢走过
去,生怕一上前她就消失了,大概过了几分钟,全身都挂满
了雪,她才又哭又笑地说:“谢谢你,谢谢你在努力记得我,
请你忘了我吧,请你忘了苏子瑶吧。”
\newline

{\centering\subsection{4}}

他试过各种办法,钱老板只字不提,可能现在得叫常
公公了。他打不过他,单是看他跟苏子瑶空手夺白刃的功
夫,就知道自己永远也没办法逼他讲什么。他想再等等,
总会有个下手的机会。

只要他还装哑巴,就算不上常公公,钱老板最近焦虑
起来,苏子瑶能找得到,别人也有可能找得到,每隔两个时
辰就出去查看一圈周围的脚印。他还担心苏子瑶会带救
兵杀回来。可是人都在中原,田独那么远,往返一趟就是
小半年,至少可以安心地把年过完。除夕夜里钱老板终于
喝多了,文思清做了一桌子的好菜,仿佛要对得起她的手
艺,哪怕小五子下了两大袋的蒙汗药,第一杯下肚就开始
额头冒汗,钱老板也坚持着菜吃得差不多,才轰然倒下。

文思清吓坏了,她看着小五子把钱老板扛进卧房,又
去厨房拿了绳子菜刀,以为他今晚就要杀了钱老板。她拼
了命地抱住他,求他不要干傻事。直到他把她推开,将她
反锁在门外,隔着门跟她保证,不会杀了他。

他把他绑在床头,结结实实捆了三圈。怕他不服软,
他找出捣火铁棍,放在炉火里烤红,右手提着棍子,左手脱
下钱老板的袜子。脚底已经有了烫疤,两指宽的一条,像
是匕首烙上去的。他看看右手烧红的铁棍,又看看那道
疤,想不起来,但他确定,这是自己干的。

这时钱老板醒了,似笑非笑地看着他。这多少激怒
了小五子,举起铁棍在他眼前晃着,警告他不要不信,这次
我要烫瞎你。钱老板眨眨眼睛,他当然相信,你小五子就
是这样的人。

反正还是那些问题,你不是肉店老板,我也不是杀猪
伙计,你是谁,我是谁,把我藏到这来,到底对我干什么?
钱老板没有回答的意思,他把烙铁靠得近一些,感觉脸上
的汗毛都在滋滋作响。

“我不杀你,不管你是谁,这两年你确实没动我,反而
对我还不错。我只求你告诉我,我是谁? 这是我的事情,
你应该让我知道,我答应你我不跑,不管我是谁,我都在田
独被你看着,行吗?”

他声音都颤了,钱老板依然不为所动。外面有人放起
了鞭炮,大年初一的早上,田独虽然人少,地处偏僻,可每年
鞭炮总是放得特别多,仿佛声声响响都想让中原听到。噼
里啪啦地响了十几分钟,钱老板看看窗外的天光,转回头看
看眼下的烙铁,终于要说话了,那么尖的声音,不男不女
的。他说:“你是嘉和三年五月初七生的,今年二十五岁,无
父无母,没有兄弟,百花谷的事就别问了,我早晚让你知
道。新年了,咱们也过个年,去外面把鞭炮放了吧。”
\newline

鞭炮声放到一半哑了下来,炮捻子落到地上被雪扑
灭了。他想过走,往南方去,一路独行看看哪里才是百花
谷,可是他一没银子,二没武功,怕是没进山海关就被饿死
打死在路上。最想不明白的就是这一点,他都已经是少谷
主了,为何他除了杀猪摇色子,一点功夫什么的都不会?
他冒出个奇怪的想法,就从摇色子做起。

色子自然要在赌场摇,小五子答应过文思清戒赌的,
好好开肉铺赚钱,给钱老板养老。现在看起来全是扯淡,
早晚有一天他会杀了常公公。大年初一赌场人不多,两三
桌的赌客几钱几钱地押,更像是找个地方取暖。他每个桌
子巡视一遍,有对师兄弟似乎有些功夫。小五子坐过去,
隔着桌子跟他俩说他来做庄,一起玩两把。趁他俩不注
意,换好灌铅的色子,扣过骰盅摇了三圈,开底之后他
赢了。

年纪小些的端不住了,嚷着在老子面前出千,我剁你
一只手,拔剑就要弄小五子。年长那个按住他肩膀,冲小
五子笑笑,说这把多押点,还是你做庄。小五子点头继续,
示意他们看看这三个色子,将骰盅罩在上面摇起来。想要
什么点数,都可以照口诀摇的,这次别是三个六,险胜就
行,一套动作下来应该是俩四一个五,开了骰盅,居然是三
个一。

小五子数钱给他们,要他们再押。这次他慢点摇,就
奔三个六去,竖起耳朵听色子。他知道问题在哪了,三个
六刚刚要落好 里面的色子自己动起来,交成三个一,他再
去摇三个六,色子还是要多动两下,交成三个一。他迟迟
不开,盯着对面的四只手,注意到年长那个人的右手搭在
桌面,时不时地用内力震动桌面,将色子变点。年纪小那
个催他快点,小五子笑道再摇最后一圈就开。小五子他重
新奔三个六摇,对方刚要发力时,他忽然跳起来,抓住了那
个人的右手,叫道:“兄台,这么玩是要剁手的!”

那个人手就这么被小五子按着,等了半天也不见小
五子出招,反手扣住了他的脉门。小五子只是想找人打一
架,这就是他奇怪的想法,一直使不出功夫,也许是没遇
见强敌,这两个人刚刚好,生死攸关之际没准就把功夫逼
出来了。

可功夫迟迟使不出,反而一次次被这两个人羞辱。
年长的师兄交手两回合,就知道小五子只是个赌场混混。
这时那个师弟来劲了,好像难得找个人肉靶子练,一次次
地把小五子放倒,再踢出赌场大门。小五子又不服,每次
站起来还往赌场里冲。最后一次这个师弟连给他三掌,又
坐在小五子脸上,结结实实放了一个屁。

赌场里的人都笑了,师弟举着手跟大家一起笑,捡起
小五子输掉的银子跟赌客们分。小五子躺在地上,好半天
才支撑着站起来。肯定是有肋骨断了,从衣服里摸进去,
左边第三根骨叉已经支出外面。他让他们俩等着,他回去
拿钱,一会儿好好赌一把。

他扶着肋骨,一路踉跄回到肉铺,翻箱倒柜把这两年
的积蓄掏出来,抄起杀猪刀又折回到赌场。他把杀猪刀揣
在怀里,一进门就把这一百多两银子洒在桌上,喊着全押
了,一人一骰盅,直接比大小。年轻点的笑道没这么多钱,
你懓慢押,慢慢输。

“没钱没关系,赌命。”

师兄弟相视一笑,年轻的说他先来。小五子规定这
次来硬的,谁也别玩猫腻。年轻的点头同意,年长的主动
把手拿开,抱着腰看他俩怎么玩。这时有个要饭的拄着拐
过来讨钱。小五子本来说没有,转身过去的时候发现他是
个瞎子,一时有点难过,喊住要饭的,跟他说:“你别走,这
把赢了全给你,输了你帮我收尸。”

他和那个师弟,一人一个骰盅,摇好之后,他让对方先
开,两个六一个五,一共十七点。小五子将自己的骰盅露
出一条缝,三个六。

他没有作弊,这回是真赢了。他在琢磨对方会不会
乖乖让他宰。他盖住骰盅,手伸进怀里捂住胸口,说:“你
赢了,桌上的钱都是你的了。”

师弟笑话他,看把你心疼的,小心肝都扑通扑通的。
他站起来躬着身子,双手伸过来划拉小五子面前的银子。
小五子看准了抄出杀猪刀,手起刀落,将他左手剁下来。

师弟的手腕开始喷血,小五子第二刀去砍他右手,被
他师兄挡开。暂时没法理会小五子,他连忙撕下布条,系
在他左手腕止血。抓不到另一只手,小五子将留在桌上的
左手连剁三刀。看着自己的手被剁成肉泥,师弟也顾不上
止血,右手拔出剑就朝小五子捅过来。头一刀被他用菜刀
格开,第二刀砍向他的肩膀。这次小五子挡都不挡,转身
就往外跑。年长师兄一个跃身堵住门口,后面师弟的剑也
直抵他背身,小五子侧身躲开,又往赌场里面跑。

赌场早就乱了,小五子拿每个赌客做挡箭牌,刚躲到
人身后,赌客马上就跪下了,有的赌客直接倒在地上装
死。一时间除了他们仨,只剩下那个要饭的瞎子看不到情
况,顶着桌角摸桌上的银子。小五子拉着他肩膀躲后面,
突然感觉一阵阵的内力传过来。他看眼乞丐,没事似的还
在用手指数着散碎银两。两个发疯了的师兄弟从后面绕
过来,一左一右朝他劈过来。小五子无处逃遁\footnote{原文“逃循”},乞丐低声
说,先砍左边的大杼穴,再去点右侧的大肠腧。小五子完
全不知道他说的是什么位置。乞丐换个方式说:“左边猪
颈肉,右边大里脊。”

这是小五子强项,就是活猪冲他跑过来,他都一砍一
个准,加上刚获取的内力,左右两刀便将师兄弟砍翻在地,
靠在墙角哼哼唧唧。乞丐把银子收好,作揖道多谢公子施
舍,点着拐杖走出了赌场。年长些的看出来这乞丐非同寻
常,可等他日小五子落单再算账,要小五子报出山头姓名,
待伤养好后前来赴会。要是早半年,不用对方间,小五子
自己都往外说,不服来钱记肉铺找我。他和死太监两条贱
命也就算了,现在多了个文思清。但不报山头真的太怂
了,他朝门口看看,那个乞丐已在雪地里走远,变成了一个
黑点。他转回来对师兄弟二人摇着食指说:“你们两个,不
配打听我名字。”
\newline

{\centering\subsection{5}}

他出去找了好几圈,这回雪地里连个黑点都见不着
了。田独没乞丐,自己活得都费劲,哪还有闲钱施舍别
人。中原要饭的也不至于要到田独来,那就是为什么人而
来。一个瞎子,跑这么远来,有一阵他感觉找的是自己,既
然他都是百花谷少谷主了,没准百花谷还是江湖中数一数
二的门派。冷静一下他明白不是,要是找少谷主就在赌场
门口等他了,用得着他在大雪里兜三圈?那么是常公公
吗,田独镇还会有什么大人物呢?

常公公不在店里,文思清也说,没见着什么要饭的。
他点点头,肋部疼得厉害,他要上楼躺一会儿。楼梯爬到
一半,脚一打滑,他从一楼半滚了下来。

钱老板晚一点回来,扒下他的衣服,先看看伤势,确定
只是硬伤,不是什么高手所为,他捏着折成两半的肋骨,忽然
发力,将骨接合到一起。小五子也没叫,咬着牙忍了半天,想
着杀猪剔骨那一套手法也是他教的。缓过来一点,他喘着
粗气问道:“你以前在宫里,到底是杀猪的,还是看病的?”

钱老板没搭理他,用纱布将肋骨缠好,起身拍拍小
五子说:“好好躺着吧,两个星期你都别想下床。”

文思清怕他闷,给他弄了一箱书。小五子哪看得进
去,有时候盯着书名都能睡着。后来他就卧床上嗑瓜子,
可他从来不吃这东西,香香腻腻的嚼两口就反胃,他只是
把瓜子仁剥出来攒者,晚上留给文思清一大口吃完。

要躺十五天,他怕等能出门的时候,乞丐早离开田独
了。他想过,和钱老板聊聊这件事,田独镇还有哪个深藏
不露的狠角色。可是从哪里聊起呢,他连自己是谁都不知
道,聊那些人都不知道是敌是友。

他还在犹豫,钱老板先出事了,有天快天亮才回来,脚
步沉重,直奔小五子房间,告诉他猪圈底下有一个地窖,一
会扶他进去,他身上有一个方子,让文思清这几天把药凑
齐,等他七天后上来服用,要是七天后他没能自己爬上来,
就地给他竖一块墓碑就好了。小五子问他,墓碑也得有名
有姓,你叫常什么呀?钱老板说了几次“沈”字,随后一大
口血吐出来,瘫倒在床边。

钱老板姓钱,常公公姓常,这回立个墓碑又他妈姓
沈!文思清把猪轰出去,拨开上面一层干草,果然露出一
个地窖的门。底下漆黑冰冷,将油灯点亮,说是一座地宫
也不为过。老家伙天天不在店里,原来是跑猪圈挖这个来
了。底下有半亩地大小,正中间摆着一个红色的冰床。小
五子把常公公放上去,手在冰床上化点冰水,手指被染红,
凑到鼻子闻一下,不是血水,像是某种草药的味道,估计是
某种植物熬出来的红汤,一点点铸成的冰床。

那就等七天,文思清跑了附近三个镇子的药局,唯有
一味红参全都断货。药局老板说这东西山上有的是,只是
季节不好,大雪封山,谁都上不去,等开春雪化了,你要多
少,我卖你多少。

等不到开春,既然谁都上不去,文思清就自己上。小
五子拦着不让她出门,老家伙不杀他就不错了,怎么可能
冒着命救他?再说七天之后,他既然能从地窖里爬出来,
少吃点药还能死回到地窖里去?说不动文思清,小五子答
应,再等两天他肋骨接好了,陪她一起上山。

他以为文思清答应了,一觉睡醒发现她悄悄出门了。
只到山腰的话,来去两个时辰,他想如果晚上不回来,就上
山找找。可半个时辰不到,他脑子里就闪现过十几个雪崩
坠崖的面面。他拽出纱布给自己肋部绷紧,打个死结,抄
起菜刀上了山。

他一直找到半夜,手脚都使不上力气,脑子里那十几
种死法差点都发生在自己身上。爬到山顶看到几株红参,
他揪两把揣在身上。那就换一条路下山,没准回去的时候
文思清正做好早餐等他呢。下来的山路上,他看到石缝间
两只老虎仔嗷嗷待哺,这让他有种不祥的预感。沿着脚印
寻找,一只成年母虎堵在山洞口,远远望过去,困在山洞里
的正是文思清。他把菜刀掏出来,轻轻走过去,距离老虎
三十米远的时筷,他大叫一声。老虎嘶吼一声,转过身子
朝小五子冲过来。他让文思清快跑,离开山洞,自己扎稳
马步盯着老虎的颈下。好像在等钱老板的手势,活动开
了,解决吧,猛虎跃起的一瞬间.他一刀斩进去,顺势将老
虎挑到头顶。开膛破腹,老虎内脏瀑布一般泻出来,洒到
他的头顶脸上。老虎还在雪地上奄奄一息。小五子忍不住
吐了出来。

他捧一把雪擦脸,头发里净是老虎的血腥味。文思
清扭伤了脚,小五子背起她,半步半步地下山。小五子怕
她睡着了着凉,一路编着笑话讲给她听。他自嘲道,自己
这个也不知道什么本事,是个两条腿的我都打不过,可四
条腿的,哪怕狮子老虎我都不怕。文思清好半天没说话,
他担心她睡着了,手贴着后背捏了捏她的大腿。文思清脸
蹭蹭他的肩膀,手停在他脖子上勾得再紧一些,低声说:
〝我听着呢,一句都没忘。你答应我,小五子,不管你过去
怎么样,不管你以后跟谁好,你这辈子一定要娶我一次,好
不好?”
\newline

{\centering\subsection{6}}

冬去春来,三月过后转四月,五月初七他给自己过了
二十六岁的生日。钱老板一脸嘲讽的表情,不过生日当天
还是煮了两个鸡蛋当做礼物。文思清去山里兜兜转转,找
到了那只死老虎,扒下虎皮给小五子做了把椅子。每天小
五子坐在虎皮椅上切肉收钱,好不神气。”

六月开始田独进入极昼,街上的人也开始多了起来。
每年夏天小五子都几乎不用睡觉,接近三更天才黑下来,
躺上一时辰天就大亮,人们又出来活动了。今年多了一帮
从中原来的侠客,骑着马在田独兜了两圈,挨家挨户地找
昆仑公子,找太子,见没有线索,继续往北找去了。他们不
是一起的,各找各的,估计从京城组团一路往北捋,捋到第
三年终于干到田独了。好像有两伙人意见不统一,在赌场
还打了起来。人一个没伤,倒是把赌场砸个稀巴烂。

小五子后来听何员外说,这两伙人一伙是五公主的,
一伙是三王爷的,他们都找太子,但目的不同,五公主的人
要救大子,三王爷的人则要杀太子。

“那皇上呢,他想杀,还是想救啊?”

何员外瞪大眼睛,觉得这孩子不可理喻,只知有汉不
论魏晋,皇上在昏迷啊,都快三年了。小五子摇头,这可怪
不着他,他的记性也就是两年多,过去的事谁知道。何员
外来了兴趣,真的一点都不知道?没受过什么伤?他眯着
眼睛看小五子分割切肉,似乎想找找有没有哪门刀法的底
子。直到身后的钱老板故意咳嗽一声,何员外才识趣
离开。

那两伙人走后,田独镇又恢复了平静。赌场被砸后,
小五子也没地方去了,没事就自己拿色子在猪肉案板上
摇,左手和右手赌。有天他大老远就看见一个老熟人,往
肉铺这边跑。正是被他砍掉一只手那个师弟,估计是来寻
仇。小五子把文思清支走,抄起杀猪刀,只等动手。

那人慌慌张张,左顾右盼地往前跑。经过钱记肉铺,
见到是小五子,一时间满脸的杀气,随即扑通一声跪在地
上,说后面有人追杀,请求大侠庇护。

“我成大侠了,”小五子一乐,“干嘛\footnote{原文“干吗”}追杀你啊,只剩一
只手了,还能出千呢\footnote{原文“出千哪”}?”

看着自己仅存的一只手,他恨不得上来跟小五子拼
命,可是后面还有劲敌。他继续跪着,边回头边说:“不是,
是两个姑娘。”

那得看看热闹,小五子让他把剑交出来,躲在肉案底
下。他一刀一刀使劲在案板上剁猪爪,远处过来两个白衣
姑娘,身上没剑没刀,只是手上托着一盆仙人球,见到小五
子也不打听一下,完全无视,任凭他叮叮当当地剁肉,从门
前匆匆走过。小五子不高兴了,人家都追过去了,小五子
喊住她们:“两位姑娘不买点肉啊?”

其中一个看样子是师姐,让他别捣乱,她们有急事
儿。小五子在后面喊,不管追什么人,聊两句再追也不
迟。年轻一点的奇怪了,低声问师姐,杀猪卖肉的也敢过
来和我聊两句,我长得真有那么难看吗?师姐安慰她,一
个杀猪的懂什么?其实年轻的这个反而漂亮,师姐倒是个
丑八怪,也不知道谁安慰谁。师妹折回来,走到小五子面
前,问他:“你怎么知道我们在找人?”

小五子盯了她半天,说看面相看出来的,一只手的那
个,对不对?师姐看出小五子知情,从后面跟上来,抢话说
那是她师弟,偷了师父的钱,她们奉命来剁他一只手。还
剁手?小五子笑起来,又不是我家养的猪,四个猪爪,人哪
有那么多手给你剁啊?

师姐让他闭嘴,随后和师妹对一下眼神,问:“你们家
是不是有后院?”

“有啊,是猪圈,刚才有个一只手跑过来找工作。我看
他都那样了,也就干干喂猪的活儿吧,赌是不行了。”

这跟赌有什么关系?师妹不解,跟着师姐冲进后院。
钱老板正在猪圈躬身喂猪,师妹冲过去,手先拍一下仙人
球,连刺带血地对着他后背就是一掌,嘴里还喊着:“贼子,
看你还往哪儿躲!”

钱老板此时大病初愈,吃痛回身,师姐又是一掌过去,
钱老板拿起盆格挡,盆被一掌击碎,猪食溅了他一脸,更看
不清是谁了。师姐师妹两人一起动手,一掌掌拍在钱老板
身上。直到打累了,她们发现好像年纪差得有点远,相视
一愣,师妹抱怨道:“这么大年纪了,还喂什么猪啊?”

两个姑娘耷拉着脑袋回到店前,师妹低声对小五子
说:“后院有个老头擅自喂猪,我们帮你教训了他一顿。”

“你们把我老板打了?”

“为什么老板喂猪,你卖肉啊?”师妹又不解了,“把这
仙人球送你,就当我赔不是了。”

她递过仙人球,小五子并未伸手去接,摇头道:“仙人
球你留着吧,长得这么好看,把你名字告诉我就行。”

师妹皱了皱眉,手掌拍了一下仙人球,接着给了他一
掌。小五子没顾上喊疼,右手抬起就给了她一耳光。师妹
瞪着他,扔掉手中的仙人球,伸手捏住小五子喉咙。

文思清刚好从外面回来,拼了命地要去救小五子,脖
颈被师姐一把抓住,挣脱不动。于是嘴上开始撒泼,从未
见她如此失态,一时间把骂女人最脏的话全抖落出来了。
钱老板用衣袖抹着脸走到后面,袖子里握着暗器。他盯着
师妹的手指,倘若她一发力,宁可暴露身份,也要先将她
击毙。

师姐提着文思清劝阻:“师妹,这人不会武功,就算了
吧,师父不是说了吗,一次出门最多只能杀五人,你已经杀
了七个了。”

“一人杀五个,咱俩加起来十个,你再让我一个?我先
取了他性命。”

“不行,我这名额还要留着杀我那小白脸和他那三个
小贱人呢。”

师妹翻眼皮算了算,提醒她,全杀完就十一个了。师
姐说那就杀三个,刺瞎一个。

“上次那一只手还不知道是不是你剁的,这次又说要
刺瞎双眼,就知道你舍不得杀他。”

两三句话师妹消了火,放开小五子,跟师姐离开。文
思清跑过来解开他上衣,查看他的伤,帮他拔掉仙人球
刺。除了点皮外伤,也没什么大事。一只手从案板下面战
战兢兢爬出来,也不知道该谢该怨,让她们过去,不就完了
吗,非要叫回来撩几句,总之多谢你救我一条命,但断只手
的仇,还是要报的。小五子说,好啊,那我们就好好算算,
一只手值多少命,算你半条命吧,我救你一条命,扣掉半
东,还剩半条,以后别让我见着你,不然我要你半条命!

话讲得恶狠狠的,总觉得哪里不对,又不知道怎么反
驳,逃了一天的路,饥肠辘辘,银子都跑丢了,索性你再给
我点吃的,算我欠你半条命加一顺饭。小五子挥手让他快
走吧,欠太多怕你还不起。一只手还挺懂规矩,说了句后
会有期,就要告辞。小五子叫住他,还有向话要问他:“你
那小师妹叫什公名字?”

一只手慢慢走回来,手撑在案板上,一字一句地告诉
他:“吴思若。”

小五子正回味这三个字,一只手抓起案板上的猪爪
就跑。小五子拎刀追出去,连追两条街,撞倒街边的何员
外。何员外要小五子扶他起来,说我这是不差钱,你这么
跑,要是真撞到个穷光蛋,像我这个岁数,你们肉店搭进去
都不够赔的。

一只手一边回头笑,一边朝前跑。他太饿了,生猪爪
也忍不住舔了两下,举着猪爪对小五子挥舞,正得意间,撞
到了前面两位师姐的后背。

小五子看得目瞪口呆,他看到小师妹抢过一只手的
猪爪揣进自己的行囊,师姐拿起绳子要反绑他,却不知道
只剩一只手应该从哪里绑起,最后气得师姐连打了他几个
巴掌,又心疼地抱着他哭起来。
\newline

{\centering\subsection{7}}

小五子好几天都吃不下去东西,文思清都气死了
她抱怨小五子就想着那个小狐狸,茶饭不思。倒是偷偷有
想一点,但真的是没胃口。后来他发现钱老板其实也没吃
东西。五日不食,小五子瘦了一圈。文思清做了一桌好
菜,两个人就不动筷子,弄得她酸溜溜地吟诗作赋,为伊消
得人憔悴。钱老板拿起筷子,将每个菜都夹过一遍后,难
得地说了一段话,那两个姑娘是大漠仙人的弟子,还好火
候设练到家,等几天毒性过去,就可以恢复正常了,如果碰
到高手,比如大漠仙人出手,他能让你持续不吃不喝,直至
干枯而竭,死的时候形同枯槁。

好大的本事,小五子问他,这把仙人掌扎手上,再拍过
去的掌法是什么掌,钱老板叹了口气说,仙人掌。小五子
以为他听错了,重问一遍。钱老板还是回答仙人掌,仙人
是仙人掌的仙人,掌是掌法的掌,仙人掌是江湖三大毒掌
之一,就是借助仙人掌不吃不喝也能茂盛生长的道理,练
掌的过程不断地提取仙人刺里的毒素,再施予对手体内,
这两个姑娘还是初学者,出掌之前还要从仙人球借刺,练
到大漠仙人的程度,已是满手倒刺。

“另两个毒掌是什么?”

也不知道是没胃口,还是不想回答,钱老板只是摇了
摇头,又把每盘菜夹了一遍。小五子握住他的筷子尖,盯
着钱老板问:“其中一个,是不是叫断魂掌?”
\newline

{\centering\subsection{8}}

后来小五子想明白了,人活的就是一股气,一种魂魄,
终其一生,一贯至底,有人把它人为地断掉了,这就是断魂
吧。他问钱老板,如果大漠仙人使的是仙人掌,那么断魂
掌是谁的绝学?钱老板不说,岔开话题讲,你肉切得不对,
精肉要带点肥,肥肉一点不能有瘦肉,肉皮别剃,带皮跟着
卖,照你这么弄,一只猪少卖七八十贯。

小五子打量着钱老板,好像三年里初次见面似的,从
头到脚瞅了一遍,摇头冷笑道:“你没这本事,我要找的不
是你。”

找到也没用,这是小五子更不明白的地方,啥本事没
有,凭什么大人物能给他一掌断魂掌?不是百花谷少谷主
吗,靠什么当上的,难道谷主是他爹?苏子瑶怎么说的,谷
主有令,找到少谷主,不惜性命也要把少谷主带回去。不
用带,他早晚自己过去。钱老板那天要死要活地下地窖,
到底是谁伤的?还有,那个乞丐来田独究竟要找谁?断魂
没有用,事情总会露出狐狸尾巴,要是一死了之就算了,所
谓的重新开始,根本不可能。

倒是有人在夏天死了,何员外的老管家,门缝里看人
的那个,估计活到岁数了,在躺椅上摇着摇着就掉下来摔
死了。何员外要大办,远近都发了请帖,又不是死了爹,老
管家而已。当然没请小五子,他接到通知,丧事那天挑一
头上好的肥猪送到何府。

扛一头太麻烦,小五子一大早就赶着活猪过去,何府
早起来了,上上下下都忙着筹备葬礼。新来的管家更跋
扈,让他到后院杀猪刮毛,可别脏了厨房。今天先忍了,找
个日子总要弄一下这个新管家。小五子在后院把猪杀好,
开膛破腹,掏干净内脏,喊人拎壶开水出来,他要刮毛了。
喊了三声,没人答应,他只好自己把猪扛进厨房。

厨房瞬间没人了,小五子巡视一圈,每个活儿都是干
丁一半就撂下,是不是何府有点名做早操的习惯。他不管
这些,洗洗手准备找新管家算账。这时大厅里有人说话
了,狠巴巴地问:“何振生,你师父向问和躲到哪里去了!
搞这么大的丧事,棺材也是空的吧?”

话末说完,他跃过去一掌劈碎了棺材,里面躺着的是
一个纸人,一时间纸屑乱飞。怪不得跟亲爹似的操办,原
来是假的,办给仇人看,小五子想等会再出去,别钱没要
来,惹得一身骚。外面有人回答了:“我师父早知道,师兄
弟四人之中出了一个版徒,让我留下来,誓死也要看到是
哪位师伯。”

听声音好熟悉,小五子从窗缝看过去,正是大腹便便
的何员外。对面领头的把脸蒙上了,带了几个人他看不清
楚,不过何府的几十号人都站在何员外身后。那领头蒙面
的说:“还口口声声地叫师父,我看叫老乌龟还差不多,这
么缩着脑袋,能活到一百五十岁,躲到这么偏的地方,装模
作样盖起了员外府,让我找得好苦。”

蒙面人的几个弟子一起哄笑。小五子也没听出哪里
好笑,倒是何员外有意思,原来是假员外,来田独躲命的,
那起码换个姓啊,常公公还知道开个肉铺叫钱记肉铺,你
何府那俩大宇儿,生怕别人不知道你姓何。

他们还在笑,何员外被惹急了,怒斥蒙面人。貌似蒙
面人也一肚子怨气,还回去一大段话。两伙人不动手,一
来一往说了好半天。小五子慢慢捋明白了,有个沈师祖,
是何员外师父的师父,早年创立三种掌法,断魂掌,蓬莱
掌,仙人掌,三个弟子各传一掌,这样互相车制,有所畏惧,
哪个都不至于霸行于世,偏偏出了个叛徒,可能就是这蒙
面人,二十年前把三本秘籍都偷走了,带回家偷摸练,三种
掌法全部练成,这时三个师兄弟再见面就没那么愉快了,
除了寒暄打哈哈就是互相猜忌,说不上哪天就被师弟师兄
下黑手拍死,沈师祖知道出了逆徒,研发了无为掌,专破这
三大掌的,十五年前选了何员外的师父向问和做关门弟
子,向问和当时还是丐帮帮主,为此辞掉帮主之位,传给何
员外,专心修炼无为掌,到明年腊月初八,满十六年即可出
关,蒙面人当然不干,要赶在无为掌练成前除掉四师弟向
问和,为此何员外带着他东躲西藏,换了七八个地方,在田
独待了五年,本以为能撑到向问和出关,没想到还是被蒙
面人找到了。

小五子在厨房寻思,江湖的事真乱。赌场那个乞丐
他清楚了,来田独找的就是何帮主和向问和。何帮主就是
何员外,可是向问和又是谁呢?哦,闭关呢,没准何府也有
个地窖,摆个红冰床黑冰床各种修炼。那蒙面人是何员外
的哪个师伯呢?小五子就知道一个大漠仙人,吴思若的师
父,希望不是他,不然上梁不正下梁歪,给他断魂掌的也是
一个,还有个叫蓬莱掌,那个中掌后不知道怎么样。断魂
掌是断片儿,仙人拿是耗得生不如死,沈老前辈这么毒,蓬
莱掌肯定好不到哪去。

可何员外不觉得沈老前辈狠毒,还说沈师祖活到百
十岁,却为你这样的逆徒含恨而死,劝蒙面人念在师徒情
分,回头是岸。这时候蒙面人急了,吼道:“不要提那个老
贼,他不是我什么师父,二十二年前要不是他往悬崖下摔
死了我刚出生的女儿,我也不会偷那三本秘籍。”

何员外哈哈大笑,说道:“我知道你是哪位师伯了,何
必还蒙着脸,要不是你做了苟且之事,沈师祖也不会夺走
你的女儿。”

“本想饶你一命,可你自作聪明,知道了我是谁!”

蒙面人说着朝何员外扑过去,后面的弟子与何员外
的家丁也兵戎相见。现在想想,这些家丁应该都是丐帮
的。大堂里叮叮当当的好不热闹,小五子忍不住地探到窗
缝偷看。看起来蒙面人也不一定打得过何员外,两个人纠
缠不休干了十几个回合,反倒何员外逐渐占到上风,一个
虚晃顺势扯掉了蒙面人的头巾。小五子看那蒙面人还挺
年轻的,四五十的样子,何员外手握着头巾,一脸惊愕问
他,不是你?说着后心被一个蒙面弟子抓住了,低声问他,
你师父的那张九宫图在哪里?何帮主回头笑道:“你这无
名小卒,哪有资格问我师父?”

蒙面弟子瞬间移动,一眨眼的时间在他前胸后背各
拍一掌,追问他九宫图。何员外面色苍白,嘴里说着怎么
能是你,没有九宫图。蒙面人摇了摇头,发力一掌打在他
胸口上。何员外空挥了几次手臂,要抓这个弟子脸上的头
巾,终于倒在大厅里。

其他的人还在撕打,这名弟子在混乱中快步穿梭一
遍,等他从人群中穿出来时,何府所有的家丁都被击毙
了。他已经扯掉了头巾的“蒙面人”对了下眼神,蒙面人
命令弟子搜搜何府是否还有其他人。

小五子听见他们去了后院,好像有丫鬟抓着绳子藏
在井下,有人一刀斩断绳子,丫鬟尖叫着落井。听到脚步
临近厨房,小五子无处藏身,蜷缩一团,钻到大肥猪的下
面。两个弟子进来一顿乱踢,其中一个说,火灶里面有一
个。那是新上任的管家,趴在柴火堆里,死抓着火灶门不
出来,口中喊着饶命。

“那就别出来了。”

一个人说着点了个火折子,扔进柴火堆里,关上了火
灶门。里面的管家拼命拍打,两个弟子关心起地上的肥
猪。嘴馋的弟子提议,火都点起来了,干脆吃完再走。他
提了一下,没提动,要另一个搭把手。小五子身子一轻,猪
被提起来,他使劲抓着猪肚子两边,裹在里边,以让自己别
掉出来。身子晃了两下,忽然一霞,连人带猪都扔到铁
锅里。

锅还没热,隐约听见下面的新管家还踢着炉灶挣扎。
两个弟子倒是有了分歧,一个说加水白煮,一个说这么烤
熟了吃。小五子心里骂娘,你们这么讨论,有想过猪的感
受吗?下面没了动静,估计是新管家已经烧死了,锅热了
起来,小五子把自己封进猪肚子里,里面又闷又臭。那个
领头的蒙面人进来呵斥他俩胡闹,赶快离开这是非之地。
两个弟子连连认错,蒙面人身后的弟子忽然上前,一人一
刀,把他俩刺死了。

“你这是卸磨杀驴!”蒙面人吓得声音都颤了,“你答应
陪你演完这出戏,你送我们马帮一张九宫图!”

那个弟子冷笑几声,伸手去掐他的喉管,剩下的弟子
闻声而来,一个个拔剑喊着,放下我们帮主。他掐着帮主
的喉咙,提起他身子抡了一圈,那些弟子各个脸上留下帮
主的鞋印,不出三秒钟,鞋印迅速发黑,弟子们倒在了厨房
里。而帮主被放下的时候,喉管早己捏爆,喷出来的血浆
黏在那个弟子的手上。他蹲下来,用帮主的衣服把手擦干
净,收剑出了门。

终于没人了,该死的都死了,小五子在猪肚里打算再
数十个数,要是没有动静,就赶快逃命。数到六时他就从
铁锅蹦了出来,踩着死人磕磕绊绊,举一盆冷水浇到头
顶。铁锅里的肥猪已经滋滋作响,煎出板油。新管家还在
火灶下面烧着,小五子忽生恻隐之心,想给他捞个全尸。
他打开灶门,蹲下来用木棍掏了半天,只钩出一块大腿的
骨头。他摇摇头,把骨头又扔回到火炉里。

大厅里横尸遍地,小五子倒吸一口气,感觉这三年杀
的猪,都没有今天见到的死人多。还好没一丝血迹,全部
是中掌而死。平静过后小五子反倒舍不得走,装模作样查
看每具尸体,可是他又不懂,煞有介事地在那儿分析,这个
是中掌,这个也是中掌,这个呢,还是中掌。一点新意都没
有,江湖的逻辑他想不明白,有刀有枪干嘛\footnote{原文“干吗”}都拿手拍?他
扶起何员外,真看不出来,平常吃得比猪还多,居然是丐帮
帮主,那门缝里看人的老管家还是他的师父。靠!都他妈
会武功,就我小五子不会!小五子剥开上衣,对着前胸的
掌印,比比自己的手掌。寻思什么掌,自己怎么就整不明
白,一掌能把人劈死?他深吸一口气,右手奔着掌印拍下
去。好像有点感觉了,他加点力拍第二下,最后使出全力
拍第三下,尸体打了个激灵,一口血吐出来,喷在小五子
脸上。

人家是一掌打死,他竟能一掌打活!小五子蹦起来,
退后一步,声音颤颤悠悠地提醒他:“你看好了,我可不是
杀你那人,我是肉铺的小五子,你肉钱还没给我呢。”

何员外指了小五子半天才想起来,说:“对对对,你是
送肉的,正好肉来了,我去给我师父烧俩菜。”何员外站起
来,貌似伤好了,腿脚利索地往后厨走,转身问小五子:“你
也别走了,留下来吃口,咱爷仨好好喝一顿。”

小五子“啊”了好几声不知道怎么接,看着何员外进了
厨房。虽然你没死没伤,可你们何府被灭门了啊,心再大,
也不至于坐在尸体上喝两盅啊。也许就是个冷血动物,自
己活命比什么都强,这些家丁都是上辈子欠你的,为你战
死都不多看一眼。得了,你们爷俩喝吧,我打不过你们,但
也绝不会和你们这种人坐在一桌喝酒。

小五子准备撤了,要插空蹦过去,才能够到大门口。
有两个人并排死在一起,小五子跳过去时踩到一具尸体的
手青,习惯性的他说了声对不起,身后的人还不满意,抱怨
道:“疼!〞

他以为是幻听,慢慢扭回头,被踩手的那个人坐起来
揉着手背,嘴里骂着:“跟你说多少回了,要玩出去玩!”

小五子半张着嘴不敢出声,赶紧走吧,转回身时发现
面前的尸体也在动。他揉揉眼睛,再睁眼时,那具尸体已
撑着手臂站了起来。大厅里的尸体陆续都醒了,脚下的上
百号尸体一个一个地站起来。他们有人傻笑唱歌,有人手
舞足蹈,有人盯着身上的血迹在凝眉思考,到最后有个女
人大哭起来,说,娘,你买这么好的布料,为什么不给我做
件衣裳?

就像被困在车水马龙的路口,前后左右都是人,都是站
起来的尸体,小五子一下子就蒙了,站在原地,一动不敢动。
\newline

{\centering\subsection{9}}

谁把一整只猪下了锅,还把火点上了?何员外把猪
从锅里拽出来,落在马帮弟子的尸体上,用刀割了一片臀
尖,葱姜蒜切好,扔到刚熬出的板油里。他想做红烧肉,把
肉切成拇指宽的小块。这是他老本行,以前在某个大户人
家做厨子。每天剩下的边角余料,他都会乱炖一锅,送给
破庙里的那些乞丐。最后还是被老爷发现,乱棍把他打出
门,叫人一把火烧了破庙,让那些乞丐无容身之处。不管
多委屈多不忿,当乞丐们提出把老爷宅子一把火烧了的时
候,他硬生生把这些人拦住了。

这事过去多少年了,二十五年总有了吧?猪肉块被
他均匀切好,用手拢在菜刀上下进锅里,灶台哗啦啦地往
外溅油,要找个锅盖焖一会儿。好久没下厨了,当上“何员
外”就没进过这个厨房,灶台右下方应该有个锅盖架,大户
人家好点的厨房都是这么摆放。他弯下腰,手臂在下面掏
着,脚底下软软的,低头看一眼,踩在了一个尸体的肚子
上。喉结上都是血,他蹲下来看一眼,是漠河马帮的刘帮
主,旁边还有四五具尸体,应该是马帮的弟子。而他在干
什么,他看看左手上菜刀,锅里的肉还在等他翻炒扣盖。
他扒着灶台边站起来,知道自己中的是什么掌了。
\newline

大厅已经乱了套,刚推开门就有两个家丁要拉着他
出去放风筝,他看着这两个小伙子,一个是何府的马夫,丐
帮的二代弟子\footnote{原文“二袋弟子”},负责和外界弟子保持联系,另一个是他收
养的义子,由于避嫌,何员外迟迟没有给他名分,他打算秋
后让他从一代弟子\footnote{原文“一袋弟子”}做起。要干净利落,少些痛苦,他咽了
口唾沫,挥剑对两人胸口各插一剑。

没人注意这边杀人了,每个人都有自己的世界,唯有小
五子看得目瞪口呆。他看到何员外屏息咬牙,含着眼泪刺
向每一个人。几十秒钟将近百人一个一个斩落,到最后大
厅只剩下他们两人。他呼吸急促,太阳穴上青筋暴突,不至
于让自己哭出来。他将剑扔在地上,背对着小五子往外走,
边走边说:“我等下交代你件事,之后你要把我也杀掉。”

这回是真死了,小五子双腿发抖,“扑通”一下坐在血
浆上。何员外再回来时,托盘上盛了两碗米饭和一盘红烧
肉,嘴里还不住地道歉,好久不下厨,手都生了,让你久等
了。红烧肉色泽不错,小五子接过米饭,却一点胃口都没
有。一口米饭一块肉,何员外说着给他做了个示范。吃到
第三块时,他恢复了理智,将托盘推翻,抓紧时间对小五子
交代:“我中了蓬莱掌,这屋子里所有的人都是,现在是时
有时无,再过几个时辰,我会彻底疯掉。”

小五子举目望过去,血流成河,流到前面的血已凝固,
挡住了后面血流的去路。他低声回应:“就算是中掌,但他
们是被你杀的。”

“这是莲莱阁老的伎俩,一百多号疯子,他们如果活
着,丐帮就毁了。”他在推翻的托盘里翻找,后来把小五子
手中的铜碗夺下来,把米饭倒掉,将碗捧在他面前说,“这
是帮主之碗,我并非员外,而是现任丐帮帮主,我师父向问
和是前任帮主,现在京城皇宫大牢里闭关修炼,腊月初八
之前,一定要把这个青铜碗交给他,将碗底倒扣头顶,抵住
他的百汇穴。帮他老人家出关。”

小五子摆弄着碗,也没什么特别的,碗底镶着一块拇
指肚大小的玉,他看得直皱眉,他不愿搅和进来,屁大点功
夫没有,还想去京城,不等进关就被人像蚂蚁一样碾死。
他摆手推辞说,我得留在田独,你师父是老人家,我们肉铺
也有个老人家,你自己去吧,把碗亲手交给他老人家。

何员外等了等,知道求不动他,转身去拾剑,剑柄朝着
小五子递过去,说自己真的要疯了,身体发肤,受之父母,
以前曾起皙绝不自尽,拜托小五子刺他一剑。小五子站起
来摇头,他没法答应,过去什么样他不知道,但是从田独开
始,他这辈子不想杀人。他想起文思清老说的一句话,有
时候他要亲她,要抱她,文思清总会扭身躲过去,她说,不
是不行,是有些事一旦开了头,就一发不可收拾了。

小五子转身要走,何员外在身后最后一次求他:“杀
了我,不然我还能疯癫地活三十年,别让我在这世上受
辱三十年。”

小五子仰头看看梁顶,转回身接过长剑。何员外用
食指从肚脐往上,一直到喉咙,画了一条线,用你杀猪的本
事,开膛破腹。可杀猪的时候手没有这么抖,他右手抓剑,
左手抓住右手的手腕,告诉他,我数三个数,你抓紧跟这世
上的一切告别。

一,何员外闭上眼睛,活了四十五年,告别只要一秒
钟;二,他睁开眼睛,眼神坚毅地说,来吧;三,小五子将剑
压低,只待起手将对面这个人挑落,何员外低头看了看地
上打翻的饭菜,抬头看着刀刃说,你要是没吃饱,我府上还
有上好的糕点。

小五子左手松开,右手依然握着剑,盯着他问:“你是
又疯了,还是怕死?”

“你尝尝嘛,吴州张知府托人送过来的,放我这儿一年
都没舍得吃。”

何员外说完就要去后厨取。小五子轻吐一口气,右
手松了剑,将地上的铜碗踢还给他,头也不回地走出何府。
\newline

{\centering\subsection{10}}

镇上就传开了,何员外在丧事那天发了癫,将何府上
下满门杀绝,一个人跑到山上当野人去了。几十个当差的
进到山里搜了三天三夜,誓要将何员外绳之以法。绳是绳
了,但始终没有以法,过去五年,大大小小的差人多少都受
过何员外的好处,眼见他疯得不成样子,不忍心他坐牢问
斩。何府还有百余具尸体,他们一把火烧了灭证,又把何
员外放回田独镇。

钱老板问小五子是怎么回事,你那天一大早赶着猪
过去,满屁股是血地回来,到底发生了什么事?小五子这
次讲得很明白,等你告诉我我想知道的,我再说说你想知
道的。钱老板点点头,笑了笑,说我明天再知道,也不迟。

明天什么样呢?他和文思清逛了一天的集市 ,买好
布料回来时,肉铺的门口已经堆了两个包裹。钱老板说,
他年纪大了,肉铺干不动了,你们去别的地方过活吧。文
思清不知所措,求钱老板工钱可以少点,但留他们住下
来。小五子知道他要什么,死太监多少也会点功夫,算是
江湖上的人,办事的手段却这么下三滥?他提起两个包裏
径自走进去,路过钱老板身边时说出三个字:“蓬菜掌。”

三个字引出更多的问题,钱老板跟着小五子上楼,看
着他铺被褥追问,何员外是什么人,闯进来的是谁,为什么
马帮的人也在里面,那些人是怎么死的?一连串问了七八
个问题,小五子把床铺好,拍了拍,对钱老板耳边说:“以后
每到一个月,我回答你一个问题。”

用不着一个月,七月中旬以后田独逐渐转凉,何员外
也从山上下来了。何府烧为平地,他住街边,喝脏水,捧着
-个铜碗,捡着什么吃什么,每次路过肉铺,小五子都会切
一片扔给他。何员外接过肉,就地坐下来啃起生肉,直到
打一个饱嗝,擦擦嘴角上的血,才心满意足地离开店铺。

有时候他会后悔没杀了何员外,至少有三次,小五子
看见他在别人的窗根下撒尿,被房屋的主人追出来痛打。
眼看要入秋,小五子让文思清做一套被褥,自己打一排木
栅,把猪圈隔出一块留给他住。有天夜里文思清吓坏了,
把小五子摇醒,哭着让他去管管。他披上衣服,赶到猪圈,
看到何员外正骑在一只母猪身上酣畅。小五子眼泪马上
就涌了出来,去厨房拿刀,一把将粗喘着的母猪头砍了下
来。何员外号啕大哭,裤子都没有提,就抱着猪头哭,说我
没本事,保护不了你。小五子都回房了,他还在哭,后来他
又回到猪圈,在死猪屁股上切了一块肉,说:“我们去把她
厚葬吧。”

他们走到河边,拢起火,把猪肉架在钳子上。火化的
过程何员外直流口水,一再催促熟了熟了。小五子用刀切
成块推给何员外,他抓着往嘴里塞,头一块烫得他吐出来,
捧在手里吹个不停。小五子问,你还记得自己是谁吗?他
不急着回答,注意力都在那块吃不下去的肉上面,感觉凉
一点了,他一口咬下去,闭上眼睛细细地嚼,咽下去的一刻
他睁开眼睛问:“什么东西,这么好吃?”

“臀尖,何帮主。”

何员外愣了一下,恍然大悟,啊啊了半天说:“臀尖肉,
我最爱吃臀尖肉!”

那就多吃点,吃饱。小五子把剩下的烤熟切好,看他
狼吞虎咽。后来他也吃不动了,看着剩下的肉发愁,忽然
想到说,这些就下葬吧。然后他就挖坑,两个拳头大小的
坑,把臀尖毕恭毕敬地落土为安,跪下来磕了三个头。再
看下去,小五子都要哭了,拍拍他肩膀,让他看着自己,告
诉他:“你是何员外,何振生,你师父是向问和,你不能再这
样丟脸地活着,你是丐帮帮主!”

他掏出刀,对着何员外的心脏一刀捅下去。何员外
眼睛睁大,不知道是回光返照,还是这一刻怕死了。小五
子从他身上摸到铜碗,确认碗底有块玉,之后他松开他肩
膀,不敢多看一眼,都不知道何员外倒下的时候,是躺着
的,还是仰着的,就转身走开了。

还是杀了人,他狠掐大腿两把,让自己别太难受。天
就要亮了,这是最正常的时节,白日和黑夜旗鼓相当,彼此
较劲看谁先被吞噬掉,过了秋天,赢的总会是黑夜。都走
出挺远了,后面有人呼喊救命。小五子脑袋嗡的一下,停
住脚步。杀个人都干不利索,应该狠狠抽自己俩耳光。

何员外躺在河边火堆旁,一只手抓着胸前的刀把,一
只伸向他,求他救救自己。小五子跪下来,“哇”的一声大
哭起来,鼻子一抽一抽地说着,对不起,对不起,让你受苦
了。没有第二把刀了,他使劲把心脏那把刀拽出来。何员
外疼得坐了起来,大口大口地喘着粗气。小五子手握着刀
把,闭着眼睛连捅十几刀。他怕他不死,他怕他再遭一次
罪。他抱起他,蹚着河水往里走,大概在水位到腰的位置,
他把何员外平放在河流中。
\newline

{\centering\subsection{11}}

田独已经不想待了,他和文思清计划私奔。出了那
么多事,死了那么多人,那之后每回经过何府,眼见被烧成
一片焦土,小五子总觉得有些未竟之事,要替何员外办
完。再说年纪轻轻,他不能在田独窝一辈子。他想马上就
走,已经是八月份,再等两三个月就要大雪封山,又是一年
厮守。文思清建议他再等等,就此南下总要做点准备,再
说现在是八月十一,怎么着也得陪钱老板把中秋过完,别
留他一个人在这儿孤苦伶仃。小五子当时只是觉得她善
良,可怜钱老板,直到很久以后他才明白,是文思清想和他
再过一个中秋,她根本没打算跟他下中原。

八月十二他们去集市挑两匹脚力好的马,肉铺倒是
有匹马,娇气死了,碰上雪天,都得是人拉马,文思清说,一
匹马就好了,她喜欢坐在他怀里,被他抱着骑;八月十三文
思清给他打包裹,装的全是小五子的衣物,她说我的就不
带了,等着你赚钱给我买好的;八月十四她把那些没戴的
首饰都卖了,塞给他做路上的盘缠;八月十五他趴在案板
上磨洋工,盼着太阳下去,月亮上来,文思清在厨房做中秋
宴,三个人她准备十几道菜,好像真把小五子当骆驼,吃一
顿能顶半年似的。

那天快打烊时发生了一件小事,一帮京城口音的人,
七八个人骑着马路过田独,不知道是什么身份,互相王爷
公子地叫着,穿得都不错,就一个带补丁的,他们喊他马长
老,小五子知道那是丐帮的。走到钱记肉铺停住了,几个
人侧头看他,小五子犹豫要不要把何员外那碗拿给他。这
时有个背弓挎箭的先走过来,听他们的叫法是六公子,非
要买五斤精肉、五斤肥肉、五斤三肥两瘦的五花肉。小五
子也是跟他们较劲,看准了一刀切,个个上秤,三块肉都在
五斤上下不超一两。六公子扔了二两银子在肉案上,盯了
他好半天,对旁边的那个王爷摇头说:“不是他,确实是个
杀猪卖肉的。”

管他是谁呢,过了今晚,从此以后,连杀猪刀都不带碰
的。不过这样也好,最后一单生意,像是对他三年肉铺生
涯的肯定,他们连肉都没拿,空赏二两银子,那就是佩服他
刀下的准头。

晚饭的时候他还在高兴,和钱老板连干了五六杯。
虽然十几个菜,中秋宴,总还是散伙饭,几杯下肚他忽然有
点动情,他说来你这儿三年了,说不上好坏,总算没把我饿
死,还有啊,以后你可得吃点好的,这么大岁数,没几年活
头了,天天他妈喝粥吃咸菜,还有肉铺的生意,就别干了,
又不是没钱花,有几个像我这么合手的伙计?钱老板也动
情,面色微醺地跟他表态,以后你要走了,我肉铺就关了,
你要是还想干,我死了就把肉铺留给你,不了,不用等我,
我明天就给你,我找个地方去死。

小五子饮尽杯中酒,去茅房上厕所。出来时他拐弯
去趟猪圈,当然不至于跟猪告别,前天买的马被他养在这
里,当时给何员外隔出的一片空地。远远看去,马头下面
露出来,趴在地上。他走过去的时候还寻思,马不是站着
睡觉的吗?可能这是匹好马,睡得香跑得快。把帘子一
掀,酒醒了一半,马背上被砍了好几刀,死在马厩里了。

回到房问钱老板还在喝,小五子举起酒坛子砸在地
上,问他马是不是被你杀的。钱老板倒是没否认,满嘴说
胡话,说马是一般,马鞍真不错,可惜卸不下来,只好杀马
割皮,再取下来。小五子揪住他衣领,瞪着他问:“你是不
是知道我要跑?”

钱老板继续装醉,说你要是觉得亏,肉铺是你的了,我
就喜欢那鞍子。文思清也不知道是好是坏,出去闯一闯挺
好的,留在她身边当然更好。她说,其实小五子天生就是
杀猪卖肉的料,今天来了一帮人,王爷公子的,要小五子三
样肉各切五斤,小五子刀刀切得准呢,弄得那六公子还说,
王爷,肯定不是他。

小五子让她闭嘴,这店白给都不要。钱老板问,那个
六公子是不是张弓搭箭的?文思清看看小五子,不知道该
不该答。钱老板接着问,王爷是不是三王爷。这回是小五
子点头了,还有个乞丐,他们叫马长老,他想了一会儿,问
道:“他们说不是他,其实就是我,对吧?”

钱老板没说话,背着手上了楼。文思清已然被这爷俩
绕蒙了。下楼的时候他手里拿着一块巴掌大的羊皮,告诉
小五子:“这张羊皮万万不可以丢掉,还有,你现在就走。”

真是喝多了,马都被你宰了,我靠什么走?

“想活命马上走,他们肯定会找回来。出去以后,别管
你过去是谁,你就记住两件事,第一件事,你过去不是什么
好人,仇家太多,江湖有一半的人要取你人头,你现在都不
认识了,要提防每个亲近你的人,没准哪个就是要杀你
的。第二件事,真到性命攸关时,你就说九宫图在你手上,
再适时拖延,想办法保命。这张图你藏好了,别放在身上,
让他们搜出来,你就彻底没命了。你要活下来,慢慢你就
什么都知道了。”

文思清半张着嘴听钱老板讲完,咚咚咚地跑回去收
拾包裹,都打好结了,想了想,把头上的簪子摘下来,准备
塞进去。刚解开一个结,包裹一沉,一支箭从下面飞进来,
将包裹钉在了墙上。他们已经来了,白天那七个人。钱老
板上前作揖,把房门关上说,三王爷、六公子,好久不见。
领头的三王爷看到钱老板一愣,说果真是你常公公,我还
当你死在官里了呢,他指了指小五子说:“都还活着,常公
公公,我三王爷跟你要个人,总可以吧?”

“那得看看你要的是谁。”

常公公话音未落,将备好的镖朝三王爷甩过去,他身
旁的六公子挥弓将三王爷罩住。常公公借机连打三只镖
将蜡烛全熄灭。房问里瞬间漆黑,小五子什么都看不见,
只听镖器乱飞,房间里叮叮当当,仿佛到处都是常公公。
有人喊着先保护三王爷,常公公则喊小五子从后门上马。
这问房住了三年,闭着眼睛都能摸到方位,黑暗中小五子
先是上楼梯,从墙上拽下包裏,接着拉起文思清往外逃,耳
边嗖嗖嗖的全是飞镖声,扶墙跑了小半个房间才想起来,
咱肉铺没后门!

可那些人都在往里冲,在正门的对面摸着找后门。“咯
吱”一声,正门开了,中秋园月,大片的月光泻进来,两个身
影溜了出來。刚踩到门槛,远处伸出一只手,拽住他包
裹。小五子往后拉,又不敢太大力,怕对方一松手,自己又
摔回房间里。

就是那个乞丐,左手拽着包裹把小五子往怀里扯,右
手已蜷成鹰爪去抓他肩膀,拳出一半忽然收回去,痛叫一
声。文思清把簪子插进马长老的手背,簪尖从手心穿出
来。马长老松开包袱,右手去拔左手心的簪子。小五子弯
腰接过包袱,拉着文思清便向月色跑去。

外面果然全是马,他们的坐骑都在门口。小五子拉
着文思清问上哪个,文思清说好看的。小五子七匹马前过
一圈,挑了匹白的。里面的人一时还出不来,常公公堵在
门口冲屋里发暗器,将月光挡在门外,将敌人挡在门里。
六公子张弓搭箭朝身影射去,飞镖连挡了三支箭,第四支
箭射到了常公公的左腿上。

小五子把文思清抱上马,看着常公公一瘸一拐地跑
出来,他要他快上马逃跑。常公公懒得理他,嘴里念叨,没
你拖后腿,我让车马炮跟他们打。他拔下左腿上的箭,箭
头倒钩带出一大块肉,常公公咬牙忍痛,告诉他,跑多远,
跑多久,都记着回来,我给你看店。说完手握箭头一把扎
在马屁股上。白马吃痛开始狂奔,也没个方向,就是撒了
欢地跑,小五子和文思清四手捆着缰绳,才不至于摔下来。
\newline

{\centering\subsection{12}}

可能是往西,白马一口气跑了一个多时辰,行进山林
深处不见人烟之地,一声长啸,倒在了泥地里。小五子抱
着文思清提前跳下马,马屁股的箭头还在流血。文思清问
这是哪啊。小五子也不知道,北方地广人稀,一个地名能
管方圆百十余里,没准还在田独。文思清仰头看看天色,
月圆而皎洁,嘴里算着,马跑一个多时辰,走回去要多久
啊。出都出来了,还走回去?小五子被这念头逗乐了,跟
她算了一下,走回去大概要十八年。

文思清瞪眼睛尖叫:“不可能那么久!”

“当然不可能。”

“那你说十八年的!”

哦,那我算错了。小五子找个石头坐下,歇一会儿。
文思清执意要走,她娘的骨灰还在店里,要是三王爷一把
火把肉铺烧了,就分不清哪个是她娘,哪个是猪了。而且
她压根就没想跟他南下,她知道他怎么想的,他去找他的
“瑶”,找他的梅兰竹菊,要是他过去花,姑娘多,南下一趟,
不得拉上一车女孩回田独?百花谷少谷主,听着就八九不
离十。她才不要跟着去,当个碍眼的,拖油瓶的。

“再说,你这么机灵,遇到事肯定能逢凶化吉,有我在
就是拖累你。”她看小五子摇头,补充道,“你真的特别好,
也就是这两年遇到你,是我文思清的福气,要是早几年,你
都看不上我。”

本来挺深情,说说又跑偏了,文思清开始抱怨,赌场第
一眼你就没看上我,赢都赢了,还说下次要赌人就带个好
看点的,就是嫌我丑,不管了,丑也要回来,跟一起睡一年
多了,不能说扔就扔。小五子连忙打断她,别,别这么说,
那是一间房的两张床。

“以后拼起来,不就是一张床了?”

小五子有些犹豫,文思清又不高兴了,果然嫌我丑,我
都说成这样了,你还犹豫。说完她转身就往山下走,小五
子跟在后面解释,真不是在犹豫,他想歪了,他在想,那俩
床不一样高,拼起来有点怪。文思清回头“扑哧”一笑,示
意他下山再说。

后来两个人没怎么说话,踩着月光往下走。小五子
偶尔拉住她的手,担心她滑下去。前方越走越亮,晨曦之
时从山腰就可以望到田独的轮廓。文思清在一条小溪前
蹲下来,喝了两口水,告诉小五子,溪水尽头就到南方了。
喝过水后她掏出小刀,早准备好的,在手臂上刻了一
个“五”字,血从这四划涌出来。小五子没拦住她,摇头说我
叫不叫小五子,还不一定呢,万一我跟钱老板一样,好几个
名字,能在你胳膊上凑首诗。

“以前叫什么我都不管,反正你在我这儿,就是小五
子。”

她把小刀递过去,说该你了,刻个“羚”字,别哪天再中
一回断魂掌。小五子拿刀比画半天,跟她商量刻“儿”字行
不行,“羚”字笔画太多了。

“我就要‘羚’字,我要刻‘文思清’。”

文思清抢过小刀,把他胳膊拽过来,摸着上面“五”字,
有点心疼,不忍心落刀,放下他的衣袖说:“你顺着溪水走
吧,我想通了,你就算找着了你的苏子瑶王子瑶,我也不
怕,她们跟你没关系了,你上辈子是她们的,你要记得,你
这辈子就是我文思清一个人的。”

总之是告别,叮咛的话一辈子也讲不尽,她不愿再说
了,掩面离开,都不敢回头看小五子蹚过那条小溪。

\newpage