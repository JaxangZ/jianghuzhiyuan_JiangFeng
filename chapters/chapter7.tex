\section{柒}

{\centering\subsection{1}}

小五子用手指点着,还有七个人轮到他,他在队伍里往后看,扛包裹的,推车的,抱孩子的,浩浩荡荡七十人都不止,而且这还只是一扇门,嘉峪关的士兵开了四个口给百姓出关。深秋十月,越往北越冷,走出塞外应该已是天寒地冻,每天都有上千人出塞离家,朝廷到底干了些什么,让这些子民弃了大好河山,背井离乡去塞外受苦挨冻找活路?

他从南京出发,走了小半年才到嘉峪关,风声最紧的时候,还在扬州的赌场躲了十天。那些金条还在,以
他的赌技,输出去费劲,可他也不敢大赢,每天对付点儿打尖住宿的费用也就够了。天生的本事,几根金条做本钱,能在扬州白吃白喝一辈子。中秋过后南北的赌客陆续回家猫冬,再玩下去太惹眼了。小五子打算撤了,他把本钱收好,多出来的银子,分一半散给赌场里的博头、柜主和账房,剩下的一半雇艘大船,沿着长江西去。他走走停停,大山大河的地方就让船停一天,上岸转一圈,直到重庆府才改陆路北上。

都知道他昆仑公子混进丐帮,从山海关进的中原,这次他要绕一圈回田独。边塞二十六关十八路,他算准不会有人在嘉峪关守着他。很奇怪,亲爹亲妈不记得,
从居庸关到铁门关,从函谷关到阳关,倒是记个门儿清。前面队伍还有五个人,好半天才过去两个,守城士兵拿着画像,在每个人面前比划一阵儿,确定不是才放行出关。估计是找他,昆仑公子。画像这东西,小五子根本不担心,以前那张通缉令在肉铺边上挂了两年多,都没看出来画的是他。多少表示点儿尊重,他蹲下来抠一块泥,搓成泥球按在嘴角上。痣是有了,但不是说脸上长痣,痣上长毛吗,他拽根头发揪几段,裹在泥巴里搓第二个泥球,犹豫要不要换个痣上去。应该真一点,跑了一个秋天,八千里路只剩下最后一道关,走出嘉峪关,往东往北,就可以回到田独了,他确定文思清会在那里等他,笑靥如花地站在肉铺门口,没准还有钱老板,告诉他刚刚杀了一头猪,就等着你小五子回来一起过年呢。他提醒自己进镇子以前要好好洗个澡,走了半年的山河和冰川,别狼狈得像条野狗一样回到她身边。

归心似箭,队伍却卡在第三个人那里,守城士兵拿着画面对比了半天,眉头紧锁,依然想不明白。士兵找来长官,指着面前的少妇,说李大人,虽然她是女的,可是她长得和昆仑公子实在太像了,我们该不该扣住她?真是个难题,长官尽量放松,不愿让下属看出自己也被难住了,他反问下属,这上面有写昆仑公子一定是男的
吗?下属摇头,说写了面部特征,写了身高臂长,但确实没说昆仑公子的性别。那就按照规定带走喽。下属咚咚咚地点头,李大人果然有勇有谋智慧过人。他手臂一挥,叫两个士兵把少妇绑起来。

少妇张大嘴巴吓蒙了,挣扎着问他们,画像上的人满脸胡子,我怎么就像他了?下属可不管,李大人这么要求的,他让人把少妇送到李大人行营,请李大人验验是不是昆仑公子。

小五子回头看着她被押走,脑子里始终响着一句话,英雄不在本事,在胆识。那就上吧,他跨出一步喝道,你们要干什么!卫兵们打量他问,你要干什么?小五子没回答,他被另外一个黑衣男人分了心,他顺着视线往后找,很快在隔了三个人之后找到了另一个白衣服的。两个他都见过,一个黑衣,一个白衣,一个漂亮一个丑,漂亮的那个满脸刀疤,丑的那个皮肤特别好,连痘都不长。想不起在哪儿见过,但肯定是这三个月,八千里的逃亡路上。辽阔河山里的芸芸众生,同一个人见过两次就已经很奇怪了,两个人第二次出现,一定有问题。

先排着队,眼前这么多当差的,估计也不敢怎么样,反正他在前面,一旦通过,出关就找个地方藏起来。他眼睛往前看,耳朵听后面脚步。前面就剩一个人的时
候,大门关上了,卫兵上了城楼,说去别地儿排队吧,我们要午休了。小五子前面那个不干了,说我知道你,你刚接班一刻钟,过了三个人就要午休?卫兵打个哈欠,说不是我要午休,是这扇大门该休息了。

你争不过当差的,别看你是男的,惹急了也说你像昆仑公子,送到李大人营房去验货。百十来人就地解散,混到另外三支队伍后面。小五子低着头,去最远的那个,他抢得慢,几乎是队尾,左右没见着黑白两只鬼,队伍往前走了几步,脖颈后面有人喘着粗气,那两个人又站在后面了。

小五子低头看脚面,是见过,在黄鹤楼,本不该去那么惹眼的地方,可管不住嘴,那可是天下第一楼,就算被人认出来,从楼顶上推下去,也要吃顿好的再摔死。这两个人分坐两桌,每人点两盘菜。当时客人多,店小二建议小五子和别的客人拼桌。小五子赶忙拒绝,硬着头皮点了一桌子菜,说自己要宴请朋友。于是店小二找黑衣白衣商量,他们也是不答应,跟小五子一样,各自加了几道菜。事情到这也没什么,直到把菜上齐,小五子发现两人面前的八菜一汤是一模一样的。黄鹤楼美味千百种,一道两道撞车都不应该,八道菜一样,十有八九就是来盯梢他昆仑公子的。小五子放下筷子结账走人。出门右转进了一个巷子,进家客栈他开间二楼上房,从窗户盯着门口。两人倒也没跟出来,反而细嚼慢咽了半个时辰,并排下了楼,不认识一般,相互不说话,在黄鹤楼门口,一个向左,一个向右,头也不回地分道扬镳。

可能是下战帖,比武前的较劲,或是某个秘密帮派在接头。是不是过于小心了?小心驶得万年船,用不着一万年,保我半年到田独就好。现在看起来并非太小心,当然是冲他来,人家把气都呼到脖颈上了。他回头直视他们,两人反而左顾右盼不看他。他闭眼盘算,先排着吧,跑到哪儿,这俩货都得跟屁股后面。

下午太阳上来暖和一些,路面都化成泥浆,蹬腿出去能把泥点甩到前排的后脑勺上。天色渐暗,泥浆又冻得梆梆硬。终于排到头了,下一个就是他,小五子摸摸嘴角上的痣,指肚感受着痣上的毛,向前跨出一大步。守城士兵斜眼看他,举一天画像,早用不着这个了。他让小五子抬头,说你把嘴角上那泥点擦掉。小五子说这是痣,你看,上面还有毛呢。把门的没接茬,把画像重新打开,看一眼画上人,看一眼小五子,皱着眉头,让旁边士兵请李大人过来。

看样子要出事了,小五子看着他走过去,拉着李大
人胳膊比划,之后两个人朝这边走过来。后面人催他快点,退是退不出去了,身后全是人,关外倒是空旷,就一门槛,迈过去就是一大片冻硬了的泥地,但是弓箭手都在城楼上呢,守了一天百无聊赖,可算逮着一个拉弓射箭的目标了。

后面人还在催,小五子回头看一眼,居然是白衣男子骂骂咧咧的,话不停嘴,小五子祖宗十八代都被他问候一遍,后来把黑衣男子都骂急了,转身问你骂谁呢。白衣服直翻眼皮,说谁排前面我骂谁。黑衣服跳起来,越过中间几个人头顶,扇了白衣服一巴掌。白衣服愣了一下,捂着他那又丑又光滑的脸,推开前面几个人,一脚朝他肚子踹过去。黑衣服先挨了一脚,第二脚有所准备,双手抱住他大腿。白衣服一脚被抱,另一只脚蹬着地,伸手去搂黑衣服脖子。场面有些混乱,两人扭成一团,滚在地上。小五子看不明白了,分明是臭无赖打架,看起来也不会什么武功,怪不得从武昌汉口,一路跟着他到了嘉峪关,都不跟他动手。

离老远就听李大人喊怎么回事。把门的站起来,指着小五子说,李大人你看看,这人像不像?李大人瞪大眼睛,但不是看小五子,指着地上翻滚的黑衣白衣发火道,我问你这是怎么回事,火烧眉毛了,你还堵着门口不抬屁股?把门的吓着了,连忙起身让人把寻衅滋事的两个人抓起来。小五子还挡在他面前,把门的冲他吼起来,赶快给我滚出去,不然连你也抓起来。一个识眼色的下属推了他一把,小五子连人带包裹摔在了门槛外。大门在身后缓缓合上,里面传来李大人的喊话,因为这两个人,今天谁也别想出关!

小五子捡起包裹爬起来,摸摸嘴角的痣,竟然还没有摔掉。他仰头看一眼,城门上的弓箭手陆续收弓撤岗。背对着大门他走出几步,面前一片塞外的苍凉,雪片从空中落下来,飘飘荡荡,他大步往前走,要赶在雪下大以前找到过夜的地方,哪怕只是一个树洞。
\newline

{\centering\subsection{2}}

睡到一半他想明白了,他们在装,三脚猫是装的,吵架也是装的,两个人保他顺利过关,别被官府抓走。李大人关不住他们,从牢里跑出来,昆仑公子早晚是他俩的。之后他就睡不着了,裹在树叶里翻来覆去。嘉峪关以北一片坦荡,寸草不生,小五子一直走到天黑,也没见个树洞山洞。听说还要往前小一百里才有个辉山镇,又
累又困,他把掺着白雪的树叶干草拢成一堆,钻进去对付一夜,万一明早还没冻死,就去镇上加几件衣服,雇辆马车往田独去。

他睡到凌晨出发,走到下一个天黑才到镇上。天寒地冻,腿都冻得打不了弯。他犯懒找家客栈,跟店老板说开间最好最大的客房,有三个火炉的那种。身上还有几根金条,痛快点花掉,没准哪天死到金条前面去。

进了房间,他又是一觉。夜里醒来从窗户看去,外面又下雪了。他下楼让小二做碗面条。小二进厨房转了一圈,回来说面条要等,面要现和。小五子问他,炒菜米饭馄饨水饺,什么快?小二挠了挠头,说面条快。

那还说什么呢,面条。他让小二去做,他坐在这里等好了。小二哼着小曲出去,没一会儿背袋面粉回来,卸在厨房问他想吃什么口感的,有嚼头的还是软和点的?就是面条,正常什么样,他吃什么样。他说着去关门,看见风雪里一黑一白从老远走过来,夜色里白的乍眼,黑的看不见,不过挂了一身的雪,感觉白衣男子牵一个雪人往客栈走。

他掏出碎银子放桌上,说自己肚子不舒服,先不吃了。小二跑出来拉他胳膊,说面我都和好了,这么晚你让我卖谁去。小五子皱眉望着他,你谁也不该卖,我面钱给你了啊。

可是面条没人吃啊,小二依依不饶,抓着他胳膊不撒手,早告诉你时间长,你说可以等的。小五子把着楼梯扶手,甩不掉他的手,就闷头往上走。小二也是能扛五十斤面粉的体格,胳膊抓不住就去抱他大腿。

门吱的一声开了,一阵冷风吹进饭堂,人还没进来,白衣男子就问小二弄点吃的。小二松开手,和小五子一起回头看。黑衣白衣对了个眼神,黑衣服手臂一展,做了个邀请的姿势说:“这位公子,留下来一起吃点吧。”小二这次要一百个确认,你们是三个人,吃面条,而且谁也不许走,明确过后他满心欢喜地去厨房和面。小五子靠在椅背上坐他俩对面,身子都快瘫桌子下面去了。他先表态,他说:“我知道你们在找我,跟了我几个月。”

俩人没说话,他感觉这两个人在控制,白衣人张了几次嘴,都被黑衣人按住了手腕,好像有什么浓烈的情感克制住,别爆发出来。后来白衣人还是忍不住了,不顾黑衣人反对,起身把椅子踢开,扑通一下跪在地上,大声痛哭,说少帮主,你真不记得我们了吗?小五子腾地跳起来,但不是扶他,而是向后退了两步。黑衣人刚才
一直拦着他,此时叹了口气,事已至此的叹息声,也跟着跪了下来,埋怨白衣人道:“叫你不要相认,这不是给昆仑大哥平添负担吗?”

小五子让他俩快起,有事慢慢讲。白衣人把筷子掰成段,摆出一副简易地图说,咱们本是昆仑门下,世代住大漠以西的昆仑山脉下,当然你是我们少帮主,前几年你涉足中原,得罪了不少武林门派,直到你消失,江湖上的人找不到你,昆仑派也找不到,老帮主急火攻心,带着昆仑派几十个弟子穿过沙漠,一路往东,跟我们说找不到少帮主的话,就永远不回昆仑了。白衣服说一半,黑衣服接他话往下说,昆仑派这几年分散在各地找你,汴梁的武林大会我俩也去了,恨自己本事不够,没能力把你从大漠仙人和蓬莱阁老手里救出来,只好一路跟随,在金陵你用计甩掉二老,我们跟着少帮主你的路线,一直到这里。

“二老都被我甩掉了,你们俩反而没甩掉?”

“我们也曾疑惑,以昆仑大哥的本事,自然早发现我们俩一路在跟踪你,”黑衣服说,“但是白师弟说,没准昆仑大哥想起了同门情谊,才没戳破我们。”

“他叫白师弟?”小五子左右看看,“那你就是黑师哥?”

“不,我是白师哥,他是白师弟,我跟师弟碰巧同姓白。”黑衣服凑近半个身子,低声说,“昆仑大哥回想一下,大漠蓬莱二老带你走的一个月,尚且有不少寻仇的,你独自北上的三个月,是不是一个仇家都没见到?”

白衣服终于说话了:“为少帮主扫清障碍,是我们应该做的。”

前面什么昆仑山脉昆仑派,小五子差点信了,给你昆仑公子这名号,三岁小孩都能猜出是昆仑派的,真正有想象力的身份是百花谷少谷主,再不济也是丐帮帮主。两人一个喊少帮主,另一个却叫昆仑大哥,故事没编就算了,口径先统一下行吗?但后两句倒不假,一路没仇家,他还以为自己藏得好,殊不知这黑白配一前一后替他开路垫后呢。

那他们找他到底干什么?先陪他们演一会儿,小五子问,我爹怎么样。白衣服摇摇头,眼泪又要涌出来了,哽咽着说老帮主身体不大好,天天盼着你回来。小五子看出来了,白衣服最能演,老帮主少帮主,磕头下跪的,入戏还挺深。面条上来了,等半个时辰就两碗白面,黑衣服问菜呢,弄点酱油也行啊。小二说酱油得等,没有现成的。

那可有得等了,豆子发酵都不知道几个月,小五子说就这么吃吧,他秃噜着面条说,回去转告我爹,孩儿有件要事得办,事情一结束,我马上回去。黑衣服摇头,咽下嘴中的面条说,还是跟我们回去吧。白衣服的筷子早掰没了,他去别桌找筷子,在小五子身后那桌说:“少帮主,你不知道老帮主有多想你。”

小五子点点头,他明白了,不是寻仇,也不是什么故人,他们是受雇带他见一个人。
\newline

{\centering\subsection{3}}

这次小五子不想跑了,也跑不了,身边多两个人照顾他也挺好。半年里风餐露宿,这十几天还胖了一点。慢慢发现这俩人也没多大本事,跟大漠蓬莱那几个人没法比,就是跑得快,骑马技术好。通常都是黑衣人一匹马在前面,白衣服和小五子驾一辆四驾马车,在一里开外跟着。还是有寻仇的,横刀立马,问车里面坐的是谁,这时候黑衣服去跟他们交涉,下马作揖。仇家自然下马还礼,只要仇家脚着地,白衣服驾着四匹马就跑,根本不给他们上马追赶的机会。

估计约好的,每回都是,白衣服往前跑三十里,再右转跑十里,然后就驻营等黑衣服赶上来。只剩他们俩,白衣服还要演,比如小五子问他,昆仑派的绝招是什么?白衣服深深叹口气,说自己年少时偷懒,把昆仑一点绝学得皮像肉不像,看着好看,使出来却杀不了人。他找块空地,抡胳膊要一通,问他现在的昆仑一点绝能打几分?

昆仑一点绝?问你昆仑派绝学是什么,就起了这么不负责任的名字。小五子说能打一百分,二百分,断魂掌之后,我是一点不会,绝也不会了。这时候白衣服又要演了,感同身受的那种难受表情,望望天,望望地,眼眶湿润,劝小五子别难过,老帮主不是一直告诫咱们嘛,英雄好汉,武功在其次,最重要的还是人品,你要是做了一个好人,哪怕没武功,被人活活打死,那也是死了的英雄好汉!多愁善感,一般都要演到黑衣服过来会合。黑衣服马都不下,持着鞭子说:“走吧,那边等着交货呢。”那边是哪里呢,嘉峪关往东日夜赶路,过了乌海再往前叫沉狮谷,这天难得住了店,次日清早居然没有催他上路。黑衣服出门办事,留白衣服在客栈里看着小五子。白衣服说,沉狮谷就是昆仑派的老巢,他想想又补半句,暂时的老巢,老帮主年纪大了,白师哥先去招呼一
声,让他老人家有个心理准备。

那交货就在沉狮谷了,听名字有点熟,但江湖上唬人的名字,无非就是狮虎熊豹。黑衣服一直到午后才回来,后面跟着一老一少两个人。老的也不算太老,四十多岁,一起来的小伙子喊他齐师叔。他进来就问人在哪呢。也不等白衣服介绍,目光锁定在小五子身上,奔到他面前,弯下腰,几乎是贴着脸又看一遍,起身说:“很好,你们俩谁跟我算下账?”

果然是卖小五子。脏活累活黑衣服干,用脑子的事要白衣服来,他们到门口算账。黑衣服留在房间,盘腿坐在小五子旁边,盯着他问,不需要给你点穴吧?小五子说不用,我哪也不去。黑衣服点点头,但还是不放心,拉起小五子的手,捧在手心里。一时间小五子有点害羞,跟他强调我真不跑。黑衣服说,我知道,所以没点你穴,说完还摸摸他的手背,冲他笑笑,那意思是你放心,大家好聚好散,我们不折磨你。

他们在门口讨价还价,小五子听懂了,这职业有点像保镖,不过保的是人,你要找谁,管他活的死的,哪怕是躺在墓里面,他们俩也能把棺材挖出来,完好无损地给你送到家。黑衣服握着小五子的手说:“早看见你了,反正你往北走,就想等着过了嘉峪关,再跟你说。”

小五子努力抽出手,点点头:“你们倒是图省事。”两人在外面吵起来,齐师叔喊道,说好的价钱,临时翻倍?白师弟说,你们没说这人这么惹眼,路途又远,躲了多少仇家,才送到沉狮谷,再说你们这一单,我们兄弟俩跟了三年,翻倍都是少跟你要了。齐师叔冷笑,说当初价钱是你们开的,说找昆仑公子,你们就应该知道这四个字的分量,你们干三年,我们不一样等三年。白衣服也笑,说齐师叔的话在理,他问屋里白师兄准备怎么样。黑衣服说掌控之中,说完还不忘摸摸小五子手背。白衣服说,不行就弄死他,当我们没来过,昆仑公子分量重啊,我们哥俩就是扛着他尸体,各门各派要份子钱,也能要得比你多。

原来为这个摸的手,黑衣服拇指扣住小五子手腕,手臂一阵酸麻。再不叫命就没了,七分痛,小五子十二分地叫出来。外面两人不说话,估计是互相瞪着,看谁先服软。最后是钱袋落地的声音,齐师叔加了钱,他说:“把人带出来吧。”

送上车的时候白衣服还要演一波,他先把小五子绑在马车座位上,扯一块布蒙上他眼睛,叮嘱他回去要懂事,孝顺一点,老帮主这几年为你操碎了心。小五子说好,两位师弟在哪里,下次我去看你。白衣服愣了一下, 黑衣服抢到他前面,说:“江湖之远,何必再见。”

马车开走了。原来小伙子跟来是干力气活儿的,他 在前面赶马。齐师叔坐到小五子旁边,说你别见怪,路 不好走,绑着你怕被颠出来。

果然够颠簸,起车就往下冲,五脏六腑都被震得重 新排一下位置。进谷之前,齐师叔在旁边发出一阵怪 叫,也不知在跟哪个禽兽打招呼。后来没声了,车子不 再冲得那么狠。把车停下来,小伙子把他抱到一个房 间,解开绳子。齐师叔说你先休息一下,晚上还有好多 事等你干。可这也不是床,摸起来就一块铁板。小五子 点头,说齐师叔太客气了,还盼早点见到贵帮帮主。

等半天没人说话,估计是出去了,他把眼罩拿下来, 反而吓了一大跳,睁眼一片漆黑,一丝光亮都没有,就好 像有人给他套了一个更大的眼罩。他脚趟地往前走,一 直摸到墙,房间里是空的,没门没窗。他捋着墙走,都是 实墙,摸不到暗门机关。摸完四面墙应该是一圈,他站 住想一下,走回去再摸一次墙角,不是直角,比直角大一 半,这是一个六面墙的蜂巢一样的房子。

他们怎么进出呢,他回到房子中央仰躺下来,头枕 着铁板,身下一阵阵凉气,想起钱老板那张血冰床,挖在 猪圈下面,当时就是他扶钱老板下去的,那么重的伤,睡 几天就好了。我也睡,睡一觉就好了,快睡着时他自言 自语,对啊,这也是个地窖。
\newline

{\centering\subsection{4}}

这一觉睡得够饱,睁眼时也不知道是白天还是黑 夜他揉揉肚子,还不算太饿,翻身趴在铁板上继续睡。 半睡半醒间听见有人喊他,老鼠似的偷偷摸摸,用那种 气声呼唤:“昆仑公子,你在哪儿?”

气声是听不出男女的。小五子站起来,脚跺着铁板 喊:“在这儿呢,在你下面!”

外面人不喊了,小五子头顶窸窸窣窣的,真像是老 鼠在打洞。没一会儿天窗被撬开,一根绳子扔了下来。 小五子抓住绳子往上爬,上面的人着急了,连忙说;“等 一下,你太重啦,都快把我拽下去啦。”

这回是真声,一个小丫头的声音。小五子仰头等 她,外面已经黑了,原来一觉睡到了夜里,他听见她一路 小跑远去,又一路小跑回来,在天窗探出半张脸,笑着 说:“这回好啦,我把绳子绑树上啦。”

虽然不会武功,但他也不是残疾,抓绳子上窜总没问题,片刻之间爬出天窗。他张望一圈,地窖在一片园林之中,三十步外有个挂灯笼的房子,几个佣人在里面进进出出。小五子把绳子收好,天窗合上,将草垫子盖到天窗上。小丫头十四五岁,就是个小孩子,肯定跟他中断魂掌以前没关系。小五子问她,是谁让你来救我的。小丫头微微一笑说:“到时候你就知道了。”

“嗯,”小五子趴到草丛里,看着周围的情况,低声说:“我们怎么出去?”

小丫头指着挂灯笼的房子,说前门进去,推开后门就可以出去了。小五子眯眼看过去,又有两个人从里面出来,一个抱着猪头,另一个装了一推车的猪下水往外运。看样子是个厨房,大半夜的还在赶做酒席。显然不能从那儿进出,不然一会儿厨师就要抱着他的脑袋出来了。小五子摇头问她:“你是从哪里进来的?”

小丫头指指左边,又指指右边,后来她也编不下去了,从草丛里站起来拍衣服上的土,冲小五子努嘴说:“算了,我逗你玩的,我不是来救你的。”

小五子没明白,他问那你这是怎么回事。

“没怎么回事,”小丫头说,“帮主叫我带你过去啊。”

这太伤人了,小五子深吸一口气,他都要哭了,噙着眼泪质问她:“你逗谁都行,我就那么好逗吗?”

没想到他反应这么大,小丫头也不好意思,但既然问出来了,她承认道:“你是挺好逗的。”小五子不想搭理她,起身反向走,小丫头在后面喊他:“你就跟我走吧,跑又跑不了。”

小五子一口气吐出来,站在原地。很快小丫头到他身后,让他前面右转,草丛里穿过去。她让小五子走前面,果然是从厨房进去,厨师伙计停下手里的活儿看小五子。小丫头呵斥他们干活去,昆仑公子也是你们看的吗?这么小的丫头,在谷里的身份可不低,小五子扫了一眼,厨房的肉还挺多,两扇猪挂在铁钩上,还有一头被剖开的牛放在案板,不过没主食,也没素菜,这倒有些奇怪。

从后门穿出去要走段石板路,隐约能看到远处房间的窗户透着光。尽管不想说话,可是还是好奇,小五子问为什么厨房这么远,什么好菜端过去都凉了。小丫头在后面偷笑,她说那些肉又不是给人吃的,根本就没热过,你还担心凉掉。

他问不给人吃,那是给谁吃的?小丫头又笑起来,提醒他走直线,掉下了石板路,可就不大安全了。本来
两侧就看不见,现在更是阴森森的疼人。他把油灯放低,低头看脚下的石板。左侧忽然一声嘶吼,小五子吓得往右跳一步,这时右面又叫了起来,左右都有,小五子不敢再跳了,把油灯放在石板上一动不动。

小丫头又咯咯咯地笑了,说放心走嘛,它们在铁笼里。小五子问是什么东西。小丫头说:“你想啊,我们叫什么谷?”

沉狮谷,小五子慢了脚步,贴在小丫头前面,恨不得拉起她的手。不能老想狮子,他问小丫头叫什么名字。小丫头说她叫小玉,本来该叫大玉的,但她打小不喜欢,她跟她娘商量,让她先叫着小玉,如果有妹妹了,就得把小玉这名字还给她,自己改回大玉,结果一直到死,她娘都没能给她生个妹妹。

小五子想告诉她,其实我没那么在乎你叫什么,小丫头自己讲了五分钟,小五子默不作声。眼看走到挂灯笼的房前,小玉把门推开,一阵香气扑面而来,银梳铜镜红纱帐,小五子一时结巴起来,这这这地好几次问,这是姑娘的闺房吧?他打定主意不进去,小玉再小,那也是个女的,况且真的太小了。

小玉先迈门槛进门,说我哪有这么好的福气,这是我们小姐的房间,历来没什么人来沉狮谷,我们也没准备像点样子的客房,还请公子委屈一下,在小姐房间沐浴更衣,帮主还等着见你呢。

进这房洗个澡,还不知道谁委屈谁呢。他往里走几步,小姐不在,床下面摆着一双青色绣花鞋,沐浴间在最里面。小五子挠挠头,问你们家小姐是哪位。是帮主的女儿啊。小五子说我知道,我是问她姓什么叫什么。小玉捂住又要笑,这半个时辰就是在没完没了地笑,是在笑他乡巴佬一样问来问去的吗?小五子做出一个打住的手势,说回答完你再笑,我陪你笑。小丫头看出他恼火了,嘴上憋住,眼睛却还在笑,她说:“你问小姐是谁,小姐是你夫人啊。”

小五子让她慢点说,用普通话,夫人在你们这儿的方言是什么意思?小玉仿佛刚知道,夫人原来是方言,她还挺认真,说夫人就是老婆、媳妇、相好的,接着抬高半个声调说:“公子今晚要当新郎倌了呀。”

小五子再次看看小姐的闺房,心跳得有点慌,问她:“你又在逗我玩?”

“没有,知道你开不起玩笑,我再也不敢逗你了。”

“我怎么就开不起玩笑了?”他皱眉问,“你们大老远请人抓我回来,关在地窖里,再让你一路看着我,这不是
新郎倌应有的礼遇啊。”

也不知道小姐看上他哪儿,面前这个人笨死了,什么都要问。她再跟他讲一次,小玉说:“因为我们怕你这回又跑啦。”
\newline

{\centering\subsection{5}}

小玉催了几次,小五子没应声,他泡在木桶里一直想着,想明白之后他从水里出来,也没穿他们准备的新衣服,将原来脏衣服一件件穿上,踏进沾满泥点的靴子,他叫小玉进来。他说:“你杀了我吧,我不能娶你们家小姐。”

小玉望着他,一开始以为是报复她,开那种莫名其妙的玩笑,确定小五子是认真的,小玉就反复讲,我们家小姐有多漂亮,比我好看一百倍一千倍,你要是过了这个村,就是再走上十万八千里,也找不到这个店了。小五子摇头,管她如何美貌,但我心里有别人了。

“你心里有谁,我帮你杀了她,不就好了吗?”

真是养狮子的地方,小丫鬟讲话都这么任性。他不想再纠缠这些,只问小玉,婚我是不结的,你杀不杀我?小玉哪敢杀,你死了,我们家小姐嫁谁去?小五子说随便嫁给谁,沉狮谷这么大的庄园,如果你家小姐真如你说的这么好看,还愁没有人娶吗?小玉眉毛一挑,可傲娇了,背着手说,想娶我们家小姐的人都得排长队,比满大街要饭的还多。小五子愣了一下,但还真能感受到,想娶她们小姐的人应该挺多的。

“只是,我们帮主点名要嫁给你昆仑公子。”

“还不是你们小姐要嫁?”

这是要图他点什么,武功没有了,他还有什么价值。他往怀里摸到九宫图,昆仑山庄保过他的命。也许有人真信了他有好几张,把女儿舍了,拜过天地,这九宫图就是沉狮谷的了。

小五子点点头,也不是认同什么,每次想明白一件事,都会不自觉点点头。他一句话不说,转身就往外走。小玉反应迟钝,看他出了门,才意识到这是要跑,赶紧追出去,从后面点了他的穴,拽着他肩膀拖回闺房。她说我又不杀你,你瞪我干吗,婚还是要结的,你要是死了,或是跑了,我今晚不就要喂狮子啦?

她把他放到床上,犹豫要不要给他换新郎倌的衣服,拿衣服比划两下说算了,把你这脏衣服脱下来,小姐要吃我干醋呢。她在屋里找绳子,要把他绑起来,跟小
五子承认自己是第一次点真人的穴,没想到真把你定住了,但一会儿穴位冲开了,你又要跑了。她先绑脚,缠了十几圈,却打了个一拽就开的蝴蝶结。然后她问,双手放在前面绑,还是背过去绑?小五子说放前面吧,结婚而已,何必上大刑似的。

“你看,你还是愿意说话的。”

她听他的,双手缠前面绑个蝴蝶结,让他等着,背可背不动你,出去晃了一圈,将厨房喂狮子的小推车推进来,把他抱进车里,告诉他:“我们快走吧,没准已经晚了。”

傍晚从地窖里出来,折腾到现在已是深夜了。这个点儿结婚,小五子甚至怀疑,你们家小姐是不是早就死了,拉着我办冥婚陪葬。小玉忙捂住他的嘴,求他不要瞎说,提醒他别忘了你是谁,那么多人找你寻仇,白天搞得大张旗鼓,什么人都来,我们倒是不怕,只是这婚也结不利索啊。

小五子点头称是,然后发现小玉没骗他,穴位果然被冲开,他先解开双手,在推车里前倾一下,把脚上的蝴蝶结拽开。随时可以跳下车的时候,他又不想跑了。这姑娘脚快,轻功好,反正跑出去也要点个穴再抓回来,就在车里跟坐轿子一样也挺好。

后来还真坐轿子了,小玉推了一刻钟,终于快到办婚礼的大厅,远处的一个轿子停在门口,他看见盖着盖头的新娘下脚,看不到脸,从身形看还真不是颐指气使的胖小姐。

新娘被搀进门后,小玉去招呼脚夫们过来,新郎倌也要坐轿子,别偷懒,一直抬到大厅里去。上轿之前,小五子看了一眼大厅上的牌匾,左右两个灯笼将三个金字照得反金光。小五子心头一紧,问小玉:“你们帮主姓乔吧?”

小玉点头。小五子说,你们家小姐是比你漂亮,别说比你,比文思清、吴思若和苏子瑶加起来都要好看,因为她是乔文君啊。他不想上轿了,直接走过去,朝“狮吼帮”三个大字走过去。他知道,应该娶她的,三年一去杳无音讯,怎么都要给她一个交代。
\newline

{\centering\subsection{6}}

确实没请外人,齐师叔做主持,进来一个报一个,百十来人都是狮吼帮的弟子,连唱喏都是本门弟子,狮吼功练到多少多少级的张贤秀。大厅的正位摆着两张太
师椅,乔帮主抱着外孙坐右手边,那是小五子的儿子吧,孩子没生下来,他就消失不见,弄得乔帮主无处辩解,怪不得他现在看小五子的眼神跟要冒火似的。左手边椅子上放个牌位,写着“狮吼帮帮主夫人乔李氏”。阴阳相隔,这算岳父岳母了。认真想起来,自己父母应该也在场,十有八九不在人世了,立两个牌位列在高堂,可是上面写什么呢?张王氏,李赵氏?小五子连自己姓什么都不知道。

狮吼帮是江湖大帮,这么多弟子,一路报下来要小半个时辰。小五子看新娘,她坐在椅子上,脸在盖头下面,低头看着脚尖。那就这样吧,别去想文思清,也别惦记吴思若,苏子瑶也对不住了,孩子都两岁多了,别再问我到底喜欢谁这种话了。

一一报完名之后,齐师叔把大门关上,意味着宾客到齐,再来的算不速之客。但还不能马上拜堂,走了好几年,他们得编个故事把狮吼帮、乔帮主的面子找回来。乔师叔拉来一个老太婆,说婚姻大事自古就两条,头一条是父母之命,这第二条,我们来听听媒妁之言。

媒婆磕磕绊绊,背稿子似的,半文半白把故事讲完。她说各位都知道,乔帮主只有一个女儿,家无男丁,几年前乔夫人还在世,托她为女儿找个如意郎君,能当半个儿子用的女婿,自此撮合了他们俩,郎有情,女有意,新婚在即,不巧夫人身患重疾,卧病在床,郎中诊断,唯有苦寒之地三千年的高丽参才能救活乔夫人,我们这位昆仑公子二话不说,当晚就前往东北,走遍长白山寻找高丽参,这一走就是三年多,直到今日午后,昆仑公子终于带回来了这根三千年奇参,可惜乔夫人一病不治,早已仙逝,没能赶上这大喜的日子。

乔帮主频频点头,其余弟子起哄一样带头喝彩。不知道为什么,故事编得越离谱,小五子就越觉得乔帮主这几年过得不容易。几个孩子抱进来一个老树根,说是三千年的高丽参。这就有点过了,还好没怎么做文章,装模作样走个过场。场上全都是狮吼帮的人,百十来个人关起门来自欺欺人,之前要有多羞耻。

齐师叔掐着时间,说时辰已到,两位新人开始拜堂。小五子和乔文君并排站一起,背对乔帮主,面朝紧闭的大门。齐师叔先喊一拜天地,小五子腰弯下去了,乔文君说再等等,宾客都到齐了吗?大家互相看着,该来的都来了,不该来的都是要找昆仑公子寻仇的。齐师叔清清嗓子,又喊遍一拜天地。这时大门突然打开,门外没有人,一支箭从外面飞进来,连同上面的红条幅扎
在房梁上。条幅上写着“西北六公子恭祝乔姑娘大婚”,没他小五子什么事。乔帮主一跃上去,摘下房梁上的箭,对着大门口说:“六公子前来恭喜乔某,何不进来喝杯喜酒?”

这是在亮狮吼功,他声音硬邦邦的,仿佛可以用锤子敲。帮里的弟子都练过,只有小五子头晕要倒,小玉赶紧拖椅子过来,坐下来时还看见那句话的回音,像被敲碎一样,每个字都在大厅里飘来荡去。那些字越来越轻,慢慢落到地上。远处传来六公子的笑声,他说乔帮主无意邀请,我也就不便叨扰,昆仑公子好福气!

这是在说我吗,小五子撑着站起来,六公子的笑声越来越远,盖头下面的乔文君说话了:“既然没有人来,我们就开始吧。”

索性把大门全敞开,两个人对着门外一拜天地,转回来冲着乔帮主和牌位二拜高堂,然后都转半个身,要夫妻对拜。弯腰下去,小五子一阵阵想哭。他要喝酒,把自己喝得酩酊大醉。开始大家还放不开,看小五子一杯一杯灌到肚里,觉得昆仑公子果然豪气,纷纷向他敬酒。小玉提醒他少喝一点,西北六公子这一去,不知道还会带什么人回来。那就让他们来吧,他不怕寻仇,不怕折磨,管他一刀捅死还是千百刀地去剐他,这些他都不怕了,可总还剩点什么让他心生恐惧,他怕的不是恨,怕的是爱啊。

他怕自己不爱却要厮守,他怕自己深爱却要离别。那就这样吧,别等我了,文思清;很高兴认识你,吴思若;而苏子瑶呢,不管我之前与你如何,在这里跟你说声对不起。他抱着酒坛摇摇晃晃,从一桌走到另一桌,每一个朋友他都去抱人家肩膀,希望对方是仇家派来的,掏出匕首一刀捅进他心里。

他失望了,只是酒越喝越多,视线越来越花,最后瘫坐在地上看人们相互碰杯。乔帮主把外孙带过来,说这孩子暂时随他姓乔,叫乔彬。乔帮主要孩子喊他一声“爹爹”,叫出来的那一刻,小五子号啕大哭,他哭着要去抱孩子,吓得孩子直往外公怀里钻。乔帮主说他喝多了,要齐师叔扶他回房。

背出大厅,小五子还死攥着酒坛不松手。他问怎么搞的,他怎么就是孩子他爹了,怎么之前就没有拜过堂。齐师叔冷笑,放他下来让他自己走。他看着小五子每迈出两步就往地上摔一次。他说,怎么做的你自己不清楚,几年前连乔姑娘一起,你抓了五六个狮吼帮的人,别人帮派被你放出去,都要少胳膊掉腿的,唯有我们狮
吼帮,被你关了三天,毫发无损地出了昆仑山庄,以为你昆仑公子要跟我们交朋友,可真是交啊,三天里你对乔姑娘干了什么,让她出来之后给乔帮主生了个外孙?

齐师叔还在苦笑,黑暗里一丝苍凉。小五子又一次站起来,请齐师叔早点休息,自己坚持往红灯笼的房子走。从未有过的唾弃,过去到底是个什么人,没有了记忆,人会变好吗?他跟踉跄跄走到房门前,手指着头顶的灯笼数了几遍,每次都不一样,四个,八个,六个,七个,他双手向前一推,进了房门。

到了洞房他清醒一些了,揉揉眼睛,看见乔文君在床边等着她。还剩最后一个程序,以新郎的名义去掀她盖头。站起来的时候双腿打颤,他深吸一口气,一步步朝床边走去。乔文君让他先别过来。他说我知道。但他什么都不知道,只是站不稳,双腿打绊扑到床边。乔文君再一次警告他,她说但凡你碰我一下,我一定杀了你。

“我知道,”他在床边站起来,“我不碰你。”

乔文君自己拿掉盖头,盯着小五子。尽管不喜欢,但眼前的样子还是让他有些痴了。她示意他后退,再退一步,没关系,别动就站稳了。然后她皱眉看着他问:“你知道什么?”

“我知道我禽兽不如,对你做了很多恶行,这场婚事就是给狮吼帮一个交代,没关系,我做了那么多错事,你怎样都行,怎么解恨怎么来,你杀了我吧。”

“你没做错什么。”

“你是说当时你是自愿的?我不记得了,当时什么样?”

乔文君笑了,让他别抖,坐下来再说,要喝杯茶解解酒吗。小五子摇头,坐在椅子边上等她说话。她把耳环镯子摘下来,一件件放到首饰盒里,目光似乎回避着他说:“没什么当时,我跟你几乎不认识,孩子当然不是你的,只是彬彬的父亲我不能讲,你名头又大,当时又消失了,他们逼问我,到底是谁的孩子,我自然而然就想到你了。”
\newline

{\centering\subsection{7}}

乔文君答应他,一旦有机会,肯定帮他逃出去。说这话时是新婚的第七天,一大早丫鬟们就把点心送到房间里,那时乔文君刚起床,把椅子上熬一宿的小五子叫醒,让他到床上继续睡。刚换地方小五子一时睡不着,
休息不好胃也烧得慌,侧卧在床看乔文君吃桂花糕,听她承诺道:“你放心,我下次再出沉狮谷,就想办法把你带出去。”

小五子眨眨眼睛不说话,他难受好几天了,天冷得地上没法睡,每晚窝在椅子上,一星期下来浑身就像散了架。这六天他没出过门,天天都是饭菜送进来,在房里吃。首先乔文君不相信他,怕他跟乔帮主把实情都讲出来,再就是小五子自己也没想好该怎么办,狮吼帮的人看他的眼神都不对,都当他是淫魔,正是玷污了乔姑娘,才做上狮吼帮的女婿,要是澄清呢,跟乔帮主告状,说你那外孙不是我的,我昆仑公子跟你们家没关系,说这些能怎样,乔帮主会拍拍他肩膀,说委屈你了,然后把他放了吗?不会的,既然你跟我女儿没那个,就在这儿杀了你吧,不是我女婿,你就是武林公敌啊。

逃出去是最好的办法,跟乔姑娘出去办事,逮机会就往北跑回田独。他问她哪天再出去。乔文君说不上来,有事才能出沉狮谷,没事她爹不让她随便往外跑。大概何时呢?乔文君还是不知道。那就反着问,你上一次出去是什么时候?

“昆仑山庄的武林大会,半年前。”

“我知道,再上一次呢?”

“也是昆仑山庄,三四年前。”

小五子倒抽一口气,继续问:“再再上一次呢?”

乔文君还在回味那一次的出门远行,她说那次出去时间可长了,差不多小一年,去了好多地方,光你那昆仑山庄我就去过两回,头一次是被你抓过去的,后来跟她爹爹汇合,先去了黄山,又去了少林寺,还去了京城,八月十五那天去昆仑山庄,结果你却出事了。

“不用讲那次了,”小五子打断她,“我问你,上上一次是哪年?”

乔文君被问住了,仔细想了想,回答他:“我就出去过那两次。”

“你活了二十多年,只出去过两回,然后你告诉我,下一次出去就帮我跑?”

是啊,乔文君也捋清了这道理,前半生平均十年出去一次,以后相夫教子,也许二十年都不会再出沉狮谷了。不能让小五子在这椅子上睡下半辈子,当然自己也不会真的嫁给他。心情一下子很糟糕,她让小五子先睡,别和她说话,她要好好想想。

想了一上午,一直到中午饭端进来,她告诉半睡半醒的小五子,我们不会永远困在这里的,用不了几年他
就一定来沉狮谷,把我和彬彬带走,他答应过我,事情一办完,就会把我们娘俩接走的。小五子翻个身,背对着她问,他是谁?问完他就知道了,当然是孩子父亲。火炉里发出噼里啪啦的烧柴声,冷风顶着窗缝往里挤,感觉又要下雪了。他说刚睡着的时候想起一个事,你说你这辈子就出去过两回,上回就不说了,第一回出去,你就急急忙忙跟刚认识的一个男人生孩子?

乔文君不说话。小五子看着火炉里被吹乱的火苗,他说这是个什么样的男人,有了孩子不敢认,还能让你死心塌地地等?乔文君还是不吭声。小五子车轱辘话问了好几遍,终于把乔文君问急了,甩脸说:“我就是个贱种,是荡妇,满意了吧?”

就这么个屋檐,一旦吵架两个人都不舒服。小五子又躺一会儿,确定睡不着,他披件外套坐在门外的台阶上看风雪。没多久乔文君也出来了,拿了两个小马扎,一人一个并排看着白茫茫的世界。小五子张了几次嘴,最后讲出一句真心话:“你说孩子是我的,置我于此,我不怪你,因为你以为我死了嘛,反正我名声也不好,本来就要下十八层地狱,多个淫魔的称号,也下不到十九层了。你跟谁好,也不关我的事,我只是觉得你很好,我希望你命也能好一点。”

“你当时就是这么说的,一模一样,”乔文君挑起一个树枝,在雪上胡乱画着,“你说我很好,希望我过得好一点。”

小五子扭头看着她:“真的假的?”

“真的,不是我栽赃你,是你说的。你说,日后乔帮主要是逼问你孩子是谁的,就说是我昆仑公子的好了。”小五子忽然有些激动,站起来在雪地里走了几步。乔文君问他:“是不是都想起来了?”

“没有,一点没想起来,我高兴的是,不管我是昆仑公子,还是小五子,我这个人没有变。”

雪越下越大,他在雪地里走了一圈,回来时裤腿都是硬的,神清气爽,他说我大概知道孩子父亲是谁了,我当时知道吗?乔文君点点头。这让小五子有点不明白了,我没变,难道他变了吗?他要再确认一次:“是恭祝你大婚的那个吗?”

乔文君笑了:“单祝我一个人,还说什么昆仑公子好福气,酸溜溜的。”

“为什么是他呢?”

“你们俩当时有个计划,如果不是你出事,中了断魂掌,我早就嫁给他了。你们关系很好的,你仔细想想,以
他的箭法,在昆仑山庄,怎么可能一次又一次地射偏,他根本就不想杀你。”
\newline

{\centering\subsection{8}}

万一十年出不了沉狮谷,乔文君可以一直聊六公子,但他们在一起的时间加起来才三五天。有时小五子怀疑,这个六公子是假的,只有那三五天是真的,可就是一开头,后面所有的六公子,都是她这三年的思念里幻想虚构出来的。他不想听他们如何相识相爱,私定终身,这一块没假,至少在她一次又一次的描述中,已经修订得天衣无缝。他要听乔文君讲别的,关于六公子与他昆仑公子的,找找里面的破绽,进而判断她是在骗人还是被骗。

可惜除了爱情,乔文君对六公子的了解少得可怜。六公子告诉她要办一件大事,事情办成就来接他们娘俩。为此还留了定情信物给她,信物之简陋,乔文君自己都无法修饰,她拿出一块羊皮,递给小五子说:“还千叮万嘱不能丢。”

“这个是不能丢。”

小五子掏出自己身上的九宫图给她看,乔文君愣住了,对比两张羊皮,原来不是随便从一只羊身上扒下来的,原来真的有意义。

两张平放在桌上,刚好可以严丝合缝。乔文君找出针线,将两张缝一起,说本来就该是一张,合起来送给你吧。

小五子不要。乔文君说,这事连累了你,就当是赔偿,有一天六公子把大事办成来接我,总不会要我还定情信物吧。小五子笑了,把两张九宫图收好,问乔文君,六公子到底要办什么事。他问了几次,乔文君才承认,讲不出口,她也不相信六公子口中的大事。

“他要当太子,继承皇位。”她说,“我是相信他,但不能因为我信他,他就这么骗我!”

乔文君不信的事,小五子反而要认真想想。打从他在田独杀猪卖肉的时候,门口贴告示的巡捕就说过,昆仑公子罪大恶极的事情还不是残害武林,官府追缉他,是因为他从宫里把太子劫走了。后来知道自己是昆仑公子,他也曾想过,太子被我劫哪去了?钱老板是太监,好像叫常公公,自然和太子失踪有关系。除此之外,他再就不认识从宫里出来的了。假如六公子真是太子呢,不对,他帮三王爷做事,卧薪尝胆伺机篡位,可是三王爷
瞎吗,自己侄子不认识?所以说,六公子所谓的大事,十有八九是在哄她,真的只有一种可能,辅佐三王爷登基,然后给他当干儿子,做太子。

乔文君说,这件大事是要小五子帮他一起做的。小五子问,帮他做什么?乔文君也不知道。那知道什么呢,六公子本名刘世钦,之所以大家都叫他西北六公子,是因为他父亲刘冠英当年一手撑起了西北镖局,虽然镖局在山西大同,靠着刘冠英的一身功夫和豪爽性格,交了不少朋友,黄河以北的货物往来,基本上都要跟西北镖局打个招呼。六公子自然行六,刘冠英五十岁才有的他。上面五个哥哥,最小的都要比他大十多岁,大哥要比他大三十岁。刘老爷子做镖局的,头五个儿子练的都是外家功夫,唯有六公子练的是弓箭,是江西五老峰王氏的门下弟子。

她问小五子:“你知道他五个哥哥是怎么死的吧?““我听说,是被我杀的。”

乔文君笑道:“真是的,有机会我也想中一次断魂掌,自己干过的事,还要听别人说。”笑过之后,她认真说:“其实不是你杀的。”

小五子也没特别惊讶,他清楚自己几斤几两,这辈子杀猪还行,杀人的本事真没有,何况还是一次杀五个,除非他们是四条腿,跟个畜生一样跑过来,自己手里刚好有把杀猪刀,手腕一拉剖膛破腹。在田独老虎都杀过,人嘛,就一个何员外,还是于心不忍才下的手。

乔文君说:“刘冠英是你杀的,西北镖局的掌门人,七十多岁的老爷子,不知怎么就得罪了你昆仑公子,那是早先的事情,你带着人血洗了大同,当时你还要杀六公子,可惜他不在,跟着三王爷去了京城。五个哥哥当时都在场,打不过你昆仑公子,你也没动他们,就说想报仇来昆仑山庄找我。他们知道打不过你,给老爷子办过丧事,都投到了三王爷门下。”

小五子越听越不对劲,打断她说:“先不管我能不能做出这种事,只是我没有武功,我确定,换多少年前,我都没本事杀人。”

“你是昆仑公子,但你不代表昆仑公子,你先听我讲完,我一会儿再跟你解释。”乔文君说,“你一直没杀掉六公子,他那五个哥哥也没机会找你报仇,就这么相安无事地过了一两年。后来你改主意了,不但不杀六公子,还想跟他合作什么事。潮阁寺的事你应该听说过,京郊的一座破庙。”

“我知道,我听人说,我是在那儿杀死了他五个哥
哥。”

“对,那天你中了断魂掌,在潮阁寺落脚,不巧被六公子和他五个哥哥围堵在里面,你寡不敌众,我后来听六公子说,唯一能帮你的常公公还被你绑起来了。”

“谁?哦,哑巴钱老板,不知他人在哪里,出去之后我要找他聊聊。”小五子说,“后来他们是怎么死的?”

乔文君让他自己想,兄弟六个堵到庙里杀他,五个死掉了,剩下一个所谓落荒而逃的就是六公子,你想,你没本事杀他们,那是谁杀的他们?

“六公子杀了他亲哥哥?”

“不杀他们,那该死的就是你。”

为什么,问了乔文君也不知道。小五子提着水壶去烧水泡茶,一直盯着壶盖不说话。水开以后他拎着热水回来,往茶杯里放一把碧潭飘雪,倒水时他示意她要吗,乔文君摇头说谢谢。她去茶几上拿一块玫瑰糕吃起来。等茶的时候他问:“昆仑公子是谁,我又是谁?”

“你是谁我不知道,昆仑是谁我也不知道,因为你们一直在隐藏身份。”乔文君咬一口玫瑰糕,细细嚼完才继续说话,“我被你们抓去过一回,和我三个师哥,从咸阳带到昆仑山庄,和很多门派一样,被你软禁了几天。这三年在沉狮谷我就一直在想,你不是昆仑公子,你现在不会武功,那时更不会,昆仑公子不是你,是很多个武林高手,以你昆仑公子的名义去执行任务,不知道是不是你网罗的,但昆仑公子不是一个人,他是个组织。”

“百花谷?” “什么?”

“没事,后来这些人去哪儿了?”

“别的我什么都不知道了。”

嗯,茶叶渐渐落下去,小五子喝下第一口茶,碧潭飘雪,怎么看也不像窗外的飘雪。不能再等乔文君了,他要自己想办法出去。先不回田独,搞清楚自己是谁。从田独出来一直就没个目标,被人追,他跑,被人打,他躲,每一天都随波逐流。但这不是小五子啊,没本事不代表没骨气,田独的赌场被人出千羞辱,他尚且知道拿菜刀报仇,现在跟过街老鼠似的,除了躲山洞就是躲树洞。他不想这么过了,迎上去,把那些伤害过他的人们,一茬一茬地找回来。
\newline

{\centering\subsection{9}}

十一月份,乔帮主找他聊了一次。他希望小五子学
武,过去的事不谈,往后咱们都是一家人了,他盼昆仑公子能够踏上正轨,以后带着狮吼帮,做些让武林中人称赞的事情。

这话挺明显,意思是我死后,狮吼帮就是你的。他要小五子拜师,教他狮吼功,从气运丹田练起。小五子委婉谢绝。肯定不是真教他功夫,他快三十了,傻子都知道,练什么都来不及了,乔帮主要的是师徒名分,现在是我女婿,还不便说你什么,等做了我徒弟,以后处处可要管着你了。

乔帮主让他再考虑一下,这可是狮吼功,一般弟子练不到的,他们也就是练练拳脚功夫,传男不传女,乔文君都没有练过。小五子心想那就好,没练她脾气都不怎么样,要是练了大嗓门,就成纯种母老虎了。小五子想说,正因为是狮吼功,更不用考虑了。他偶尔见过他们练功,扎起马步,掌心向上,有多大冤屈似的,冲着山谷又喊又叫。尤其师兄弟对练,面对面就是吵架,喊一两个时辰不带喝水的,看谁先把对方吵倒。

来不及拜师学艺,他在计划逃跑,昆仑公子不是一个组织吗,也不见谁来救他。有天晚上他把想法跟乔文君说了,他说等不了你下次出门,天天坐着睡,我也睡够了,你告诉我谁守大门,出门怎么走,出去混好了,我带着六公子回来看你和儿子。

“你出不去的,”乔文君劝他,“你别做傻事。”

“你别管我傻不傻,告诉我怎么出去就行。”

“你出不去!我怎么告诉你?我都没出去过!”小五子冷笑,睁眼说瞎话:“你明明出去过两回。”

“那是我爹带我的,我自己出不去。”

好吧,随便你。小五子决定靠自己,每天天不亮他就出门,贴着高墙在园子里瞎溜达,一直到晚上才回来。连走三天他把园子摸得门清,现在闭着眼睛都能找到厨房假山后花园,连之前关他的地窖都找到了,可是大门在哪里?高墙被他溜十几圈了,也只是围成一圈的墙。难道真出不去?不可能,没有人会造一圈实心墙,把自己封死的。

他去跟小玉打听,不好直接问,声东击西地绕了一大圈,你们平常怎么买菜买衣服啊,客人怎么进沉狮谷啊,你知道正常的生活应该有门这个东西吗?

“我们有啊。”小玉说。

“不是每个房间里的门,是连接你们和外面世界的大铁门。”

“有啊,两扇大铁门,打开就出去了。”

“你没开玩笑?”

“早不跟你开玩笑了,知道你开不起玩笑,我干吗自讨没趣?”

那就好,大铁门,总能找得到。这天他又捋着墙走了一圈,墙,墙,墙,从墙一直走到墙。那铁门在哪里呢?回去的时候他几乎死心了,晚饭饱饱吃了一顿,拉起两张椅子就开始睡觉。睡到他猛地醒了过来,他知道哪里有大铁门了。对着黑暗他发了一会儿呆,他在想把铁门设在那里的可能性,没准真是这样。他起身到床前,月光下最后看一眼熟睡的乔姑娘,推门走了出去。

他先去厨房,挑两把合手的刀揣在怀里,然后从后门穿出去,踏上两侧养狮子的那条路。他摸着铁栏杆,这些不是铁笼,是铁门,小玉所说的两扇大铁门,打开就出去了。他往上看看,因为铁门用狮子把门,铁门不是很高。他抓着栏杆爬上去,两只脚跨到铁门外,双手抓着铁栏,只要跳下来,就是逃出沉狮谷了。

不知听见声,还是闻着味了,两只睡了的狮子正努力醒过来,抻了抻前腿,走到栏杆下面等着他。他们也不叫,仰头张着大嘴,没一点恐惧,自信满满地等他跳下,送到嘴里来。

小五子两腿挂住铁栏,腾出两只手去握刀。闪出刀光的那一刻,狮子开始低吼了。真奇怪,狮子一害怕,小五子也害怕了。他把刀握紧,低声讲三遍,我小五子从来不怕四条腿的。越讲腿越抖,第三遍讲到一半,小五子鞋底一打滑,掉下去了。

两只狮子各自退半步,小五子脸朝下摔在雪地里,还好刀没脱手。两只狮子试探着向他拢过来。小五子吞下一口雪,咬牙站起来,双手举刀指着左右两侧的狮子。雪地里三个生物十条腿,大家都是不进不退,小五子突然从两只狮子中间穿出去,背对着铁门向前跑几步。狮子一前一后追上来,这就是他要的,听声音都知道他们在什么位置,他一个转身,朝扑过来的狮子一刀下去。

从脖子往下,前一只狮子直接开膛破肚,溅了小五子一脸的血,血盆大嘴还没有合上,就倒在了大雪里。第二只狮子一步步往后退,小五子转身跑几步,狮子就保持着距离跟在后面,不敢冒然前扑,但也不放过小五子。换平常还好,边走边等机会,可是现在这么大的雪,一脚伸雪里,还要拔出另一只脚才能迈出去,怕是走不出二里路就没什么体力了。

要速战速决,小五子转身面朝着它后退,右手的刀
贴着腰边,左手举刀在头顶,退到第三步,他保持住姿势仰躺在雪地上装死。他知道狮子会过来咬他喉咙,确保他已死。如果是从他身上踏过来,他就挑起腰旁右手的刀剖它的腹,如果绕到头顶,他就挑起左手刀,去割他喉咙。

狮子也不再吼叫,悄无声息地观察,四周静得一塌糊涂。虽然他还睁着眼,可只能看见头顶的下雪天。下雪把夜空染得一片血红,雪片落进他眼睛里,融成泪水流到眼角,他眨眨眼睛,听不到狮子的脚步声,无论从哪里过来,总该有踩在雪里的咯吱声。时间慢得可以在心里数数,他听见有人在上面喊:“小五子,小心头顶!”

一个黑影从他脑袋上扑过来,他左手扬起,一刀插进它的腹股沟,却怎么也提拉不起来。狮子一口他咬住右臂,腹股沟的刀拔不出来,右臂动不了,小五子左手去换过右手的刀,向它脖子上捅去。狮子吃痛松开他右臂,小五子侧身打滚,翻下斜坡。狮子没有跟上,窝坐在雪里喘着粗气。

左手还有一把刀,右臂咬得都见着骨头了。小五子撑住站起来,看一眼铁门,已经有七八个人站在门里面,刚才说话的是乔文君,乔帮主在她旁边,再旁边有小玉和齐师叔。没时间跟他们说话,小五子爬上坡,跟残喘的狮子对视。它身上中了两刀,一刀是脖子上的,看起来是皮毛之伤,腹股沟那刀狠些,右后腿几乎掉了一半。它三条腿站起来,后面嵌着刀的那条腿几乎是悬在半空,一声声低吼,不知是拼命还是哀求。

这回是小五子要出击了,他朝狮子扑过去,它伸出两只前爪迎击。这是虚招,在它背上捅刀没用,他一个急停躲到它身子下,刀插进它腹部,手腕使劲往上挑,一直到脖子,把它身体彻底剖开,那些心肝肺胃肠肚,泄洪一般呼在他脸上,喘气都是血腥潮湿的味道。

他等狮子死透,从它身下钻出来。双腿发软,在雪地上跪了一会儿,抓两把雪擦擦脸,起身看着铁门里的人。两只狮子都被他宰了,可这时铁门打开,跑出一两步被逮回去,就真没意思了。

他左手拿刀,右手抱着左手,说乔帮主、乔姑娘,诸位后会有期。看起来没人要抓他,乔帮主冲他点点头,问他跟百花谷什么关系,你千岁刀练得不错。小五子愣了一下,说就是杀猪的功夫,哪来的千岁刀。乔帮主笑笑,转身走了。乔文君说保重,跟着她爹离开。小玉还想跟他开玩笑,她说:“昆仑公子,早点回来,你这一身的血,我去烧水给你洗澡。”

小五子心里数三个数,一,二,三,转身就跑。自由以后他什么都不怕了,管他前方还有几只狮子,哪怕是鬣狗狼群他也不怕了。雪地不好走,一脚深一脚浅,踉踉跄跄行动缓慢,管他多慢呢,每迈出一步,至少还是向前走。

两侧都是山崖,中间一条小路够他向前跑的。已经是清晨,沉狮谷不像田独那样中午天才亮,但起码还要再跑一个时辰才能看见日出。速度虽慢,但他大步往前。跑步时他想出了沉狮谷先去哪里,文思清,吴思若,苏子瑶,钱老板,南海真人,三个女人先不考虑,钱老板肯定什么都知道,找他问清楚,然后找南海真人去报那一掌断魂掌。

前面路逐渐变宽,他忽然想起,杀猪这本事就是钱老板教的,吊起来不行,要把猪放出来,跑起来杀,这就是千岁刀啊。钱老板也好,常公公也好,他是百花谷的人了。既然是武功,肯定是要冲人来的,上次杀老虎,这次宰狮子,什么时候他有胆量对人下手呢?

跨过小溪他停下来,喝一口水,抄起刀继续跑。跑跑自己还乐出声来,谁说我什么都不会,以后人送外号千岁刀小五哥。太阳就要上来了,已经有光从崖顶的林子里透出来。前面又变窄了,估计绕过这两座山,就是一条阳光大道了。

他提一口气,告诉自己跑三千步再停。转了个弯有人在前面等他,越跑越近他看到是小玉。一定有条捷径,令小玉跑到他前面。他先放慢脚步,离小玉几百尺的地方加速冲起来,从她身边超过。小玉在后面喊他:“昆仑公子,水已经烧好啦,锻炼得差不多,就早点回去休息吧。”

小五子脚下不敢停,说你先回去,我随后就到。他猛冲两里地,见小玉没追上来,他跑得更快了。两个尸体摊在前面的雪地上,他放慢脚步走过去,靠近尸体他几近崩溃,原地转了一圈,明白自己再也跑不出去了。就是被他杀死的那两只狮子,沉狮谷,这是个山谷,是个圆圈啊。他想放声哭出来,想出山谷光杀狮子不
行,还要会轻功,上得了悬崖。那也不管了,既然出来了,就再往前跑吧,哪怕跑死在外面,也不回去洗个安逸的热水澡。

跑吧,小五子,打从出田独,你就一直在跑,这次让你在沉狮谷跑个够。没有希望,他反倒跑得更畅快了。他把刀收起来,甩着胳膊跑在阳光下,还是一样的路,前面变宽,再往前是小溪,他水也不喝了,继续跑,再前方路面变窄,继续往前两侧的悬崖高至上千尺,小玉还在原地等着他。

“我不回去,你放心,我肯定不回去!”他冲小玉大吼,加速超过她。

已经跑了两个时辰,他清楚下一个时辰的路线,变宽,小溪,变窄,小玉,再回到这两侧千尺的悬崖。他找有积雪的地方踩,要每一脚都是脚印,每一脚从雪里拔出来,每一脚踏进新的雪面上。有一脚下去他踩进雪下面的绳圈里,绳子迅速箍紧他的右脚腕,整根绳索向上提,小五子嘴里喊着我不回去,右脚套在绳圈里倒挂在空中,像一桶井水一路上升,一直升到上千尺的悬崖,吊在铁架旁边。

四周没有人,面前一个小木屋。冷静下来小五子明白了,这不是狮吼帮在抓他,这是山顶猎人自制的陷阱。木门打开,有人从小木屋里出来,穿了一身的野兽皮毛,头顶一个狼头的帽子,看样子要在山顶度过这个冬天,见到绳上挂的是人,她也很意外,慢慢往这边探。貌似是个女猎人,这么大风雪,还是倒着看,小五子也看不清楚。走近时她问了一句:“少谷主?”

说话间起风了,小五子脑袋朝下,在绳子上摇摇晃晃,偶尔刚要看清楚,又被风吹了半个圈。他看不到,但知道都有谁叫过他少谷主,想仰头望去,却是深渊,他听见女猎人在后面泣不成声。她说天啊,本以为要再等半年,到春天才有机会救你。她哭着去抱住他,摸他倒着的脸问:“你是怎么跑出来的啊?”

\newpage