\section{玖}

{\centering\subsection{1}}

苏子瑶知道那一黑一白两个怪物有问题,陆路转水路地跟了一路,就是冲着小五子来的,但不知道是吉是凶,明明要杀他,却把前面的绊脚石全给清除了。过了嘉峪关,他们终于动手了,两个人把他掠上马车,日行八百里向北狂奔。一直到那个小旅馆,他们请来了狮吼帮齐师叔,苏子瑶才明白,这事跟狮吼帮有关系,乔帮主要小五子。

他们要小五子做什么呢?苏子瑶跟到悬崖顶就下不去了。看着大门在前方合上,齐师叔的马车仿佛坠落一般向下俯冲,一直等到天黑,谷里依稀闪着灯笼,鞭炮和唢呐声时不时传到谷顶来,该不会是搞什么活捉小五子的庆典吧。晚些时分六公子也来了,那么高的功夫,一身白衣踩在雪上都不留脚印的。她看到他一个人进去,又看到他一个人出来,连他也没能抢出小五子,那留给苏子瑶的只能是等待。

她掏出一个树洞,垫上枯叶睡在里面。她等了一天,两天,三天,一个星期过去她该考虑现实问题。吃的还好,奇形怪状的树上还有一些奇怪的果子,有毒没毒总要吃了才知道,睡一觉还活着,那这种果子可以继续吃。到十一
月这种果子都没得吃了,树上光秃秃的,北方已经彻底进入冬天。得过且过,没有了果子,还可以挖草根,直到一夜狂风大作,她在树里冻得瑟瑟发抖,四周隐隐发出咔嚓咔嚓的断裂声。快天亮时轰然一声巨响,树干拦腰折断,雪片直接打在脸上。苏子瑶从敞顶的树洞里站起来,她知道想尽办法也要造一间屋子了。

这很奇怪的,她跟踪一个人,保护他,那个人被抓到谷里,她居然要造个屋子守在门口。头几天她还边砍树边问自己,真的要在这里过冬吗?回百花谷通报谷主,可能是更好的办法。造一个屋子要二十七根木梁,她多砍三根作为备用。三十根木头摊在悬崖时,她也不再犹豫,无论如何都要把屋子造好,大雪封山,她回不去了。

一间屋子四面墙,一个顶,天气太冷,她把时间倒过来,每天正午睡觉,夜里拼命盖屋子,才不至于让自己冻死。十一月中旬屋子盖好了,屋子里点上火堆,她暖洋洋地睡了一天一夜。

猎人生活终于开始了,没有弓箭,她在山上绕了两个时辰看到一头孤狼。她看狼不怕她,就想起小时候长辈们告诫的,不要和狼对视,狼向你走过来时,站着别动,不要后退闪躲,要装作你一点都不怕它。这些她都反着来,她看着狼的眼睛往后退,那只孤狼向她走几步,盯着她,忽然转身逃了起来。这就不好办了,路面平坦也许能追上它,可此时雪有小腿那么深,前脚踩进去,后脚要使力才能拔出来。她提剑追狼,踩着雪,绕开挡在面前的每一棵树,踏过结成冰的河流,追了两个时辰,终于把它逼到悬崖。孤狼身后是上百米的悬崖,她求它不要跳崖自杀,不敢再往前,向后退几步。狼在崖边一动不动,一人一狼,对峙到天
黑。到最后苏子瑶也耗不起了,索性拔剑冲上去。孤狼向后退两步,后腿险些踩空,回头望一眼深渊,露出獠牙朝苏子瑶扑过来。

禽兽是不会自杀的,夜里苏子瑶吃着烤狼肉,想明白了这件事,实力再悬殊,也存在不是你死就是我活的可能性,为什么偏偏是我死?她用剑将狼身一片片地切下,扔到火里烤熟。还有一个显而易见的小事,苏子瑶现在才想到,你要吃狼,狼本来也是要吃你的。

一顿吃了一小半,剩下的一大半她扔在雪地上,提剑守在不远处。夜里果然还有只狼闻着味来了。从这只狼吃下第一口开始算,苏子瑶趴在雪坡上,数着一二三,跳起来将狼头斩落。

小半个月斩了七八只狼,狼皮剥下来,衣帽有了,被褥也有了,将狼牙嵌木棒上,连新的武器都有了。十二月有人从沉狮谷出来,带来的消息是昆仑公子娶了乔文君,成了乔帮主的女婿。那个人姓刘,谷里负责食物储备的,从始至终都没有看出苏子瑶是女人。也是她故事编得好,说自己家有妻小,得罪了朝廷,满门抄斩,来到北方只想做一名猎户。那个人带酒来,苏子瑶请他吃狼肉,三五杯下去,得知两人早就有事,乔文君的儿子都是昆仑公子的。苏子瑶醉得更快了,再喝上两杯,就倒在狼皮上不省人事。

醒来时一些事反倒想通了,好像小五子说过,人是不会变的。她喜欢昆仑公子,不管他做了什么事,昆仑公子还是她的昆仑公子。这么想心情反而更好了,她要弄更多的狼皮,不止是狼皮,还有狮子皮、老虎皮,再打一把椅子,套上虎皮坐上去,每天威风一百遍。就算只做一名猎人,这个冬天也要把生活搞起来。

可是打不到狼了,血肉残骸放到雪地上,几天都没狼理会。也许要新鲜活的。她把自己当诱饵,到处去躺,看着白蒙蒙的冬日阳光,躺到手脚冻僵了,也没一个四腿禽兽过来闻闻她。

又要满山寻找猎物,她提着剑,像是找野兽化缘似的,学各种兽叫,低头找脚印。连出三天空手而归,天空一只秃鹰嘶鸣而过,她仰头看着,再看看周遭漫山的白雪,她琢磨着,她真的可能是方圆百里地上唯一没有冬眠的动物呢。

她把狼骨残骸收起来,化了雪水熬汤喝,算着剩下的这点狼肉狼骨,够不够她熬到正月的。事实上她已经记不住日子了,一天又一天,但她知道腊月一定没过去,沉狮谷的灯笼好久不打了。

要换一个打猎方式,找不到,追不到,就看看有没有死
耗子撞到她这只瞎猫。她挖干草,搓成长绳,套一个圈,挂根骨头扔到谷底。悬崖四周她扔了十几根,在屋里外造了一个绞盘,十几根绳全都系在上面。只要是有动物踩进绳套里,绳子动一下,她就使劲收紧绞盘。每天醒来第一件事是火堆续柴火,第二件就是盯着绞盘上的绳子,然后就没有第三件事了,一直到睡觉。

这根本不是办法,你不能钓鱼一样地钓狮子老虎。狼肉早吃没了,那些狼骨也已经煮了三四次的雪水喝汤。是不是要对这个世界说再见了,真是讽刺呢,她五岁练功,练了快二十年,虽不是一流高手,但行走江湖总不是无名小辈,到最后却要饿死在这里。

事情总会有转机,如果生下来都是一个奇迹,那么活下去也不应该有多难。这天早上绞盘终于动了。她抡着胳膊转绞盘,确定是野兽,越转越吃力,有一阵没力气,绞盘还往回放了几圈,一直拉到顶,从门缝能看到一个黑影吊在桅杆上。她将绞盘固定,提刀出去。

推开门的一刻她失望了,那是个人,就算饿死,也不能人肉入口。她放下剑,绕着倒挂的男人走了一圈,哭着去抱住他,摸他倒着的脸问:“你是怎么跑出来的啊?”
\newline

{\centering\subsection{2}}

看来是要在山顶过年了,大雪封路,一个人出不去,两个人照样没有办法,往大了说,就是来一个军队,一样在冰天雪地的悬崖上遭罪。

吃的总还有办法解决,雪地上两只狮子,够他们吃上一阵子的。他们捱到天黑,实际上是小五子一觉睡到夜里,苏子瑶把他套绳圈里一点点放下去,卸下绳索他摸黑找了一大圈,希望沉狮谷的效率别太快,别当天就把尸体清理干净。靠近铁笼的地方,他找到了两具尸体,一公一母,也不知道哪只口感好一点。公狮子略轻,他拽着尾巴走在雪地上。慢慢看到绳子时,黑暗中一声低吼,月光下一个黑影罩在他头顶。

他们又弄来一头新狮子?这一点沉狮谷效率够快的,一天没狮子浑身难受。小五子不敢回头,拖着狮子匀速前行,可面前的阴影却越来越大,狮子加速上来了。他右手抓尾巴,左手摸摸身上,没带刀下来,只有两条路可以选择—拖着死狮子逃跑,或是扔下死狮子跑得更快。他脚步加快,还是没能摆脱身后的阴影。他跨着步子往前跑,死狮子在身后的雪地上趟出一条道,但凡有希望,就不要松开右手的狮子。单脚已经踩进绳套里,左手拉着绳
子,冲上面的苏子瑶喊着:“快拉我上去。”

小五子催她快点,绳子开始往上提,身体一下子倒挂起来,脚套在绳圈里,腾出两只手死命拽着尸体尾巴。追上来的狮子朝倒悬的小五子扑过来。他闪开左手,狮子扑了个空,从尸体上翻了个跟头。等它做好跳跃姿势,再一次扑过来时,只咬到死狮子的头。绳子在一点点上升,狮子不松口,小五子不松手,最后连它都被提了起来。离地面三四十米的时候,这只狮子也害怕了,牙床紧合,四肢在半空中一动不动。

小五子,死狮子,还有这只活狮子,都在这根绳子上一下下地升起来。上面的苏子瑶拽不动了,速度越来越慢,有那么一阵还在往下沉,定在半空一动不动。最后一个办法,你死我活,总不能空手回去,饿死在上面。小五子将死狮子尾巴缠在左手臂上,空出右手,隔着死狮子去戳狮子的眼睛。第一下没有戳到,他左胳膊使劲,把尸体往上拽,伸右手能够到它眼睛的距离,一下子戳上去。狮子紧闭眼睛,不敢咬他胳膊,也许它也清楚,但凡一松口,又咬不到小五子胳膊,自己将跌入深渊。

开弓没有回头箭,小五子自言自语给自己打气,戳不到眼睛,就用两根手指插它的鼻孔。手伸进去,狮子呼吸的热气从指尖传过来。它发出闷声,仿佛在警告。小五子咽了口唾沫,盯着自己的右手,盯着狮子的牙。

他冲上面喊,指挥苏子瑶将绳子盘住,停下来,看谁能耗过谁。狮子的声音越来越低沉,终于忍不住朝小五子手臂咬过去。小五子抽回手臂,眼睁睁看着狮子的两排獠牙在手指前一口咬空,挥舞着前爪坠入深渊。小五子右手收回来,抓住死狮子,长吁一口气,冲上面喊了几下,却发不出声音,他一点力气都没有了。

狮子肉不好吃,比狼肉还难吃,苏子瑶吃了两顿就上吐下泻。小五子说,还有狮子头,你吃吗?苏子瑶弄错了,她以为是扬州的狮子头,可现有能吃的就是血盆大口的狮子头。小五子将狮子头剖开,掏出狮脑烤成鸡蛋一般口感的食物,喂给苏子瑶。他没告诉她这是什么,他明确知道,只有一点一点地吃下这只狮子,他们才能活过这个冬天。

可是哪天过年呢?苏子瑶愈发虚弱,但每天晚上还坚持着到崖顶,看看沉狮谷的红灯笼点起来没有。夜夜都是黑的,是不是难熬的日子就特别漫长。阳光充足的一个正午,小玉突然出现在悬崖上。她和那个少年一起来的,当初和齐师叔把他带进沉狮谷的耿直少年。小五子忘了他名字,或者根本就不知道他名字,两个人一人一辆车两匹马,装满了蔬菜和牛羊肉,还有沉狮谷的点心,一股脑卸
在他们的小木屋前。

小玉说,他们从昨天夜里出发,在山上找了一上午,还一直担心昆仑公子早已经死了,回去没法向小姐交代。

“小姐也想过来的,怕谷里的人察觉。这不是,一有机会就让我送东西来啦。”

小玉讲话的时候一直盯着苏子瑶,她说怪不得拼了命地往外跑,原来山上也有个小姐姐。小五子开始没接茬,小玉就停不住嘴地夸奖苏子瑶,说她漂亮,说她年轻,说着说着语气都有些酸溜溜的了,说我们家小姐还真没法跟苏姐姐比,也没她这么好的福气,我们家小姐只配给昆仑公子养孩子!

小五子适时打住,他说苏子瑶不是我的,你们家文君小姐也不是我的,之所以跑出来,原因不便多讲,但绝没有伤乔姑娘的心。

不伤乔姑娘,但似乎伤了苏姑娘,一直到他们告辞,苏子瑶都没多说一句话。回去的时候他们只坐一辆马车,留下一辆车两匹马给他们。小五子一路送他们到山腰。他问他们什么时候可以出发,离开沉狮谷。小玉说再等几天,过了正月,雪慢慢化掉,山路就好走多了。

“过了正月?”小五子皱眉问,“现在不是才腊月么?”

“年早就过啦,今天都正月十六啦!”

“那谷里,怎么没有挂灯笼放鞭炮?”

“原来你在等我们。”小玉告诉他,以前年是过的,狮吼帮身处塞北,常年苦寒,年反倒要过得热热闹闹,有滋有味,“今年没张罗,还不是因为你?”

小五子指着自己,瞪大眼睛问:“是因为我跑出来,搅了大家兴致吗?”

“要知道你跑了,也就罢了,问题是他们不知道。”

“他们是谁?”

“就是江湖上跟你有仇的那些人,听说你做了狮吼帮的女婿,一个个过来寻仇。我们乔帮主嘴又硬,绝不肯承认你跑了,不在谷里。结果三天两头,一轮一轮地跟这些人比试,弄得帮主也伤病在床,大家也就没心情过年了。”

小五子往谷里看过去,没有红灯笼,但所幸也没人披麻戴孝。送走小玉和耿直少年,天已经黑了,闭着眼睛也能摸清木屋周围几百米的路。待得够久了,正月都要过去了。去年是怎么过的,在田独,和钱老板、文思清,三个人炒了几盘菜,喝了两坛酒。那时他还有情绪,钱老板不肯说他是谁,他下药给钱老板,绑在床上问。现在知道了又如何,昆仑公子的日子可比小五子惨多了。

他们还带来了酒,两口大锅可以炒两个小菜。小五
子把饭菜做好,打开一坛酒。喝到第二坛时,苏子瑶也过来陪他一起喝。两个人没话,她还没有痊愈,不敢多喝,但总觉得应该陪着小五子,就当是补个除夕夜。喝到后来,苏子瑶搬来第三坛酒,问他还喝吗?他摇摇头,大口喝光碗里的酒,看着苏子瑶说:“我想明白了,我不是昆仑公子,以后谁喊我昆仑公子也没用,我就是小五子。”

苏子瑶低头想想,反而用第三坛给自己倒了一碗酒,喝下一小口,问:“所以,你不会和我再有什么瓜葛?”

小五子没回答,手在双膝间搓着,说已经正月十六了,有车有马,储备也有,我们也该准备出发了。

“去哪里?”

他不回答,她又问不出口,好像过了一百年那么久,她问:“去找文思清?”

小五子起身去喂马,一直走到门口,回头跟她说:“对不起,苏子瑶。”
\newline

{\centering\subsection{3}}

到了腊月,即使是河南少林寺也开始下雪了。第一场暴雪已经到脚踝那么深,这天早上文思清将积雪清扫干净,照例到厨房去熬粥。两位师父,就算沈师父不喝,八光师父总还是要吃饭的。八光走进藏经阁的院子,对着四周皑皑的白雪伸了个懒腰,哈欠打了一半,他半张着嘴巴低头看着地面,仿佛被水浇过一般,院子里一片雪花都没有。他发了一会儿呆,将剩下的一半哈欠打完,冲进厨房。

文思清也在发呆,冲着咕嘟咕嘟的白粥,时不时往灶膛里扔点树枝,扇两下扇子,火不能太大,但也不能让它灭掉,一点点文火将白粥温熟。悠闲的冬日时光,八光在门口站了一会儿,他问:“沈老前辈今早出阁了?”

“没有啊。”

文思清头也没回地回答,手头的树枝有点长,她决定掰一半扔进去,估计用不着另一半,粥就可以上桌了。文思清起身拿出三个碗,在桌上摆成一排,计时器一般默数着五四三二一,端起白锅,依次将粥倒进三个碗里,不需要勺子,每次就要溢出来的时候,她手握锅把一收,一碗粥刚好盛满。三碗装满,锅里还有半碗,显然文思清有点失望,今天又没算准。

“八光师父,喝完你那一碗,把剩下这半碗也喝掉吧。”她说。

八光点点头,刚才的话还没有问完,他说那院子的雪是谁扫的。

“当然是我扫的,难不成是你早起梦游扫的?”

“真不是沈老前辈?”

文思清抬头笑起来,说他好大的胆子,现在都想要沈老前辈扫院子了。

“你用什么扫的?”

“当然是扫把,难不成·····.”

这次她说不下去了,难不成也得是扫把啊,不然扫雪用什么呢,用抹布,用簸箕吗?可是八光没听她说完,跑回到院子里,低头找宝贝一般检查着地面。文思清端粥出来时,他已经双手撑地面,趴在地上看了。文思清看过一眼,随口问一句,这是练什么功?脚下没停,一直走到藏经阁前,将白粥放在门口台阶上,对里面喊一句:“师父吃饭啦!”

沈老前辈没应声,倒是八光站起来,瞠目结舌地看着她,挡住她路线。文思清以为他又犯病了,要他让开。看他眼神那么痴,像是动了邪念,八光咽了口口水,喉结就像是小老鼠走了一圈后,说:“师姐,你现在已经是当世前五的高手了。”

文思清皱眉,喊声师姐,蹭着学艺也就算了,怎么一天比一天狗腿。她绕过傻掉了的八光往厨房走,说你先喝粥吧,不然凉了。摸到厨房门时,沈老前辈在阁里笑了两声,说八光真的是后知后觉,思清的功力,怎么到今日才有所察觉?八光冲着藏经阁鞠躬说话,说之前只是知道师姐功力大涨,方才看到院中地面,一片雪花都不曾残留,才知师姐已是当世前五。

沈老前辈沉默一阵,不让他喊师父,八光倒是听了,但是老喊文思清师姐,总觉得怪怪的。文思清站在厨房门前,说师父快喝粥吧,我又不和人打,排第几又有什么用?

“真要是排名,当世前五未必,”沈老前辈说,“我四个徒弟武功还在思清之上。”

“那师姐不是刚好第五?”

文思清笑起来,说:“八光师父,你这么排,要把师父放哪里啊?”

八光拍了一下脑门,纠正说不是第五,是前面有五位高手。沈老前辈沉吟一阵,说还有一个人要在思清前面,百花谷谷主。文思清听得直点头:“是啊,小五子可是他们少谷主呢。”

“那就是第七,也不错了。”八光跟领导总结似的,结束这一场排名大会,进到厨房喝粥。可沈老前辈过不去,一整天都在思考这件事,到晚上的时候,他说自己不出阁了,已不能算当世之人,他四徒弟向问和,不知无为神掌有没
有练成,但就算有练成的一天,也不能伤及任何人,总之是人不犯我,我不犯人的本事,如此说来,思清确实已是当世前五。

文思清没感觉,但是沈老前辈来了斗志,从此每天要文思清加练,每日只能睡两个时辰。他自己更是不睡觉,用这两个时辰想想,明日要怎么让她往下练。八光几十年梦寐以求的就是跟沈老前辈学艺,真到苦的时候他反到缩了,每天在房间里睡八个时辰都嫌不够,倒不是怕苦,年少学艺的时候,比这个苦多了,他是被文思清打击,他发现自己怎么练,也比不上文思清,一套新学的动作猛练几十遍,都没有师姐当天打得好。

少林寺也要过年,张灯结彩,净虚净空兄弟俩跟着忙里忙外,把豆腐青菜做出花来,光是豆腐就要调出各种肉味,素鸡、素鸭、素红烧肉,连豆浆都要反复配比,做成牛奶的味道。主要是哥哥新学会一个道理,做事不在多,黄牛拉地一辈子,也是被杀了吃肉,做事要准,准到让自己无可替代,等到大家习惯了你的厨艺,离开你不行的时候,自然就会有人教你功夫。

可是这些没人吃,除夕夜做了一桌子,太像鸡鸭鱼肉,闻着就犯恶心。摆出来无所谓,多少有点过年的气氛。沈老前辈还是没出阁,听着大家在院子里吃饭闲聊。过了午时,文思清对长辈叩头上香,先是对八光师父,一个头磕下去,急得八光跪下来跟她对拜。之后是沈老前辈,怕他听不到,文思清有意磕得响一些。最后是自己父母,没有牌位,只是那个骨灰罐。文思清磕着头说,父母在上,思清现在过得很好,请二老黄泉之下不必挂念。

之前的叩头净虚净空都跟着,到文思清的父母,他们就不好磕头了。祭拜过后,哥哥还走近些去看罐上的名字,问文思清:“人家的牌位都有名有姓,为什么你父亲这写着文大人啊?”

文思清不说话,八光让兄弟俩赶快把桌上的假鸡假鸭收拾掉,回菜园子。子时已过,丑时钟响的时候,文思清向沈老前辈请安,说自己先回,请师父早点休息。她知道师父不睡觉,但这句话总要说的。换平常沈老前辈会应一声,表示听到了。这一次沈老前辈问她:“文之兴是你父亲?”

文思清愣了一下,说家父叫文再兴,师父可能记错了。沈老前辈叹息一声,说原来他名字都改了,之兴,再兴,极尽讨好之意。文思清想反驳,张了几次口,又觉得没必要争这件事。反倒是沈老前辈关切起来,问道:“他已经过世了?”

两朝宰相株连九族,满门抄斩,当时那么大的事情,文思清已不知从哪里讲起。她只是点了点头,又怕沈老前辈看不到,“嗯”了一声。

“也是,我都已经百岁有余。”

“家父在世时,与师父相识?”

“何止认识,几十年了,至于恩怨二字,谁对谁错,就很难说清楚了,”沈老前辈话风一转,问道,“可你既是宰相之女,如何又流落江湖?”

文思清笑起来,早就不是啦,文家得罪了朝廷,前几年就已经满门抄斩,几百条人命当街问斩,她这条性命,还是母亲拿出全部家当,贿赂当差的,把她裹在尸堆里逃出来的。

沈老前辈叹了口气,隔着一道门都能听出他几十年的伤。文思清等了一会儿,说自己先下去了。沈老前辈叮嘱她先回菜园,这几天不用再上来练功,他累了,他要休息几天,等过些时日歇好了,他会让八光叫她上来。

文思清说师父保重,趁着夜色出了院子。真的累了吗,下山的时候她想,师父之前可是从来不睡觉的,而且一歇就要好几天,他年纪那么大了,死亡那个绕不过的拐角,可能早早就在那儿等着他了。该不会是要去了吧,少林寺的说法叫圆寂,回到菜园,躺到床上,她不免担忧起来,她怕再也见不到师父了,真是的,早知道刚才是最后一面,是永别,她还有好多话想说出来的。
\newline

{\centering\subsection{4}}

这一天在扬州还是喝酒,吴思若知道,就算是夜夜笙歌,这也是最后一夜。师父大漠仙人的计划不是如此吗,让她吴思若去服侍蓬莱阁老,换来一张九宫图,交易完成,一切就可以结束了。

把蓬莱阁老当客人,就照着紫竹院的流程走,敬酒,听曲,相谈甚欢,酒过三巡扶客人回房,上床一同休息。看起来是顺其自然的结果,只不过客人早就把钱给了老鸨。他们师兄弟有些奇怪,话没讲透,确实也不方便明说,师兄我要睡你徒弟,或是师弟,我把徒弟送你睡。两个人就是心照不宣,干杯喝酒,讲着不痛不痒的话,彼此看着对方干笑。后来仙人说了一句,说我技艺不精,我徒弟吴思若一直仰慕师弟的身手,想晚点去你房间,跟你学点什么。

这倒简单了,原来不需要假模假式地培养感情,说我跟你学东西就行了。她懒得再喝酒了,只等着曲终人散,去阁老房间熬过这一晚。这时阁老反倒没话找话,问她多
大了,练了几年的功夫,到底是他阁老的哪一种本事吸引了她。吴思若不想跟他聊下去,她想一句话结束话题。她说自己其实没学过几年功夫,之前一直在杭州紫竹院来着。

显然知道那里,不然不会愣这么久,他问她去那里做什么。她说赚钱啊。

“赚什么钱?”

“你在装傻吗?”吴思若反问道,“当然是赚客人的钱。”

“青楼的姑娘?”这句话不是问她,转头去问大漠仙人,“你拿青楼女子打发我?”

阁老不干了,起身要走。大漠仙人好说歹说,把他留下来。阁老跟他师兄吵了起来,听了几句吴思若明白了,一把年纪,满脸的褶子,他要的是感情,居然相信真会有年轻姑娘爱慕他,委身于他。吴思若冷笑两声,自己喝起酒来。仙人还在跟他解释,说我这徒弟生性是放荡了一些,但是骨子里还是个单纯姑娘。

“她若不是打心里爱上了你,怎么可能不收钱,就去你房里?”

“九宫图不算钱?它比钱还值钱,无价之宝!”

妈呀,就这罗圈话,还指望有人爱他。吴思若打断他们的争论,直截了当问阁老:“你是不是嫌我脏?”

阁老被问住了,看着她说我不知道,多看几眼他也知道,吴思若太美了,闷头喝了一杯酒说:“确实不知道,我没碰过妓女。”

感觉心被扎了一下,吴思若也不说话了,两个人不喝酒也不出声,并排坐着看前方,就好像前面曲子弹得有多好听一般,眼睛都是直勾勾地看着前面。不能怪他,谁让自己是紫竹院出来的,血淋淋的事实,没准小五子比他还要过激的。

“不然,你从这里面选一个,带走吧。”吴思若指着弹曲的姑娘说。

阁老摇着头,目不转睛地看着前方。吴思若笑了,不该笑,但真是有些可爱,这么大岁数了,行为处事竟然还有少年感,小孩子气。阁老长吐一口气,也不知道对谁说,只说我困了,要回去睡了。吴思若问,要我扶你回去吗?

“随便你。”

这什么意思,看着他起身,她想明白了,阁老不是困,是困惑了,他希望她来做决定。吴思若看着大漠仙人,他冲她点点头,说拿到九宫图,明天一早我就去竹林,把这些人都埋了。

“我知道你不会埋的,”吴思若说,“但你知道,我那时
一定死在你面前的。”

“我知道。”

不然就死吧,人生最后一次妥协。她大步跟上去,但没有追上阁老,离他几尺远,跟着他走。黑夜里,两个人一胖一瘦,始终保持着距离,穿过整个庭院,直到彻底听不到身后的评弹声。

阁老没锁门,进来就开始翻东西。吴思若跟进来之后,把大门锁上,阁老已经将九宫图找出来,扔到她面前。他说我答应拿九宫图换你,说话要算数,至于你,拿了这张图,随便你怎么选择,看得上我,你留下来,看不上我,拿上你的九宫图走,现在已经是你的了。

还真有点喜欢上他了呢,你说到做到,我吴思若凭什么就反悔?她解衣宽带,阁老倒是羞涩起来,打了手势,希望她背过去。外面的衣服是带子,内衣是绳结打的扣子,直到肚兜褪下,整个后背都露在阁老面前。她问可以转过来了吗?阁老在身后没说话,只听到他粗重的呼吸声。她说不然就熄掉蜡烛吧,又不好一直这么站着。阁老还是没说话,吴思若放下手臂,无所事事地看着大门上的雕文。“你父母是谁?”

“啊?”

什么意思,这种事情要聊父母助兴吗?她说她无父无母,被师父收养,在大漠长大。阁老又没说话了,持续发出奇怪的声音,过了一会儿吴思若反应过来,他是在哭。她转身问他怎么了。阁老瞬间崩溃起来,尖叫着让她穿上衣服。

“是因为我背后那一小块胎记吗,”她问,“要是讨人嫌,我转过来就好了。”

“我让你穿上!”

真没想到,整个晚上羞辱的顶点居然在这里,脱掉的衣服要一件一件穿上。系那些扣子要比解开更繁琐。她咬着牙,背对着他,把衣服穿好,转回身时她彻底惊呆了。阁老的脸已经哭花了,那一脸的褶子都往外溢着眼泪。他睁大眼睛,又看了看她,慢慢冷静下来,自言自语说:“我明白了,明白了,真的是狠毒!”

吴思若问他说的是谁。

“你师父,我终于明白,他这二十年究竟在想些什么。”吴思若在他面前仿佛是空气,阁老眼神空荡荡的,“你把这九宫图拿走吧。”

“就这样了?”吴思若问着,把那张九宫图折起来握手里,看着阁老。

“千万别给你师父。”

吴思若摇头说:“这不行,我要给他,才能换回我要的东西。”

“他根本不是要九宫图!”阁老喊起来,“他就是要你和我发生苟且之事。”

“什么苟且之事?”

阁老让她赶快走,东南西北,出门随便往哪里,永远不要再和他见面,这个仇他早晚要报。可吴思若来了脾气,就是刚才那个词,整个晚上都在被你羞辱,一次比一次狠,她拔剑出来,明知没有用,但绝不想服软了。她剑尖对着他逼问:“什么苟且之事,你把话说清楚。”

阁老脱口而出,我是你······话到一半他又咽下去了。他说不能让你知道,你这么年轻,总要活下去。

“你是我什么人?”

吴思若剑尖又往前伸过去,抵住他喉咙。阁老没有闪躲,血顺着剑尖往下流,望着她痛哭起来,哭了好半天,说出一句话:“我是你亲生父亲啊。”

吴思若摇晃了几下,烛光里干笑几声,松开剑柄,转过去面对着门,受伤一般,一步一步地走出门口。
\newline

{\centering\subsection{5}}

文思清算着日子,说是歇息几天,十几天过去了,也不见沈老前辈唤她上山练功。没人管她,文思清也没放下功夫,一样只睡两个时辰,醒来就开始练功。本来她对功夫没什么兴趣,一开始是因为不拜八光为师,谁知道他会不会又变为扒光,后来跟着沈老前辈,也是八光为了偷师,催着她跟沈师父请教练功。

学了一个多月,背了各种武学心法,她就知道昆仑公子的功夫是假的,怎么伤的那么多人不知道,但不可能是他们传的那样神乎其神,腾空转一圈,能戳瞎十几双眼睛。不是因为小五子失忆,忘了功夫,是他根本没功夫。这样她反而有了动力,现在是天下第七,以后练到第一第二,小五子就再也不用见他们就跑了。

师父没传话,八光师弟倒是每天下来一趟。他现在逼她喊师弟,你喊他师父,他跟你急。他说好些时候,他都以为师父圆寂在藏经阁了,一整天没声音,门口的饭菜一直摆到天黑,也不动一口。有时候他受不了,想进去看一眼,担心师父真的不声不响地死在里面,刚一推门,听到师父“唔”的一声,知道他还有口气,退出门外,去把门口摆着的那些换成热饭热菜。

“但也只是有口气,年岁大了,也差不多了,”八光耷拉
着脑袋说,“从此以后,你必须叫我师弟,师父不肯收我为徒,倘若哪天他不在了,你喊我一声师弟,江湖上也知道,我是沈老前辈的弟子。”

文思清看着他,想不明白他到底图什么,你也几十岁的人了,在少林寺也待了这些年,武功再高,每天也只是三顿饭一张床,要那些名分图什么呢?

再过来时,八光心情好多了,他说今天师父说话了,还一气儿吃了两碗饭,他收下空碗,问师父还要不要加菜时,师父在里面说,虽然后三十年是文相负我,可前三十年却是我负文相。

“文相就是你父亲吧,师父想通了,”八光说,“原来他一直在回想这几十年的恩怨。”

文思清睁大眼睛,不明白父亲和师父到底什么恩怨。她自己都没见过父亲几次,父亲快七十才得的她,印象里父亲就是个老爷爷,所有人都说他一人之下,万人之上,那一人是嘉和皇帝,那师父和父亲是什么关系呢,以前也在朝廷里做大官吗?

想不了那么多了,新年过后少林寺忽然热闹起来,一下子进来好多俗家弟子,听说都是交了学费进来的。多了些银子,少林寺的伙食也好了起来,虽然还是青菜豆腐,但是汤少了,筷子往里一摇,准能夹到一块豆腐。

最高兴的还是净虚净空兄弟两个,突然这么多小字辈的弟子,两兄弟一下子就变成了寺里的前辈,带着这些师弟们东走西看,每天都把师弟们领进菜园子,来拜见武艺高超的小姐姐。他们说,小姐姐本来是昆仑公子的女人,之前一点不能打,来少林寺练了几个月,已经是天下一等一的高手,之前那些恶人,现在都不敢来少林寺挑衅了。还好不用她露两手,兄弟俩说什么,那些俗家弟子们就信什么,一副高山仰止的表情望着小姐姐。

文思清才不想见这么多人,她躲在房里不出来。园子里已经没法练武了,有天中午八光过来,说今天是元宵节,希望文思清去看看师父。可是看什么呢?隔着藏经阁的门,八光还让她不要说话,看到师父在里面就好了。

下午文思清上了山,搬把椅子对着藏经阁坐下来一动不动。暮色将至的时候下雪了,一片片雪花落到头顶,落到嘴角,舔起来甜甜的。师父在藏经阁咳嗽了两声,喊了文思清的名字,说你功力又长进不少,坐了这么久,才听出你也来了。文思清站起来,鞠躬说本来不想打扰师父,只是元宵佳节,看看你就走。她等了一会儿,觉得该告退了,走到门口时,沈老前辈说:“你进来吧。”

第一次进藏经阁,推门进去里面几乎是全黑的,只看
到黑暗深处些许的微光。沈老前辈说,你往里走,我为你点了蜡烛。文思清在两排经文之中越走越深,里面的光越来越强烈。

“原来藏经阁有这么深。”

一直到最深处时,文思清回头看一眼,从进门口开始差不多已走了上千尺。她转回头,第一次见到了师父的真容。文思清慢慢坐下来,等待师父说话。过了有一会儿,师父说:“不必了,你下去休息吧。”

文思清没明白,指着自己问:“是说我吗?”

沈老前辈摇了摇头,说:“八光奉了茶放在门口,我让他不必麻烦了。”

“刚才有人说话吗?”

文思清向门口方向看过去,长长的走廊只是一片黑暗。哦,她明白了,这么远的距离,她听不到,但师父听得一清二楚。那就能解释,有时师父为什么一天都没声音了,因为我们在外面听不到,只有他想对我们说话时,才能听到师父的声音。

沈老前辈说:“知道我叫你来是做什么吧?”

文思清点点头,又摇了摇头。隐约知道,但真的讲不清楚。

“聚散有时,一晃你跟我学了近百天,适合你的,师父都已经教授于你,接下来就要看你自身的悟性,能把这多少变成自己的本事。”沈老前辈说,“但是你跟师父一场,出去自称我的徒弟,总得让你见我一次,知道我长什么样子啊。”

文思清眯眼笑起来,沈老前又点几根蜡烛,把四周照得明晃晃的。

“你坐近一点看。”

文思清凑过去,师父一身灰袍盘坐在地,看起来很高很瘦,原来师父不是和尚,头发还在,最多算带发修行,全白的头发与胡子连成一片,一双眼睛还是被烛光映得大大的。

沈老前辈问她看好了没有,因为长期在阁里,眼睛不适应强光。文思清说再等一下,再看五秒钟,就可以永远记住师父了。她一边望着师父,一边心里数着,果不其然,心里数了五个数,沈老前辈一挥手,连之前的那根小蜡烛一起,全给挥灭了。

周围一片漆黑,一点光都没有。文思清寻思着,好半天没说话。沈老前辈问她怎么了,我不习惯光,你是不是也不习惯这么黑?文思清说不是,她只是有点奇怪,师父怎么会像画里的人物。沈老前辈问她哪幅画。她想不起
来了,但千真万确见过这么一幅画,这几个月在少林寺,之前在田独,再之前被卖来卖去地辗转漂泊,要是真有这么一幅画,一定是在文府见过。可是文府挂出来的一百多幅画,她张张了如指掌,绝没有一幅画,画的是师父,那又是哪里见过呢?

她问沈老前辈,跟家父到底如何相称:“我确定在父亲那里见过您的肖像。”

“如何相称?”他慢悠悠重复着文思清的问题,回答着,“我与文相早年间以君臣相称。”

“家父既为文相,那师父一定就是皇帝了。”

文思清也没有特别惊讶,她父亲文再兴做过两朝宰相,辅佐过三任皇帝,说起圣上,虽不说是司空见惯,但起码不会大惊小怪。让她高兴的是,她想起在哪见过他了,仁丰皇帝的画像上。当年他们抄家时,给父亲强加的诸多罪名之一,搞来这一幅画像,硬说在文府搜出来的,说父亲私藏前朝皇帝画像,对本朝有逆反谋权之意,当然还有其他罪名,当差的李大人宣读奉天承运,皇帝诏曰,叽里呱啦的,说得父亲浑身都是罪。

师父自己说了很多,说当时怎么不理朝政,钻研武学,丢了江山,任由刘子林父子一路打进京城。

“刘子林就是当今嘉和皇帝的父亲,”沈老前辈说,“他带着他几个儿子,打得我朝节节败退。”

一场恶仗连打三十六日,倘若不是文相死守太原,不要说一个多月,恐怕一个礼拜即被攻陷。眼看气数已尽,他不愿做亡国皇帝,愧对列祖列宗,硬要退位做太上皇,把皇位继承给太子。刘子林父子已经在山西势如破竹了,他还在给太子举办登基大典,搞退位仪式。

新皇登基不到一个月,刘子林果然破了太原,一路带兵打进了京城。攻城那一夜,紫禁城乱成一锅粥,皇后娘娘们争着往车里装首饰,太监们搜罗着金银细软往宅子搬,混乱中只有苏皇妃什么都不要,一再地追问皇上去哪儿了。后来有人在后山发现,皇上投河自杀了。

仁丰太上皇看着大家争抢财宝的场面,愤恨羞愧,一气之下将后宫这些人,皇后、妃子、宫女、太监,二百五十多人全部杀死,唯独留下了一直还惦念皇上的苏皇妃。

“本该我是皇上,他是太子,如果要自杀,也应该是我。”沈老前辈说,“我该自尽谢罪,却以他日复辟为由,带着苏皇妃逃出京城,又苟活了几十年,一直到现在。”

黑暗中他叹了口气,说刚逃出来时确实想着王朝复辟,那时还认为,如果武学修为足够,还可以号令天下,一举反攻京城,将刘子林父子剿灭。他收了三个弟子,打算
练就断魂掌、仙人掌和蓬莱掌,再依仗九宫图的路线,打回紫禁城。只是几十年过去,号令天下还没做到,三个弟子反倒内讧起来。他将弟子赶出师门,渐渐知道自己无力回天,彻底放弃了江山。

“所以说,我父亲是背叛了你。先给你做宰相,后来又给嘉和皇帝做丞相。”

沈老前辈点点头,说:“过去了,也想明白了,毕竟我有负于文相在先,机缘巧合,又让我收了你做关门弟子。除了无为神掌,毕生所学已全部教给了你,接下来就是看你自己的造化,今晚所言,你自己知道就好,切不可说与外人。”

文思清明白了,这次是真的永别了,她冲他磕了个头,也不知道这头是磕给师父,还是前朝的皇帝。她起身朝门口走去,一片漆黑,从黑暗走到黑暗,一直到门口,夜空中的点点星光,反倒显得明亮了起来。
\newline

{\centering\subsection{6}}

冰天雪地的季节,继续北上田独,小五子也知道不合适。苏子瑶建议先南下中原,一边走一边打听,倘若没有文思清的消息,那么就到南京百花谷,谷主会给他备好盘缠马车,那时已春暖花开,大家再一同做回田独的打算。

小五子问,如果他是少谷主,那百花谷谷主和他是什么关系?他在肉铺的钱老板呢,也就是你叫他常公公的那个,前年冬天听你跟他说话的口气,什么百花谷谷主有令,知道他也是百花谷的,可是他为什么把我掠走跑出来?他到底是敌是友,大一点说我昆仑公子,和你们百花谷又是什么关系,你们是真对我好,还是要弄我?

他一气儿问了好多问题,苏子瑶一个都没回答,她说不急这一时片刻,等到了南京,谷主自然会一五一十地讲给你。

人推车尾,马拉车头,他们大概三天才绕出沉狮谷,出去的一刻小五子不忘看一眼,谷底悬崖,上下困了几个月,自己此生应该不会再来了。过去这一年发生好多事,从苏子瑶出现,到遇见吴思若,到何员外被灭门,他还亲手杀了已经疯癫的何员外,自己成了丐帮帮主,被三王爷六公子追杀,被马长老控制,参加武林大会,发现自己竟然是当伙计时一直想成为的那个人,昆仑公子,被大漠仙人和蓬莱阁老挟持,从船底逃跑上岸,往北又撞到黑白二鬼,带到沉狮谷逼婚,入了洞房,老婆乔文君又是假的,儿子是六公子的。是啊,他回头望着越来越远的沉狮谷想,放过他吧,这
么下去,真要扛不过去了。

塞外没问题,日行三百里,一个人都见不到,四外荒凉一路畅通,过了嘉峪关,人开始多了起来。两人知道不能这么大摇大摆地赶路了。苏子瑶说要进城里喂马,顺便装扮布置一下。她把他扔在郊外的一个凉亭,自己驾着马车去了市集。中午出发,太阳落下去了,还没有回来。小五子一直蜷缩在凉亭里等待,说是凉亭,简直就是一个大风口,正月的边塞夹杂冰雪的寒风将亭子里吹个通透。小五子又饿又冷,开始还在凉亭里边走边跺脚,抵御冷气,后来连跺脚的力气都没有了,双手插在袖子里,蹲坐在座位旁一动不动。

偶尔还是有人经过,骑马的,赶路的,也许还有昆仑公子的仇家,往这边匆匆看一眼,知道有人冻死在亭子里,事不关己,继续赶路。晚一点小五子闭上眼睛,因为睁着眼睛,眼珠子都冻得难受。醒来时天已经黑了,苏子瑶还没有回来。小五子想站起来,双脚已经冻得没知觉,他双臂抱着柱子,一点点往上蹭,站起来时朝市集方向望去。

远处有个大胡子男人赶着马车过来,看眼凉亭里的小五子,扬起马鞭喊了声“驾”,眼看就要和亭子擦身而过,小五子喊住了他:“这位大哥,请留步!”

大胡子男人双手勒住马缰,马前蹄上扬把车停住,男人坐在马车上打量着他。小五子把后半句说完,他问他有没有吃的,救济兄弟一口。

再简单不过的意思,大胡子男人还是好好琢磨一下他的话里有话,问他到底有何居心。能有什么居心,就是太饿了。小五子指指自己肚子,表示饿一天了。胡子大哥迟疑一下,跟要扒光他的眼神一般,又把小五子打量一遍,回头掀开帘子对车里说:“娘子,那我们就在这儿稍停一下,看看他是否是黑松寨派来的奸细。”

里面也不知道说了些什么,胡子男人连声说好,说你不必下车,在里面歇着就好,若是黑松寨的人,我直接把他料理了。说完他还转头瞪了眼小五子,跳下马车将马缰拴上,拿着一个包裹走进亭子,在小五子对面坐下来。

小五子对他笑笑,极尽讨好之意。胡子男人白了他一眼,打开包裹,里面有一只烧鸡,两斤酱牛肉和一壶捆好的酒,烧鸡还是热乎的,刚开口还一层层地冒着白气。胡子男人撕下鸡腿,一口咬下去。

光掉哈喇子可不行,总得说点什么。小五子看看车里,问胡子男人,车里面那个是你娘子?胡子男人警觉起来,说车里面没有人,整架马车只有我自己。换以前小五子一定怼回去,说车里面坐着的,如果不是人,那就是狗。
今天嘴不能太臭,肚子还咕咕叫个不停,腿上不自觉地走了过去,蹲在男人身前,往前探个头,就能咬到鸡腿。胡子男人当他不存在,每口下去都能咬出油汁,眼看只剩最后一口,小五子克制不住了,失声叫道:“等会儿!”

胡子男人停下来,嚼着鸡腿看他,问他怎么了。

“就这一口了,给我行不行?”看他犹豫,小五子补充道,“不白吃你这口鸡腿,等我有了力气,黑松寨的经过这里,我来帮你解决。”

胡子男人盯着他,说:“你果然认识黑松寨的人!”

“根本就不认识!”小五子激动地站起来,“黑松寨那帮禽兽不如的狗东西,我怎么可能认识!不要说认识聊天,见一次都怕瞎了眼睛,听一次都怕烂了耳朵!”

这可能是他最后一点力气了,在亭子里绕着圈地咆哮,把话喊完就瘫坐在地上,喘着粗气。胡子男人公鸭一般地笑起来,有那么一阵,小五子还听到了女人的笑声。他朝车上望去,胡子男人故意咳嗽一声,撕一块鸡肉扔过来,说:“好好吃你的东西,不要东张西望!”

小五子接过鸡肉,大口咬下去,三下两下便吃完手里的肉。肚子暖一些了,他坐近一点,看着胡子男人把纸袋里的牛肉撕成一条一条的。小五子说黑松寨的人就是一群疯狗,没什么本事,还到处乱吠。

胡子男人高兴了,递给他一块酱牛肉,说原来兄台是明白人,刚才错把兄台当成黑松寨的人了。小五子说不知者不怪,本来还想客套两句,可是嘴里塞满了牛肉,说不出话。胡子男人说,这酱牛肉有些冷了,还请兄台不要介意。小五子说太客气了,冷牛肉配烧酒,越喝越有。说着他解开烧酒壶,见胡子男人没阻拦,咕咚咕咚喝下去半壶。胡子男人不言语,小五子把手伸纸袋里去抓肉,每次伸手,还假模假式地关心打听两句,来转移视线。比如大哥是怎么惹着黑松寨的啊,黑松寨派了多少走狗来杀你啊,你这是打算往哪逃啊?一次抛一个问题,胡子男人刚一沉吟,思考怎么回答,小五子的手快去快回,肉已经从纸袋里转移到到他嘴里。

他问一句,吃一口。胡子男人答了些什么,他一句也没听进去。一只烧鸡,二斤酱牛肉,整壶酒也基本都是他一个人喝掉。他将纸袋翻过来抖抖,吃最后一口肉渣,狠狠地打了个饱嗝。这时才认真点听胡子男人讲述他的故事。

吃饱喝足,该放轻松才是,可是他越听越紧张。胡子男人说他本来姓齐,是黑松寨的厨子,和黑松寨的大小姐有了感情,别看他其貌不扬,厨房的手艺可是一绝。也许
就是俘获了大小姐的胃,进而把她的心也俘获了。两个人一来二去,被黑松寨的刘寨主发现了,把他打进地牢,要拆散这对鸳鸯。就在要处死他的当晚,刘大小姐假传寨主密令,命人将他解救。两人带上两口锅一口灶,连夜私奔。他们上个礼拜跑出来的,七八天马不停蹄,想一路下江南,找个无名小镇隐姓埋名地在一起生活。

齐大胡子说完,还自我陶醉一番,憧憬一下未来。他问小五子去过江南没有。小五子点点头。

“听说那边是鱼米之乡,是不是江南什么都有,河里田里随手一抓,都是下厨的好食材?”

“是吧,我也不清楚,我以前杀猪的,但是在北方杀猪,更北边,田独。”

小五子说完,往后退两步。他在看他反应,天下人都知道昆仑公子从田独来。这胡子男人有问题,又是大小姐,又是厨子,又是地牢,又是私奔,故事讲得这么俗不可耐,一定是编出来的。黑白二鬼替师父找昆仑派掌门人的故事,编得都比他强。

毫无疑问,冲他昆仑公子来的,苏子瑶到这时候还没回来,肯定也和他有关。小五子一边打着哈哈,一边走出凉亭,说我跟大哥投机,聊了这么久,小弟还没见过嫂子呢。他说完快步往马车走,刚才听说话,知道车里的女人中气不足,那么近的距离,都没听见她说什么。小五子计划先上马车,把那所谓的“娘子”劫持,就算不是娘子,肯定也是他心爱的女人,到时候再看有没有活路可选。小五子摸着怀里,两把刀都在,他抓住一把握住刀柄,朝马车走去。

齐大胡子在身后笑他,那是我的娘子,你急着见什么?听声音还没追上来,小五子大步走过去。忽然前方一阵马蹄声,七八个人从远处骑马过来。小五子停住回头望,齐大胡子一个起身,从凉亭跳过来,把小五子推进车里,低声说:“先上车再说。”

车里没有娘子,摸起来就是一身红衣红盖头。齐大胡子解开僵绳,坐到前面赶马。他到底是什么来头?讲了半天厨子小姐的私奔,里面连个女人的影都没有。大胡子架着马车不缓不急地上了路。马蹄声越来越近,快交汇时,为首的一个人喊住大胡子,让他停一下。听声音很熟悉,小五子知道是他仇人,追查昆仑公子的。他摸起车里的红绸缎,竖着耳朵听他们说话。

为首的问大胡子是什么人,这么晚干吗去?大胡子反问他,是不是黑松寨的人,要杀要剐,你们就地解决,反正黑松寨我和我娘子是绝对不回去了。

“你娘子?”

大胡子哈哈大笑,说你们没想到吧,我和你们刘大小姐早已是生米煮成熟饭啦!怪不得故事这么俗,简单易懂,为首的听到这儿,就知道怎么回事。他盯着大胡子,说我们不是黑松寨的人,但想一睹你娘子的美貌。大胡子生气了,冷笑道:“我娘子,可不是人人都能见得!”

说完他抽了一鞭子,马蹄长鸣,车却不往前跑。他回身站起来看,只见一个矮胖的男人,双手抓着车尾一动不动。大胡子脸色突变,说话也结巴起来,说你们到底是谁,想要对我娘子干什么。为首的笑了笑,忽然拉开车帘,只见车里确实坐着一个红衣新娘,头顶还盖着盖头。他不放帘子,小五子也不敢动。为首的点起一根蜡烛,伸进去里面晃了晃,最后将蜡烛留在里面,放下帘子说了句:“蜡烛就送给你娘子,做贺礼吧。”

说完他冲车尾的矮胖子点点头,胖子松开双手,马车还往前溜了点车。几个人上马向北继续赶路。大胡子也不急,还是慢悠悠地赶着车往南走。小五子揭开盖头,头探出车外,看着大胡子背影,说句谢了。他挥舞着鞭子笑道:“没猜错的话,你应该就是昆仑公子,刚才那饿死鬼的样子,可一点都不像。”

小五子手握着刀柄,问你是什么人?大胡子又抽了下鞭子,回头说道:“怎么刚分开几个时辰,就不记得我啦?我下午不是跟你说好,去城里装扮布置一下,顺便把马也喂过。”

小五子看过去,自己怎么瞎成这样,原来就是那两匹马。他再看看齐大胡子,真是的,粘得和真的一样,脸上都贴了胶,看起来一脸的糙肉,只是那双眼睛,眨巴眨巴的,还能看出是苏子瑶。
\newline

{\centering\subsection{7}}

文思清和八光坐在港口旁的茶摊前,二月的南京春寒料峭,可长江里的船却已经热闹起来,长工们排着队在码头卸货装货,不时有客船靠港离港,上船的人北上中原,下船的人进入南京,更多的是送往等客的人群。出了少林寺,他们没回田独,文思清说,她要来南京见个前辈。八光陪她等了几天,每天问一百次,她到底要见谁,那个人还在不在南京。

文思清说:“在的,她一定在这里,只是我还没有想好,要以何种理由,去拜见这位前辈。”

他们上礼拜到的南京,从少林寺出来,十几天的路
程。本来正月十六,文思清都要收拾行装上路了,八光手足无措,在藏经阁的院子里打转转。沈老前辈让他下山去送送文思清。八光说,送不送下山倒无所谓,佛门圣地,谅歹人也不敢在嵩山撒野,只不过出了少林寺,江湖凶险,文师姐一个女孩子家,难免会被别人欺负。沈老前辈沉默一阵,批准他一路把文思清送到田独,只是八光万不可以淫心大起,破了色戒。

要送到田独,可就有得计划了。文思清一天就收拾好行李,八光自己的行装却三五天都收拾不完。仿佛要出门远行,再也不回少林一般,八光把能带的全都装进去,水瓶水壶都悉数往里装。有一块抹布,他实在装不下了,估计也是再不想干擦桌子的活儿,他连洗了三盆水,把它洗干净,恭恭敬敬地叠好,放到桌子上。

文思清见过这抹布,之前是黑的,几盆水洗白后反倒引起她注意。她拿起来查看,一块奇形怪状的羊皮,两个巴掌大,握在手上刚好可以抹桌子。她问他哪来的,八光冲藏经阁努努嘴,说师父送我的,之前都是布的,一使力就烂,师父送我个羊皮的,用了好几年了,结实耐用,主要是特去油,不管是桌子上什么油,一抹就掉。

“这是九宫图啊。”文思清拿着抹布说。

“我当然知道九宫图。”八光笑了,那意思是虽然我十多年没出少林寺,但江湖上的事,我什么都知道,如果这个是九宫图,我还至于拿它当抹布嘛?

文思清走到藏经阁,将抹布放在门口台阶上,说弟子和八光将九宫图奉还您老人家。

“你拿走吧。”沈老前辈在阁里说,“我几十年前还给过文相一片,也不知有没有传给你。”

原来九宫图是按片算的,文思清说,父亲母亲都没给过她这个,可能是家破之时被人抄走了。沈老前辈说了声“嗯”,要她多保重,踏入江湖一切小心,如果遇见他那三个心术不正的弟子,不要与他们攀同门交情,也不要和他们有不必要的争执。

“你本事和他们差得还远,我不想日后他们挟持你来威胁我。”

文思清回答:“弟子明白。”

沈老前辈说:“我门弟子,有两个人可以信任,一个是丐帮的前任帮主向问和,我授了无为神掌给他,你尽可以叫他一声师哥,他也会拿你当小师妹待。另一个人是百花谷的谷主,我二人虽无师徒的名分,可我毕生的武学也传授了她不少,我之前跟你说的,那个发现皇上不见了,跳河自杀的苏妃,就是她。”

原来苏妃是百花谷谷主,小五子是少谷主,通过她总能找得到小五子。文思清冲藏经阁叩首,望师父保重身体,希望有生之年,可以再见一次师父。

“无需再见了,”沈老前辈说,“下山就是要闯荡外面的世界,倘若只为了再见一次师父,那岂不是哪也不去,留在寺中即可?”

文思清愣在原地,师父在阁里说了最后一句话:“去吧。”
\newline

{\centering\subsection{8}}

行到苏州的时候,一脸胡子的苏子瑶遇到了吴思若。那天他们入住当地最大的客栈,还是楼下赌场楼上住宿的那种。小五子一身新娘装扮,不方便进赌场,即使是躺在楼上听着过瘾,也坚持要入住这一家。

前几夜有些不愉快的事情,苏子瑶已懒得卸妆了,就粘着一脸的大胡子,躺在小五子身边。越到夜里人越多,楼下熙熙攘攘,开大开小的声音时不时传到楼上客房,良辰美景,却和一个假大胡子共处一室,平躺一张床。小五子侧对着苏子瑶,看着她的胸脯一起一伏,半起身伏在她身上,求她一件事。两人距离不过半尺,苏子瑶眼神慌乱,磕磕巴巴地说,有什么要求尽管吩咐,压在我身上干什么。

“我身上还有块金条,”小五子把金条拿出来,放在她胸口上,“你去帮我换成银票,在赌场输掉。”

苏子瑶起身,瞪大眼睛看着铜镜,可能全是这一脸胡子惹的祸。

“你输了钱,我就当是过瘾了。”

她从床前拿起金条,穿上外套下了楼。出门之前,小五子还在房间里喊:“不用出去换的,一般赌场都给兑银子!”

还真值不少银子,打杂的小工忙前忙后地把银子搬到赌桌上。那就输光吧,她一把一把地往桌上推银子。对面有个赌客一直在赢,右手压注,右手收钱。苏子瑶看着脸熟,直到有个献殷勤小工过来,说要帮他换成银票,他抬起左手让他走开时,苏子瑶想起这个人在武林大会见过,这个人一只手,混在丐帮里,小五子身边,是吴思若的师弟。

反正都是输,她一边压银子,一边四处张望,看看吴思若在不在这里。回身看了一大圈,吴思若却从她面前,一只手的身后走过来了。她喝了不少,摇摇晃晃地搂住一只手的脖子,坐到他腿上,冲他耳边吹气。吴思若越亲密,
一只手就越紧张,之前赢家的气质都没了,连续几把押错,看牌的小哥都懒得伺候他,一个个跑到大胡子身后来了。

他俩怎么跑到一起去了?苏子瑶皱眉看着吴思若发酥,一只手不断劝师姐,别这样,让五帮主看见,我又要欠他一条命。吴思若撒娇说,就是要让他看到,明天我们去南京,要让百花谷的人都看到,我跟你在一起,他小五子永远没戏。

一只手吓坏了,把手头的钱押完最后一把,匆匆上了楼。吴思若留在她对面,低着头喝酒,偶尔抬头,看见苏子瑶在看她,指着“他”问:“看什么看?”

苏子瑶冲她笑笑,将桌上的银子兑了银票,剩下的碎银子打赏给小哥,起身上了楼。

小五子一直在等她,见她进门就夸她,赌技不错啊,一根金条用了一个多时辰才输光。苏子瑶把银票扔过去,小五子看到上面的数字目瞪口呆,你怎么可能会赢,是输了不给钱,反倒把对手打劫了吗?

苏子瑶没回答,说我困了,先睡了。她熄灭油灯,背对着小五子,面朝窗口躺下去。小五子兴奋得一时睡不着,拿起银票翻来覆去地看,最后将银票捂在胸口,做起美梦来。

而苏子瑶一直没睡,睁眼就能看到窗外的月光。她想吴思若要干吗,就算昭告天下,自己和小五子没任何关系,总不至于找一只手那样的做垫背。快天亮时,小五子反而睡得更沉,呼噜声一次比一次响。有人走出客栈,将马从马厩里牵出来。苏子瑶起身从窗口张望,她看见吴思若跳上马,离开客栈,向南跑去。一只手从客栈里追出来,喊着:“师姐,等一下,我陪你去南京,还不行吗?”

一只手掏剑进马厩,随便砍断一匹马的缰绳,骑到马背追了出去。睡梦中小五子“唔”了一声,不知道又梦到了什么好事。马蹄声渐远,苏子瑶躺下来想了想,摇醒小五子,她说:“我们现在出发,去南京吧。”

小五子半梦半醒,说本来就是要去南京。苏子瑶坐起来,看着窗外,一只手也不见了踪影,清晨雾气升起来,苏子瑶右手摇摇小五子,她说:“这次去南京,肯定会碰到很多人,很多事,至于你是做小五子,还是回来做昆仑公子,百花谷少谷主,一切都取决于你了。”

小五子又“唔”了一声,坐起来揉着眼睛,问她刚才说什么。苏子瑶看着他,笑了笑说:“我说,我们出发吧。”

\newpage