\section{拾捌}

{\centering\subsection{1}}

挨过几天,八光伤势越来越重,但他执意不肯回少林。有一天他跟文思清请求一件事,说老看你抱着那个盒子,我一直想,我死后能不能也烧成灰装在盒子里,也让你这么成天抱着。

文思清瞪大了眼睛,仔细想了想,盒子里面装的是我娘,把你再放进来成什么事儿了。

“是啊,”八光问她,“那你以后就抱俩盒子?算了,这也太麻烦了。”

之后就不再提这件事了。有一次,他觉得自己不行了,就要死了,他说有件事压在心底,一直想跟你讲,但是不敢说。文思清问他什么事儿,八光想想,摇摇头说算了,他还是把这些话带到墓里去吧。文思清生气了,威胁他,你要是不讲出来的话,你死了我也不给你烧纸,也永远不会想起你。那些想起你的,也都是恨你的,被你欺负过的女人,你自己看着办吧。

八光缓缓说道:“那天我去妓院找小五子···...”

“这事儿你讲过,你还说,全天下的女人在你看来都是皮囊,但是我一直不明白的是,既然你已经觉得全天下的女人都是皮囊,为什么还认定自己无法持戒,说自
己愧对少林,就是不肯回去?”

八光道:“我绝不会再犯淫戒,可没想到却又犯了情戒。我不敢说,我一直觉得我不配说这个,也不配犯这个情戒,喜欢上这个人。我那天是说觉得全天下的女人都是皮囊,但我没说全,我想说的是除了你,全天下的女人都是皮囊。”

文思清没接话,不知道怎么应对,大中午的背身过去,说自己要睡觉,其实泪眼依依地看着身边的草地。过了一会儿她以为八光睡着了,打开盒子,将她妈妈的骨灰一点点撒进河里,说:“娘,你放心走吧,有一个人会一直陪着我,一直对我好的。”

八光醒来,拦住她,问她在干吗。

“你死了之后,我就把你的骨灰放在盒子里,天天捧着。”

八光这辈子也没有感受过这种温暖,震天动地一般哈哈大笑,随后气息变得微弱。他说自己活不了几天,问她,自己死后要作何打算,是不是去找她的小五子。文思清叹了口气,说自己早没有和他在一起的念想,之前在百花谷待过半年,回到宫里,便已不会也不能再当小五子是自己的爱人。话才说完,文思清一阵恶心干呕,她忽然意识到,自己也许怀孕了,怀了小五子的血
肉。不会去找他,日后先去见百花谷谷主,再听她的安排吧。

文思清打开骨灰盒,在盒子的底部,看到一张羊皮,掏出来。八光笑了,说:“我就知道世界上凡是抹布,就一定有一对,你看我的。”

说着拿出自己的抹布给她看。文思清问他哪儿来的。八光说:“是沈老前辈以前给我的。我说这么好的抹布,你怎么舍得给我。他说,本来无一物,何处染尘埃,他再也用不着擦桌子了,就给我了。”

文思清纳闷了,说:“我去过他的藏经阁,除了尘埃就是尘埃啊。”

八光哈哈大笑,笑声越来越小,终于圆寂。
\newline

{\centering\subsection{2}}

三王爷和五公主不好杀,六公子要把他们流放到北方。路上这两个人各揣心事,三王爷惦记着什么时候能把五公主甩掉,自己往南跑,五公主惦记着什么时候能把三王爷给甩掉,自己往汴梁跑。行至关外,北风夹杂着雪花,让大家异常寒冷。

三王爷看着自己的鸟和蛐蛐都被冻死了,伤心起来,他说能不能绕道去南边,再去买些鸟和蛐蛐?公主知道他也不愿北上,两人花钱买通押送人员,带着车队,向南走去。

二人对各自目的心照不宣,公主对三王爷道:“三皇叔,从我父皇,到我,到小五子,到六公子,你斗了四任皇帝,其实也累了吧?”

三王爷说,到今天才发现,遛鸟和斗蛐蛐是这么有意思,当皇帝有什么好的啊,天天都担心被人推下去。

“皇叔若打算与世无争过太平日子,侄女给你指一个好地方,那边四季如夏,有阳光、沙滩、椰树、海浪,还有好多大乌龟,皇叔可以到那边去尽享天伦之乐。”

三皇叔问她这好地方在哪。

五公主答道:“我不是在海南待了一年吗,随便找一个海岛住下,天高皇帝远,六公子才没兴趣带兵去打你。”
\newline

一眨眼,小五子与吴思若等人已在农家生活了半月有余,有时候帮忙做些农活。一只手时不时跑到农村夫妇那去吹嘘,说他们的主子小五子可是上一任的皇帝。夫妇俩面无表情,一只手恨不得揪着他们耳朵说,皇帝
啊,我们家主子是皇帝啊。看他们还没反应,一只手问道:“你们知道上一任的皇帝叫什么吗?是什么年号吗?”

别说上一任,连这一任夫妇俩都不知道是什么年号,也不知道当今皇上孙天奇。一只手气得冲他们大叫:“你们这帮小民,草民,就知道放牛种地,天大的事都不知道,还能有什么大出息!”

小五子傍晚对吴思若叹息,说:“我们沈家的天下,先是被他们孙家夺了下来,然后我们又杀了他们孙家的太子,之后他们又要斩我示众······上面打打杀杀,你看这些百姓,几十年来还是一年四季的耕种,吃饭,生活没有变,倒是我们沈家和孙家这样抢来抢去的,又有什么意义呢?”
\newline

次日清晨,大家被远处一阵马蹄声吵醒,几个人迎出去,看到是五公主来了。

可能是嫉妒,吴思若看见她很不乐意,讽刺她:“是不是你皇兄让你过来找我们的啊,我猜啊,明天就得有百万大军杀过来了。”

公主不接话,看了眼小五子,转身就走。吴思若怕小五子回头怪她,嘴上又不服软,继续说:“你现在走也没用啊?人家都知道我们住这儿了,你任务完成了就要跑,是吗?”

公主这几个月,女儿死了,又被软禁,一路上颠沛流离,早就气坏了,转身大骂吴思若:“你到底是想怎样?你就一奴才,伺候我的宫女!”

吴思若笑道:“呦,拿宫里的身份说事儿?那不还赶快给皇后请安?”

小五子出来打圆场,说道:“行了,都不要胡闹。五公主先在这里安顿下来吧,其他的事以后再从长计议。”

吴思若点点头算是默认了,但心里还有点不服气:“那这请安你也给她免了?平身两字我可憋了半天了。”
\newline

夜里小五子睡不着,到屋外走一圈,看见远处天边被火光照得亮堂堂的,把耳朵贴在地面上,只听见杂乱的马蹄声和脚步声。小五子将所有的人叫起来,说是有官兵追上来了,众人手忙脚乱地,也来不及收拾东西,先躲进山里。有人想起吴思若白天说的,怎么那么巧,前脚五公主刚到,后脚官兵就来了呢?

众丐帮弟子也跟着起哄,小五子急了,警告所有人,从现在开始,你们有谁再敢怀疑五公主,我就跟他以命
相拼。

山里毕竟不是久留之地,可大家也不知道接下来还能到哪里躲藏,吴思若说:“你还记得当年方丈夸下海口,说把昆仑公子放在这放心,整个武林都打不进来这句话吗?”

“那是他吹呢,少林什么本事,我还不知道啊?一个李准驸就把他们吓个半死,各个去练闭气大法。”

“去试试吧,万一真有高人,方丈说漏嘴了呢?”

隔天到了少林。方丈不让进寺门,一再跟大家解释,现在是香火淡季,粮食短缺,你看我的弟子都在练闭气大法呢,还是请各位施主另投别处吧。小五子道,不就是钱的事吗。浑身上下摸了一遍,转身问皇后有钱吗。

吴思若笑道:“后宫开支紧缩,我也没有啊,公主总有吧?”五公主不说话。向问和慌了,总不至于向丐帮拿吧?

众人既然上来了,一时也不愿下山。那就守在门口。一日一夜过去,大家明白不拿出银票,方丈是不会开门的,可要退下山也来不及了,山下围了一队人马,大概有千八百人,喊着冲啊,杀啊!小五子仰天长叹:“我今日命丧于此!”
\newline

{\centering\subsection{3}}

眼力比较好的一只手,忽然喊道:“这不是李准驸吗?”小五子定下神仔细看,还真是,气得跑过去,一脚把李准驸踹下马来骂道:“你什么玩意儿,整我是吧?”

李准驸赔笑道:“我这不是开玩笑呢吗?”

小五子道:“你看我笑了吗,你仔细看看,我笑了吗?”

“我把银票带来了,”李准驸说,“你当年不是问我,抄文宰相家的那些银子哪儿去了吗?我哪敢回答你啊,都被我······保管起来了。我在南海也保管了不少,那些刁民一个个都想跳海自杀,我沿着海岸线修了一千里长的铁栅栏,谁都不许给我自杀,都给我活着干活!”

小五子把银票都接了过来,问他:“公主不是说你死了吗?”

李准驸转过去质问五公主:“你说我死了?我成全你们俩,让你去找那兔崽子,你说我死了?”

小五子踹他一脚,问他:“说谁呢?”

李准驸不敢回答。小五子问五公主:“当时你是怎
么讲的,你明明说你杀了他的。”

“你想想你当时怎么问的?我能怎么答?”五公主说。
\newline

太平不了几天,一大早的就被喧哗声惊醒,原来是六公子派了五千精兵围住少林,要少林交出人来。方丈过来求小五子他们:“这回不是银子的事了,你们还是赶紧出去吧。”李准驸拍着胸脯跟方丈保证,别怕,我一会儿带人把他们打跑就没事了。下山不到半天,李准驸就丢盔弃甲地跑回来了,边跑边喊敌军太厉害了,人数又多,还吩咐方丈赶紧把寺门关紧,能撑一会儿是一会儿。

方丈气坏了,一群人赖在佛门清净地不走算什么意思,关键是里面还有女人和乞丐,真是太不尊重佛祖了。李准驸提醒他,他们可是给寺里缴了银子的。可那时候不知道你们是朝廷要捉拿的要犯呀,再这样牵连下去,少林寺都没了,还要银子有什么用?

方丈越说越激动,指着小五子问,你说,是不是这个道理?小五子不说话,他又指着李准驸问一遍。李准驸也不吭声了。方丈又冲着殿门外站着的人喊,还有你,你来评评理?小五子最先发现不对,他顺着方丈说话的方向看去,文思清一个人孤零零地站在那里。吴思若也发现了,喊了一声:“文姐姐?”方丈这才明白过来,原来这人不是和小五子一块儿赖在这里不走的,可是是谁呢?身边的小和尚慧根连忙提醒他,就是你从武林大会上带回来的那个女人。方丈说:“哦,是她啊。”其实还是记不起来。

文思清一声不吭,将两张羊皮扔给小五子,转身就走。小五子追上她,文思清警告他:“不要再跟过来,否则我不客气了。”小五子心里一阵酸楚,眼泪都快下来了,只好呆呆站定在原地看着文思清的背影越来越小,最后消失不见。

九张羊皮已经收集到八张,小五子让吴思若帮他把羊皮缝合起来,只差最后一块就齐了。小五子就和吴思若分析,这张会在谁手里?不过多想也没意思,况且大家连这张图是用来干吗的都还不知道,如果是神功秘籍还好一点,万一是张藏宝图,搁手里也没有用,况且方丈现在连银子都不要了。

分析到半夜也没分析出什么结果,躺下才一个多时辰,小五子就被人轻轻摇醒了,睁眼一看是钱老板。小五子奇怪道:“你怎么来的?”

钱老板道:“我趁天没亮偷偷摸上来的,不过我上来
的时候看见官兵陆续在集结待命,怕是马上就要进攻了。”钱老板一边说一边从怀里往外掏东西,“我这次来,主要就是将这最后一块羊皮送过来。”

最后一块羊皮居然在钱老板手上?小五子奇怪了,问他:“你怎么知道这是最后一块羊皮?”

钱老板沉默半晌,正要回答,屋外忽然喊杀声大作。李准驸跌跌撞撞冲进来报信,说不好了,五千精兵攻进山门了,咱们赶紧想办法突围吧。小五子叹一口气,说你听外面这喊杀声,恐怕远远不止五千人,官兵一定是又有了增援,咱们才这几个人,要如何才能突围得出去?被小五子这么一说,钱老板也仔细听起外面的动静来,他眉头一皱说不对,怎么这时候还有和尚有心情念经?还真的是,一开始这诵经声虽然微弱,却绵绵不断,仿佛就在耳边,而且慢慢地越来越响越来越清晰,而喊杀声却越来越弱,终于安静下去。小五子和钱老板、李准驸出了屋子,只见外面横七竖八躺倒了一地的官兵,全都昏迷不醒。方丈站在空地中央,指挥着僧众把这些昏迷的人挪到一边,把打坏的门窗、佛像清理干净。

“果然厉害,”小五子好奇地问方丈,“你这是什么武功?念念经就能击败这么多军队,简直比乔帮主的狮吼功还厉害。”

方丈瞪了他一眼,道:“废话,我要是有这本事,还用得着求你们自己下山吗?”说完他摸摸自己的后脑勺,自言自语道:“这是咱们少林的功夫吗?”

负责清理的净空忍不住了,他把身上背着的人扛到墙角放下,还伸脚踢了两下把人摆正,炫耀道:“施展这功夫的高人在后山,是八光师兄的师姐的师父,等于就是我师父了。”
\newline

小五子带着吴思若和五公主,在藏经阁前面跪了几个时辰,终于从里面传出一个声音:“请门外的沈公子进来。”

沈公子就是小五子了,他推开藏经阁的大门走进去,就是当年文思清走过的那条路。见了沈老前辈他再次跪倒,说感谢前辈救命之恩,不知这是何种功夫,如此不可思议。沈老前辈说这是无经咒,不立文字,见性成佛,至于救命之恩什么的,沈公子不必放在心上。

小五子随后奇怪了,打从进少林寺以来,人人都叫我小五子,你怎么知道我是沈辟朝?沈老前辈缓缓说道:“贪恋红尘也惭愧,但总得让你知道,我就是你的太爷爷。”

沈老前辈给他讲了个故事,当年有个皇帝,被敌人兵临城下,眼看皇位不保,竟然做出许多荒唐事。这皇帝逼他的儿子即位,又自己为自己办了一场假国葬,从地宫逃跑出去。不用说小五子也知道沈老前辈是在讲他自己—直至他的儿子亡国,他从地宫返回皇宫,抱出当时的皇后,以及孙子沈志基,他开始精研武功,独创了几套功夫教给三个弟子以及自己的儿媳妇百花谷谷主。由于他只想着励精图治复辟王朝,缺失了对弟子的管教,使得师门不和,令他灰心丧气。沈老前辈又说他一生做错过两件事:第一件事是当年自己贪生怕死,弃了朝廷,弃了天下;第二件事便是,惭愧没好好教育四个弟子。晚年,他收了向问和做他的关门弟子,用十年的时间教会他无为神掌的诀窍,希望能弥补以前的过错,接着便出家闭关直到现在。

小五子问:“那向问和就是您最后的弟子了?”

沈老前辈说:“还有一个弟子,学了不少真本事,她叫文思清。”

小五子明白了,难怪当年向问和说那些公主的死因,大部分都是沈老前辈的手法。也就是说,文思清学了沈老前辈的功夫,杀了一十三位公主。小五子问道:“既然太爷爷你知道地宫之路,武功又已经大成,能不能带我们打回皇宫?”

沈老前辈沉默不语。

小五子又问:“那个九宫图怎么看,总得告诉我吧?”

沈老前辈还是不说话,小五子跪久了,偷偷抬起头观察,才发现沈老前辈已经坐化。
\newline

{\centering\subsection{4}}

自从六公子的精兵在少林被沈老前辈的无经咒击溃,各地流言四起,说那一天是佛祖显灵,预示本朝气数将尽,会有人取而代之。李准驸这几年以寻找太子为名游山玩水,从地方官员那里没少捞钱。趁着这机会,他拿出毕生搜刮来的金银财宝招兵买马,凑齐十万大军,还不等操练,就浩浩荡荡地向皇城进发。

六公子坐在朝堂上,右手支着头,听取各地信使的汇报。一个多月来就没几条好消息。一开始还说敌人都是乌合之众,不成气候,不出三日必将匪首捉拿,押送京师。三天后就说讨匪大将军轻敌冒进,中了圈套全军覆没。过一个礼拜又说有不少守军受流言蛊惑,纷纷改旗易帜。后来六公子听得火了,杀了几个总是汇报坏消
息的信使,这下没人敢说话了,信使要上殿前不是装死就是诈病,要不就编一些无关紧要的新闻试图蒙混过关。

早朝的时候六公子问官员,关于战局还有什么良策?文武百官面面相觑,谁都提不出什么好办法。六公子沉吟许久道:“好在皇城守备还算坚固,当年只靠九门提督的区区人马就能把三王爷的重兵拒之门外,看来至少再撑上一年半载还是可以的。传旨下去,今日开始各部都退回京师拱卫皇城,我要册封皇后。”

殿下群臣顿时议论纷纷,丞相站出来劝皇上不要胡来,大敌当前,应以国事为重。六公子反问:“有谁觉得这场仗我会胜?”

除了少数还在坚持拍马奉承的官员,大部分的文官武官,此时只有沉默。

六公子道:“看来败局已定,你们任何人有疑虑,我即刻批你们辞官返乡。但是我六公子当上皇帝,把这件大事办成了,却没能娶到乔文君,就白活这一回了。”
\newline

九宫图拼好之后,上面什么也没有,一片空白。小五子用尽办法,火烧,泡酒,用血洒上去,往上刷米汤都不管用,只是一张缝好的普通羊皮而已。小五子没事就盯着那张九宫图,茶饭不思,吩咐谁都不要来打扰自己。有一天李准驸冒冒失失闯进来,小五子把他一顿痛骂:“都说了不准进我屋里,你能有什么要紧的事情?”李准驸一脸委屈,瞅瞅地上的图,抱怨道:“还以为你真有什么了不得的秘密呢,不就是一个地宫图吗,有什么好看的?”

小五子警觉地问道:“你凭什么说它是地宫图?”

“你也不想想我是干什么的,我是九门提督啊,天天就守着这个啊。但是我就是感觉它像地宫,里面什么样我从来没走过,据说里面各种机关暗器,稍微走不好,可能全军覆没,都不用埋,直接死地底下。”

小五子问他:“这就是一张羊皮,什么都没有,你怎么看出是地宫图的?”

李准驸把羊皮要过来仔细一看,说:“原来这是缝线啊,我刚才眼花了,以为是画的呢,当我什么都没说,那肯定就不是地宫图了。”

李准驸说完出去找别人了。小五子盯着这张羊皮,专门看缝线处,他忽然明白,这些缝线就是通往宫中的路线。一时眼花,这些缝线慢慢有了颜色,整张羊皮仿佛着了色的一幅画,展现在小五子面前。他一下子明白
了,这些羊皮不是随便扯的,他们就是按着线路裁下来的,散落在江湖,等待有朝一日有人能重新凑齐他们,入主皇宫。
\newline

攻城那天,李准驸作为开路先锋,骑着马躲在大军的后面。按图索骥穿过地宫,大门缓缓打开,兵不血刃就进入了皇城。一眼望去,皇城里面没有一兵一卒,却是三三两两地横着数十具尸体。李准驸后面一声大吼,让大军让开一条路,冲到最前面,假模假样地对着尸体一阵砍杀,看看又觉得有些熟悉,怎么这些尸体都是以前当官时的同侪?李准驸收剑入鞘,左看看右看看,叹息道:“张大人,王大人,我过去没少给你们塞银子啊,还想着以后你们能提携我呢,怎么你们反而死到我李准驸前面去了?”

皇城已被攻下,眼看皇宫就要沦陷,六公子的册封大典却要如期举行。九路大军浩浩荡荡地攻进皇宫,越来越近,六公子在奏乐声中搀着乔文君走向皇位。皇宫大门顷刻被撞开,起义军蜂拥而至,进来后看到皇帝正在大殿进行册封大典。为首的小五子站住不动,打手势要大家停下来,不知道六公子这回唱的是哪一出。

六公子站在台阶上,皱着眉头,对不远处的小五子说:“皇城外守备也算坚固,你竟然这么快就能攻进来?”

小五子道:“你一直想把九宫图据为己有,恐怕万万也想不到,这九宫图不是什么武功秘籍,而是从地宫进入皇城的路线图。”

原来如此。六公子点点头,他算是明白了,造化弄人,自己抢不到九宫图,今天反而要因它而死。

小五子倒过来也问他一个问题,起义军还没攻进皇城里,怎么文武官员就先死了一地?

“一群忘恩负义的窝囊废,食君之禄,到了紧要关头居然各个都想辞官保命。我把他们全杀了。”六公子顿了顿,继续道,“当年你册封吴思若的时候,我可是对你行君臣之礼,而且安安静静地把大典看完,现在我封乔文君为皇后,礼尚往来,总可以吧?”

看起来真的讽刺,皇宫沦陷,改朝换代的日子,变成了一场婚典。小五子让众人放下刀剑,陪着六公子把这最后一场大戏唱完。

册封的礼仪之中,乔姑娘眼含热泪,她等这一天等了许多年。大典即将要结束的时候,乔姑娘走到一个侍卫身前,忽然从侍卫那里拔出一把剑,转身刺入六公子心脏。

小五子他们全都愣住了,起身过去,也不知道是该救还是不救。只见乔文君对六公子含泪道:“我杀了你,是因为我要给我爹报仇。”

说完她将剑从六公子的胸口拔出来,剑尖对着自己,插入自己的心脏,继续说:“可是我太爱你了,我还要为你报仇。”

闹闹冲过来,抱住他母亲的腿痛哭。六公子侧卧在地,奄奄一息道:“闹闹来,喊声爹。”

闹闹结巴半天,只喊了一声父皇。六公子失望摇头,抓着闹闹的胳膊,大喊着:“不要父皇,叫我一声爹!”

闹闹被吓得哭了,憋了半天还是叫了一声父皇。六公子双眼发空,吐出最后一口气,一动不动。

乔姑娘嘴角挂着血,抬头在人群里寻找着。小五子知道她在找自己,上前几步,俯身听她说:“五哥,我求你件事,求你不杀闹闹,把他抚养成人。”

小五子点点头,乔文君对他笑了笑,像累极了似的,合上双眼。

乔姑娘随六公子而去。一场变故却还没完,远处一个小太监急匆匆跑过来,见六公子死了,也不知道接下来会是谁手握大权,只好扯着嗓子朝天大喊:“嘉和皇帝醒啦!”

赶到寝宫时候,嘉和皇帝还在咳嗽,咳嗽完他拿丝绸手绢擦擦嘴角,瞅了一圈,指指小五子,又指指五公主,就这俩人他认识。然后对小五子说出醒来后的第一句话:“沈辟朝,我不是已经批准你带着五公主走吗?”说着他看看宫殿四周的装饰,说:“你们俩在这儿成亲,成何体统!”

小五子疑惑道:“你当时批准我?带着你的五女儿私奔?”

嘉和皇帝纳闷了,不是你提出要带五公主离开的吗?
\newline

{\centering\subsection{5}}

就在那一年的八月十五晚,大火的前夕,小五子本来是想带着五公主跟嘉和皇帝摊牌,把一切都讲明白的。他执意要往皇上的寝宫走,五公主在后面拉着他,说:“我错了,你说得对,我们应该再忍一忍的。”

小五子不管不顾,说自己一天也不想再忍了,他说他就是一个棋子:“我太累了。我想马上就带你走!”

五公主哭道:“你不可能带我走的,跟父皇讲明真
相,你就会死在这儿的!”

五公主还要拉他,小五子使劲推开,问她把车备好了没。五公主说:“备好了,其实我们现在就可以逃掉的,用不着跟父皇讲。”

“敢做不敢当,那不是我沈辟朝。”他在父皇寝宫门前对五公主说,“我如果半个时辰还没出来,你自己赶快逃。”

五公主转身泪眼依依地离开,忽然背后有人跟她打招呼,转头一看是苏子瑶。苏子瑶客客气气地问她:“太子去哪儿了?”

五公主回答:“皇兄去父皇的寝宫了。等他出来,我让他去找你好了。”

苏子瑶笑道:“太子去皇上寝宫,五公主您哭什么呀?”

“可能是刚刚看一个话本,太入戏了吧。”五公主找个借口急匆匆离开了。她怕苏子瑶,也怕她望向寝宫时的眼神,希望她不会从自己的话里察觉出什么破绽。
\newline

到了寝宫,小五子跪在皇上脚下,说道:“父皇,儿臣今天有些事要对你讲,讲完之后,要杀要剐听凭父皇发落。”

皇上说:“我舍得你死,五公主还舍不得你死啊,沈辟朝。”

小五子一愣。皇上从床上坐起来,小五子想起身帮他披件衣服。皇上摆一摆手,让他继续跪着。

“你当我今天才知道?我自己的儿子自己不认识?打我见你第一眼我就知道,我儿子肯定不是你这副德行。实话告诉你,这些我都不能讲,让你来的是常公公。我三十六个女儿,没儿子,这事儿太医肯定掺和进来的,把你们都除掉的话,我那三弟还在虎视眈眈。衣食住行,我就连喝碗粥都有可能被你们毒死。我能怎么办?我一个儿子都没有了,我就得看着你们几伙人作。还好你这孩子本性不错,知道能让我颐养天年,没急着想杀我。我也就认你这个儿子了。你就这么走了,我怎么办?你在皇宫能帮我稳住常公公、太医这些人,三王爷那边也不敢轻举妄动。这样吧,今天晚上你先把公主安顿到一个稳妥的地方,三日后,我发国丧吊唁五公主,然后给我几年的时间,我把他们一个个都除掉,之后封你个南海王西北王什么的,你就去和我女儿幸福生活吧,我呢,太医一除,想办法生个一儿半女,也用不着你来接班。”

一席话间,小五子长跪不起,他没想到自以为的秘密,其实嘉和皇帝早就洞若观火。那就这样吧,再忍忍,只要能和五公主在一起,等上几年又算什么呢?小五子谢过皇上,退出寝宫门口时,听到身后皇上一声闷哼,回头一看嘉和皇帝已经歪倒在床头,苏子瑶面无表情地垂手站在旁边。小五子想过去救皇上,背后又来了一个人,对小五子说:“让开!”

小五子求道:“请你放过他吧。”

那个人道:“放过他?一掌打死他,你来做皇帝不好吗?”

小五子摇摇头:“我不能做皇帝,愿意的话,你就做你的皇太后,垂帘听政好了。”

那个人看苏子瑶一眼。苏子瑶摇头道:“我没有身孕,太子根本不理我,天天和那个五公主腻在一块儿。”

那个人说:“如果你执意如此,我愿意等,等你重头再来。”

那人说着,挥掌打向小五子。这掌法手势小五子以前见过,他绝望地摇头,道:“别给我断魂掌,五公主还在马车上,我不能忘了她。”

那人笑道:“你又不会死,无非是重生一次。”

断魂掌结结实实地击中,小五子顿时倒在地上。她吩咐苏子瑶:“给皇帝一掌,别让他死,也别让他活,让他一直躺着,等太子三年。叫常公公把他带走。”

一炷香的时间,常公公和太医被召唤进来,常公公扶起地上的小五子,吩咐太医不要张扬,先给皇上治伤。

常公公扶着小五子出去,一到外面就被一个太监认出来。常公公借机杀掉他,将脸划花,扔进池塘里。他直奔皇宫大门,遇见一个老熟人,常公公和他攀谈几句,说:“这就是三王爷要的人,帮忙把宫门打开,让我们出去。”

那个人笑道:“常公公眼力果然好,我藏了这么多年,也知道我是三王爷的眼线,我跟你一起走,有福大家享嘛!”

大门打开,常公公背着中掌的小五子出来,这一次他不再是太子,又变回了他的亲生儿子。
\newline

{\centering\subsection{6}}

四年的时间,就像一次轮回。小五子又一次回到皇宫,只不过这次已不是冒名顶替太子的身份回来,而是真正打回来的。嘉和皇帝透过窗户看到整个皇宫拥满
了起义军,问道:“你们都是谁,跑到皇宫来撒野?”

五公主说:“父皇,您已经昏迷了四年多了,中间已经换了三任皇帝了,我都当上一任了。”

“那我那三弟呢?”

五公主回答他没当上。皇上说那就好,现在谁掌权啊?我还能不能回来啊?五公主指指地上的六公子,半个时辰前他是皇上。嘉和皇帝眯眼看了看,说:“有点眼熟,这不是我三弟的人吗?他都当上皇上了,我三弟还没当上?现在还是不是孙家的天下啊?”

皇上背后有一个女声传出来:“今天是,明天就不是了。”

嘉和皇帝一回头,又被击中一掌。可怜他才醒过来不到半天,这回真的死了。嘉和皇帝的身子慢慢倒下去,把原本躲在他身后出掌的文思清现了出来。文思清收了掌,说一句:“谷主请。”

向问和皱眉看着好整以暇从人群里缓步走出来的百花谷谷主,对她道:“自从二师兄、三师兄也死了之后,我知道那个逆徒就是你了。”

谷主道:“你知道得有点太晚了,这样很好,至少没耽误我的事。”

小五子问道:“何帮主一家三十多口人都是你杀的?”

百花谷谷主点头。

“当年我那一掌也是你打的?南海真人是你杀的?阁老和仙人也是你杀的?”

谷主等他一口气全问完,又点了一次头。小五子看看谷主,又看看站在谷主旁边的文思清,她的肚子已经现形,看得出已经有四五个月的身孕。谷主吩咐文思清道:“你去把那个太子杀了。”

文思清向闹闹走去,小五子跨出一步挡在她面前。谷主在后面命令道:“谁挡你的路,你就杀谁!”

钱老板冲过来护住小五子,哀求道:“母后,你毁了我一生,我整条性命都给你没关系,求你放了我的儿子吧!”

谷主根本就不看钱老板,命令道:“我再讲一遍,谁挡你的路,你就杀谁!”

文思清回头看看谷主,推出一掌将钱老板击倒。谷主在后面道:“杀了他,把他们父子俩都杀了,你肚子里的那个才是太子,他们俩不要当皇上,我还要当我的太后。”

钱老板奄奄一息,只求一命换一命。文思清深吸口
气,又下一掌,杀掉钱老板。可小五子还不逃,怎么那么笨呢?文思清的手悬在半空,不知道这一掌要怎么打下去。

谷主又在背后催促:“你别忘了,是谁害你家破人亡?”

于是文思清又试着去想,她的父亲如何被当街处死;她又被转手卖给几个不同的主人,吃了多少的苦。可又有一些回忆总是不受控制地硬是插进来,那些在田独的日子,那一天小五子把她从山上背下来,她在他的背上睡着了,做了一个永远也不想醒来的梦。要是还在梦里该多好,文思清低头看着自己怀着身孕的肚子,忽然间一声大哭,用尽全身力气向天击出一掌,转身掩面而去。

谷主冷笑一声,道:“都是没用的东西,那就只能我自己来了。”

在场没有一个人能是谷主的对手。向问和从始至终不敢多说话,他知道自己武功尽废,无非是白白送死。可他又想起师父对他的期望,当年是如何尽心地教他这招无为掌,无论如何也要誓死一拼,不能再苟且偷生。想到这里他豪气顿生,不由地向前迈出一大步。

谷主讥笑道:“向师弟的无为掌,就不要出来献丑了,无为,无为,无所作为,连你们丐帮都知道。我且饶你一命,你就苟活下去吧。”

向问和道:“虽然我资质太浅,师父的功夫我越学越差,对不住他,但总得接你一掌再死,否则无颜面对三位师兄和师父。”

“那就了你心愿,先杀了你吧。”谷主同时施出三掌,蓬莱掌、仙人掌、断魂掌合力向向问和打去。向问和脸色苍白,只能伸出双臂迎击,却毫无自信。两人对掌半分有余,双方岿然不动。

小五子和吴思若都觉得不可思议,以向问和蚂蚁都打不死的本事,同时接下三掌,怎么能支持这么久还没倒下?就在这时胜负已分,谷主忽然癫笑起来,吐出一大口鲜血,一步步走过去抱起沈志基,疯了一般地说:“孩儿啊,娘对不起你,娘给你唱歌。”

向问和看着自己的双手,终于领悟明白,师父果然英明,这无为掌本不能杀害任何生灵,却可以将任何施于你的伤害原封不动地还回去。无为,无为,无所作为,又无所不为。

谷主在众人注视下,抱起沈志基,踉踉跄跄冲进寝宫内堂,大家跟上去,谷主在帘子后面说:“你们别进来,
我一会儿就好,我一会儿就好。”

等了有一刻钟,一时间没有一个人说话。忽然谷主说道:“我好了,你们进来吧。”

小五子手臂下压,他一个人先进去,过了一会儿其他人见里面没有动静,也陆续进去,只见小五子静静地站在那里一动不动,顺着小五子的视线,他们看到谷主已换上凤袍,怀里抱着身着龙袍的沈志基,枯死在那里,就像一枝插在花瓶里很多天,干枯了的花朵。
\newline

{\centering\subsection{7}}

转眼行德三年,这是小皇帝闹闹的年号。那一次风波过后,小五子坚持皇位该由六公子和乔文君的儿子继承。这一年冬去春来,万物复苏,小五子向少年皇帝闹闹请求,想带五公主和吴思若出关,去田独看一看。少年皇帝看完小五子的奏折,问五公主:“姑姑,你觉得如何呢?”

五公主道:“跟我有关,皇上还是请三王爷给点建议吧。”

三王爷去年已经从南海回到皇城。他整日无所事事,早就没有了当皇帝的念头。说谁走都行,就是别把他再发配走。他连忙说道:“你们说了算,都行,都行。”然后低着头看从南海带回来的乌龟在地上乱爬,自言自语道:“你们可不准回南海,都在这儿陪着我。”

春光三月,小五子带着吴思若及公主二人一路北上,三人走走停停,差不多到夏天才到田独镇。店面和街道都没有变样。小五子指着前面那间破旧的房子说:“那就是以前的钱记肉铺,现在连牌子都没了。”

院子里荒草丛生。一个穿龙袍的小男孩从里面跑了出来,大概三岁的样子,后面跟着一个女人。小男孩说:“娘,大热天的,我不想穿这么多,热死了。”

跟在男孩后头出来的是文思清,但是看上去却和以前判若两人。她看看小五子,又看看吴思若和五公主,好像根本不认识他们似的,只是呵斥她儿子道:“沈定坤,你要记得,你是太子,一定要扭转乾坤。”

三个人看了心里都不是滋味,吴思若说:“要不然跟她商量一下,把孩子抱回来吧?”

公主道:“孩子没了,他娘怎么办?”

小五子叹了口气,说:“咱们谁要是会断魂掌就好了,一切可以重头再来。”

\hfill(全文完) 