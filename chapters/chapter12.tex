\section{拾贰}

{\centering\subsection{1}}

小五子决定上京城,想带走的人不愿意北上,而不想带走的人,却时时磨着小五子,能不能跟他一起走。八光师父就是这样,去不去京城不重要,但他一定要离开百花谷,只要能远离这儿,从谷里出去,上了平地,随便去哪儿都行,回少林寺他都愿意。

这么弄小五子就不明白了。百花谷,顾名思义,全是女人,你要真是和尚,离这儿远点,我敬你一句得道高僧,可你是八光啊,田扒光啊,在这百花丛中,好比鱼儿回到大海,为什么还要走呢?

小五子把八光问住了,他瞪大眼睛,一动不动,过了好半天,才结结巴巴地辩解:“我现在真的是和尚了,就算之前六根未净,来百花谷之后,我完全能做到清心寡欲了。”

八光是抓着索道下来的,中了百花谷的花香之毒,静养了一星期才苏醒过来。刚睁开眼睛的时候,发现自己动不了,这还不算意外,房间里处处透着不对劲。躺在床
上,为什么这么香,连蜡烛都能烧出香薰的味道,为什么床这么软,为什么帷帐都是粉色的,除了自己一动不能动,这一切都似曾相识,好熟悉啊,这不就是他以前经常干的事吗?瞄准一个目标,锁定她家的位置,等到华灯初上的时候,再从窗口溜进去,这就是女人的闺房啊。

他听见门口有人笑,两个女人的笑声,也不知道头一个讲了什么,那么好笑,另一个姑娘笑个不停,最后竟然笑得上气不接下气,发出了娇喘之声。隔了一个帷帐,加一扇屏风,再加一道门,八光都能看出来,那姑娘花枝乱颤的身影。

笑声越来越近,两个姑娘甚至推门进来了。八光意识到自己只有两个小指能动,他咬牙屏息,两个小指发力,硬生生地把自己半个身子撑了起来,向后靠在床头,大口喘着气。两个姑娘说着话,收起屏风,走到床前,看到八光满头大汗,那一副无助又惊恐的样子。

“八光师父,你醒啦?”

“你赶快休息,”另一个说,“我们来帮你上药。”

她们俩赶紧跑过去,各抓着他的一只脚,把八光从床
头又拽了下来。那可是用两个小指,一点点蹭上去的,感觉就像是一只乌龟爬了一冬天,终于从院子的这一头爬到那一头,过来一个孩子,简单粗暴地把乌龟又拎回去了。

他八光也有这般无力的一天,也许是死了。脑子里有这种念头,两个姑娘的样子也变得模糊起来,声音全都听不真切,嘁嘁喳喳的。头一个女孩打来一盆水,另一个女孩将一壶药液倒进去,投好毛巾,先擦他的脸,擦耳朵,擦脖子,然后把扣子解开,脱掉他衣服裤子,头一个女孩又拿出一条毛巾,两人一上一下,分工明确,像擦桌子一样,在他身上抹了起来。可他毕竟不是桌子,看着面前的女孩,大口喘气,大口咽口水,一时间所有的羞耻感都浮上心头,急得晕了过去。

再睁开眼,还是在床上,看窗外天已经黑了。他检查一下,不只是小指,整只手都可以动了,脖子也能扭了。两个女孩眨巴着眼睛看着他,问他:“八光师父,好些了没?”

八光皱着眉,这次是两只手可动,发力将自己撑起来,打量这两个女孩,问道:“我是活着,还是死了?”

“当然是活着,”一个姑娘说,“如果你死了,我们还有必要,天天给你上药吗?”

另一个姑娘只是笑,口中回味着:“怎么会以为自己死了?”说两句又愈发觉得好笑,一时半会都没法停下来。

八光想起来了,刚才就是这姑娘,花枝乱颤,从头笑到尾,白天还以为是多好笑的笑话,原来是笑点真的低。八光问:“这是在哪里,我躺在这里有多久了?”

头一个姑娘说:“八光师父,这里是百花谷啊,你中了花毒,已经有一个星期没下床啦。”

他想起来了,知道自己莽撞,贸然入谷,才有如此下场。八光点点头,也不知道说什么,只说:“你不用每句话,都叫我八光师父的。”

“知道了,八光师父。”

之前笑点低的姑娘笑够了,这句话倒一点没听出好笑,她拿起一个药壶,说:“差不多了,我们得帮你上药了。”

八光连忙做出打住的手势,让她等会儿,问她是什么药。

“八光师父,这是解花毒的,防止你全身麻痹而死,”不怎么笑的姑娘说,“我们把毛冬青和威灵仙捣在一起,再调和薄荷脑和樟脑,混在清水里揉搓擦拭。”

“每天都搓?”

“当然要每天都搓,一天还要三次,每次要小半个时辰呢。”

“都搓哪里?” 

“就搓你啊,还要搓哪里?”笑点低的姑娘把话接过来,自己笑起来。

“我全身都搓?”

头一个姑娘点点头,说:“八光师父,你这状况是麻痹全身的,我们肯定不会漏下任何一个部位。”

八光还有话要问,想了半天,却实在问不出口了。爱笑的姑娘打来一盆清水,药液倒进去,投着毛巾说:“问那么多干什么,正好你现在醒着,我们擦拭一遍给你看,不就完了吗?”

说话间,爱喊八光师父的姑娘,又拽起他双脚,把他拖回床前,脱下他衣服。身上几乎全裸的时候,他无力反抗,大口呼吸,任凭她们在身上擦拭揉搓,任凭自己老迈的躯体展现在她们面前。前后持续半个时辰,两个姑娘有说有笑,好像面前的不是他的身体,而是一张桌子一面墙,完全是在劳动。八光仰躺在床上,看着棚顶,眼泪从眼角冒出来。几十年的淫贼恶名,即使在少林寺呆了十几年,也未能铲除邪念,而此刻,就在这两个姑娘面前,一个爱笑,一个爱叫八光师父,在她们摘果子、挤羊奶一般的劳动氛围中,他的淫心却彻底戒除了。
\newline

{\centering\subsection{2}}

一只手当然要在一起,欠了那么多条命,跑了找谁索命去?主要是他自己要跟过来,江湖险恶,这个五帮主,又是昆仑公子,又是少谷主的,那么多人要取他性命,他倒是福大命大,一路活到现在,别说是被剁手,连根手指头都没掉。吴思若和文思清也一起上路,先别问选谁,走一步看一步,大不了在京城将事情料理完,就近找个地方出家,小五子想。

临出发的时候,小五子去见了钱老板。两人一时没话,钱老板叫人准备晚宴,说是要给小五子送行。小五子让他别忙乎,他们一会儿就要出发了。“而且,我不是来跟你告别的,”小五子说,“我是来叫你,跟我们一起走。”

钱老板干笑,那种一看就是在朝廷混了几十年的假笑,他说:“少谷主,你就别逗老夫了,我一把年纪了,在百花谷混混还行,偌大的京城,可不是要把我走迷路喽?”

小五子没说话,叹一口气,斜眼看着他。知道你在敷衍,可你稍微,稍微认真一点儿敷衍啊,你说你在京城会迷路,跟我在这儿装乡巴佬,可你是常公公啊,你在京城呆了二十多年呢。钱老板也反应过来,自己这谎撒得不接上下文,他躲过小五子的眼神,转头往门口看,找话题说:“这
已经快五月了,怎么风还这么大,按理说该下雨才对,结果刮着风,大太阳还顶在头顶。”

他说了半天,小五子也不接茬,他转回来,看着小五子,有些为难地挠了挠头,“唉”了一声,说:“算了算了。”最后他冲厨房的方向喊了起来,告诉他们:“别准备了,没有饯行饭,少谷主一会儿就走!”然后他顿了顿,尖着嗓子喊道:“我跟他一起走!”

以前他不说话,装哑巴,他那嘴巴除了吃饭,也没见过干别的,这回从钱老板变成了常公公,吊着嗓子说话,小五子才知道,他这么爱演,原来嘴这么碎。一路上嘴没闲着,吃饭的时间都没有,有时候小五子都担心,他这张嘴,一直说话不进食,别饿死在半道上。

他们四月初七出发,小五子,文思清,吴思若,一只手和八光,以及嘴碎的钱老板,分坐三辆马车。当晚他们就出了南京,不到十天就出了江苏,然后再往北进了山东,经过泰山、德州,转眼就进了济南府。本来是不打算多待的,住上一两天,周边转转就直奔京城。

可能是济南府太好玩了,大大小小上千个泉眼,就算是周边,一时半会儿都转不完,再加上天下菜系出鲁菜,吃一顿就走,大家都觉得不过瘾,小五子决定多待两天,反正向老帮主也不知道躲在北京什么地方练功,与其在京城瞎转乱找,不如在济南府先玩个够。

旅行是别人的事,要说小五子最喜欢的,还是去赌两把。第二天,他就摸到了当地最大的一家赌馆。走到门口,他却有些怯了。他想起苏子瑶了,想那时自己不敢下去赌,让苏子瑶替他赌了半宿。他不能再碰这个了,哪怕只是为了悼念苏子瑶。可他又舍不得走,站在门口,听里面哗啦哗啦的骰子声,那简直是人生最美好的音乐,声声打在他心上。

他回去找一只手,拿出赌本,叮嘱他不许出千,替他去玩几把。一只手伸出他左胳膊,小臂到头,光秃秃的也不见手掌。这几年怕人嘲笑,一只手永远都是左手插兜,不让人看出来自己哪里有问题。他对小五子挥了挥左臂的袖子,说:“五帮主,我都这样了,人家不耍我,就已经烧高香了,我还能出什么千啊?”

他其实不想玩,跟小五子恰恰相反,一只手一点赌瘾都没有,当初在田独,不是为了赌,是为了赢钱骗钱去的。一只手被拉到赌场,满脸的不情愿,小五子说:“你进去不要贪心,不多玩,就押十把,一把二十两银子,前五把跟庄,后五把跟闲。”他拿出二百两银子,再次叮嘱他:“老老实实的,只押十把,不涨注,也不减注,每次只用二十两,看你一
会儿出来,是四百两,还是一文都不剩。”

一只手脸上没有一丝兴奋,木然地点着头,进去了。小五子在外面听声数着,每次开骰盅,都有人兴奋欢呼,有人发火骂娘,各种声音一起冒出来,唯独听不到一只手有什么动静。以前真没看出来,小五子想,他有这么淡定。一直数到第十把开骰盅,一只手出来了,面对面也看不出是喜是悲。小五子问他:“怎么样,输了赢了?”

“输了。”

“输了几把,还剩多少?”

“全输了,”一只手摊开右手说,“一两都没剩。

”小五子皱着眉,转着眼珠问:“十把全输了?”

“对,十把。”

“每把二十两?”

“对,就照你说的,不押二十一,也不押十九。”

“是前五把押庄嘛?”

“嗯。”

“后五把押闲?”

“没错。”

“那不应该啊?这么巧?”

小五子有些怀疑,审视着他,不自觉地上前去搜他身。在他衣服上拍两下后,一只手来情绪了,大声质问他:“是不是觉得我没押,一把没玩,直接把你银子匿下来了?”

一只手甩开他胳膊,转身往大路上走。小五子跟在后面,反而不好意思了,跟他解释自己错了,道理他才想明白,真要骗他钱,也得剩个二十两、四十两给他,一两不剩,这有点假,肯定不是骗子能干出来的事。“可是,十把全输,这个更假嘛,而且还是换着押的!”

一只手停下来,表情跟小五子一样困惑,说:“我是想骗你钱来着,赢了多要点儿,输了就少拿点儿。可是我也不曾想,开十把骰盅,一把都不中啊。”

这应该是真话了,小五子看着他笑了,问他:“你本来要骗我多少?输了你也拿,要不要脸?”

“输了,我就少抽点,肯定给你剩二十两。”

“如果赢了呢?”

“给你剩四十。”

“哦,你赢钱,最后还算我输一百六?”

两个人说完哈哈大笑,接近中午,他们二人找酒楼吃桌上好的酒席。一只手问他:“输了那么多,你还能吃上好的酒席,你是把棺材本都用上了吧?”

小五子愣了一下,回答他:“还真是棺材本,只不过这
是劫来的棺材本。”

本来他想讲,当初怎么被大漠仙人和蓬莱阁老挟持,怎么劫了一个丧葬队,怎么把他关在棺材里,就在棺材夹层处,他发现了这三十多根金条。可是一只手不打听,眼神飘忽在想事情。店小二每上一道菜,还报一次菜名,不一会儿,整个桌子都摆满了。小五子拿起筷子,对齐在桌子上磕了磕,跟一只手说:“别想了,吃饱了再说。”

可一只手还在想,想了半天,告诉小五子,他说他刚才在赌场,对面一直坐着一个人,两个女人坐他旁边,左拥右抱的,看起来是个当官的,一个叫他李大人,另一个又叫他李驸马,没准还真是娶了公主,成了驸马爷。

“驸马爷调戏民女?”

“那不重要,主要是他也在押,一次押五百两,跟我押的刚好相反,”一只手说,“我是先五把庄,再五把闲,他是先五把闲,再五把庄。”

“你要说什么呢?”小五子问。

“我要说的是,他十把全中了,而我十把全赔了。”

“所以呢?他五百两,你二十两,你觉得他在弄你?”

小五子夹着菜,嘴里咔嗤咔嗤的,他示意一只手吃东西,这事就算过去了。但一只手不甘心,筷子都不拿,努力回想,似乎要把赌场的十把骰盅全过一遍,最后得出结论说:“有人在帮他捣鬼。”他身子前倾,看着小五子,“他看起来一个人,其实不是,后面那些押注的,看热闹的,跟着起哄的,其实都是他的人。”

“他不是左拥右抱吗,怎么又一个人了?”

“不是这个意思,我是说,他不该单枪匹马地来啊。”小五子对他笑笑,摇着头,不知道一只手怎么了,钻到里面出不来了。

“你想啊,他是李大人,驸马爷,怎么可能自己跑过来赌?况且我明显能感觉到,这些骰盅摇骰子的时候,有人动过的。”

“怎么动?”

“吹气,从桌板下面震桌子,用暗器击打骰盅,反正他们都是高手,能用的手段,比我们当年在田独的,高明多了。”

小五子撇撇嘴,任由他讲述赌场里的各路神仙,一桌子饭菜被他吃了一大半,最后他拿起毛巾擦擦嘴,下楼结账。走出酒楼门口的时候,他说:“说得这么神,那就去看看?”

“可是,”一只手为难起来,“我还没吃午饭呢。”
\newline

{\centering\subsection{3}}

一进门,小五子就看见他了,坐一红木椅子上,那两个女人果然在他旁边,一口一个“李大人”的叫着。小五子示意一只手别过去,找个角落的桌子观察一下。确实是他“一个人”,旁边的几个人都不正常。拿扇子的,扛镐的,好像真是传说中的“渔樵耕读”。还有他屁股下面的那红木椅子,跟别人的都不一样,两侧带扶手,驸马爷跟个太师一样地靠在上面。

小五子拿出一沓银票,交给一只手,跟他说:“你先上去,跟他反着押,等你输光了,我再过去。”

他让一只手快去,自己坐在原地看着。显然,李大人不记得一只手之前玩过,或者是不在乎,除了中注收银子,眼里面没有任何人。荷官先摇骰子,举着骰盅在身前转了几圈,放在桌上,告知在场的人开始下注。绝对没错,那几个“渔樵耕读”全是在“看热闹”,没一个往上押的。一只手离老远看看小五子,不想暴露,小五子故意挡住脸,往别处看。直到一只手抽出一张银票押上去,小五子看回来,荷官等了一会儿,差不多的时候,喊了一声:“买定离手!”

他摇了一下骰盅上的铃铛,掀开罩子,把点数展现给众人。有人欢呼雀跃,有人捶胸顿足,那几个假冒的书生农夫也跟着起哄,可是你们一文都没下嘛。李大人赢,从太师椅上起身,把桌上的钱全都揽在怀里。两个女人叽里呱啦,说你这把赢这么多,也不分我们姐俩一点儿?李大人不高兴了,怒斥她们:“钱都给过了,还好意思跟我张嘴要,叽叽歪歪!再多嘴,把你俩卖进青楼去!”

两个女子叽叽喳喳地说:“李大人,我俩本来就是青楼女子啊。”

李大人挠挠头,想说点更狠的吓唬她们俩,指着对面的一只手说:“再说话,把你俩从青楼赎身,就卖给他!”

两个女人吓坏了,有一个看两眼一只手,吓得哭了出来,另一个年长些,一声不敢吭。没人说话,摇骰开盅也来得快一些,转眼一只手又连输三把,加上李大人旁边的女人还在哭,他心情急了起来。五帮主说的,把这点银票输完,他就过去。他数数手里的票子,还有七八张,一股脑全拍在桌上。李大人摸了摸银票的厚度,冲他笑了,说道:“就这点儿了吧,要不然你跟我一起押?”

“不必,我偏爱跟你反着来。”

“那我押闲。”

李大人嘴上说押闲,却把银票放在了庄上。一只手拿起银票,正要落闲位。这时候小五子得过去了,在一只
手旁边拽把椅子坐下来,假装不认识他,跟一只手说:“这位兄台,赌牌摇骰子嘛,最怕是有牌气,有骰子气。要不然这样,我们合作一下,不瞒您说,我对骰盅其实研究很多年了,略有所通,可惜一直没有赌本,我今天就拿你这几张银票做赌本,赢了钱,咱们二一添作五,怎么样?”

一只手拿着银票,瞪大眼睛看着他,估计以他这样的智力,得反应一会儿才明白,小五子是在跟他装不认识,他要配合着把这场戏唱完。一只手想了想,中间还皱了皱眉,最后说道:“不行,你爱找谁找谁去。”

小五子满脑子问号,心里说了一万遍,在座的谁帮个忙,帮我把这货打死。他冲一只手眨巴着眼睛,陪笑道:“给个面子,哥,借小弟一半?”

一只手轻蔑一笑,还哼了两声,说:“没面子,谁认识谁啊,跟谁套近乎呢?兜里没钱,你他妈凑过来干嘛和牙?”

小五子深吸一口气,抬屁股把椅子挪离一只手远点。他手伸自己怀里摸,就一点儿碎银子,一张票子都没有了。那点碎钱加起来没一两,他还有些自嘲地问荷官:“这点儿不让押,是吧?”

荷官没说话,一只手抢过来说道:“别说押不了,没钱,你就不该坐这儿。”

小五子又摸摸怀里,后悔没带杀猪刀来,不然就把他另一只手也给剁了。起身要走,又不甘心,他只好坐在桌前搓着双手。

折腾一圈,一只手反倒冷静了,手头的银票,不但不全下,还抽出两张,让人换点小票子来。这是在磨小五子呢,一次押个二三两,看他能在这儿坐多久。小五子坐在旁边,搓了几把手,也不在乎谁赢谁输了。他起身准备先撤,背对赌桌的时候,李大人在后面喊了声:“少侠,请留步。”

小五子站住不动,看看在场的所有人,起身的就他一个,况且,全场他最年轻。但是,少侠?打从记事起,从田独到济南府,还没人这么叫过他。他转回来看着李大人,指了指自己,问他:“是我吗?”

李大人点点头,伸手示意他先坐。小五子慢慢坐下来,眼睛不离开他片刻。李大人拾起一沓银票,推到小五子面前,笑道:“少侠若有雅兴,尽可拿属下的银票去玩。”

他左右的两个女子不干了,头一个也不哭了,说:“你好偏心哦,我们在这儿陪着你傻坐了一天,竟没有一个陌生人拿的钱多!”

头一个连撒娇带抱怨,另一个就放肆多了,握紧拳头,要去捶打李大人胸口。一时间李大人羞得满脸通红,
连喊几声:“住手!”

青楼女子才不管这个,一个捶胸,一个去拿赌桌上的银票,赌馆似乎成了窑子。混乱中李大人清咳两声,小五子能听出来,这是个暗号或是指令。果然,之前拿扇子的那位丑公子从后面蹿出来,一手一个,抓着两个姑娘的后脖颈,把她们拎出赌场。

李大人看着他们出门口,确定没人再烦他,就开始盯着对面扛镐的农夫,也不咳嗽,扛镐的一开始没领会,盯了好半天,才知道你是在叫我,那就暴露吧。农夫走到李大人身前,放下铁镐,躬着身子等李大人指示。李大人低声
跟他说了几句话,农夫点头说,明白,起身也离开了赌馆。

农夫都出去了的时候,小五子注意到,他把镐忘在了桌前。反正暴露了,也不用装种地的了,可是李大人要他到底去干嘛?他冲小五子微笑,说道:“那少侠就收下银票,咱们玩上几把?”

小五子拿过来银票,扫了一眼,五百一张,差不多小一万两。他抬头看看,银票给了他,李大人桌前是空的了。小五子笑着说:“那怎么行,钱都给了我,你拿什么玩啊?”

李大人怕他有顾虑,弯腰拽了一下太师椅的坐垫,原
来下面还有一个暗盒,抽出来厚厚一沓,全都是大数额银
票。虽说是驸马爷,可你这是把国库搬来了吧?李大人
拽出十几张,说:“您尽管放心玩,属下这边还有的是。”

为什么自称属下呢?小五子打量着他,基本可以确
定了,李大人认识他,以前认识,没中断魂掌的时候,可是,
你认识的是昆仑公子,还是百花谷的少谷主呢?

“那我拿您李大人的钱玩,赢了怎么办,输了又怎么
办?”

“赢了,您尽管拿走,如果您不小心输了,属下再给您一些钱,做回去的盘缠,就当是属下攀高枝,跟您交个朋
友。”

言必称属下,那就先玩着。李大人抬手让荷官摇骰子,骰盅放下,众人买注。李大人问他:“少侠想押哪里?”

刚一上手,也听不出骰子点数,小五子抽出一张五百两的银票,押了个闲。

“那我就押庄,免得你中了,没得抽。”

同样五百两,李大人放到“庄”字那一处。一只手这回打算多押,拿三张二十两的银票在算,一阵思量,他决定跟李大人走,把银票放在他五百两的上面。

荷官喊着,买定离手,然后摊开双手,给大家看一眼,自己手上没活儿,拨了一下骰盅上的铃铛,揭开罩子,所有
人看过之后,大声喊了一句:“闲中!”

在场的全都跟着李大人押,这是他今天第一次输。小五子反倒兴奋不起来,转着眼睛思索,问题出在哪儿?

“愿赌服输。”

李大人笑着把银票推过去,这笑容有点假,小五子想了有一会儿才绕明白,他在假装强颜欢笑,就是说,他输得可开心了。第二把还是输,第三把也输,既然李大人能故意赢,自然也能故意输。小五子全听过一遍,摇骰子没问题,放骰盅也没问题,大家押注的时候,没人碰过那东西,荷官喊过“买定离手”,摊手给大家看,手是干净的,然后他拨了一下骰盅上的铃铛。停!这个铃铛有问题。

小五子伸手示意,先别揭罩子,他把押注的银票拿起来,跟对方商量:“李大人,这样,我改主意了,我们两个对调一下,怎么样?”

李大人哈哈哈的假笑,说:“少侠说了算,属下怎样都好。”

小五子把两张五百两的票子换了位置,让赌局继续。李大人冲荷官点点头,荷官又要去碰那个铃铛。小五子抓住他手腕,叮嘱他:“别碰铃铛,直接揭罩子。”

荷官为难,手停在骰盅上方等待指令,直到李大人又点了点头,荷官揭开罩子,喊了声:“庄中!”

果然如此,玄机就在铃铛上,第四把小五子终于输了。李大人收下银票,一副假开心的样子,还冲小五子竖拇指,说道:“少侠果真了得,这么多把,才让属下侥幸赢了一次。”

难过的是一只手,前三把都跟着李大人押,一直输,第四把看出门道了,要么是五帮主厉害,要么是李大人故意输,反正他改跟五帮主,把钱全投进去,这次却一把输没了。接下来换他搓手,搓完手心搓手背,连看好几把,鼓起勇气,跟小五子商量:“少侠,借我点儿银子使。”

“你谁啊?”

“我?”他看着小五子说,“我是一只手啊。”

小五子拉起他左臂,撸下来袖子,没有手的手臂萎缩得像个小拳头。小五子跟没见过似的,大惊小怪:“还真是一只手,哪儿去啦?”

“那个,被你剁了。”

“开什么玩笑,我都不认识你。”

小五子说完,就不再理他,继续跟李大人赌。一只手跟挂在阳台上的咸腊肉一样无所事事,在旁边看了几把,默不作声地起身走了。

一只手走后,小五子又玩了半个时辰,他知道李大人
有问题,冲他来的,只是想不通,为什么要故意输钱给他?反正先赢着,有钱拿,他也懒得戳穿铃铛的问题。只是寄望于一只手聪明点,那么明显的事情了,把他支走,就是要他回去搬救兵嘛。

半个时辰过去了,一只手不见,一个时辰过去了,还是不见他带人回来。倒是之前出去那农夫回来了,把立在赌桌旁的镐扛起来,跟李大人说:“事情都办好了。”

什么事情呢?小五子跟着李大人,往门口看过去。外面多了个侍卫队,十几个人的样子,背对着大厅守在门口,透过人缝,还能看见一顶红轿子停在街面上。又是这一套,他小五子太熟悉了,点了穴,或是绑了人,把他塞进轿子里劫走。

可这把还是赢,小五子把钱揽过来,抽一张五百两银票,随便押个庄,押个闲,一张一张数着桌面上的盈余,一厚沓银票揣进怀里。荷官摇骰子,放盅,喊注,拨下铃铛,开骰盅,管它庄中,闲中,想都不用想,小五子中。流程走完,小五子突然起身往外走,头也不回地边走边说:“这把不要了,改天请你吃饭。”

话还没说完,他已经走到门口了。十几个侍卫跟一群石狮子似的,双手背过去挡着道。小五子装作跟自己没关系,怼着胳膊肘往外挤。居然没人要拿下他,还真被他挤出去了。他再往前走几步,绕过红轿子,拿扇子那个丑男人,不知道从哪冒出来的,一下子挡在了他面前。小五子想想,转身往回走,扛镐的农夫又一次挡住他。

李大人小跑着出来,满脸热情地说:“这位少侠,你急什么,咱们先吃个饭再走。”

小五子看着他,质问道:“怎么,李大人,赢了钱,就不让我走吗?”

“哪里哪里,就是觉得少侠一表人才,属下想拉着少侠,去我那里小叙一下。”

“改日再说!”

小五子转向找出路,自然又有人挡面前。看样子跑不了了,束手就擒吧。奇怪的是,他们又不抓他,只是背手挡着,一步步地逼着小五子后退。他再转身,其他人也围上来了,众人把小五子围成一个圈,只留一个豁口是通往轿子的。小五子先顺着退几步,再试试脚下不动,这些人就像移动的墙,用胸膛顶着小五子走。要不然打一下看看呢?反正要是被掠走,按李大人这种把人卖到青楼的喜好,他小五子往后也没什么好事。

小五子深吸一口气,右手握拳,一拳打在面前的侍卫身上。他没武功,自然也谈不上内力,这一拳打下去,能不
能打实都不知道。这时奇迹发生了,被打的那个侍卫“啊”的一声惨叫,飞出去十几丈远,结结实实地摔在地上。小五子惊到了,抬手看着自己的手心,这是怎么了,神龙附体吗,还是吃了什么大力丸?

他再试一次,这次换左手出拳,第二个侍卫飞得更远,叫得更惨。他两手摊开一起看,一定是百花谷,他想明白了,在那里住上十多天,闻着沁人花香,相当于别人苦练十多年。感谢谷主,小五子微笑着想,从此以后,我也不讨厌你常公公了。

“那就对不住了!”

他一下子信心爆棚,握紧双拳,一拳一个,将两人打飞,转身去打身后的几个人。只是他们人太多,每打倒一个,就会有新的人进来补位。但这很过瘾,小五子跟踩了风火轮似的,在人群里忽左忽右,一挑三四十人。他一边出手,一边咆哮:“还有谁?”

可是人怎么打不完,打倒那么多,剩下的还是能把他围成一圈,似乎还越打越多。直到一拳已经打出一半了,他忽然停住,看着侍卫的脸,说道:“我见过你。”

侍卫懵在原地,踮着脚尖,不知何时跳出去。

“我打飞过你,是不是?”

侍卫点着头,有些含糊地“呃”了一声。“我打飞过你几次?”

“这次能飞,就是第八次。”

小五子放下拳头,叹了口气,转半个圈,把每个侍卫的脸都看过一遍。确实,个个都被他打飞好几次,他要面前的侍卫让让,看着周围的街面,打飞那么多,现在地面一个人都没有,就上一个打飞掉的侍卫,正拍着屁股上的尘土,一路小跑着,往这边赶呢。

小五子在人群里找到李大人,皱着眉问道:“你到底是什么人,谁的驸马爷,摇骰子陪着我输,在这儿陪着我打,你到底在跟我玩什么?”
\newline

{\centering\subsection{4}}

一只手确实傻,完全不知道小五子身处险地,看不出李大人另有所图,一路骂娘走回到住处。刚好赶上他们在吃晚饭,文思清跟客栈借的厨房,做了一桌子饭菜。他找副碗筷,盛满白饭,坐下来跟他们一起吃。

中午就没吃,光琢磨赌馆作弊的李大人,饿了一下午,胃口大开,三口两口吃掉一碗,他端着碗去厨房转了一圈,回来说:“锅里没有了,哪儿还有饭?”

吴思若看着他,越看越不对劲,问道:“你自己回来的?”

“是啊,怎么了?”

“小五子呢?”

“他有钱不借我,我又没钱,我就自己回来了。”

“以后你跟我说话,带上师姐两个字,记住了吗?”

“记住了。”一只手心不在焉地回答,忽然又想起些什么,补充道,“师姐。”

文思清挺好奇的,跟他打听:“你们去哪儿啦,他不借你钱?”

一只手本来要说赌场来着,脑筋一转,觉得可以小小报复一下,他说:“逛窑子呗,不然什么地方还用的着花钱啊?”

吴思若问他,哪家窑子?一只手也不知道济南府哪家青楼有名,他先说虚的,说里面姑娘好看,一个个可有风情了,门脸还特别大,镇宅的东西也奇怪,东边立一石狮子,西边立一关公。

八光打断他,问道:“醉生楼,是不是?”

“什么?”

“那家店是不是叫醉生楼?”

“真有啊?”一只手随便说的,这么一问,自己反倒含糊了。

“是醉生楼,”八光回想着说,“但你方位弄错了,不是东西向,门口的南边是关公,北边是石狮子。”

大家都停下来,看着八光。弄得他有点难为情,最后自己给自己垫了句话下台阶,他自言自语说:“没想到醉生楼,这么多年还没变。”

既然真有醉生楼,一只手索性放开了编,他说,门口看着豪华,里面其实不贵,一个姑娘五两银子。“我跟五帮主借,五两银子,他都不给,我说二两半也行,我就一只手 摸姑娘,跟她们商量商量,只付一半的钱。二两半,他也不借。你知道他说什么吗?你们知道他说什么吗?"一只手问两遍,也没人接茬,干脆就自问自答,“他说,他多五两银子,宁可找俩姑娘,也不借我。”

“那是他为你好。”八光宽慰他。

“哪儿为我好了?”

“有些事开了头,就一发不可收拾了。”

文思清咳嗽一声,八光识趣,不言语了。吴思若似笑非笑,看着文思清说:“我不知道妹妹怎么想,我的建议是,先让小五子回来。当然,你做主。”

文思清也这么想,可是她一女孩子家,进窑子找男
人,那不是肉包子打狗吗?让谁去合适呢,她跟钱老板商量,说:“钱老板,您最年长稳重,小五子以前也是您的人,就拜托您跑一趟吧。”

钱老板提醒他,肉铺钱老板是假的,自己只是个宫里的,进去一张嘴,听嗓子,人家龟奴、老鸨,都得把他赶出去。文思清转回身求一只手,说:“不然,我借你五两银子,你去把小五子找回来?”

“这才是肉包子打狗,有去无回。”吴思若说,“你给他拿五两,明天他都不一定回来。”

“那找谁呢?”

吴思若对八光撇嘴,说:“这不是现成的吗?连醉生楼的石狮子都知道。”

八光连忙摇头,往后退。文思清也感觉这是个好办法,说:“那就你去吧,正好这也是一次难得的考验。你要是经受住了,以后也不用怀疑自己了,你要是没经受住考验,那就放弃吧,以后也不用老折磨自己了。”

两个女人劝了八光好半天,吴思若说:“我们不聊考验、劫难什么的,你就是去找小五子回来,有这么费劲吗?”

一只手乐呵呵地补充道:“你快去吧,一会儿五帮主都完事了。”

钱老板在一旁看热闹,八光被说动,决定出门的时候,他还赶过来,塞给八光五两银子,说:“高僧,你要是实在忍不住的话,就把这五两银子花出去。”

当然不能花,八光攥着银子,推门出去。闭着眼睛,都能走到醉生楼,他在门口徘徊一阵儿,在石狮子和关老爷之间走了进去。

他站在大厅中央往上看,从一楼到三楼,里面莺歌燕舞,打情骂俏,那么熟悉,这就是田扒光的舒适区啊。他不断提醒自己,我是来找人,把小五子找出来,我就离开这儿。这么暗示果然有用,他一路向上,把每个房间都过了一遍,没见到小五子,那就赶紧走吧。下楼的时候,有个姑娘倚在门口冲他笑,问他:“你怎么了?”

“没怎么?”

“那你这么慌张,干什么呀?”姑娘说,“没怎么,你就慢慢下楼啊。”

八光放慢脚步,一步一个台阶地往下走,后面的姑娘还在笑,声音传过来轻飘飘的:“不然进来喝壶酒,歇一下吧。”

好像有回声,似乎从山谷里传出来的声音,在他脑海里荡来荡去的。楼梯一节节往下,就要到一楼时,他忽然折回来,走到姑娘面前,把五两银子交给她,歇一下就歇一
下,怕什么呢,让我进去喝壶酒。

姑娘等八光进来,她把门关上,还是倚在门口,只不过这次在门里,八光在房间里。她笑着说,自己没和和尚同房过,看你的样子老当益壮,让人期待呢。姑娘开始脱衣服,就在八光面前,一件件地把衣服脱下。八光望着她直咽口水,脸上却痛苦得满眼含泪。

就让时光定格在这里吧,后续的发酵要到两天之后,文思清再次见到他,八光好半天一声不吭。她问八光,那天经受住考验了吗?八光点点头,也不多说话,像是回避这一话题。

文思清当然很高兴,说:“那天钱老板给你五两银子的时候,让我担心坏了,生怕你破戒,挺过来就好,那你现在把银子给我吧。”

八光看起来悔恨不已,低声说:“我花了。”

“花哪儿了?”

八光不说话。文思清生气了,指着他的鼻子怒斥道:“那你还跟我点头?当初下山的时候,你怎么跟师父保证的?他说,你还得五十年,才能六根清净,我看你啊,一百年都不够,你还是别当和尚了,继续当你的田扒光得了!”

八光低着头,他只说不是这样的。

“那是怎样的?”文思清问,“你倒是说啊。”

他没办法说,但那一天在醉生楼的情形,真不是这样的,以后可能讲给你听,可能永远不会讲出来,但我现在只能说,事情绝不是你想的那样子。
\newline

{\centering\subsection{5}}

八光去醉生楼找小五子以后,一只手越编越亢奋,最后把青楼描述得漏洞百出。最早当然是吴思若发现有问题,这里面她再熟悉不过了,一只手连青楼和窑子都分不清楚,追问两三句,就把一只手给问住了。

一只手只好承认,跟青楼没关系,他和五帮主一天都在赌馆,里面有一个自己带椅子来的驸马爷,带了几个大内侍卫,故意输钱给五帮主,一把就是五百两,十把五千两,按八光的算法,是一千个姑娘。“姑娘真有那么便宜吗?”一只手问。

“这么大的事情,”吴思若质问,“你说他去逛窑子了?”

“是他赶我走的,赢了那么多,一两都不借,硬生生把我赶了出来!”

“他是让你回来,通风报信呐!”钱老板急得喊了出来,“故意输那么多,肯定有问题!”

他们赶车过去,到赌场的时候,天已经有些黑了。离老远就看见小五子在赌场门口,一个一个地打李大人身边的侍卫,每一下刚刚碰到对方,就像小五子有多高深的内力一般,弹出去老远。

打了一个多时辰,这些人还在配合小五子,既然请不动,又不愿强行把他绑走,似乎这是最好的办法,配合他打来打去,只等着小五子精疲力尽倒下来,把他抬到轿子里。小五子明显累了,之前还是碰一下再弹出去,如今出拳无力,隔着空气侍卫们就往外蹦。他大口喘着气,自己给自己打气一般,凛然道:“还有什么人,都给我上来!”

几个人都挺纳闷的,小五子什么时候这么大本事了?这么好打,一只手也想上去过过瘾。跑进人群里,看准一个侍卫,使足力气,一掌击过去。哎?他没飞出去,脚步都没移动一下。可能发力不对,一只手蓄力,再打第二掌。

这些侍卫本来就陪小五子玩了一下午,摔了百十来次,正愁有气没处撒,看着不知从哪冒出来的一只手,简直是自己跑进来的出气筒,总算可以正常打上一架了。他侧身躲过一只手的第二掌,一掌击在他后背上。一只手晕晕乎乎地,就要倒在地上,旁边另一名侍卫急忙喊道:“别把他打死了!”

是啊,打死他,就没出气筒了。头一名侍卫亡羊补牢,一只手就要倒下去的一刻,扳住他的肩膀,把他从地面捞起来,对着他前胸又来一掌。这次一只手是要仰躺下去,可还是被侍卫提了起来,前胸后背都打过了,换个新鲜的,侍卫把他抛到空中转圈。

转圈也分好几种,抛得低,但转速快的;转速没那么快,但是抛得很高,一时半会儿下不来的;最刺激的是第三种,高空大风车,抛得又高又快。大风车的过程里,一只手一阵阵恶心,感觉自己有好几次,还没来得及吐出来,又给咽下去了。

有时在空中都能看见,小五子又如盖世英雄一般,轻描淡写地将这二十多名侍卫,一一击倒在地。这不符合武学精神啊,一只手想不明白,怎么五帮主今天运气这么好,赌桌上大小通吃,出来打架,怎么还能一拳干倒一大片?

李大人一再地拍手,那口气听起来就是阿谀奉承,他说:“少侠果然是少年英雄,属下这二十多位侍卫,还不比少侠的一根小指头。”

“照这种打法,别说二十位,”小五子边打人,边说,“就是两百位,两千位,也一样给你打回去!”

李大人对着小五子行礼作揖,恳求道:“属下在这里,替我这些下人们,谢少侠不杀之恩!”

我倒是想杀他们,小五子想,个个跟能无限复活,几百条命一样,反反复复地折磨人。他看到大家都来了,忽然感觉很露脸,很想显摆一下,一拳一脚出得还像那么回事,那么侍卫们似乎飞得更远了。

文思清和吴思若看着很不解,不知道这个李大人,葫芦里装的是什么药。钱老板倒是一直在皱眉,他认识这个李大人,其实再熟悉不过,只是没听说他娶公主,什么时候成驸马爷了?哦,他叫李准驸,据说名字自己改的,好像原名叫李准基,要么就叫李准隆,改了准驸这个名字,那是下定决心要娶公主了,估计这三年已经实现了,只是不知道,是嘉和皇帝的哪位公主。

钱老板不想他认出自己,也看出来李准驸没有伤害小五子的意思。他转身要走,李准驸却认出了他,在后面喊他:“常公公,请留步。”

钱老板只能回头装糊涂,问:“这位大人是?”

李准驸毕恭毕敬地跑过去说:“常公公,您贵人多忘事,我是九门提督小李子,以前每个月都给你上两回贡的,好几年不见了,这笔钱花不出去,都给你留着呢。”

钱老板想想,也没法否认,嘴上不说话,但其实已经默认了。李准驸笑眯眯的,继续说:“我以为当年中秋之乱,您惨遭不幸了呢,没想到您施的是金蝉脱壳之计。”他说着,看了看正在激战的小五子,低声说:“太子这几年,一直跟你在一起的?”

“当年昆仑公子把太子从宫里挟持出去,”钱老板问道,“你一定知道吧?”

“当然,当然,不瞒您说,我也是足足找了三年,今日才老天有眼,让我遇见太子。”

钱老板看看赌场,笑道:“这三年一直在赌场里找,一定很辛苦吧?”

李准驸有些紧张,辩解道:“都知道太子好这一口,当初离开京城的时候,我就下令说,别的地方不用寻,去哪儿守着赌场就对了。那常公公您呢,怎么太子就一直在你身边?”

“我一路追踪,最终从昆仑手里救出太子,当时想过再加上太子,”常公公说着,指了指自己的太阳穴,“这里受了点伤,怕敌不过三王爷。”

李准驸回头看看小五子,说:“我见他第一眼就看出

“我当时也是,跟你这般为难,又没有李大人这么大的本事,只好将太子暗中保护,只等见到李大人,由您迎太子回宫。”

几句话说得李准驸飘飘然,频频点头道:“以后还需要您在皇上身边,多替微臣美言几句,虽然圣上现在听不着,但总有醒来的那一天,还要您继续提携小李子。”

“我年事已高,只想在江南养老安乐,宫里恐怕是回不去了,以后太子还需要你来辅助。”

李准驸回头看着激战正酣的小五子,问钱老板:“那微臣现当如何是好,还请常公公指教。”

“事到如今,正好有我在,太子对我还有三年的信任,不如就地把真相告知太子。”

李准驸点头称是,朗声要众侍卫停战退下。小五子早看见他在和钱老板说话,知道他们相互认识。他看见李准驸朝他走过来,忽然下跪叩首,大声道:“九门提督李准驸,叩见太子!”

小五子有点蒙,看着李准驸后面的钱老板,谁知他跟着跪下了,说:“太监总管常公公,叩见太子!”

满脑子都在嗡嗡作响,他看着文思清,看着吴思若,看着一只手,大家表情都有点震惊。小五子一直在想,如果我是太子,可我又是昆仑公子,是我弄错了吗?三年前中秋夜,太子不是被昆仑公子劫走的吗?

\newpage