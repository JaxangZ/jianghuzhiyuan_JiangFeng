\section{拾陆}

{\centering\subsection{1}}

出嫁前夜,五公主在宫中大闹,小五子让太医开些镇定的药方,让五公主服了之后早些休息。她在夜里几次哭醒,却没有力气起来,据说到出嫁的轿子上,都是被太监宫女们抬上去的。

一晃又是半年,秋去冬来,快到正月,乔文君带着闹闹来了一次皇宫,她带来一个消息,说是乔帮主上个月病故了。正是六公子那包所谓的哑药,折磨了乔帮主大半年之久,终于令他撒手人寰。虽然只是做过假夫妻,小五子还是要求后宫对乔姑娘行贵妃之仪,对闹闹以太子之礼。

当然,二人没有同房,以礼相待。有一次乔文君问小五子,他是怎么从沉狮谷跑出来的,那个猎人又是谁?

小五子叹息不语,又开始想念苏子瑶,想起自己的
断魂掌和苏子瑶的死都是拜南海真人所赐,一气之下,把刚粘合的那个桌角又拍掉了,说:“我过去见这个老贼只占口舌上的便宜,三个姑娘哪个死让我来挑,我跟个孙子似的,闭着眼睛让他杀!我七日内,必然南征拿下海南岛,为苏子瑶复仇!”

小五子知道她难处,一个女人带着孩子,不好行走江湖。他有个想法,想把闹闹留下来,自然不是做太子,先养大了再说。他承诺给闹闹找最好的老师,定会把他抚养成人。几番承诺,乔姑娘含泪离别自己的亲儿子。

送走乔文君,他要一只手去请个老师来。刚过半个时辰,就传一只手,问他:“找老师的事情怎么样了?”

一只手完全是懵的,辩解道:“这不是你才跟我提的事情吗?”

“半个时辰了!”小五子发飙,“现在就给我去找,我再给你半个时辰的时间。”

过了半个时辰,一只手领来一位老先生。小五子装模作样询问几句,便封他为太子太傅,从即日起,不管太子身在何处,需每日伴读。

狮吼帮的人还在京城等着乔姑娘。他们对乔文君说:“没了乔老帮主,我们这些乔帮主的弟子最后的任务就是,把你平安送到沉狮谷,以后我们弟兄几个,混迹江湖,各安天命,就不再麻烦乔姑娘了,但凡你有事,只要在沉狮谷插一面狮吼帮的旗子,我们就是赴汤蹈火,也会赶来相助。”

乔姑娘劝大家别走,她说:“我爹我娘创建的狮吼帮绝对不能毁在我的手里,我们现在就一起回沉狮谷。”

出城的那天,京城下雪了,乔文君回头望着漫天飞雪,心想这一年经历了多少的事情。她盼望往后的日子能安省一点,能看到闹闹一天天地长大。
\newline

{\centering\subsection{2}}

吴思若被救出来以后,就跟上了救她的那个蒙面人。一路走了几日,蒙面人很少说话,一直未向吴思若表明身份,有几次赶她回去,让她不要跟着自己。吴思若说:“你救我一命,起码得让我知道你是谁,以后有机会,才能报答你,你要是什么都不说,我就一直跟着你好了。”

有两回蒙面人试着甩掉她。吴思若都想尽办法跟上,甚至还使上了大街上喊抓贼这类手段。

蒙面人一路向南,一直走到大路尽头,坐上了海边的客船。吴思若就让后面的渔夫驾船跟着他。

行船三个月,两艘船停靠在海岛,蒙面人眼看甩不掉她,反而要她跟自己去个地方。之后穿过两道山,差不多日落时分,他们来到山脚的两座墓前。蒙面人才揭
掉面纱,吴思若认出来了,原来是蓬莱阁老。

阁老指着一座墓说:“跪下来磕头,这是你娘。”

吴思若看他眼神坚定,知道所言应该不假,跪下来恭敬地磕了三个头,记住墓碑上的名字,吴淑珍。

她说:“我跟我娘一个姓,这我从来没想过。那我父亲呢?”

阁老不语,吴思若看明白了说:“你认识我父亲,所以那天你害怕了,你怕你以后见着我父亲不好交代。”

阁老命令她以后不准再提这件事。吴思若问他:“旁边那个小墓叫章志瑶的是谁?”

阁老沉吟道:“是你。”

吴思若有些懵了,看着上面的日期,转身问道:“我二十二年前便已经死了?”

说完这句,树林里传来一阵大笑,蓬莱阁老对着树林喊道:“你也跟了够久的,该出来了吧!”

树林里笑的是大漠仙人,吴思若出于惯性,正要叩拜师父,阁老扶一把吴思若的肩膀,内力传来,令吴思若身子躬不下去。

大漠仙人笑道:“二师兄的内力果然日见精湛。”

阁老道:“我和你以后不再有师兄弟关系。”说完又对吴思若道:“以后你和他也不再有师徒关系,再也不要叫这个禽兽师父。”

大漠仙人哈哈大笑,道:“说我是禽兽,可与你二师兄相比我还差得远呢!上一次我把我最美的女弟子献给二师兄,本来想问问这个吴思若伺候得是否到位,但是这两个月见你又从法场救她,又把她一路带回海南,看来你们真的是处出感情来啦!”

阁老闻此,向前拍出一掌。大漠仙人闪身一躲道:“伺候得好不好还没说呢,别忙着灭口啊!”

二人越打越凶,但是彼此都顾忌对方的神掌,并未拼尽全力。

吴思若劝道:“一个是我师父,一个是我救命恩人,有事情慢慢讲,何必动手?”

大漠仙人笑道:“你那相好的,就是怕我们坐下来慢慢说嘛!”

树林里又传来一个内力深厚的声音,道:“两位师弟,跑到海南来,也不到我府上坐坐,忙着在这切磋什么功夫?”

说着,一个人影飞了过来,拉起两人的左右手将二人分开。

吴思若见是南海真人,“啊”了一声,南海真人冲她
笑了笑,说:“上次在南京我没杀你,这次倒惹得我二位师弟大动干戈,唉,那位苏子瑶死得可惜啊!”随后真人冲着墓园朗声道:“向师弟,你也看了很久了,那就快出来吧!”

然后向问和拍拍手,一步一步地走过来。

南海真人问道:“我刚才一直在寻思,要是他们俩真的出了杀招,我若不跳出来,向师弟可会出手相拦?”南海真人话都说完了,向问和却还有几十米没走完。

向问和笑道:“我二师哥和三师哥平日关系那么好,肯定打不起来,我就是刚练成无为神掌,想借机揣摩一下仙人掌和蓬莱掌的精髓,要不大师兄你也跟他们玩一会?小弟再揣摩揣摩断魂掌的精髓?”

南海真人说:“咱别耽搁时间了,师妹还在我府上候着呢,她就知道你们今天要来,让我出来迎接你们。”

几人施展轻功先行离去,吴思若施展不出,落在了后面,向问和说:“我陪姑娘慢慢走。”

蓬莱阁老信得过向问和的人品,便说:“在寿南山下万龟滩上等你们。”

吴思若最后一个到万龟滩,见了真人的第一句话就是:“小五子说得没错,你还真在这养龟,生意越做越大。”

蓬莱阁老提醒吴思若道:“别乱说话,这些龟都是大师兄用来练断魂掌的,一只乌龟活了一二百年,被大师兄在壳上拍那么一掌,昨天在哪下的蛋都想不起来了。”
\newline

最后一个来的是百花谷谷主,吴思若见过她,只是从未猜到,自己与她还有这样的渊源。众人见过百花谷谷主,一起到祠堂祭拜了沈老前辈的灵位。吴思若在旁边看到一个灵位写着——爱妻吴淑珍之位,忽然激动不已,问:“南海真人,吴淑珍是你过世的夫人?”

南海真人点点头不愿与她多答。

吴思若接着问:“那章志瑶是你的女儿?”

真人又一愣。吴思若马上说:“我就是章志瑶。”

能看得出来往事翻涌,真人一时都要哭了。吴思若扑上去喊了一声:“爹!”

真人忽然抬手便要劈过去。吴思若身前冒出一个人,蓬莱阁老替她挡下了南海真人的断魂掌。

五个师兄妹并不是如何的担心,师父当年教这三掌给头三个弟子的时候,就是可以互相牵制,本门弟子中了掌,需休养些时日才可恢复内力,并无性命大碍。但是,如果中了断魂掌,又接着中了其他任何一掌,便有性
命之忧——这是当年沈老前辈为防止某名弟子叛逆造反,才有此良苦用心。但后来有人偷走了师父的三本掌法秘笈,沈老前辈才又潜心自创无为神掌,招向问和为第五个弟子,以克制这位尚未查出的逆徒。

南海真人发现受掌的是蓬莱阁老后,第一个指着的人不是阁老,反而是大漠仙人,警告他说:“二师弟恢复内力之前,你不得靠近他半步。”

大漠仙人反击道:“大师兄说得真好,你打了第一掌,就想诬陷我打第二掌,倘若偷学三种掌法的人是你,回头你对二师兄拍了个仙人掌,那我找谁说理去?”

百花谷谷主表示:“我本来想说,十二个时辰看着三师兄大漠仙人,但我一介女流,大师兄的府上,还是由大师兄自己做主吧!”

真人安排道:“向师弟你与阁老同住,我和大漠仙人一间房。”然后转身向阁老赔了个不是。

阁老并不领真人的情,直接质问道:“大师兄你那一掌要是冲我来的,我绝对不生你的气,可你那一掌打的却是我女儿,这个事我跟你没完。”

吴思若指着阁老问:“你?你是我的爹?”说完她想了好半天,自言自语道:“我现在是明白了,你为什么怕见我了,从古至今就没有你这样的爹。”

真人问仙人:“这个姑娘之前叫你师父,原来那晚是你把她抱走的。”

仙人倒是邀功道:“怎么样,大师兄,看我把这姑娘养得又漂亮,又水灵,阁老头一次见她的时候眼睛都直了,那我大漠仙人能违了阁老的意吗?当即就送给我二师兄做贺礼。”

仙人自鸣得意,说话间脸上被抽了一巴掌。打他的是百花谷谷主。她怒道:“够了,你别再讲了!”
\newline

几人聚在这,本来是皇帝带兵攻打海南岛,商量一下如何应对,因为吴思若的出现,二十来年不提的恩怨情仇,此时全摊了出来。

六个人中最年轻最痛苦的是吴思若,此时瘫坐在椅子上,情绪濒临崩溃。她对阁老说:“你那夜说过,让我走得远远的,再见到我,就一定会杀了我,你现在就说到做到,杀了我吧!”

阁老不语。吴思若跪地请求阁老杀了她。谷主兰贵人把她扶起来坐在身边道:“这么多年,既然你还活着,就有必要让你知道,就是因为你出现,我们师门四人才分崩离析,既然躲也躲不过去,索性全都说给你听。”
\newline

{\centering\subsection{3}}

二十多年前,南海真人是沈老前辈最早的弟子,他和吴淑珍生出了一个女儿,沈老前辈为她取名章志瑶;大漠仙人是三个弟子中练功最刻苦的一个;蓬莱阁老风流成性,四处拈花惹草;兰贵人则不学掌法,只是专心培育奇花异草。

直到有一日平静被打破,先是大漠仙人发现了吴淑珍和阁老之间的秘密,他偷偷告诉了南海真人,找到了后山的那个隐蔽山谷,将阁老和吴淑珍当场捉奸。一开始他们三兄弟商议,先不让师父知道。直到大师兄察觉章志瑶并不是他的亲生女儿,吴淑珍承认这是她与蓬莱阁老所生。真人与阁老闹得水火不容,沈老前辈知道此事后勃然大怒,他将章志瑶抱出来,对大家摊牌分析道:“真人想杀了这个女孩,阁老想把她养大,我现在就算强迫你们师兄弟重归于好,随着这个女孩一天天长大,你们的仇隙会越来越深。”

师父对众人说道:“我现在就把这女孩从悬崖上扔下去,生死有命富贵在天,你二人再也不许提这件事。”

几名弟子跪地聆训,师命不可违,他们眼睁睁看着女孩被师父抛下悬崖。母亲吴淑珍发了疯一般,跟着跳了下去,粉身碎骨。

沈老前辈令二人要跪到次日天亮,不许下山。兰贵人此时对大家讲,沈老前辈死前最懊悔的一件事,就是原本他已算好,多大力气可以将孩子挂在悬崖下的树枝上,不至于摔死,在这两个弟子跪地反省之时,将章志瑶找到,送到某户人家寄养,算是了却这段恩怨。可是当夜,当他下山时,只剩下吴淑珍的尸体,襁褓中的孩子已经不见踪影。

兰贵人说完对着仙人道:“没想到是你三师兄瞒着师父抱走了,有人阴险一时,真没想到,你能阴险一世。”

大师兄表示:“既然二十多年前没能解决的恩怨,拖到了今天,我还是一样的看法,我要杀了她,祭奠我夫人。”

阁老重伤在身,他哀求真人:“如果大师兄心中还是有怨恨的话,我阁老愿代犬女一死。”

吴思若冷冷地笑道:“我用不着你替我死,你还得好好风流着呢!”

似乎此言比那一掌还要痛,阁老像泄了气的皮球,反复重复道:“爹替你死······”

“这事从当年到现在,我也有些责任,”大漠仙人极尽狡诈之能事,说,“当年我不该告诉大师兄,二师哥和嫂子好的事,现在吴思若这个事,小弟做得也有点过了。我有一个两全其美之策,既然大师兄会断魂掌,那也就不要杀,也不要保,就给吴思若一掌,让这二十年的恩怨就此了结。”

众人乍一听,有些突兀,仔细想来却不无道理,眼前的这位姑娘,似乎已经生不如死。兰贵人这时候说:“当年吴思若是个婴儿,死也就死了,现在二十多岁,毕竟是咱们师门的后人,她也没做错什么,不可能说打就打。”

她转身问吴思若:“人一辈子,大多数一半痛苦,一半快乐,有人好一些,有人坏一些,但是从来没有人像你这么不幸,可能你自己想想也是吧?应该是九成的苦一成的甜。”

阁老伤感道:“就那一成的甜,还是对你们那少谷主昆仑公子的单相思,她为他差点被斩首!女儿,爹也想通了,可能,大师兄给你这一掌,对你对我都好,你以后都不会再怕面对我,在你重新开始的那一天,爹把前半辈子欠你的债一点点还给你。”

百花谷谷主提醒道:“我虽然孤陋寡闻,但是当今皇上毕竟是我百花谷的人,见到你之前,我都听说你早就被斩首了,昆仑公子也是这么想的,姑娘你要想清楚,挨了大师兄这一掌,你这二十多年的恩怨情仇就全了了,如果你不愿接受这个结果,我武功虽然不如师兄,但必然会全力相助。”

向师弟接话道:“我和师姐立场一样,当年师父教我无为神掌就是为了避免同门相残,大师兄要是执意劈下这一掌,我定以无为神掌奉还。”说完他回想了小五子的做人道理,在桌下偷偷看了看自己的手掌。

大漠仙人也劝吴思若:“你就不要再做皇后梦了。听说皇上前两天刚册封了一个皇后。”

“果真有此事?”其他人问道。

“确实如此,不信你们去打听,我若有半句虚言,让我也中大师兄一掌。”

这是吴思若遭受的更大一次打击,听完大漠仙人的话,她涌着眼泪摘下小五子送她的一只银镯。阁老问她:“这是为何?”

“他既然有了皇后,我失忆之后也要重新做人,也不用拿这个镯子睹物思人了。”吴思若转身对南海真人说,“大师伯,你动手吧!”

吴思若看着南海真人发力,双手捉住桌腿,一动不
动,掌就要劈过来的时候,她隐约听到大漠仙人在她耳边说了一句话:“我说小五子册封皇后,半句不假,但是他册封的是你,他是以为你死了,所以他追封你为皇后!”

吴思若忽然后悔,想要躲闪,可已然来不及,她头晕目眩,想起了一切小五子的好,说了句,小五子,我对不起你!泪如泉涌倒在地上。
\newline

{\centering\subsection{4}}

小五子是在春天力排众议,挥师南下的。此次讨伐海南岛的开路先锋是李准驸。小五子命五公主作为将军家眷随军出征。因为吴思若,他一直记恨着她。路上,他对五公主笑道:“我就是想让你亲眼看见,你自己是怎么变成寡妇的!”

“你现在就可以杀了他,就他这个窝囊废,何必还让十万将士给他做陪葬?”

小五子叹了口气说:“哎呀,你这么一提醒我,我倒是舍不得杀他了。”

行军途中,小五子有机会又问了李准驸:“我让你查的文宰相灭门的事情,可有眉目?”

李准驸慌乱道:“文宰相家当时被杀了个精光,一只狗都没留下,朝廷也没备案,没有任何线索,无从查起啊!”

“那文思清活下来了,按你的意思,连只狗都不如?”

李准驸接不上话。

小五子提醒道:“赶快查吧,我怕你这回去南海,有去无回啊!”

李准驸吓得从马上摔了下来。

小五子瞅着他一身戎装,说:“李将军,李大人叫了半年多,还是第一次看你穿着这身衣服。”

一路上小五子各种困难险路都让李准驸先走,每回还都笑道:“这可是李将军立大功的机会啊!”

李准驸次次都硬着头皮说:“谢主隆恩,臣在所不辞。”
\newline

当夜,小五子,李准驸及公主三个人在大营商定进攻之事,小五子讲:“我此次前来只是督战,一切由李将军说了算。”

李将军一本正经地说:“最好的战术,敌不动我们就按兵不动。”

见小五子不信,他又讲了一大堆狗屁理由,气得小五子急了,一脚把他踢开,说:“你以为我们带十万大军,拿着军饷来海南岛旅游啊!听说南海真人以一敌万,好像那几大高手都在山上,我觉得你按兵不动那个计策很好,这样吧,我留九万九千五按兵不动,你带五百人前去寿南山抄小道探探路。万一你全军覆没了,咱们还能保存实力。”

李准驸知道此战必死,直看公主什么意思。五公主说:“这是皇上给你立功的机会,还不赶快谢主隆恩?”

李准驸跪下来哆哆嗦嗦讲了一大堆马屁话,领命前去。

公主冷冷道:“我记得你曾经说过,你不舍得杀他,想让这个废物陪我一辈子吗?”

小五子说:“但是我现在明白了,我要是舍得杀你,也就舍得杀他了。五公主听令!命你带五万大军正面进攻万龟滩,拿下南海真人老巢。”

公主咬牙盯着他,最终还是领命而回。
\newline

{\centering\subsection{5}}

大家本来是为了商讨如何应对皇帝南征,经过了这场变故,似乎每个人都很疲惫,有气无力地问南海真人,十万大军来袭,你如何应付?

南海真人说:“早已布置妥当,大家今天可以放心休息,明天起来,我跟大家分配任务。”

几人知道大师兄向来言出必行,也都放心回房睡觉。

谷主和吴思若住一间,大师兄晚上曾说过,三日之后,吴思若醒来,跟个全新的人一样,所以不必操心。

向问和与蓬莱阁老住一间,蓬莱阁老晚上几次想要去探望吴思若,向问和劝道:“吴思若是你女儿,但是我师姐也在那房间睡着呢,你这么过去,算怎么回事?你急什么?”

但是这天晚上,向问和醒来,看到蓬莱阁老还是出去了,估计是看吴思若去了,等了一会儿,阁老悄悄进来,躺回床上。向问和说了一句:“要是再出去的话,小心师弟对你施无为神掌了。”

阁老“哦”了一声,一觉睡到天亮。

南海真人和大漠仙人一间房,他感觉大漠仙人坐立不安,直跟他嚷嚷:“你今天不睡,弄得我也不能睡,明天没法面对大敌。”

仙人还是来来回回在房间里走。

后来南海真人干脆坐起来说:“你要是睡不着,咱哥俩喝两杯。”

倒酒时,南海真人悄悄在酒里下了点蒙汗药,这细节被大漠仙人发现了,趁南海真人取下酒菜的时候,调换了酒杯。

南海真人回来也察觉到了,故意把仙人的筷子弄到地上,趁仙人拾筷子之时,又调了酒杯。

仙人弯腰,偷偷震了一下桌角,将对方的竹碗震掉,说:“大师兄,你碗也掉了,有点远,你自己拾吧!”

南海真人说:“掉就掉吧,我干脆用手抓算了。”

大漠仙人急了,在桌上把酒杯快速乱倒,后来谁也分不清哪杯有药哪杯无药,俩人哈哈大笑,举起各自的酒杯,真人说:“喝了吧,大师兄还能害你啊!”

“我怕大师兄害了你自己。”

“那咱就赌一赌。”

俩人说罢,一饮而尽。
\newline

{\centering\subsection{6}}

次日,百花谷谷主被远处山下的擂鼓声震醒,顾不得吴思若,直接去敲大师兄房间的门,半天不应。蓬莱阁老和向问和也闻声赶来,推门而入,发现俩人还躺在床上呼呼大睡。南海真人先醒,起来之后有些不太对劲,众人看在眼里,忧心忡忡。接着是大漠仙人缓缓醒来,说起昨天蒙汗药的事情,自己一觉睡过去,毫无所知。

可是南海真人的表现,分明是中了蓬莱掌和仙人掌两掌,神情恍惚,时不时地有片刻清醒。

大军压境,众人也无暇查明真相,趁南海真人短暂的清醒,问他怎么安排。

南海真人说有一条小路可以逃走,叫众人随他前去。

阁老背着吴思若,南海真人不高兴了,说,我就带你们去,不带这个人去。一时疯疯癫癫坐在地上撒泼。阁老让大家先走,不连累他们,自己留下来保护吴思若。

百花谷谷主讲:“我不知道偷师父秘笈的那个人是不是你,你会不会用三掌?如果不是你,说明你是真伤了,你留下来毫无用处,以后你想疼,也没处疼吴思若,而且你忘了吴思若是什么身份?皇后!你守在这必死无疑,还是跟我们先走吧!”

这时候“嘭”的一声,阁老中了一记闷棍倒地。向问和拍拍手扔掉棍子,对谷主说:“师姐,别怪我出手鲁莽,事情紧急,现在我们沈家帮的逆徒是不是他还不清楚,大师兄中了掌,不能再让阁老束手待毙了,这样我们沈家帮就没了。”
\newline

仙人架起阁老,跟着真人往悬崖上冲。山路越来越窄,仙人心里直犯嘀咕,这哪是逃生之路?中了掌的真人可不管,一路冲到悬崖边,先是喊着:“冲啊,跑啊!”最后来了一句:“吴淑珍,章志瑶,我也来找你们了!”说完纵身跳了下去。

名震江湖的南海真人,就那么摔死了。大家望着他一直摔到谷底,等阁老醒来,知道大师兄跳崖之事后,叹息着说了一句:“他昨天打我的只是普普通通的一掌,逆徒不是他。”

向问和质疑道:“昨天晚上,二师兄,你趁我睡着出去了一趟,到底去哪里了?”

阁老解释,他只是想看女儿,在窗口站了一会没有进去,估计谷主也在房间。

谷主瞪了他一眼,警告他:“还好你没有进去,否则我跟你玉石俱焚!”

阁老马上说道:“仙人一直睡在大师兄的身旁,有什么异动你该知道吧,现在大师兄死无对证,你这么厉害的高手,居然拿蒙汗药来哄骗我们?”

感觉阁老和仙人都在狗咬狗。向问和说:“我现在不想错杀你们其中一个,却放过了真凶,我们先找路逃出去,回头找到凶手我必代师父清理门户。”
\newline

{\centering\subsection{7}}

李准驸出征那天战战兢兢,骑在农夫队长的马上,像个女人一样抱着队长的腰,顺着山谷进军。骑出去没多远,李准驸就说:“大家跑累了,稍微休息一下吧。”

队长质疑道:“李将军,十里路,我们已经休息五回了。”

李准驸这时引用了一大堆兵书上的名言,向他阐明保存体力的重要性,反复强调:“你们死不足惜,但还有九万九千五百的大军,等着我李准驸来统帅。”

说着,上面掉下来一个人,李准驸喊道:“有暗器!”

众人后退十余米。队长看清楚下来的是个人,感慨道:“李大人,他们居然拿人当暗器,来袭击我们!”

李准驸让大家不要动,等敌人暗器用光,我们再过去。许久不见上面再有动静,他让队长过去去瞧瞧是什么人。副将汇报说,是南海真人。

“死了没有?”李准驸低声问。

副将回答:“已死。”

李准驸大喊:“小心南海真人,快快撤退!”自己却持剑冲上去,对着尸体一顿乱砍,手忙脚乱中,还不小心割到了自己的腿。
\newline

五公主这边损伤最多,几位高手不识小路,都是从五公主这边强行突破。转眼间,仙人和阁老二人即将五公主擒获,百花谷谷主上前说:“此人与我有旧交,且是昆仑公子的亲妹妹,二位师哥交给我来处置。”

二老放下公主,继续杀敌。百花谷谷主对公主道:“我心中一万次想杀你,但此时昆仑公子在宫中,还需要你多多扶持,先留你一条性命,待昆仑公子坐稳了皇帝这个位置,再杀了你不迟。”

五公主冷笑道:“他已把我许配给李准驸,我也无法继续扶持公子,留我何用?”

“五公主手段高明,不管如何打压,相信你总有翻身夺权的那一天。”谷主说完,一掌将公主推了出去,公主稳稳地落在马上,谷主下山而去。

五公主继续赶路,带人行至寿南山,里面仅剩空城一座,得到禀报,说有一女子昏倒在后院厢房,公主过去一看,正是那个死而复活的吴思若。她让人把吴思若抬回陛下的大营,走至一半,她改了主意,让人把她送回自己的营房,一点消息也不许漏出去。
\newline

李准驸带着五百士兵便诛杀南海真人,一进大营就眉开眼笑,时不时地暗示小五子,自己有多厉害。这时,有一个快马加鞭的密使进了大营,将一份密奏双手献给小五子,小五子读完将密奏仔细收起来,脸上一点表情也没有。

李准驸看得好奇,问道:“陛下,难道我夫人也传来捷报?”

小五子摇摇头说:“不是,这上面写的是,我该怎么赏你。”

“赏什么啊,”李准驸自言自语,随即醒悟道,“平定南海,乃国家之大计,臣之责任,微臣不该领陛下赏赐。”

小五子赞赏道:“李大人,你自从娶了五公主,见识是越来越长了啊,那就照你说的,不赏了,请退吧!”

李准驸迟迟不起,忍不住问道:“陛下真不赏啊?”

“赏!朕不但要赏你,还要重重地赏你,等你跟朕班师回朝,朕在京城有一屋子的宝贝,二十件全都赏给你。”

小五子带着一只手前去搜索寿南山,一只手贪吃,在乌龟窝里翻了半天,最终在龟蛋下面找到了一张九宫图,想起小五子一直在收集这东西,随即拿回去请功。

清缴的时候,小五子在一间房子里发现了他曾送吴思若的银镯,他把镯子拿在手上,愣愣地看了好一会儿。吴思若没有死,她曾经在这住过,小五子想。
\newline

公主从吴思若身上搜出一张羊皮,自己存了起来。吴思若醒来后,公主发现她前言不搭后语,完全不知道自己的过去,明白她已中了断魂掌。如此施掌之人南海真人已死,吴思若再没有治愈的希望,正琢磨着,外面有人传令:“皇上驾到!”

公主让人将吴思若安顿好,然后慌忙起身迎接。

小五子没好气地问公主:“吴思若到底有没有被斩?”

公主本来想将吴思若送还给小五子,了却他们兄妹之间的恩怨,几句话激怒起来,公主反唇相讥道:“如果我问你,小顺子死没死,你怎么回答?死了就是死了!”

送走小五子,公主对吴思若说:“你就是我的宫女,名叫子柯,这次随我出行途中摔了脑袋,以后你还得在我身边伺候着,随叫随到,不得离开我身边半步。”

公主出门之后,吴思若自言自语道:“我还以为我是什么富贵命呢,原来就是个伺候人的宫女。”
\newline

{\centering\subsection{8}}

班师回朝,途径百花谷,谷主的态度让小五子一时难以分辨是敌是友,但总还是潜在着危险。他沿途下一封诏书,要百花谷一个月内解散,至于谷中的奇花异草,尽献于皇宫,否则百花谷会是第二个寿南山。
\newline

回到宫中,小五子就本次征讨,论功行赏,唯独没有赏李准驸。小五子在朝廷上说:“退朝以后,朕要亲自赏你,还记得那二十件宝贝?”

李准驸兴高采烈,心情大好。退朝后小五子带他来到一处小黑屋。李准驸还一再地吹捧,说这一看就是藏宝贝的地方。

走近一听,屋里传来狗群的吠叫声。李准驸说道:“这么多疯狗守着,这宝贝谁敢偷啊,问陛下,有多少只狗啊?”

小五子答道:“二十只。”

“那正好,一只狗守一件宝贝。”

“数是没错,但你没弄明白,宝贝是什么,这二十只狗就是二十件宝贝。”

小五子把李准驸推到小黑屋前,让人打开锁,一脚把李准驸蹬了进去,然后关上门,里面传来阵阵惨叫,过了两分钟后,小五子让人看看,这人死没死。李准驸被架出来时血肉模糊,倒还有一口气在。

小五子狠狠地说:“你这个人挺好玩的,朕很喜欢你,无意杀你,只是深仇大恨,朕保不了你。朕问你一个问题,你如实回答,朕就赐你死个全尸。你要是胆敢骗我一句,以后公主跟我要你尸体的话,到这二十只狗的肚子里找去吧!”

小五子继续说:“嘉和十年,七月初九,文宰相全家婢女家丁及后人,七十六人被杀,十六人被卖作家奴。抄得财产两万四千两,悉数入你囊中,你当时一个小小的九门提督,量你吃了雄心豹子胆,也不敢带兵进去,你现在告诉我,是谁指使你干的?”

李准驸在被咬烂的衣服里掏出一张手谕,说:“陛下,上次你让我查文家的案子,我就整日坐立不安,这道手谕我随身带着,我确实受了宫中贵人的指使,您可能是贵人多忘事,但是你看看这道手谕,你自己的字迹,总该认得出来吧?”

小五子双手发抖地接过手谕。

李准驸跪在地上说道:“文思清的父亲与三王爷结成同党,不断质疑你的太子身份,企图将你废储,甚至宣称找到了你本非皇子的证据,到后来,逼得陛下心急气躁,连夜下了这道手谕给我这个最不起眼的九门提督。我伪造谋反罪证栽赃给文相,使文府上下满门抄斩,为陛下除此大患,得到了陛下的赏识,至此,平步青云,一路到今天的驸马。”

小五子看着他说,手中的刀依然没放下。

李准驸跪地讲述:“当年文武百官皆反对立储,本来老皇帝无子,您只是老皇帝在山西征战时的私生子。百官皆以为,即位者当是三王爷,且这些官员近十年来,收受三王爷的拉拢贿赂,形成三王党,其中文家势力最大。满朝四品以上官员无一不听令于你,我李准驸虽懦弱无能,但我们这些五品六品的小人物却是太子你在这风雨
飘摇之际能够信任的人,为你清除异己,扩张自己的势力,铺平你的登基之路。”

小五子看着手谕,落款为昆仑公子,问道:“所以当我要报复和残杀这些人时,不方便说自己是太子孙天奇,只落名为昆仑公子?”

李准驸点点头。
\newline

小五子想到,以前曾对文思清讲过,进宫第一件事就是养几十条狼狗,天天不喂食,就让它们饿着,等我把这个人给逮着,直接扔进狗屋,喂饱了为止,我让他骨头都不剩。

小五子提着刀,让人把小黑屋的门打开,众人劝阻,连李准驸都求道:“陛下,不如就让我替你一死。”

小五子让众人退下,狗在小黑屋里叫个不停,小五子把门打开,大步走进去。

过了许久,里面的狗吠声停止,小黑屋的门被推开,小五子浑身血迹,拎着刀走出来。对跪地的李准驸说:“你去养伤,朕错怪你了。”

小五子回想起那一天,他在山谷把文思清从老虎洞中救出来,对她讲,我小五子是个两条腿的就打不过,这些四条腿的不管多凶多狠我都不在话下。那时候真好,那天真好,路虽然泥泞荆棘,但走着走着,文思清就趴在他背上睡着了。
\newline

{\centering\subsection{9}}

李准驸被弄得半死,公主想要见皇上,几次都被拒之门外。那就把气撒在子柯身上吧,吴思若中了断魂掌,浑浑噩噩,说话做事都没规矩,正好被公主找到借口毒打一顿。

子柯身体养好后,不想在皇宫待了,想出走。可是皇宫太大了,都不知道出宫的路该怎么走,子柯就在皇宫里转啊转,迷路了。那天甚至与皇帝擦肩而过,小五子只见她的背影,没看到她的脸,随口跟身边的太监吩咐道:“这个宫女是新来的吧,一点礼节都不懂,问问主子是谁,让主子好好管教。”

子柯这次被打得更厉害,她始终想不明白一件事,就问其他的宫女:“我们为什么要一直留在宫里伺候这个主子?”

一个宫女回答:“因为我们从小就被送进宫里伺候公主。”

子柯继续问:“那你永远在这里伺候公主给人家当奴才吗?就没想出了皇宫,自由自在随心所欲地过日子吗?”

很多宫女从小就被告知,你要一辈子伺候主子。乍一听,子柯的问题,还真的让她们思考了一下,自己这辈子该干嘛\footnote{原文“干吗”}。可这种问题想得脑壳疼,她们回答,给公主伺候好了,就有机会伺候皇上,给皇上伺候好了,就有机会被恩宠,被皇上恩宠了,就有机会升为贵人,贵人做好了,就有机会做妃子,妃子做好了,就有机会做贵妃,贵妃做好了,就有机会做皇妃,皇妃做好了,就有机会做皇后,那就是一人之下万人之上了。

一番话把子柯听得头都大了,目瞪口呆地叹息道:“加油吧,祝你成功!”

大家拼了命地想当皇后,谁能想到,这个饱受公主凌虐的子柯,就是当今被追封的皇后啊。
\newline

公主本来想折磨吴思若,结果不出一个月,吴思若搅得众宫女情绪不稳定,公主房中乱作一团。有一天,公主在后花园训斥吴思若,吴思若早就学会了左耳进右耳出的本事,瞪大眼睛诚恳地望着你,其实一句话也没听进去。公主远远见到皇帝过来,让众宫女带着吴思若赶快走,自己迎上去请安。

小五子问五公主:“李准驸的伤好后,你就应该离开皇宫了吧?”

“我家夫君不知道得罪了谁,被人放疯狗咬了个半死,怕那个人再放疯狗,叫我来宫中避避风头。”

小五子装糊涂问:“这是谁干的?竟敢对驸马爷下毒手,哥哥帮你出头好好查查。”说完他就岔开话题,问道:“你刚才那宫女远远一看,挺面熟的,叫什么名字?”

“这个丫头叫子柯,哥哥你后宫佳丽三千,妃嫔无数,该不会连我的宫女也要抢走吧?”

小五子叹息道:“都说皇上嫔妃无数,怎就我身边冷冷清清的呢?”
\newline

{\centering\subsection{10}}

那日几大高手下山后,彼此谁也不分开,在山谷静坐几天几夜,待阁老养伤。百花谷谷主说:“就这么无所作为地查下去,也不是办法,不如大家各自先回去休息,总有一天,逆徒会露出马脚,我和向师弟肯定要诛杀此人,以告慰师父在天之灵。”

大漠仙人和蓬莱阁老彼此咬定叛徒就是对方,向老前辈心中却想着,可不要查出来,待我回去好好研究一下,无为神掌到底是怎么个无为法,再来清算这一切。

而百花谷谷主,则刚接到谷主信使禀报,当今圣上勒令百花谷解散,将谷中那些剧毒无比的奇花异草统统献给皇上。其他人表示,南海真人既然已亡,逆徒查出来之前,谷主千万不要跟朝廷对着干,况且他皇上就是你百花谷的少谷主,有事还好商量。

“听说向师弟就是在小五子的帮助下出关,”百花谷谷主说,“如果方便的话,请向师弟做个人情,帮去说说话。”

向问和婉拒,说:“不管怎样,咱们在皇帝眼里,都是前朝余孽,我这人情再大,也大不过孙家的天下。”向问和顺便感谢了一下师姐,说:“师姐果然细心记得我心肺相反,便反复提醒小五子不要失手,让我送命。”

百花谷谷主点点头说:“做姐姐的,这些都是应该的。”

向问和下山时,特意找了根结实的木棍,其他人惊异,你丐帮又不是当年的丐帮,何必东施效颦,学洪七公弄一个打狗棍。向问和无奈道:“自从练就了无为神掌,出手必是杀招,但是有些人罪不至死,拿根棍子教训一下便足矣。”

向问和向众人告辞,打算集结丐帮弟子前往田独镇,祭祀前任帮主何振生。
\newline

阁老一路北上,查到吴思若在宫中当宫女。他觉得自己的闺女,给人家当下人使唤,传出去脸往哪放?进了京城,他每天在皇宫外的大树上像猴子一般窜来窜去,查看宫中地形,寻找女儿的位置。

守了十几天,真能见到女儿的机会屈指可数,他索性借机偷了套太监服,易容混入宫中,每日在宫中的赌场寻找机会打探消息。宫中高手如云。但更难的是那个叫子柯的宫女根本不知道这个假太监就是她爹。策略一时没想到,屈辱倒是受了不少,那些小太监嘲笑他,得活得多没出息啊,一把年纪了还跑到宫中当太监。

有回给皇上跪安,他有想过,跳出来跟皇上讲,你要找的皇后就给公主当着宫女呢,但随即一想,这么大的事,公主可不敢瞒着,肯定是皇上嫌弃了我们家吴思若的身世,下放到公主那的。没有办法,阁老天天买醉度日。

父女俩就这么误打误撞,一个当了太监一个当了宫
女。不同的是,吴思若一心想着出去,带着希望;阁老则早已绝望,醉生梦死,觉得每天能看上女儿两眼已经足够了。有天阁老喝多了,瘫倒在花园的灌木丛里,一个相识的小太监路过,要拉着他的腿拖回房,稍一使力,拽掉了他的裤子,惊呼一声冲着太监房大喊:“快来看啊,原来他是带把儿的!”

阁老惊醒,一掌击向小太监,小太监顿时疯掉,更多的太监赶来围观,见小太监疯言疯语,但是刚才那一声喊叫,确实是小太监的,大家起哄让阁老脱裤子验明正身。阁老一着急连给这些太监一人一掌,没打到的人一边跑一边喊有刺客。阁老忙向公主寝宫方向跑去。

大内侍卫好几百号人,将阁老困在吴思若的房间内。吴思若被阁老拿住,一开始还说:“你拿我当人质是没用的,他们早就烦死我了,我又没犯死罪,杀又没借口,所以你要是杀了我,就等于帮他们办好事。”

阁老心中一动,望着她,觉得此时她还是那个嘴上不饶人的女儿吴思若,他过去要抱着她,吴思若东躲西藏,问他一掌把人打疯掉,是什么功夫。阁老哭道:“我就是来把你带走的,你是我女儿!”

大兵破门之前,她终于相信了面前这个老头就是她的父亲。
\newline

小五子早朝听说宫里抓了个刺客,让人带上来提审,一看是阁老,乐了,打趣道:“阁老,还真是煞费苦心,潜伏几个月啦?现在都不会站着尿尿了吧?”

一只手禀报,阁老是来宫中抢一个宫女的。小五子听说阁老要找的是自己的女儿,问道:“阁老的姑娘还在我的宫中?还真是让我蓬荜生辉!”

阁老呵斥小五子,说他忘恩负义,见异思迁。小五子也没听明白,就说把阁老的姑娘带上来,给我瞧瞧长什么样。

带上来之前,一只手对小五子说了几句悄悄话,说一夜之间,宫里多了十多个疯太监,问他该怎么办。小五子说:“你既然能开个赌场,那你就再开个疯人院吧!治好了算你大功,治不好,你也住进去吧。”

这时候听到一个熟悉的声音:“奴婢子柯叩见皇上。”

一时间吴思若以前说过的所有话,和这一道声音混成一片。小五子忙让她平身。子柯迟迟不肯平身,道:“家父罪大恶极,奴婢不敢起身。”

小五子声音都颤了,几乎可以确定她就是吴思若,
连喊几句:“平身,平身!”

后来干脆把她扶起来,含着眼泪就要抱她,带着哭腔喊着:“吴思若,是我啊,小五子啊。”

吴思若慌慌张张问:“小五子是谁?您不是皇上吗?再说,谁是吴思若啊,我是子柯啊!咦,我怎么连个姓都没有啊,爹,我姓什么啊?”

“没事,我以前也没姓,就叫小五子。”随后他明白了,转向问蓬莱阁老,“断魂掌?”

蓬莱阁老点了点头,那些士兵还在押着他,小五子在大堂,反复走了几圈,对着那些士兵喊:“放了!把国丈给我放了!”
\newline

到了议事房,小五子还是像刚才那样反复踱步,问道:“谁干的?”

阁老回答:“南海真人。”

“你当时在场?” 

阁老点点头。

“那你让他打这一掌?你是她亲爹!”

蓬莱阁老回答:“她当时也是自己想挨这一掌。”

“吴思若挨掌之前说过什么?”

“她说告诉小五子,我对不起他。”

“她是对不起我!”小五子冲阁老吼,“你们在场的,谁他妈对得起我了?要不然你们就把她杀了,我也就死心了,弄成这样,送到我面前算什么!”小五子撸起袖子,放到阁老眼前,“你看看这些,我怕自己忘了,以前刻下的字,看看这个瑶字,苏子瑶!我对她毫无感觉,谁知道我们俩以前什么样!”

小五子冲阁老喊了一通。阁老问:“陛下要是看着心烦的话,请允许我把吴思若带回去。”

“你敢!朕这个月就娶她。”
\newline

小五子找公主发了一通火,他说吴思若没死,是朕以前错怪你了,但是你他妈把她藏起来做宫女!

公主冷冷道:“你满口除了文思清就是吴思若,你从来没有想过我的感受,是吗?”

“我他妈凭什么想着你啊,我他妈三十六个姐妹,你以为你是谁啊,我大婚之后三日内,你就跟李准驸去南海。”

“去南海干嘛\footnote{原文“干吗”}?”

“我封他南海王了,行不行,我现在就封他,你立马给我滚蛋。”

公主要哭了,看着他说:“你怎么可以对我这么狠?”

小五子说:“哭什么哭,赶快给皇后请安去!”
\newline

宫女正在给吴思若试皇后的婚装,帮她试衣的两个宫女,其中一个刚好是要从公主的宫女一直爬到皇后的那个励志姐,另一个是以前常常被吴思若质问你凭什么要永远伺候你的主子的那个宫女。感觉这几个宫女的精神已经恍惚了,一个认为,公主的宫女怎么可以一下子当到皇后,另一个人认为,子柯怎么可能一翻身比她的主子公主都高上一级。

大家都恭喜着吴思若,可她此时还如在梦中,她想不通,事情怎么会来得这么假,那个皇帝他才见我一面,就要被强制嫁给他。宫女劝道:“那可不是一面啊,你可是被追封的皇后啊!”

吴思若眯着眼睛,左思右想,怎么也找不到宫女那种兴奋的感觉。

公主向她贺喜,做宫女这几个月,子柯还是头一回见到公主对人行礼。两个女人假意寒暄了一阵,把宫女支走。吴思若问道:“你早知道我是皇后,以前我就想不通,你警告我,千万别让皇上见着我,你说,我脑子摔坏之前惹怒过皇上,是你拼了力保住我的命,如果皇上再见到我,非斩了我不可。这些是不是你讲的?”

公主没否认。

吴思若问她为什么:“你是不是一直想杀我,我失忆是不是你弄的?”

“我要想杀你,就直接把你弄死好了,何必还让皇上把你认出来。至于你失忆是谁弄的,我也听说了,是你自己,你想忘记过去的一切。”
\newline

{\centering\subsection{11}}

婚礼大典。除了文武百官之外,小五子请了些武林人士,做了一年的皇帝,看到这些人不禁感慨万千。三王爷带着六公子送来贺礼。

众人其实近不得皇上,都是远远地望着,能和皇上说上话的也就是吴思若一人。小五子一直想不通,为什么现在跟吴思若,与以前跟她在一起的感觉,那么不一样。后来他想明白了,不用担心,不管记忆失去了多少,人还是没有变的,他们就是天生的一对,她总会爱我的。

婚礼开场之际,有侍卫报信,百花谷派人送来贺礼。小五子问,什么贺礼?来了多少人?侍卫回答,都是些
奇花异草,但是来的就是一个女人。小五子心中大喜,一个月前,要求百花谷解散和将植物进贡的事,谷主都照办了。他问侍卫来的这个人年纪有多大?侍卫回答是年轻女人。小五子明白了那就不是谷主,他让侍卫把花草先存放在稳妥的地方,搜搜这个女人身上是否有武器和毒药,再放她进来。

可进来的女人让小五子大惊,那些武林人士,乔姑娘、方丈及三王爷也都吃了一惊,此女子正是文思清。
\newline

文思清恭敬叩首道:“听说陛下今日大喜,百花谷香主文思清代谷主前来贺喜。”

小五子愣了一下,问道:“你怎么来了?找你找得好辛苦!”吴思若低声问小五子:“既然你找她找得那么苦,我这时候是不是应该装作吃醋的样子才有皇后的样子啊?”

小五子没回答她,吴思若觉得有点折面子,高声道:“皇上皇后已领百花谷心意,文姑娘请回吧。”

文思清看了看小五子,说道:“那在下这就告辞了。”又对吴思若说,“吴姐姐,你赢了,我这就回去。”

小五子失声叫出来:“你别走。”

吴思若偷看看小五子,发现他眼神都在文姑娘那里,奇怪了,你这个皇帝怎么这么苦情呢,前两天你看我就是这表情吧,你怎么瞅谁都这样啊?文思清还继续往宫外走。

吴思若低声跟小五子说:“你看她根本就不想走,进来的时候一眨眼,出去的时候得一炷香,你看现在还没走到第五根柱子呢。”

吴思若朗声道:“文姑娘,先不要走,姐姐记性不好,想问问你,陛下还认识几个像你这样的姑娘?”然后又低声对小五子道:“你看,刷的一下,又回到第一根柱子了,这姑娘真好玩,我帮你把她留下了吧。”
\newline

晚上小五子和吴思若进了洞房,见小五子要解她衣服,吴思若一下就慌了,羞涩道:“陛下,差不多就可以了,还要来真的啊?我现在才见你两三面,何况你还是当今圣上,我这心态还没从宫女调整过来,你要是个杀猪卖肉的,估计我还觉得咱俩挺般配的,没准儿就从了你。”

“我过去就是卖肉杀猪的。我跟你一样,也中过断魂掌,你睡一觉醒来,发现自己是个宫女。我比你还惨,我是一觉醒来在猪圈,我老板过来催我杀猪。”小五子把
备好的银镯子拿出来戴在吴思若手上,说,“这是我过去送你的。以前也有个姑娘,就像我苦恋你这样,苦恋着我,我跟那个姑娘什么心情,我全记得,所以我理解你对我的感觉。听你的,咱就慢慢来吧,一切都会好起来的。”

当晚两人和衣而睡,吴思若想着小五子的话,想恢复哪怕一丁点关于他的记忆,可她什么都想不起来。
\newline

小五子次日见文思清,问她百花谷的情况,怎么就忽然间就成了百花谷的香主了。

“百花谷已经解散了,只是沈总管给我的封号,他说······我来见你,不能比苏子瑶苏姐姐的职位低,昨天你要是真让我回去,我都不知道去哪儿。我在百花谷等了你那么久,都不见你来接我,我没有怪你,全天下都知道你很忙,你在忙着追封吴姐姐为皇后嘛,忙着替苏姐姐报仇嘛。”

小五子知道她吃醋了,过去哄她两句。文思清问他:“你一个封皇后,一个替她报仇,皇上你答应我的事儿可曾上心?那个人查出来没有?”

小五子脸色大变,结结巴巴地给文思清编了一个故事,说那个人早就被他五马分尸,恶狗分食。也不知道文思清信了没有,她反过来撒娇问:“那陛下什么时候册封我为皇妃?”

小五子以为撒个谎,会让一块石头落了地,可不知怎么的,心里更难受了。
\newline

{\centering\subsection{12}}

转眼半年有余,武林中风平浪静,只是宫中接连出现怪事,大公主,二公主,三公主,七公主,九公主,十一公主直至三十五公主,接连有十三位公主意外死亡,她们或是出外巡游遇险,或是睡觉时心脏骤停,或是骑马打猎时被山贼乱箭射死。小五子苦苦追查没有任何线索,本来是一个个意外事件,但是集结到一块发生,这其中必有玄机,小五子加强对其他公主的防护,之后一个月,竟再不见意外发生,如果有凶手的话,必然还会继续动手,这突然的停顿让宫中出现了各种鬼怪传说。

小五子曾召集武林的一些前辈,来宫中商议,并向他们说明各个公主的死因。向问和对着其中五个公主的死因沉思不语,小五子单独留下他。

向问和讲道:“令我百思不得其解的是,这些都是沈
老前辈的上乘功夫,早已失传,就连我这个关门弟子也没有学到。”

临别前,小五子问了他,无为神掌的功夫练得怎么样了。向老前辈沮丧道,此门掌法练得越深,功夫越弱,现在连个蚂蚁都拍不死。小五子叮嘱他记住那八个字,嘴上高调,手上低调。小五子说:“你不能死,也不能示弱,那个逆徒,全江湖唯一惧怕的就是你。”
\newline

出嫁南海的五公主像个不速之客,忽然回到皇宫。小五子虽然不时嫉恨五公主,但心里总觉得对她有些亏欠,生活在李准驸这样的窝囊废身边,五公主一定度日如年。就在五公主回来的前一天夜里,他还梦见五公主杀了李准驸,以至于第二天,见到五公主他还神情恍惚,直接问她:“你真把他杀了?”

公主盯了他几秒,点了点头,道:“杀了。”

小五子很懊恼,叹息道:“李准驸虽然笨了点,窝囊了点,马屁拍得也有点甜得齁嗓子,但总还对我有十二分的忠诚,你就这么把他杀了,不就等于是朕害了他吗?”
\newline

公主和皇帝冷了几日,有次主动找到他质问,那些死掉的十几个姐妹是怎么回事?小五子表示,不都记录在刑部了吗,还找我问什么。

“你离开那三年,宫中没有出现一次这样的事情,自你娶了皇后和妃子,接二连三地出意外,你不觉得该查查这两个女人吗?”

吴思若当时就在他身边,反唇相讥。皇帝帮衬皇后,公主吃了一鼻子灰,悻悻离开。

在宫里住了几日,意外又开始找上五公主了,比如房梁掉下来险些把她砸死,比如本该她乘坐的马车,马儿失惊,拉着马车在街上横冲直撞,公主一边加强戒备,一边让人秘密跟踪吴思若及文思清。

吴思若最先发现自己的宫里出现了奸细,问清楚后联合文思清到小五子那里告了一状。小五子把公主叫来,狠狠地怒斥一顿,让她回她的海南岛当她的寡妇。

公主刚出城门,小五子跟一只手骑马带人追了上去,小五子把公主拉到一边悄悄讲:“我知道你是对的,我也知道有人要杀你,所以我必须要把你骂走,你现在回海南岛也是无依无靠,我让人给你找了个地方,把你安顿下来,一旦查出真相,我会立即接你回宫。”

快马加鞭,两天一夜,三个人行至汴梁。小五子对
公主道:“我的记性不好,他们告诉我,这是我过去的藏身之所。其实我也不记得,哥哥过去是怎么待你的,过去我们俩产生过一些误会,我有待你不好的地方,原谅哥哥,毕竟是亲兄妹,我不会把你抛弃的。”

公主有些感动地对他说:“别说了,你我之间又岂止是亲兄妹这么简单。”

小五子带她先进了昆仑山庄。其实小五子自己也没进来过几回,反而公主一进来倒是轻车熟路,就好像住在自己的寝宫一样。小五子问她:“我的藏身之处,你如此熟悉?”

“我过去经常来,和你一起来。”

小五子让一只手留下来照顾公主,有什么消息可直接密报他。

公主送别他时,掏出一张羊皮说:“这是在寿南山,吴思若受伤时,我在她身上找到的。其实本来想马上给你,也想一块儿把吴思若给你,只是,我一时无法说服自己,再加上你的态度,就压了下来,我知道你一直在收集这个,你要集全了拿它做大事。”

小五子接过来说:“其实这东西没用,我好几张了,拼来拼去就是一片空白。”

“也许收集全了,你就知道了,你要相信我,昆仑公子,过去的事,你都不记得,但是你要永远永远地相信我,我五公主绝对不会伤害你。”

小五子开玩笑说:“你这个妹妹真有意思,我对你软一点好一点,你就对我梨花带雨的,你典型的吃软不吃硬啊。”

“你为什么叫小五子?”

“因为我失忆前在手臂上刻了个五,其他都是百花啊,瑶啊,这些女人名,那这个五字肯定是我的名字,小五子嘛。”

五公主盯着他的眼睛,一字一句地问他:“你有没有想过,你这个五,是我五公主的五?”
\newline

{\centering\subsection{13}}

小五子回宫之后,贴身太监给了他一张红布,上面是一些数字,小五子盯着这些数字,问太监哪来的。太监见其他宫女太监在场,悄悄说了一句话。小五子说知道了,然后一个上午都心事重重。

小五子先到吴思若房里走了一圈,全程一语不发,直勾勾地盯着她。吴思若问他怎么了,神神叨叨的,就
你这个劲,我过去怎么会喜欢上你啊?而且还是个杀猪的?小五子说今天累了,我先去休息了。

之后进入一间寝宫,对着纱帐里正在睡觉的女人坐下来,然后翘起二郎腿道:“我知道你根本没睡,我这有一个红布兜,上面有一些数字,这上面画了叉的数字是一,二,三,七,九,十一直至三十五,一共有十三个数,这半年里,依次死掉的公主是大公主,二公主,三公主,七公主,九公主,十一公主直至三十五公主,接连十三位公主。这其中还有一个没有画叉的,你在上面画了无数个圈,就是画不了叉,这个数字是五。起来吧,或者就躺在那,给我讲讲为什么。”

小五子把肚兜还给她,问文思清:“一十三个,我以前就有疑惑,拿到这组号码,我就更不明白了,我三十六个姐妹,为什么单单挑这些数字的公主来杀。”

“因为这十三个,加上五公主,她们都姓孙。”

小五子站起来苦笑道:“这些号码姓孙,其他的不姓孙吗?”

文思清肯定地回答:“不姓孙。你有三十六个姐妹,没有哥哥没有弟弟,就你一个皇子,你想过吗?”

“我知道,因为老皇上无子,才把我从太原召回来当太子的。”

“你有没有想过,嘉和皇帝在外面生了你这个儿子,回到宫里,却生了三十六个女儿,这不奇怪吗?她们都是被常公公,也就是你的钱老板,掉过包的平民家的女婴。”

小五子下意识地重复了句:“掉了包?”

“这些名单上没有的数字,四,六,八,十,一直到三十六,本来应该是皇子,但是当年的常公公为了保你做太子,宁愿自宫来到宫里一路做到了太监总管,皇帝最亲信的人。每次有妃子怀孕,太医因为常公公的授意,都会诊断为女胎。临产期一到,生下来男孩儿,即被抱走埋掉,同时送进早已备好的女婴替换。”

小五子问:“那你为什么单挑真公主来杀?单挑姓孙的来杀?你杀的都是我亲姐妹,我叫什么?我叫孙天奇,我也姓孙。”

“你真以为自己姓孙啊?三十六个公主,不管是真是假,都不是你的姐妹。”

“那我是谁?常公公为什么要保我做太子?”

“你是昆仑公子。”

小五子叫道:“我知道我是昆仑公子!常公公为什么要保我,他跟我是什么关系?”

文思清告诉他,常公公叫沈志基,当年孙家打入皇宫的时候,他还在襁褓之中,被皇后百花谷谷主抱出宫。沈志基成长于南京,二十三岁那年,生了一个儿子,叫沈辟朝,也就是复辟皇朝的意思。次年,他听说,孙家皇帝在太原与一余姓女子生下一名皇子孙天奇,且嘉和皇帝尚无子嗣,沈志基笼络嘉和派给余姓女子的太医,将自己的儿子沈辟朝与孙天奇掉包,并杀掉余姓女子,谎称暴毙。沈志基自此下狠心,阉掉自己,去宫中当了太监。二十一年间,他成了皇帝最相信的人,并联合太医将每一个出生的男胎掉包为女婴,保你做太子直到登基。

小五子半天缓过神来,问:“那我是孙天奇?还是那个沈辟朝?”

“你也知道,早几年你做太子的时候,以昆仑公子的名义冷酷无情残杀无数,为什么?如果你是孙天奇的话,不管是不是私生子,你总是嘉和皇帝的亲儿子,用不着心虚,不必害怕。正因为你不是孙天奇,你是沈辟朝,才不得不有众多恶行,你问我常公公是谁,那我告诉你,他是你父亲,确切点说,是你的父皇。”

“你是常公公派来的?百花谷派过来的?你怎么可能一夜之间对我一点感情也没有?”

“就算我对你没感情,但是也没杀了你对不对?”

小五子盯着她,问道:“你父亲爷爷的事情,你全知道了?”

文思清回答:“你都不敢跟我承认,我这次来,是百花谷派过来的第二任香主。”

小五子问:“第一任是谁?”

“去年南海真人让你选一个你最爱的女人,并且把她杀掉的,苏子瑶。你真以为你俩青梅竹马?你真以为她爱你爱到不惜为你去死?她是可以为你去死,因为你就是她的任务,你没欠她那么多,她一生都是为了你复辟这件事活着的。”

小五子仔细回想了一下,自言自语道:“那她对我多少还是有些感情的,我能看出来,但是你已经变得很冷酷。杀了所有姓孙的人,再想办法让我知道,我就是沈辟朝,安心做我的沈家皇帝,就是你的任务?”

“我还有第二个任务,谷主怕你像过去一样不肯当皇帝,已留了后手。”文思清说着,指指自己的肚子,低声道,“我替他们怀了沈家的龙子,也就是你的儿子。”

\newpage