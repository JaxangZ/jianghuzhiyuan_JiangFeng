\section{伍}

{\centering\subsection{1}}

那就不是小五子了,是昆仑公子,醒来在河边洗脸,
看着水面上的自己都想抱拳作揖,久仰久仰,恕在下有眼
不识昆仑公子。多了他就不敢想了,当小五子就要在文
思清和吴思若之问纠缠,昆仑公子竟然还有俩,苏子瑶和
乔文君。他冲着河水长叹一口气,转身对阁老和仙人说:
“洗好了,我们走吧。”

被他俩擒走也不算坏,抛开感情不谈,真自由了命都
保不住,大会那么多人,恨不得镶上獠牙啃他两口。二老
看起来也不打算杀他,要活着带到南海真人那里。只是
路上实在无聊,两人不搭理他也就算了,他们都不说话,
互相有心事的样子,就那种漫长的无言,时光有多久,沉
默就要多久。

可是晚上他们却充满着互动交流。头几天小五子还
没被绑起来,怕他夜里跑了,仙人和阁老把他夹在中间
睡。睡到半夜小五子被摸醒,阁老在后面把手伸到衣服里
抚摸他后背,满手老茧,摸在背上像澡堂子搓澡,只是速度
更慢,一寸一寸地往下摸。手就要伸到裤子里的时候,小
五子扭了一下屁股,仙人在前面也把手伸进来了。他先从
领口进,从脖子往下摸,检查完右胸,再检查左胸,手掌在
小五子心脏的位置停下来,感受他心跳,时不时还要捏两
下,另一只手从肚皮上进去,拇指压在肚脐上,四根手指以
肚脐眼为圆心划了一个圈。小五子努力挣脱,可前后身都
被他们掌力吸住。他屏住呼吸,忍住不吐。忽然四掌将他
翻转,这回换阁老摸前面,仙人去抚摸后面。

持续一两个时辰,来来回回翻了四五次,小五子都要
呻吟了,等到掌力稍微松下来,他猛地坐起来喊:“你们俩
一把年纪了,到底想要干什么!”

仿佛馋嘴被发现,阁老马上翻过去,背对着小五子打
呼噜。仙人的手还在摩挲着他后背,在后面嘿嘿地笑,说
师弟,你不是在背着我验他的伤吧?

“不错,明人不做暗事,这一断魂掌正是打在他膻中
穴偏下一点。”

“嘿嘿,膻中穴是不错,可惜当时有人拦了一下,失了
七成力道,我看他真正的一掌是背后左侧的风门穴。”

“是吗?”阁老翻回来,眼珠子比萤火虫还亮,又摸了
摸他的胸口,轻捏两下,点着头说,“果然如此,把他翻过
来,我要好好摸摸他后背。”
\newline

白天好一些,三个人面无表情,就当昨晚的事没发
生过。大漠仙人永远都在捻佛珠,二三十颗串成一个链
子,只要不是在睡觉,不管他是在吃饭,在赶路,还是在
杀人前,杀人后,手上肯定有个链子慢悠悠地转。而且
还不是一颗一颗地捻,跳着捻。小五子有一次看明白
了,他先捻一颗,摸准了,圆圆的,下一次直接摸两颗,第
三次三颗一起过,第四次四颗,一二三四,加起来十颗
了,链子上的佛珠捻了一半,还要第五次五颗,第六次六
颗,一直到第二十四次,中指指节顶着出发那颗,拇指一
颗颗捻着查,第二十四颗刚好回到出发那颗,一整圈链
子,一个轮回,他又从第一个开始捻。小五子就要疯了,
有事没事都强迫症似的盯着他捻佛珠,一直等他捻完二
十四颗,抓紧时间揉揉眼睛,歇一下,好等他从头再来。

蓬莱阁老就好很多,他不捻佛珠,也不看他三师兄捻
佛珠,他看打身边走过的年轻姑娘,眼珠子发亮,满脸期
待。小五子开始以为他是淫贼,可惜还不是,白瞎那么好
的功夫,就是坐着不动,满脸期待地把迎面来的姑娘硬生
生地看成背影。一不用钱,二不用强,老以为哪个姑娘能
被他吸引,在他这儿停一下,没话找话问个路什么的。姑
娘一靠近还特显摆,眼巴巴地想露两手,把筷子插桌子里
面,或是用手劲把金元宝捏成小兔子。问题是您都六十
多了,要么别想,想就敞亮的,花钱去窑子,或是当个采花
大盗,这么大本事,你就是当第一淫贼,武林里能惩治你
的也不超过仨。头发没几根了,还老觉着自己潘安宋玉
谁家姑娘主动倒贴。

一次还真有个姑娘过来了,十八九岁的样子,一脸稚
嫩。那时他们在客店里等面条,小姑娘跟着她十几个师
兄从外面进来。为首的年轻人在二老面前作揖鞠躬,说
自己是崆峒派第十六代掌门人,想跟两位老前辈借一个
人。说着还满腔怒火地指着小五子,说我们祖孙三代都
被这小贼给害死了。大漠仙人刚捻完第六个佛珠,一二
三四五六,一下子要捻七个了,生怕这会儿错了,他停了
一下,说:“你想把他借走,可你怎么还啊?”

“晚辈想把他拉到我父亲爷爷的牌位前,手刃了这小
贼。”

“然后你怎么还呢?”

少当家的也明白这么还不合适,但是大仇未报,只好
觍着脸说,我跟您借活的,杀死了之后还您二老全尸,再
打副棺材当利息。仙人斜眼看他,不说话了,聚精会神去
捻下面七个佛珠。小五子满脸憧憬地看着少当家的,心
里想着带我走吧,现在就杀了我吧。店小二把第一碗面
端上来了,他望了他们好半天,叹一口气低头吃面。

阁老倒是一直盯着小师妹,把筷子掰成十几截,不声
不响全都拍进桌面,心想她怎么这么害羞,见到喜欢的男
子都不敢抬头直视。后来他着急了,但他又不是主动跟
姑娘说话的人,他只好跟那少当家的没话找话:“你们崆
峒派三代人是怎么被他害死的?”

少当家的又自我介绍一遍,说他是崆峒派第十六代
掌门人,他爷爷,第十四代,当年就是被这小贼杀死的,他
爹爹,报仇途中,不小心坠崖身亡,我崆峒派和你昆仑公
子不共戴天之仇!

“那第三代呢?”仙人问。

“啊?”

“你说,祖孙三代都被他害死了。”

他眼神迷离了一阵,说自己就是第三代,家破人亡,
爷爷死了,父亲死了,自己苟活于世也是行尸走肉。仙人
想了想,跟阁老说,行尸走肉,还是你来吧。第三碗面迟
迟没上,他把佛珠放桌上,拽过小五子吃了一半的碗接着
吃。小五子只剩筷子没有碗,抬头看着他们,把还没咽下
去的面条细细再嚼一遍。阁老也不掰筷子了,突然跳到
少当家的面前拍了一掌,瞬间又坐回到位子上,吃刚上来
的第二碗面,跟小五子说:“你真是造孽啊,害了人家祖孙
三代。”

他的那些师兄弟开始吓一跳,见到掌门人也没倒没
晕,那就是没事,刚要说谢前辈手下留情,少当家的呵呵
傻笑起来,梦游一般先出了客栈。有弟子明白了,这是蓬
莱掌,就算是活着也疯了,也是行尸走肉。哎哟,他们明
白了,怪不得要阁老出手。这些弟子的剑拔了一半却不
敢抽出来。

不愧是属仙人掌的,仙人半碗面就吃饱了,他放下筷
子换佛珠,问你们这里面谁来做第十七代掌门人。他们
互相看看,谁也不敢应。

“不管谁做,别再报仇了,你们回去的时候,顺便告诉
后面那些跟着的沙河帮、嵩山派,都散了吧,没本事报仇,
无非再搭几条人命。”

十几个弟子低着头走出客店,那个小姑娘气不过,没
一会儿又跑回来,右手瑞在怀里冲他们三个喊声,看镖!
随后右手甩出一把毒针,不等他们反应过来,转身就跑
了。十几根针对二老当然不在话下,阁老拂袖要接了,只
可惜小姑娘功夫不到,扔出来的时候轻飘飘的,离他们还
有好几米,一大把全掉到了地上。

小五子借两步,低头看地上的毒针,问:“后面还有人
寻仇?”

“跟两三天了,就在后面五六里。”阁老吸溜着面条
说,“你停他也停,你走他也走,又不敢上来。前面还有堵
你的,上午过去那几个小道士,应该是报信的。”他说着端
详一下小五子,“你之前多大本事,招那么多仇家?”

小五子摇摇头,过去的事也不知道,但本事再大也不
及你们俩,怎么就闹到整个武林欲杀之而后快?他说这
几天都奇怪,那天昆仑山庄那么多人,居然没一个找你们
俩报仇的?仙人和阁老相视一笑,这么简单的道理,你想
想就明白了。

没仇家,是因为不杀人吗?不可能。是他们不敢来
寻仇吧?二老摇头。第三碗面上来了,仙人推到小五子
面前,让他快吃,要赶路了。小五子用筷子挑几下面,热
腾腾地白气扑上来,他吹两口,入口还是烫,端着一筷子
面条继续吹。仙人着急了,佛珠换到左手,右手食指插到
碗里。只见他面不改色,食指在碗中发出阵阵寒气,不一
会儿一碗面结成了冰。他拔出手指,刮掉挂在上面的冰
碴,叮嘱小五子:“快吃吧,现在不烫了。”

小五子用筷子敲敲上面的冰,为难地看着这一碗面,
起身跟伙计要两个烧饼带走。他走到店外,解开缰绳把
马牵过来。一共三匹马,二老各骑一匹,照规矩白天赶
路,小五子要横着趴在第三匹马上。仙人用绳子把他绑
在马背。小五子看着地面,说我知道了,早听说你们几十
年没走江湖,没人寻仇是因为你们把仇家都熬死了。

“杀人很麻烦的,你杀一个人,他师父师母,他哥哥弟
弟,甚至他儿子过了十八年,都找来报仇。所以说,”绳
结打好,仙人把缰绳和自己那匹马系在一起,翻身上马,
慢悠悠地讲,“要杀就灭门,免得被一茬又一茬的人过來
烦。”
\newline

{\centering\subsection{2}}

他们两匹马走前面,小五子趴马背上只能看见途经
的草木、污泥和沙尘。一路上他也不消停,问个不停,南
海真人在哪啊,是在南海吗,南海在哪啊,咱们花两三个
月过去,万一人家不在家,去长白山了呢,到时候咱们北
上去长白山,人家又回南海了怎么办,后来把仙人问烦
了,下马在他脖子后面点了一下。小五子第一次被点穴,
还觉得挺新鲜的,可惜已然绑住不动,看不出效果如何。
他眨巴几下眼睛,贴在马屁股上的手指还能动,他脸蹭着
马背,悬在另一侧的双腿甩了两下。他问到底点的是什
么穴,张了几次嘴就是出不来声。完蛋了,他被点了哑
穴。

露营休息也没给他解开,看样子要一直哑巴到南海。
小五子没胃口吃饭,跑到树下去抠嗓子,指甲够到嗓子眼,
连恶心的声音都发不出来。以前在田独掷骰子听哪个赌
客说过,人被点了穴,十二个时辰会自动冲开。十贰个时
辰是一天一夜,他早早躺地上睡觉,等着二老晚点把他包
起来,要是嗓子还在,他肯定要抱怨,一个比一个抠,花点
钱住店啊,拢一堆树叶子当床算什么玩意儿?

天不亮他就醒了,手脚被捆,像个粽子一样勉强坐起
来,看着二老起身洗漱。他要有耐心,等着时辰一到“叮”
的一声,就可以说话了。从日出到日落,声音还是没回
来。他拿树枝在地上写字,问他们是永远哑巴,还只是暂
时的。一句话写半天,仙人看过一眼把这些字踩掉,然后
冲他微笑,安慰他别担心,人和人本来就不需要说话。

十几个字他又写一遍,拉阁老过来看,虽然出不了
声,还是张大嘴巴问一遍,我永远哑巴了,还只是暂时
的?阁老看看他嘴型,又低头看了半天。

“什么意思?”

“阿巴,阿巴。”

“我不识字。”

“阿巴!阿巴······”

他计划逃跑,寻找一个石片藏袖子里。白天赶路还
是他们俩在前,他在后面慢慢割绑在身上的绳子。仙人
今天的心事似乎更多一些,走了一个多时辰突然哈哈笑
了起来。

“不识字这个办法好,师父的秘笈自然不是你偷的
了。”

“我打拜师学艺那会儿就不认宇,口诀都是你和大师
兄读出来,我硬背。”

仙人长叹一口气,说:“这么多年也难为你了。”

“你说我装这么多年?后面那小子,验过他的伤没
有,秘笈就在大师兄那儿。”

仙人回头看着。小五子马上停住,石片推进袖口。
仙人摇着头说,我看也未必。小五子等他们转过去,不再
聊这话题,进入惯常沉默,试着把石片从袖口抖出来。马
上一个颠簸,石片掉了了下去,他想今天就算了,脸贴在马
背眯了一会儿。

醒来时他知道哪里不对劲了,绳子割了一半,这么大
的口子晚上肯定要被发现,那就再没有机会了。他盯着
豁口,又看看日头,时间不多了。他头伸过去用牙咬。咬
到眼泪都出来了,最后一个细绳被他咬断。他将绳子在
手掌上缠几圈,挺起上身,胳膊撑在马背上,足尖离地面
不到一尺远。他心里默数着三二一,前方就要拐弯的时
候,他手腕一松,轻轻落了下来。他先不动,趴在草从里
看着前面的二老。夕阳西下,二老的身影刚好挡住迎面
的斜阳。只要再数十个数,就可以顺势从山坡往下滚。
他闭上眼睛,尽量数慢一点,那三匹马越来越远。这时只
听一声长啼,没有了小五子的负重,那匹小马高高兴兴地
跑到二老坐骑的前面。

仙人骑着马朝这边过来,小五子在想一会儿该怎么
说,干脆反咬一口,你们怎么搞的,正睡得香呢,被你们摔
下来,要是三番五次地这么摔,也别去南海了,还不如直
接杀了我。话都想好了,忽然记起自己被点了哑穴,他瞪
大眼睛冲仙人摇头。

马停在小五子身旁,仙人捻着佛珠,在马上俯视着
他,那表情似乎很伤心,我们对你这么好,你居然要跑?
阁老后面跟过来,建议他别找大师哥了,就在这儿把他剖
了吧,受的什么掌什么伤我们看不出来吗。仙人点点头,
让小五子站起来,把上衣衣服脱了,接着他问阁老有没有
带刀带剑。

“手撕就行,”阁老下了马,手在小五子胸前比量,“先
验心还是先验肺?”

“要是断魂掌的话,直接验脑子。”

阁老两只手从他脑后摸去,在找从哪里下手。仙人
叮嘱他轻点撕,脑浆迸出来就什么都验不出来了。看到
小五子的表情,他扯一块布,说蒙上他眼睛,别让这孩子
先吓死了。

面前一片漆黑,小五子感觉到阁老的两个拇指压在
他鼻子两侧,另外八指捋着后脑勺的中轴线,找到受力
点。指甲都已经嵌进头皮了,他喉结一动,咽了口唾沫,
听见阁老说:“万一发现不是大师兄呢,是你干的呢?他
死完就是我死。”

“真是我的话,我现在就能杀了你。”

“你还不敢。”

小五子身子一轻,被提到马上。阁老在他身后上了
一匹马,轻声问他好好想一想,到底是谁伤的你。双眼被
蒙,迎面是一阵阵的风,小五子摇摇头,听见两侧树林唰
啦啦啦地响,后来风更大了,有雨点打在脸上。阁老停住
马,把他带到榆树里避雨。小五子解开眼前黑布,看着头
顶一串串的榆钱流口水。他爬上树干,让阁老递他一根
杆子打榆钱。

后来仙人也到了,两个老头并排坐在树下的大石头
上等雨停,被打碎的榆钱从树上飘落下来。仙人手捻佛
珠说,我一度还以为是你,如果真是大师兄偷的,他照着
秘笈练了二十多年,你我就算联手也不是他对手。

“去还是要去的,又不至于死在他手里。”阁老说,“把
事情查清楚,不是还有小师弟能帮咱们。”

“何府灭门的事,你听说了吧。他们跑到极北之地,
还是被大师兄找到了,怕是师弟也凶多吉少。”

“那就再叫上师妹,虽然几十年没联系,她百花谷也
会助咱们一臂之力。”

小五子在树上刚摘到两串榆钱,听到“百花谷”三个
字愣了一下,他哇哇叫了两声,把手里的榆钱扔下去。二
老接在手里,一声不吭地看着大雨吃榆钱。小五子留在
树上边摘边吃,小师弟是何员外的师父,前任丐帮帮主,
他早知道了,他们还有个小师妹,竞然是百花谷谷主。他
现在是丐帮帮主,以前可是百花谷少谷主啊。他嘴里嚼
着榆钱笑了起来,双重的关系,你们没理由杀我啊。

天黑以后雨终于停了,有个送葬队从山那边翻过来,
看到树下有人,他们又开始敲锣打鼓,一个个晃着脑袋吹
喇叭。阁老叫他们站住,问什么人死了,高兴成这样。领
头的抱拳作揖,说家中私事,就不劳二老操心了。他身后
的少妇估计看出来这两个老头不好惹,上前解释是我们
家老爷的偏房,老爷早两年就不在了,这个月小老婆也死
了,你说我们当家的能不高兴吗?说完她笑眯眯地看着
阁老,弄得阁老春心荡漾,仰头对树上的小五子说:“臭小
子,这个真不是找你来寻仇的。”

仙人一旁看着,有了主意,一跃到车上掀开棺材盖,
里面的女尸涂了厚厚一层胭脂,面色苍白,看起来比那少
妇还多几分姿色。他说你们把棺材马车留下来,剩下的
路我们帮你送。十几个人一听紧张起来了,领头的说,再
不济这也是我们王家的人,怎么随便丢在路上?仙人摇
摇头,那意思仿佛是,这事就这么定了,怎么还商量起来
了?阁老也不明白他二师兄是什么嗜好,就算那女的好
看一点,可毕竟是死人啊。

领头的扎起马步,摆好架势,说阁下留下万儿来。小
五子在树上看得直摇头,还留下万儿,他杀人杀满门,知
道了他是谁,你们今天全都死这儿。小少妇又来帮说话
了,说前辈息怒,有什么事咱们好商量。

“不商量。”仙人奇怪这有什么好商量的,“把棺材和
车给我就行了。”

领头的大吼一声扑过来,后面十几个人一拥而上。
看起来还不是寻常的大户人家,好像这些人功夫都不
弱。小五子在树上吃着榆钱,看仙人人群里几进几出,半
炷香的时间都不到,十几个人都躺在地上。

小五子揣满榆钱从树上下来,此时躺在地上的人一
个个都站了起来。他知道了,在何府他见过,仙人掌,吴
思若以前还给过他一掌,即便她功力不够,也让他个把星
期食不下咽。仙人跟领头的说,你们还能活几天,快回去
料理后事吧,人就别送了,我怕等你们到了那儿,一起死
在坟堆里。

“你是大漠仙人?”小媳妇问道,然后她指着阁老问,
“这又是谁?”

仙人说他是我三师弟,我感觉他的蓬莱掌要比我的
仙人掌凶残多了。领头的满脸恐惧,虽然此刻身体不痛
不痒,但他们都知道,自此以后不吃不喝,直至身体干涸
而竭。断魂掌,仙人掌,蓬莱掌,到底哪一个更凶残,江湖
上已经讨论了几十年,如果是你,必须选一个,你希望是
失忆,饿死,还是疯掉?换现在小五子觉得断魂掌还好,
重启的人生还会有新的美好。他想问问昆仑公子,倒退
几年,那么多难以割舍的爱与情感,你最怕的是断魂掌
吗?可能,断魂掌没有伤害到他,真正伤害的是苏子瑶和
乔文君吧。

那些人离开后,仙人要小五子把棺材里的女尸抱出
来。不小心碰到尸体的手,小五子头皮发麻。他比画着
放在哪。仙人说随便,主要是你躺进去。小五子双手发
抖地扶着棺材边,两只脚迈进去,慢慢躺在里面。棺材里
一阵阵寒意。他仰躺着看夜空繁星,仙人将棺材盖罩上,
让阁老找六根筷子钉进去,啪啪啪,啪啪啪!一片漆黑,
他一时喘不上气,他明早会闷死在这里。忽然一声巨响,
五个手指从棺材盖戳进来,小五子吃了一嘴的木屑。原
来是给他透气用的,没多久他们找地方睡觉去了,留小五
子躺在棺材里。头顶的几个洞就像是九阴白骨爪的骷髅
头,秋后晚风从洞口细细地吹进来,偶尔睁开眼睛,他还
能看到洞外的微微星光。
\newline

{\centering\subsection{3}}

棺材从外面看起来是木制的,打开盖里面还是木头,
可是怎么躺小五子都觉得是躺在一块铁板上。何况也只
有三种姿势,双腿伸直了平躺,左腿稍微弯曲地平躺和右腿稍微弯曲地平躺。小五子想,可能为了尸体防腐吧,哪怕午后烈日,棺材里面都是一片冰冷。

屁股凉尿就特别多,仙人跟他规定好的,有事敲棺材盖,敲一下是上厕所,敲两下是饿了,敲许多下就是无理取闹,没人搭理你。可是马蹄声声,一下两下根本听不到,这样小五子又不尿急又不饿,一天都在咚咚咚地无理取闹。那天下午他憋得在里面直踢腿,恨不得用脑袋把棺材盖撞起来,马车在山路上把他颠得一上一下,终于最后一次落下来时他尿了裤子,眉头舒展,长舒一口气。原来那些卧病在床的人有卧病在床的爽法,尿了千百回,躺着尿最舒服。

跟所有不要脸的事情一样,一旦开头就上瘾。白天他在棺材里睡觉,睡醒了敲敲棺材盖,没人管就进入生活不能自理模式。要是睡太多,实在睡不着了,他就想想这三大恶人,三大高手,也是三个师兄弟,到底是怎么回事。

他们心不和面也不和,生怕对方是偷秘笈的那个,练了二十多年大功告成,弄死另外两个。老大南海真人,老二大漠仙人,老三蓬莱阁老,这都是后来的封号,几十年前都在一座山上,跟着沈世祖学艺。师父对每人只教一掌,各练各的,为的就是互相牵制,彼此能有个顾忌。我虽然会仙人掌,可你那蓬莱掌也不是好受的。不过十年之后事情失控了,有人把秘笈偷走了,查不出是谁,沈世祖一气之下将三个人都逐出师门。头一个往南,做了南海真人,第二个往西,做了大漠仙人,三师弟往东,做了蓬莱阁老。

但这事没完,总有一天这人三掌练成会跑出来祸害武林。沈世祖开始琢磨,能不能开创一种掌法,比这三种都厉害,收个品行还不错的弟子传授给他,于是就连上了向老帮主的无为掌,何员外怎么带着他师父东躲西藏,还是被那个弟子找着了,蒙着脸把何府灭了门。向老帮主在京城大牢里面,腊月初八就可以出关。也不知道仙人和阁老是否欢迎他,这个他先不说。那百花谷谷主又是怎么回事,他们的小师妹,似乎沈世祖也没教她什么本事,他又怎么当上少谷主的?

算了,反正也想不明白,他弯一下左腿,打算再睡一觉,隐约觉得哪儿不对劲,闭眼睛的时候知道问题在哪了,他在棺材里吃喝拉撒好几天,主要是撒,怎么会一点积水都没有,那些尿都是从哪渗出去的?他把手垫在屁股下面捋着摸,果然正中间有一条缝,棺材底是可以开合的。那就有逃生的希望,他不睡觉了,也不尿尿了,在几
尺空间里一寸一寸地找开关。

原来这木枕就是机关,推半圈能把底板打开。他等待时机,从洞口看天色已晚,阁老在赶马,仙人在车里睡着的时候,他左手撑着身子别掉下去,右手抓枕头边拧了半圈。底板打开时他差点叫出来,下面是实的,铺了好几层的金条。

那些人不是送葬,是镖局送镖,怪不得个个会武功,豁了命保这棺材。还说什么小妈死了,当家的乐开花。小五子把金条一根根挪开,最下面是一个檀木板,留了两个拳头大的透气孔,可能那些尿液渗来渗去从这里流出去的。他颇为遗憾地摸了好半天,把这些一一复位,又躺回板子上看头顶的手指洞。

他有点悲伤,倒不是怕死,就是什么事刚有点希望,一盆水又给浇灭了。晚点他们找客栈休息,仙人和阁老开了一间上房,把棺材推到马厩里。酒足饭饱阁老下来把棺材盖打开,扔给他两个馒头。小五子坐在棺材里,吃一口馒头,就一口馒头。看着他干嚼,阁老有点心疼,进去拿了两个空盘子出来,跟他讲这家是广东的大厨,味道还不错,这个盛的是上汤焗龙虾,那个刚才是脆皮烧鹅,盘子刚刷没多久,你用馒头蘸蘸还有味。

小五子点点头,满嘴的馒头噎得眼泪都要出来了。阁老也是,似乎这辈子都没对谁这么好过,叹了口气,说混江湖就是弱肉强食,你打不过我们,按理说该把你杀了,怎么可能在你身上贴钱?小五子馒头嚼得直掉渣,努力往下咽,他摸摸底板,他想说这下面一百来根金条,随便抽出一根能把这酒店客栈都买了,但它买不来我命,我打不过你们,所以它就是你们的。

当然不能说,没点哑穴也不能说。吃完饭他平躺下来,阁老问他要不要上个厕所,小五子摇摇头,冲他微笑,那意思是不管怎么样,我都谢谢你。阁老把棺材盖扣上,抽出六根筷子啪啪啪地钉进去,对着洞口说:“那就早点休息吧。”

他睡不着,四周除了黑就是黑,感觉自己都要被侵蚀掉了。回光返照似的东想西想,但就是记不起过去。什么他都想,各种人各种事,从文思清到钱老板,从何员外到苏子瑶,甚至连关长老这样无关紧要的人他都要想一遍,揣测他能不能干掉马长老,当上丐帮帮主。想这些干嘛\footnote{原文“干吗”,在考虑之后不行就不再改这个了}呢,三年以前,他中断魂掌的那几个时辰,是不是也这么瞎想?

冥想了小半夜,他一下子明白了,他是在告别,记得的人,见过的事,在离开这个世界以前,一页页重温一
遍。想清楚以后就可以死啦,他闭上眼睛,试着不喘气,憋得不行了才吸一大口,再使劲憋住,悄悄喘两口,直到呼吸均匀地睡着。

他梦到自己过鬼门关走黄泉,牛头马面前面带路,两边藏好的妖怪时不时蹦出来冲他吼,也不碰他,嗷两嗓子又退回去藏起来。小五子不明白,这都是干嘛\footnote{原文“干吗”}呀,死都死了,还怕这些吗?牛头跟他解释,这都是阎王爷安排的,怕有些人没死透,黄泉路上就把他吓死。马面不说话,走在最前面,抢先两步把鬼门关打开,门那边居然一片白光花团锦簇。马面回头说,我们不喜欢阳气太重的人,他手臂伸向鬼门外,忽然变成姑娘的声音说:“小五子,你死了没有啊?”

他腾的一下醒了,大口喘气,一脑门子汗,那声音又来了,问他你死了吧,痛快说句话。那是吴思若,仿佛溺水十分钟,就要沉到湖底的一刻,有只手把他提了起来。他阿巴阿巴地乱叫,使劲敲棺材盖,这还不够,他想抱住她,永远不松开。可是在棺材里翻个身都不行,他伸出右手食指,从洞口穿出去,怕她看不到,露在棺材外的半截手指拼命地动。

没声音,他手指扒着棺材盖,就像坠崖的人死命抓着岩石,生怕自己掉下来摔死。我看见了,我看见了。他听出来她要哭了,手指伸直,指肚似乎成了他的脸,他慢慢转着手指,想知道转到哪里停下来,好好看看她。

有了新状况,大漠仙人出来了,质问她在马厩干什么。吴思若连诓带骗,故作欢喜说,师父原来你还活着,他们都说你被阁老杀了,我还以为棺材里面的······后半句不说了,扶着棺材盖假哭。阁老也醒了,见到这么好看的姑娘,从窗户上翻了好几圈蹦下来,问仙人这丫头是你徒弟?那该喊我师叔。吴思若盈盈一拜,一声师叔喊得可甜了,说昆仑山庄见过师叔一回,从此就天天想着您老人家。阁老激动了,闯荡江湖这么多年,终于碰到欣赏他的女人,他吞吞吐吐,好半天也没讲清楚一句话。吴思若说不着急,您若能把我师父请走,叫他别跟着我们俩,我好好听您讲什么。阁老只是笑,隔着棺材都能想象他花枝乱颤的样子。仙人声音压低,让她先进客栈,有什么事明天再说。

那就明天吧,小五子听见阁老一连串的扑腾,跳回到客房,仙人甩两下袖子,朝这边走过来。我得走了,吴思若轻声说,你放心,我肯定救你出去。可能怕他不放心,可能她自己也没信心,她又补了一句,救不出去,我陪你一起死。然后她伸出手指,点在了小五子一直在等她的
手指上。
\newline

{\centering\subsection{4}}

虽然还是老样子,每天躺在棺材里看头顶的五个洞,但没那么闷了,一是吴思若时不时弄点花生瓜子塞进来,再就是可以听她和阁老聊天来打发时间。他们也没聊什么干货,除了打情就是骂俏,可是对抗沉闷就这样,总要发生点什么,既然好事不常来,听点恶心话,刺激刺激肠胃,时光也会不知不觉溜走的。

加上吴思若,马车里已经放不下棺材了,仙人在车外吊了两根绳,把棺材悬在马车的一侧。这样小五子更舒服,晃晃悠悠跟摇篮似的一会儿一觉。阁老还是坐前面赶车,吴思若坐他旁边为他加油打气。仙人倚在马车的最后面,捻着佛珠眯着眼,看前面那俩人啥时候能上天。

吴思若勾搭老男人确实有一套,也就一套,不管阁老干什么事,她都是拍着手说你好棒,天哪,你这么厉害。阁老策马扬鞭,吴思若惊呼,呀,你这么厉害,你这手臂应该能一把抱起我吧?阁老吃饭拍筷子,吴思若装傻,筷子怎么不见啦?阁老把桌子劈开,筷子就在桌缝里呢。店小二不干了,抬棺材进来也就算了,吃一碗蛋炒饭你劈我桌子干嘛\footnote{原文“干吗”}?可没人搭理他,小二跟透明的一样。吴思若睁大眼睛,嘴巴合不上,一字一顿地惊叹,怎,么,可,能。逼得小五子任督二脉差点打通,冲她喊出来,一整天一个路数,你换个姿势行不行?

到了晚上还真有新姿势了,吴思若开始呻吟了。前后也没个过渡,就是眼瞅着天黑抓紧再跑五十里,阁老狠狠地抽了一鞭子。这时吴思若跟着轻叫了一声。阁老愣住了,上一次听到这种声音,还是给小五子检查身体的时候。他问姑娘怎么了。

“你轻点,疼。”

阁老又狠狠地抽了一鞭子,说:“不行,赶时间。”

“啊,疼。”

真听不下去了,小五子把瓜子放下,打开底板去金库转转。一块块金条腾出来,他把瓜子皮掏干净,从最底下的通气孔扔出去。不行,还不够解气,他拽根金条对着孔外扬起的尘土松了手。金条留在了路面上,可是马车已经向前跑了五十米。一间酒楼就这么被他扔掉了,还挺过瘾的。他连扔三根,时候不早,他把金条一根根归位。金库没那么满了,少了三根金条就多了几本书的空隙。他拇指食指比画金库有多高,接着比画一下自己的头。
一根一根地把金库腾出来,他合上底板想,以后没准会有用。

再上来车速变慢了,阁老也不抽鞭子了,他在和吴思若商量,你师父也被我赶走了,现在只剩我们俩了,晚上你来我房间切磋武艺吧。小五子鼻子一酸,他想起以前在丐帮的时候,也是叫吴思若晚上来我房间,那时说是给讲讲你身世。时过境迁,他宁可死了,也不想吴思若把身子卖出去。

可吴思若不知道,她还在跟阁老讨价还价,她说没用的,我师父就在前面等着我,只有他死了,我和你才能远走高飞。阁老不说话,好长时间吴思若也不敢多嘴,大概有半个时辰,阁老喊前面骑马的二师兄上来乘车,同时低声对吴思若说:“哪怕我真杀了二师兄,就剩咱们俩,这小子你也救不走。”
\newline

{\centering\subsection{5}}

后来吴思若要走了,照顾好他最后一餐,她说她没办法,她要去找人帮忙,然后她又说了那句话,你要是死了,我跟着你死。小五子坐在棺材里吃糠咽菜,他想说你谁啊,咱们算什么啊,还带殉葬的?我相信我要是死了你会难过,但最多俩月,日子往后过,结婚生子哪样都不耽误,过个十年二十年这会成为你炫耀的资本,跟儿子说当年有个小伙子特别喜欢我,可惜他命不好,后来被你爹给杀了。小五子看眼阁老,也没准,什么事都可能发生。

真离开那天小五子还是舍不得,他想说你别死,我不领你情,好好活着得了。可是他哑巴,说不出来,想写下来又没纸笔,只能不理她。他怕多看她两眼会让她动了情,以后真要寻死觅活的。他躺到棺材里不出来,对她伸进来的手指无动于衷。直到确定她已走远,他才敲了敲棺材盖。

后面路程还挺顺利,没寻仇的也没救人的,三个人都一声不吭地往南走,从汴梁到黄陂,从黄河到汉江。进了汉口他们改长江水路,马车不要了,棺材还得留,歇下来搁在甲板上,里面金条早被小五子扔得差不多,就剩几块堆在金库角落里。

刚一上船他有点晕,随着浪花晃晃悠悠。以前没坐过,小五子确定,二十多年来他第一次坐船。船上食物紧张,能分给他的就更少了。反正他也没胃口吃,赶上风浪大的时候吃什么吐什么。从汉口上船,还没到九江他就已经开始虚脱,持续的昏迷。眼睛都不敢睁,面前全都是
花的。偶尔二老关心他,怕他死在船上,把棺材盖打开让他晒太阳,他都会捂住双眼,求他们把他送回到黑暗里。

不过耳朵还没坏掉,不时能听到水浪、船夫的号子,以及迎面过来的船冲他们鸣笛吹号角。大概跑了半个月,江上的号子多了起来。掌舵的说他们到南京了,再跑个一天,就能从江宁换船出海了。小五子感觉天气应该不错,阳光肯定刺眼,四周都是号子声,那些即将上路的和终于抵达的船夫们互相吹号角致意。掌舵建议他们靠岸补给,等出了海可就再没有加水补粮的机会了。仙人点头应允,船慢慢进港,迎面一艘花船挡住了他们的航路。

不是纸扎的,是真的花,从船舱到甲板,爬山虎一般包满了整艘船,小五子在棺材里都能闻到扑鼻的芬芳。花船上出来一个女人,后面站着如琴如诗两个丫头。那女人问,对面船上可是大漠仙人和蓬莱阁老两位前辈?声音有点耳熟,小五子把耳朵侧过去分辨,就快想起来的时候,那女人接着问:“老谷主要见怪呢,怎么二位路过南京,都不来百花谷喝茶赏花?”

那就对了,原来是上辈子的冤家苏子瑶。仙人作揖称谢,指着棺材说,他们要赶着送葬出殡,多有不便,还请谷主师妹不要见怪。

“我们老谷主可见怪了呢,”苏子瑶举袖遮嘴咯咯笑了起来,“送葬出殡,怕棺材里还是个活人吧?”

仙人说姑娘果然好眼力,里面的确是活人,我们打算到了墓地现杀现葬。阁老不耐烦了,句句围绕半死不活那小子,他在一旁耍那么多功夫,甲板都要站出坑了,花船那姑娘也不看他一眼。他抢话说,活人又如何,你们百花谷要是不满意,我现在就让这小子变死人。阁老说完就朝棺材劈过来,苏子瑶脸都吓白了,连喊三声且慢,质问他:“你可知道,这位公子是百花谷的什么人?”

这算威胁吧,阁老可不吃这套,随便他是谁,弄死了再说,他抬起手臂冲棺材中间往下劈。这时一个花篮从船上向阁老后背掷去,阁老回手挡了一掌,花篮被打落,击碎的花瓣飘得满天都是。阁老抱怨,小师妹原来你也在场,为何只让小孩子和我说话?一个老妇人伴随着花瓣轻飘飘地落在甲板上,她头顶着一尺多高的双凤翊龙冠,一身红罗袍拖在地上都看不到脚面。她上前两步,漫不经心一般站在棺材和阁老之间,笑盈盈地说:“三师兄,二十多年没见,你怎么还是这么大的脾气?”

那就是百花谷谷主了,小五子知道是来救他的,苏子瑶不是问棺材里面是百花谷什么人吗,少谷主啊。小五子使劲敲棺材盖,谷主说话时扫了一眼,知道棺材盖被钉
死了,她一掌拍在盖子上,小五子随着棺材往下一沉,六截筷子被震出来,谷主将棺材盖推开一半。天气晴朗,一道阳光照进棺材里,小五子手挡额头眯着眼睛,他还是看不清谷主,面前一片明晃晃的光。谷主右手抓住他肩膀,准备把他抱出来。这时仙人出手了,一掌拍向谷主,她松开小五子,腾出右手挡住这一掌。小五子又被摔回棺材里。

这次是真的晕了,再醒过来,他看到仙人和阁老在联手围攻谷主,三人出掌之快,掌力在船上形成一道密不透风的墙,弄得苏子瑶一直找不到空隙上船救人。那些从汉口来的船夫、厨子和掌舵的,早都吓傻了,半个多月棺材一直在甲板,里面躺着的居然是活人。他们都躲到船尾,讨论是现在跳江逃命,还是等等看,毕竟这些人师兄师妹的叫着,没准是所谓的切磋武艺,意思意思就收手了。

两位师哥也确实没使全力,仙人挥着手掌劝师妹先退回去,有什么事慢慢商量。谷主接他话说,把人先给我,一切都好商量。但她已然撑不住了,她知道退回花船,就别想再商量了。阁老打一会儿忽然惭愧了,说咱们两个大男人怎么一起打起师妹来了。


“不对不对,”他跟谷主说,“我刚才攻了你十八掌,现在还你三十六掌。”

他转过身跟仙人打起来了,一二三四五地数着,还见缝插针地说,有什么本事都使出来吧,趁小师妹在,看看到底是谁偷的。仙人一再骂他糊涂蛋,手上的功夫不得不加快。仙人快,阁老也快,一边打一边数。数到三十六,眼看仙人撑不住了,阁老一甩手不打了,退到旁边看热闹。

这时候刮风了,有雨点打下来。谷主渐渐势弱,身上已挨了两掌,脚下一滑,双手撑在甲板上才免于摔倒。她打算搏一下,大家五五开,回头看了一眼苏子瑶,一掌向下拍在甲板上。一声巨响,船头往水里倾斜,眼看着要沉船,谷主拍下第二掌,整艘船都散了架,一下子碎成上万条木板坠进江水里,棺材仿佛一艘孤船浮在风浪之中。苏子瑶盯着棺材,她知道谷主的意思,谷主拖住二老,她去棺材里救人。

船夫、厨子也都纷纷坠水,随便抓一个板子向岸边游去。谷主和仙人踩在一根木板两头伺机出招。大漠仙人嘛,钻沙子骑骆驼没问题,一碰水可就不行了,他摇摇晃晃也只是不掉到水里,哪里还顾得上还手。

阁老不高兴了,拽起帆布铺在水上指责师妹,你这
就不对了,大家打打玩玩,把船击沉了做什么?他长期住海岛,面朝着大海看蓬莱幻境,水性要好得多,脚点一下木板,都能在水面连迈三大步。他踩着帆布,把师妹也拉上来练练,难不成你的长江比我的黄海还要凶?打两下就知道师妹不行,他要慢点打,收点力,难得在江上打一会儿还挺过瘾的。

狂风推江水,江水推棺材,棺材摇摇晃晃向东漂流,一个水浪打过来,整个棺材翻到长江里,棺材口朝水面扎下去,散开的棺材盖从水里拔了出来。苏子瑶好容易追上了棺材,抓着棺材边却无处使力。她用背顶着棺材,憋一口气,大叫一声将棺材正过来。她看看身下的江水,老天爷保佑他还在,她上下牙打战,撑直双臂,从水里跃到棺材上面看着里面。

苏子瑶哑着嗓子喊谷主,听声音那边天都要塌下来了。谷主朝那边望过去,掌势渐收。阁老点点头同意罢战,拽起她踩着木板向那边迈过去。一里多水路他走起来比船还要快,最后一大步他带她到水面的棺材板上,看苏子瑶倚在棺材里掉眼泪。

“他不会水,”苏子瑶哭着说,“一点都不会。”

谷主点点头,咬着牙床看四周的江水,也不知少谷主沉在江底的哪一处。阁老也有点不好意思了,搓着双手说:“算了,就当我杀的,有什么仇,有什么气,找我撒好了。”

“我再问你一遍,你可知道他是百花谷的什么人!”苏子瑶冲他吼,“他是少谷主,是我们谷主的亲孙子!”
\newline

{\centering\subsection{6}}

日落时分,秦淮河上一户渔家收网归船,小姑娘指着水面问她父亲,河上为什么有一个棺材。她父亲低头不语,拉网上船,刮风下雨一整天,网里面没什么鱼,今天就算白干。小姑娘又问了一次,怎么棺材会跑到河里面?她父亲抬头看一眼,说可能是喇嘛在水葬吧。

他也不知道,秦淮河上从来没见过,就是以前听说书的讲过,有的喇嘛不埋不烧,要是死在内地,就把尸体放在大街上,等秃鹫老鹰吃掉,可是南京城里哪有秃鹫老鹰,连老虎豹子都没见过一只,扔一个月都放臭了也没人管,后来朝廷禁止天葬,他们就改水葬,死人放在席子上,顺水一推送进江河湖海。不过用棺材的真是没听说过,他起身望了一会儿,河上的棺材漂漂荡荡,好像从很远的地方来呢。

小姑娘说要去看喇嘛,一个猛子扎到水里,再出头时已是二米开外。他从渔网捡些小虾小蟹扔进铁锅,盛些河水把火点着。水烧开女儿回来了,爬上船说棺材是空的,里面什么都没有。他扭头看过去,说不是啊,棺材里有个脑袋露出来呢。小姑娘睁大眼睛,真的哎,刚才怎么没有呢,有头发的喇嘛,还在动呢。

他醒来的时候先吐一大口水,平躺的身子有一半浸在水里,有个女孩在上面问有人吗,是不是被吃啦?四周一片漆黑,他想起来这是金库的暗格,棺材落水时他躲到里面去的,当时水浪太大了,水从半拳大的通气孔涌进来,他一只手顶住那个孔,另一只手脱掉衣服把孔塞住。随着棺材在江中的几个翻滚,他人在暗格中彻底晕掉了。睁眼时就是这个女孩在问,是不是被吃啦?

他不敢出声,屏住呼吸,一直等到那女孩离开,一大口水吐在胸前。他要确定安全,一点声音都没有,他打开隔板,上到棺材里。看天色还没有大黑,他扶着棺材边坐起来,水面的宽度似乎不是长江,棺材顺势而流,早就从长江下游的右岸拐到了秦淮河。两岸都不着边际,这么漂着也不知道何时才能靠岸,刚才那小姑娘已经又回到渔家,看样子不是仇家,也许可以跟他们借身干衣服,讨口饭吃。

还是稳妥为上,小五子没人管,想杀昆仑公子的人可多呢,那些武侠话本,说的不都是背着深仇大恨临水而渔吗?他坐下去,靠在棺材一头,就这么漂着总能到岸边。命是保住了,他要想想以后怎么活,赶快跑吧,离江湖远远的,可江湖不是一个地名,到处都是江湖。那就一直往北跑,田独就没什么江湖,要是那也不安生就再往北,总有没江湖的地方。他忽然明白钱老板是自己人,应该是钱老板把他带到田独的,他不让他离开,处处管着他,就是怕他遇见仇家。这么说自己太傻了,天天多大委屈似的瞪着钱老板,此生若是有机会再见他,真该磕两个头,跟他说声对不住。

他坐不住,肚子饿得咕咕叫,有小鱼从棺材旁边游过,他弯腰去捞,差点掉到河里去。月上梢头,秦淮河反倒热闹起来,河面上停了十几艘花船,都是假花挂在船舱上,跟百花谷的比可差远了。两岸人头攒动,男人喊着价看谁能上看中的花船。

小五子明白了,这是选花魁,古韵凌波十里欢,风摇画舫雨含烟,夜游惊艳思八艳,情洒秦淮不夜天。还有那句更有名的商女不知亡国恨,隔江犹唱后庭花,他每次听到都是一脸坏笑。花魁还没选出来,小五子先中了
新郎官,棺材不受控制,朝着船头撞了过去。

还好身边有金条,不然看老鸨破口大骂的架势,要将棺材拆了再把他扔回到河里去。他手举一根金条,老鸨赶紧笑脸相迎,冲上说别叫价了,喊来喊去都是银子,人家这位官人可是带金条来的,棺材棺材,升官发财。两个龟奴把他抬上船,给他烧了热水泡澡,换了身新袍。小五子还是无法说话,冲他们比画要吃饭。龟奴说句得嘞,把他抬到花房闺床上,把桌子搬到床前,三趟两趟就摆满了一桌子佳肴,看菜品都舒服得想呻吟。小五子这辈子也没吃过什么好东西,在田独天天都是猪肉炖粉条,到冬天大雪封山,粉条供应不上了,就猪肉炖大棒骨;等进了丐帮更完蛋,基本上跟狗抢吃的,好容易下次馆子,还得把菜捣烂了再吃,说丐帮弟子不能忘本;跟仙人阁老更是吃糠咽菜,馒头蘸馒头渣,还广东的大厨,闻闻盘子上还有味儿。

一个姑娘进来给他斟酒,坐到他对面抱起琵琶,问小五子想听点什么。小五子想问后庭花有吗,苦于开不了口,财主都当得不尽兴。

边听边吃,肚子快吃爆炸了,饭菜还剩一大半,两壶酒下去,小五子觉得那姑娘越看越好看。他有点晕,后仰躺在闺床上,看着影影绰绰的烛火,姑娘过来把青纱帐放下来。他犹豫就在这儿过夜吧,苦了那么久,难得对自己好一点。也就一念之间,他撑起来再喝一壶酒,他知道不可以。现在已经够乱了,昆仑山庄四个姑娘拿命来救他,他知道最终只能选一个,肯定会伤三个女孩的心,但今晚在这儿过一夜,他小五子就真的不是人了。他用手绢抹抹嘴,坐起来示意她别来服侍我,接着把手绢展开写了一个“岸”字。他要下船回去,老鸨听说之后进来劝他,怎么好现在就走,要是不满意我再叫几个姑娘陪你。小五子摇头,他发现当哑巴也挺好,省了不少口舌之争。

来时俩龟奴伺候,回时就一个龟奴划小船。老鸨她们真可以,划出去没几米,就听见她宣布继续选花魁。岸上的男人又活跃起来,小五子示意龟奴远点走,找没人的地方上岸。下船时他有些不舍地看看船上的花火,他冲龟奴挥手,转身进树林,走出两步脚下拌蒜,摔倒在草地上。他爬起来,扶着树干,两腿颤颤巍巍,好半天才迈出一步。棺材里躺了一个月,他全身都要萎缩了。但总会好的,只要没死,一切都会好的。

\newpage