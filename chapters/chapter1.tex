\section{壹}

{\centering\subsection{1}}

大火那夜是八月十五,整个宫中都乱成一团,宫女
太监们三五成群地四处乱窜,嘴里还喊着有刺客,快去
救驾。可是谁也没撞见刺客,也没见着皇上太子。宫里
说了算的依次往下排,小顺子拉着侍卫队长一直等到五
公主回来,才算是找着了个主事的。

五公主那天本来要去安顿进京朝拜的外省大臣,车
行还不到一里,有人看到皇宫上空起着火光。她让车夫
赶紧调转回宫,刚一进宫门已经有几十名宫女太监跪地
请命。小顺子和侍卫队长禀报,皇上不在寝宫,三宫六
院都找遍了,太子也不见踪影,皇上就这一个皇子,没出
嫁的公主唯有她排行最长,请五公主快快给大家下令。
五公主让所有人起身让开,走进内门她才发现,原来火
势那么大,从后花园丁香丛,沿着涌石路的苍松翠柳,一
路烧到了池塘边。她看得双眼发干,问刚才是谁服侍父
皇。侍卫队长递过来一个名单,今晚轮值的太监宫女,
一夜之间全没了踪影。

“是谁放的火,”五公主问,“到底有没有刺客?”

侍卫队长不说话,他也说不上来,火势映得漫天血
红,这么大的火,肯定有刺客,不是意外碰倒两盏油灯就
能点着的。

“宫里还有什么人?”这次她问所有人,“太监总管常
公公,人在哪里?”

没人知道,又不敢吭声,多说一句话没准都要掉脑
袋。五公主倒抽一口气,好半天不知道该下什么命令。
远处一株快被烧枯的古柏轰然倒下,火花掉到池塘里嗞
嗞作响。一阵晚风将烟雾吹过来熏得她眼泪直流。她
用食指关节抹抹眼睛,清了清嗓子,尽量让自己的声音
保持着镇定和最后一丝威严。她告诉这些人把水桶放
下,不要管火,都去找皇上找太子。

“要是父皇出了什么事,”她停了几秒钟,巡视着每
一个人,牢牢记住他们的脸,“谁也别活过今晚。”
\newline

{\centering\subsection{2}}

满地的尸体让他满脑子都是空的,旁边的苏妃讲了
什么他都没听进去。地上也不全是死人,有两个人还活
着,夹杂在一堆尸体之间。有一个是年轻人,一身素衣,
躺在地上昏迷不醒,另一个年纪大些,身着龙袍,虽然脑
后淌了一地的血,但似乎并没有死,胸前时起时伏,显然
他和苏妃还不打算杀掉他们俩。

皇上是不在寝宫,这里是尚书房,不是没有人发现
过他们,只是先前进来的人都死了。刚才找进来的是两
个小宫女,被五公主的一番话吓坏了,她们相互打着气,
要活过今晚,两个小姑娘逆着火势结伴而行,一路哭着
摸到尚书房,看到里面影影绰绰站着两个人,先隔窗喊
话,问皇上是不是在里面,太子是不是在里面。苏妃说
在的,都在的,有什么话快进来说。两个小宫女一阵窃
喜,一前一后地小跑进院,刚一推进门,大点的那个喉口
一凉,就被苏妃用匕首割了喉。小点的那个也没躲过,
转身要逃,被苏妃匕首一挥,划开她脖颈侧部的动脉,踉
跄着几步死在院子里。

又是两条人命,她们太小了,小的十四岁,大的也不
过十六岁。对着这些尸体他有点恍神。苏妃从腰间拽
出一条手帕,擦掉刀刃上的血,将手帕扔到那个昏迷年
轻人胸前,低声细语地说:“常公公,把他带走吧。”

常公公还在发愣,一时都缓不过来,本来就是老太
监,声音一发颤,显得更尖了。他问送到哪里,你们让我
把他带哪去。苏妃没理他,踩着带跟的弓鞋向门口走
去,出门之前将擦好的匕首抛还给常公公,对他笑了笑:
“百花谷。”

苏妃走后只剩他一个人了,他长呼一口气,不小心
把眼泪带了出来。也没时间悲伤,外面呼天抢地地喊着
救驾救火,再撞进几个人,也是白讨几条人命。他弯腰
将地上的年轻人扛在肩上,踏着尸体,大步出了门。人
们东奔西跑,没人注意到他,他也不知道跟谁讲,嘴里念
叨个不停,反复说着“对不起”。

房间里还有一个偷听者,一直闷在东南角的书箱
里。等了好一阵儿,确定没有人,箱子的门从上面打开,
一个老者探头探脑地从箱子里爬出来。看到这一地的
尸体有点眩晕,他先吹灭油灯,双腿发软地去抱住皇帝,
摸了下脉搏确定他没有死,手掌按住后脑的伤口。四处
张望没找到一块干净的白布,慌乱中从怀里掏出一张羊
皮,将嘉和皇帝的头包扎起来。

这些常公公不知道,悔恨不已却还要躲避凶险。逃
出去的路上被一个太监认出来。太监追问他跑哪去了,
五公主一直在找你。常公公敷衍几句就向前大步走,可
惜这太监没眼色,跟在屁股后面问,背着的是谁。没办
法了,再多条人命吧。他回身捅了他一刀,太监的生命
凝结在惊恐不解的表情上。需要偷天换日,活要见人,
死要见户。常公公把自己太监总管的衣服脱下来,换到
他身上,再用那把匕首将他的脸划花,想想又不放心,提
着他的头发甩火焰上方烧了十几秒,用力一抛,尸体穿
过大火,落到了池塘里。

人可以见一个杀一个,但总要逃出这皇宫,即使是
最小的侧门,也要有四个侍卫。西侧偏门的一个侍卫是
老熟人,姓张,四十有余,大老远就看见常公公背着人往
这边来。常公公知道对方有所察觉,又是四个人协防,
没那么好下手。他试着攀谈几句,投其所好,说这是三
王爷要的人,出去领赏一起喝酒吃肉。张侍卫端着不说
话,忽然发力将另外三个侍卫给杀了,随后一脸的嬉笑,
对他作揖说:“常公公眼力果然厉害,隐藏这么多年,还
能知道我是三王爷的眼线。”

张侍卫将门推出一道缝,刚一出宫感觉天光暗了下
了,走了半里地常公公才想明白,里面的是火光,围墙把
大火挡在了皇宫里。张侍卫带他直奔王爷府,其间还老
想验验货,看看他背上的是不是三王爷要的人。常公公
不想给,岔开话题,问他从哪年开始成为三王爷的人。

“眼线又不是我一个,"张侍卫说,“宫里一半都是他
的人。”

常公公点点头,倒吸一口凉气,当年要不是他的建
议,将太子召回,可能嘉和皇帝早就遇害了。张侍卫问
他皇上怎么样,是死是活。常公公没说话,心里想着怎
么才能解决这个张侍卫。主要是肩上扛一个,第一刀捅
不准,就是一死两命。远处传来马蹄声,轰隆隆地像整
个军队朝这边碾过。张侍卫说三王爷来救驾了。说完
还生怕常公公没听懂,一脸猥琐地在那笑。他拉常公公
站到路中央,干脆就在这儿拦住三王爷的座驾,直接交
人换银票。

踏步声愈来愈近,感觉几千人在行进,后来连地面
都在震了。再不动手就来不及了,他说你验验人吧,到
时别说我常公公误你。说完他把人放地上,故意让脸朝
下,等张侍卫翻过来。张侍卫弯腰抱起年轻人,刚看见
脸就感觉后脖颈发凉。常公公一刀从后脖扎进去,一直
穿过去,在喉咙口冒出一个刀尖。他把尸体踢进草丛,
在军队赶来之前抱着年轻人,闭眼一跳,一路滚到了半
山腰。

他把人抱得严实,倒弄得自已浑身都是伤,跌跌绊
绊到后半夜才找到一间破庙。没死就好,他将年轻人放
下来,端详他半分钟,痛哭起来。他脱下他衣服,双手抵
着他的后背,把最后一点气力传过去,为他续命。直到
自已浑身无力,昏倒在地上,才换来那个年轻人睁开
眼睛。

浑身不舒服,年轻人咳了几下,吐出一口浓血,然后
一脸疑惑地看着旁边的这个死太监。他觉得眼熟,但是
实在想不起来这个人是谁,是敌是友。他从他身上翻出
那把匕首,看着刀刃上的血迹琢磨,是坐在这里等这个
老太监醒过来,还是趁他睡着,现在就把他杀死在破
庙里。
\newline

{\centering\subsection{3}}

三王爷感觉自己一天都在赶路,他一大早就起了
床,虽没有早朝,却要把京城的几个大户全见一遍。中
秋佳节,身边的幕僚早就建议过他,皇上只有一个皇子,
真到驾鹤西去那一天,他作为皇弟,与太子的王储之争
还要指望这些大户人家的财力势力。请他们一起到王
爷府吃一顿是最省事的,但是这太惹人眼目。他得一家
一家走,一直忙到傍晚,还要去皇宫和文武百官中秋赏
月。月亮是扁是圆他根本不在乎,他只关心这些官员的
立场,打点一下自己的眼线,观察哪些人可以试着拉拢,
哪些是太子的死忠派,找个机会杀鸡儆猴。

回到王爷府已是入夜,双腿累得直打战。即使他这
般淫色之徒,这一天也是早早就上了床。他最近一直在
做梦,白天实现不了的事情,希望梦里可以黄袍加身。
三年前有过一次这样的梦,甜到笑醒,后来就一直没逮
着这美好画面。今晚的梦有那么点意思,黄陵祭祖,他
点好三炷香,死活没见着皇兄,回头一看身后百官对他
跪叩,难道他已登基?他正要低头看一眼自己是不是身
着黄袍,西北的六公子郑明宇在门口将他唤醒了。

换别人早杀头了,唯有少数几个幕僚有这种特权。
他深知少了六公子这样的左膀右臂,这辈子也就做做皇
帝梦罢了。隔着窗子他听明白了,宫里出事了,赶紧让
六公子进来,问他老东西死没死,有没有缺胳膊少腿。
六公子摇摇头,见三王爷有些失望,他补上一句,太子被
昆仑公子掠走了。听到这些三王爷来劲了,从床上蹦下
来问六公子怎么办。六公子建议他多带些人去宫里救
驾,待他将宫中占据,别说太子到时进不来,皇上的生死
也在他股掌之间。

这主意倒挺好,可是没有兵,皇上这几年每逢洪水
地震就跟他借兵,把他几十万军队裁得就剩百十名家
丁,杀个猪都得满院子追,还指望他们去宫里救驾?六
公子说他有人,这几年他在京城秘密养了三千兵马,以
备紧急状况,在河北定州还招了五万人的军队,即刻就
可以往京城出发。好像是天赐良机,三王爷赶快唤人更
衣,鞋子穿好过后他才反应过来,三千人占领宫中,明天
再有五万人将皇宫包围,登基指日可待,总有哪里不
对劲。

“那么,”袖子套一只他停下来,盯着六公子问,“这
五万三千人,是你的人,还是我的人?”
\newline

{\centering\subsection{4}}

天快亮时才知道太子被人掠走,先是有人在池塘找
到常公公的尸体,已经被烧得不成样子,五公主让人将
尸体放置到棺材里,而那些死了的宫女太监,一并堆到
御厨房后身的马厩旁。后来皇上终于找到了,在尚书房
里昏迷不醒,她问太医伤势如何,太医支支吾吾,说睡醒
就好了。只剩下太子,五公主命令下人一间间搜查,没
多久有人在皇宫后门发现一行血字,八个字从右至左
是:三年之内,归还太子。侍卫队长惊呼是昆仑公子的
手段。

“谁是昆仑公子?”五公主问。

死一般的沉寂,看表情好像那些习过武的侍卫,个
个都知道昆仑公子是什么来路,五公主追问一遍,到底
是什么人,哪门哪派,能把皇家侍卫队吓成这样?过了
好一阵侍卫队长说,昆仑公子没有门派。那你们怕什
么!五公主吼起来。没有人回答,弄得她也害怕了,颤
着声音问,为什么要三年,太子能不能活着回来?

“太子不会死,”侍卫队长说,“昆仑公子没杀过人。”

五公主松了一口气。但是侍卫队长没讲完,他说他
们不是怕死,是怕生不如死,所有见过昆仑公子的人,或被
挖双眼,或被断脚筋,总要留下点什么。五公主蒙在原地,
让人把血字擦掉,命令九门提督李准驸封锁京城所有大
门,挨家挨户地查也要把太子活着救回来。李准驸还未
领命,小顺子过来通报三王爷前来救驾。总算有个可以
倚仗的自家人。五公主要亲自迎接,她大步走向大门,侍
卫分列两侧,给她在中间让出一条路。距离宫门三十米
远,她做了个开门的手势。宫门在她面前缓缓开启,进来
的不是三王爷,十几个手持盾牌的冲锋兵在往里挤,后面
黑压压的全都是人,一个个将刀枪举在头顶。

“关门!”五公主在后面声嘶力竭地下命令。

最先冲进来的盾牌兵先后被刺死,剩下的几百名侍
卫全都顶着肩膀,拼着老命把宫门顶回去,将外面的三
千人马挡在了皇宫外。塞进门闩的一刹那,几乎所有的
侍卫都瘫坐在地上。五公主还不能倒,对着宫门喊话:
“父皇并无大碍,早已休息,请皇叔明日午后再来请安!”

话音未落她示意放箭。弓箭手呈三列登上城楼,向
人群里放乱箭。五公主看不到外面,她一直盯着城楼最
高处,询问侍卫队长今晚的哨兵是否都被买通。队长低
着头不说话。城楼上的负责人宣布叛军已散,是否开门
追击。五公主摇摇头,告诉侍卫队长今晚在城楼巡逻过
的人,全部斩首。

随后她又一次跟太医确认,我父皇果真睡醒即好?
太医强调他在宫里已二十多年,小到风寒,大至绝症,没
有一次误诊。五公主点点头,那就等父皇睡醒吧,这一
夜就让它过去吧,她对小顺子传令:“通知文武百官,圣
上偶染风寒,明日早朝取消!午后待命!”

她想了想,吩咐小顺子,过两个时辰派人将皇宫外
清理干净,所有人不得泄露三王爷叛乱的事,对方底细
不明,还不是硬碰硬的时候。

“昆仑公子那边,”她说,“查出这个人,就是把京城
插出几个洞,也要找到太子!”
\newline

{\centering\subsection{5}}

昆仑山庄在汴梁,那上面也有八个字:要务在身,择
日再聚!本来各门各派约好了在八月十五夜来绞杀昆
仑公子。那时江湖已不剩几个门派,强的不强,弱的怎
么说呢,反正比种地的农民强点。大家打打杀杀上千
年,留下来的都是苟活者以及苟活者的后代。人类的发
展就是负基因的扩散,那些最好的死士,最好的忠烈之
士,早早就将自己以及自己的基因,自绝于他们的时
代。樊於期自刎于荆轲面前,让他提着自己的头去见秦
王,一旁陪同的秦舞阳吓得两腿发软,早早就被赶了出
去。荆轲身子被剁成肉馅喂狗,脑袋挂在城楼上示众,
樊於期也是全家抄斩,两个人都没有留下后代,倒是见
势溜掉的秦舞阳,日后可以生上十个八个,将自己胆小
懦弱的基因子孙万代地传下去。适者生存,强者,都绝
种了。

三大帮派一直留存至今,少林、武当和丐帮,寺庙道
观一直承载着福利院孤儿院的功能,养活了一帮孤儿流
浪儿;至于丐帮,历朝历代都少不了要饭的,把他们整合到
一起,倒解决了朝廷的麻烦。大概化缘和乞讨是一回事,
找大户人家施舍点,千百年来也不需要与人相争,自然存
活至今。倒是其他的门派轮流坐庄,占个山头就是王,自
己买卖没做明白,还老惦记着对方盘口的那点生意。

过去是乱象,明争暗斗,昆仑公子出现之后,大家倒
是同仇敌忾地抱成一团。要说大恶倒也不算,就是自打
半年前,每个门派的老大都陆续收到一封请帖,邀请他
们八月十五夜来山庄赏月,赏一赏九宫图。江湖这种想
搏名的人多了,换个王公贵族,你就是拿个金帖来,也不
至于请得动各路名宿。只是昆仑公子送帖的方式有些
特别,在客栈在酒馆遇见一些侠游的弟子,打听你哪门
哪派,双手奉上请帖,客客气气地请你们帮主中秋赏月,
临了还提醒你别不当回事,怕你忘了,挖你一双眼睛或
是砍掉一条腿,加强你记忆。小门小派就不说了,三大
派都未能幸免,收到请帖的宗主看着自己缺胳膊少腿的
弟子,不为赏脸为复仇,总得来一趟吧。

都说昆仑公子不杀人,谣传到最后人也杀了不少。
黄山派迎客道人的师兄就是被昆仑公子所杀,只得由他
临时掌管黄山派,尸体也没有,死无对证。虎头教也有
类似的情况,不同的是教主被弄死,教主夫人上了位。

唯有丐帮算是例外,现任帮主不但没死,前任帮主
居然也还活着,只是他们都不在帮中,好像之前有个姓
向的老帮主,比乔帮主还早,辈分高,后来听说去练无为
神掌,传给了徒弟何振声,没几年徒弟也不干了,陪他师
父练功去,丐帮交于关、马二位长老打理。两人倒也不
敢争权,逼急了向老帮主出山每人五十大板。不过关长
老眼睛确实是最近才瞎的,被人下了毒,一日不如一日,
索性什么都看不见,马长老逢人就说,一定是昆仑公子
下的毒。对此关长老都是冷笑,冲马长老冷言冷语,我
是怎么瞎的,没人比你更清楚。

乔帮主跟马长老一样,好事坏事要捋清楚,有些账
要跟昆仑公子算,有些可能是借刀杀人。昆仑公子总是
要死的,但他绝不姑息欺师灭祖的败类跟着浑水摸鱼。

乔帮主掌管狮吼帮,负责在河上喊号子,押送船上
的货物,往来江河只要听他自称一声乔三,大家都会给
两分薄面。两个月前他也收到了请帖,昆仑公子托女儿
乔文君送递过来,当时惊出一身冷汗,所幸女儿毫发无
损,也许是江湖上唯一一个全身而退的。然而流言很快
也传开了,乔姑娘借宿昆仑山庄两天一夜,不但没要乔
姑娘点什么,怕是还送狮吼帮一个外孙。两天一夜到底
干了什么,这些跟女儿也不好问,他知道以她的刚烈性
子,要是真有这种事发生,怕是早就自刎雪耻了。

除了赏月,请帖还提到了九宫图。就一江湖上瞎传
的东西,说是集齐九张图,就能坐拥天下。活到五十岁
乔帮主也没见着一张,前两年听别人说,嘉和皇帝有一
张,给了太子,武林至尊沈老前辈有五张,圆寂之前把五
张图依次给了他的四个弟子,天、地、人、和,剩下一张陪
他下了葬,百花谷谷主有一张,似乎两朝宰相文培源府
上有一张,后来被抄了家,把房子翻漏了也没找到。

九宫图也就这二三十年的事,在往前数好像还有过
五行卦,十二生肖兽首,反正都是号令天下的宝贝。可
是哪有那么神,没听说哪个开国皇帝是靠攒宝贝登基
的。少林思考生死,丐帮思考饱暖,他狮吼帮乔三就思
考这些宝贝,想了二十年,终于想通了这一道理,这是朝
廷在订游戏规则,看你们武林太和平了,练了一身本事
又不打打杀杀,人丁日益兴旺,联合起来叛乱怎么办,订
个夺宝规则,让你们内部消化一下,要是哪天真是有人
踩着尸体集齐了宝物,朝廷就换个宝贝重新玩。

他是花了二十年想明白的,换道行浅点的,一听到
九宫图可就双眼放光了。能看出来,有那种决绝的帮主
就为这个来的,即使满门尽丧于此,能得着半张九宫图
做镇派之宝,也算是含笑九泉了。大家各揣心思,跑到
昆仑山庄却扑了个空。墙上留着八个字:要务在身,择
日再聚!再发请帖,是不是又要挖几十只眼睛,断几十
根脚筋啊。昆仑公子还够客气,人不在了,摆一桌子宴
席等着众宾客。看起来很美味,也没人敢以身试毒。都
是五湖四海奔这儿来的,好像就地解散也不对劲。乔帮
主提议,外面安营扎寨,到了天亮再说。迎客道人附和,
十五的月亮十六圆,没准明天就出现了。乔帮主白他一
眼,既然坐实了武林败类,怎么说话还这么没条理?

夏日傍晚蚊虫乱飞,周围人员喧嚣,乔帮主以为自
己睡不着,结果一睁眼已快天亮。乔姑娘不在身旁,找
了半天在屋后呕吐,也没吐出什么,就是胃里恶心得难
受。乔帮主皱了皱眉,还得上前将手绢递上去。他盯着
女儿,看她擦净嘴角的污秽,犹豫该不该问清楚,张嘴却
只问出:“你没有吃他们留下的东西吧?”

远处一匹快马朝这边行进,有人从京城带来消息。
少林方丈最早得知,召集各派集合商议。他说昆仑公子
果然是要务在身,今晚他潜入皇宫,行刺皇帝未遂,将太
子劫走了。这是大忌,武林朝廷,自古井水不犯河水,昆
仑作为武林人士,他的所作所为必定让朝廷与武林犯
难,自此江湖的日子不会好过。

用不着再动员,从此摒除分歧,暂停各帮各派的事
务,合力追杀昆仑公子献于朝廷,是武林唯一的活路。
大家挥拳赞同,乔帮主全没听进去\footnote{原文为“听见去”,可能是错别字},他还在想着,要是他
乔家真走到最坏的那一步,他该怎么办。这时乔姑娘从
屋后回来了,她拉着父亲的衣袖,有个请求要他应允。
她说若是有一天真抓到昆仑公子,求爹一定留他一条活
命。乔帮主瞪大眼睛看着女儿,舌下生津,止不住咽唾
沫。远处东方既白,真是的,刚出来的太阳,还没有昨晚
的月亮圆。
\newline

{\centering\subsection{6}}

醒来时还在庙里,只是被绑在柱子上。不远处的年
轻人早就醒来,光着膀子在门口烤着匕首。常公公扭头
看一眼,双手正是被年轻人的上衣绑在一起。他那么认
真,也不知道他要干嘛\footnote{原文为“干吗”},匕首已经够锋利了,他还把匕首
捆在扫帚棍在火焰上来回晃动。刀刃都被烧红的时候,
他举着扫帚棍走进来,见到常公公醒来也不惊讶,吹着
刀尖上的火花,好像那把匕首是他刚打出来的,一边欣
赏一边说:“我还以为你死了呢,一会儿我问你是什么
人,但你先别告诉我,等我把这一套刑玩够了你再讲,不
然就算你说了,我要是不尽兴,一样杀了你。”

他说完还是不放心,过去拍着常公公的胸口,去点
他哑穴。食指中指戳了十几下,常公公吸口气劝他别胡
乱点了,他不说就是。年轻人没面子,自言自语说记得
哑穴就在这一带。他弯腰把常公公的袜子脱下来,团巴
团巴塞进他嘴里。之后充满仪式感地举起烧红的匕首,
去烫常公公的脚心,用刀尖从他每一个脚趾缝穿一遍。

常公公不说话,可惜也不叫,耷拉着脑袋一脑门的
汗。他担心常公公死了,动刑固然好玩,他还是好奇月圆
之夜,怎么会和这么个老太监待在破庙里。他把袜子从
常公公嘴里掏出来,还颇有耐心替他穿回到烫伤的脚上,
用劝解的语气讲: “说吧,我也累了,你到底是什么人?”

常公公吐出一口气,看着他的后方,说出了第一句
话:“小心身后!”

有六个人站在庙门口,其中一位公子引弓射箭。年
轻人侧身一躲,箭冲常公公喉咙而来,常公公被缚住,左
右无法闪躲,只好低头用牙咬住箭头,过了好一阵将箭
和震碎的半颗门牙吐出来。

年轻人来气了,六个人又如何。他跳起来喊道,要
打出去打,别伤到我的人!说着他跑出庙,直往草垛后
面跑。五个年长些的追了出去,剩下的一个是西北六公
子,盯了常公公好半天道:“我就说,常公公这么大的本
事,怎么会烧死在宫里?”

常公公看着他也出了庙,自己被绑在柱子上叹息。
听起来外面还没打,也不知道年轻人是叫嚣还是求饶。
他喊着既然你们是六兄弟,那合葬在这庙里,谁是老大,
谁是老二,你们做个决定,我是从大到小地杀,还是从小
往大来,一个个杀?有两个不忿的,大吼两声就朝他劈
过来。常公公听出来年轻人在草垛上蹿下跳,一时砍不
到他。后来听剑法,应该是六个人将他包围在草垛上。

那就一起死!一个抵六个,值了!一阵草垛燃着的
声音,火光映得庙里都发红。后来的剑法他听不真切,
直到年轻人再次叫嚣,杀了你五个哥哥,就留你一条性
命传我威名.日后若是再让我撞见,非把你先剐再杀!

常公公一头雾水,以一敌六,不知道他是怎么做到
的。没两分钟年轻人进来了,看到柱子上的常公公,三
步并作两步跑过来给他松绑,嘴里问个不停:“常公公,
是谁害的你?”

绳结打开常公公坐在地上,脱下袜子给他看脚底的
伤口,指了指年轻人手上的匕首,低声说是你弄的,继而
仰天大哭地说:“孩子,你这是中了断魂掌啊!”

年轻人看了看手上的匕首,环顾一圈这破庙,时不
时地摇着头,他终于害怕了,几乎哭出来的声音问:“我
还有多久?”
\newline

{\centering\subsection{7}}

还有多久才能醒来,一个上午五公主问了三次太
医,你说醒来就好了,但是什么时候醒来?太医支支吾
吾,坚持说皇上的确醒过来就好了,可是何时醒来无法
预估,可能三年,可能三十年,也可能是明天。五公主打
断他:“那现在算是活着还是死了?”

“活着,肯定是活着。”

“所以,不可以立新皇!”

可能是比昨天更槽糕的一天,没有一个能让她\footnote{原文为“能她让”}透口
气的消息。九门提督李准驸带着两万名禁卫军,一夜之
间查了京城九十万户人家,还是没有太子的下落。涿州
知府在上午快报,从定州过来五万精兵,正经过涿州准
备进京。五公主下令全力阻截,并要求固安、涞水派兵
增援。直至午后前方也没有快报,小顺子向她通报,三
王爷和文武百官在宫前想向皇上请安。五公主让他传
话,皇上身体欠佳,请众官稍候。她还在等,等前方战事
的结果,等父皇会突然醒来。一直到黄昏时刻,小顺子
劝她不要等了,京城已经传遍了,都说皇上驾崩,太子被
刺,那些大臣都以为三王爷要登基,已经开始摇摆了。
五公主看着沙漏,将桌上的发簪扎到发髻里,一字一句
地说:“宣百官上朝!”

龙椅上是空的,五公主坐在一侧的偏椅看文官争
执。她在观察哪些是三王爷的人,哪些又是她未来可以
托付的大臣。两派的观点很明确:一方说皇帝还活着,
现在就是太子回来也不得继位;而另一方表示,倘若皇
帝永远不醒,又不算驾崩,活个十年二十年,天下岂不是
大乱?五公主和三王爷却不表态,两个人偶尔还面带微
笑地对视几眼。有人建议暂时由五公主代理朝政,直到
皇上醒来或太子归来;另一些就嘲笑,既然公主都能当
皇帝,那为什么单是五公主,六公主七公主八公主,皇上
二十七个公主,每个公主当一年皇帝好了。

双方僵持不下,轮到三王爷说话了,一反常态,他赞
成五公主代理朝政,如果有利于黎民百姓,他可以放弃
这个皇位。“可是,你不是吕后、武则天,你日后的孩子还
不知道姓什么,你仅仅是公主,你生不出皇子。”

那些最拥护五公主的大臣,此时也没了声音。无论
如何都不能让,倘若让他登基,不出一个月,父皇一定会
不疾而终,三王爷一定会举全国之力绞杀太子。小顺子
送来前方快报,扫过一眼她长吐一口气,开始说话了:
“三皇叔,前方刚刚剿灭五万来自定州的叛军,你可知道
这是谁的部下?”

三王爷皱眉凝思,差不多用了十几秒才确定,五公主
并没有诈他,前方已经全军覆没。梳理一番他表示自己
久疏朝政,无法推定何人有弑君之心。五公主追问,若是
朝中有人和昆仑反贼里应外合,推翻父皇,是否该斩立
决。三王爷连连点头,直言叛贼为何人,请五公主明示。

五公主伸出右手食指,指着每一位大臣的脑袋,在
三王爷头上停留片刻后,指到他身旁刚刚最张狂的两位
大臣,宣布过他们的罪行后,唤侍卫进来,斩首示众。
\newline

{\centering\subsection{8}}

年轻人在吃面,不知道为什么,这家面馆的面条这
么长,嘴角的一根面嘬不到头。常公公在桌对面跟他讲
述,江湖上只有两人会断魂掌,一个是沈老前辈,另一个
是沈老前辈的弟子南海真人,南海真人已经六十多了,
沈老前辈应该百岁了,真人那时掌力火候未到,中掌者
只是片段性失忆,要么一掌打死了,要么伤好之后恢复
了记忆,总之远远达不到他师父的功力,听说他后来跑
回到南海修炼,算起来也该练成了。

“练成什么样?”年轻人问。

“十二个时辰,中掌之后你只有十二个时辰料理后
事,这期间你的记忆还是时有时无,当时辰一到,你将彻
底断点,过去的一切你一无所知。谁对你来说都是陌生
人,你可能会被仇人利用,可能杀你的爱人,你失去的不
只是记忆,你所有的感情都没了。”

年轻人停住筷子,看着面汤,低声说:“你不要走,你
要告诉我,哪个是我该爱的,哪个我该杀的。”

“没有用,两个时辰前,”常公公摇着头“你差点把
我给杀了。”

年轻人不说话了,面汤摇摇晃晃,隐约能看见自己
的脸。常公公接着告诉他,失忆跟死了一样,就像投胎,
谁能记得你上辈子怎么过的,爱过谁,恨过谁,但万幸你
还活着,万幸你还在我身边,我不会让你活得那么羞耻,
莫名其妙地给哪个仇人当家丁走狗。

年轻人把碗朝前一推,面汤从碗边溢出来。“你谁
啊?”他环视一圈面馆,敲着桌子说,“那么多空桌,你跟
我挤一桌?”他端着面站起来,坐到旁边桌前,“没钱吃面
你说话,跟谁套近乎呢?”
\newline

下午他又回来了,坐在客栈窗外发呆,桌上摊着纸
笔,想到什么写什么,有些想不起来的,就使劲抓头发,
弄得常公公都一阵阵心疼。后来他说你别写了,写了也
没用,三更一到,我就把那些信烧掉,以后我想让你怎么
活,你就怎么活。

他不接话,害怕他说的是真的,把已经写好的信放
进衣服的最深处。常公公笑了,问他藏得住吗,烧掉了
你什么都没有。他站起来,抄起匕首将常公公抵在墙
角,威胁现在就可以杀了他。

“你杀了我吧,赌一赌你再睁开眼能碰见谁。”

刀尖都已经划到喉咙,他将匕首甩了出去,浑身发
抖地大吼两声,将客房每一个物件都砍碎,怀揣着自己
的信,下了楼。
\newline

他要刻很多字,他要刻瑶,他要刻百花,他要刻五,想
了想他又刻上断魂,他得知道自己是怎么失忆的。匕首
划在手臂上,每一刀下去,都涌出血滴连起来的笔画,夕阳
从树林折射过来,映得血滴晶莹剔透。最疼的时候他揪
一把山坡上的草攥在手心里,牙齿咬得咯咯响。常公公
坐在草坪上整理包裹,哪些带走,哪些扔掉不用了。太监
总管的衣服是不能再穿了,不过料子真好,他在想改成什
么合适。再刻就要死人了,他放下匕首,吹干上面的血,冲
着常公公较劲:“到时候你得砍我胳膊了吧?”

常公公抬头看看,小臂上的血糊成一团,什么字都
看不出来,他不去理会,继续思考那个绸缎料子都能干
点什么。

“你打算让我下半辈子怎么活?”年轻人问,“带我去
哪?”

“不知道,我还没想好。”他放弃了,把衣服塞进包
裹里,先带着再说,“上面要我带你去百花谷,重新塑
造你。”

“塑造成什么?”

常公公看着他,一时间觉得他的脸还挺柔和的,说:
“杀人机器。”

“我很残忍吗?”年轻人仔细回想,过去仿佛拆解成
小碎块,三五成群地从他的记忆里离家出走,“没想好是
什么意思?你还想怎么塑造我?”

“杀猪,找个偏远点的地方,你在肉铺当一辈子伙
计,杀一辈子猪,把你那火暴脾气发泄在猪上,别再踏进
武林一步了。”

太阳就要掉进山沟里,他眯着眼睛对视着阳光,后
来跟想明白了似的,把上衣穿上,盖住小臂的血字,站起
来拍拍屁股说:“带我去百花谷吧。”
\newline

夜里下雨了,他忽然惊醒出了客栈,常公公还在睡
觉,他去偷马。马都已经牵出来了,他又忘记了要干嘛\footnote{原文“干吗”}
去,自己在雨中站了几分钟,将马牵回到马厩。回到客
栈他把常公公摇醒,用哭腔哀求道:“我什么都不求你
了,你就答应我一件事,带我去见她,让我告诉她这一
切,别让她在那儿一直等着我。我要是真失忆发狂了,
你就把我随便扔个地方,不用管我。但是我得让她知
道,我死了,这辈子结束了。我不能让她一直在那儿等
着我。”

常公公还没完全醒,不紧不慢地把蜡烛点上,问道:
“要我带你去见谁?”

他盯着蜡烛,眼神茫然,抓着常公公的肩膀哀求道:“去
见谁?你告诉我,你肯定知道,我应该去见谁,她是谁?”
\newline

{\centering\subsection{9}}

她见不到谷主,跪在帷帐前听谷主训话。谷主年纪
大了,说话一多气力就上不来,通常两句话之间总是要
停顿许久。她说常公公不可能叛变,百花谷谁叛变也轮
不到常公公,之后停了一会儿又说,本来我是要你带少
谷主回来的,为什么要转交给常公公?说完她又不说
了,冲帷帐外挥了挥手,让她去找回来。外面的女孩没
听清,她清清嗓子又讲了一遍,找回来,我命你走遍天涯
海角,也要把少谷主找回来。

她听明白了,起身后退。这时谷主叫住她:“苏子
瑶,他是不是你相公?”

“是。”

“那就对了,你去把你相公找回来。”
\newline

方丈他们到了,正在外面候着。五公主说从御厨拿
出最好的点心招待他们,她想再跟父皇说会儿话。这是
九门提督李准驸的主意,他说昆仑公子来无影去无踪,
长什么样都画不清楚,就是派八十万大军也不一定能找
到这个人,解铃还需系铃人,何不请些武林前辈组一个
同盟来找人。名字他都想好了,救太子,杀昆仑,同盟就
叫寻龙屠狼。听起来不错,不过也就是形式感十足,反
正也没别的办法,五公主看着李准驸苦笑: “你倒挺会起
名字,李准驸,你名字是你自己起的吗?”

此时她和嘉和皇帝共处一室,她也不知道该说什
么,父皇二十八个孩子,还要操心朝政,也不剩什么时间
给她了。她感觉过去二十年,也没有这几天跟她父皇相
处的时间多。她让人把父皇抬起来,她来喂吃的。全都
是流食,米汤和菜汁搅在一起。她舀出一勺,吹一吹,张
开父皇的嘴轻些灌进去,然后将他的头向后仰,食指中
指捋顺他的喉咙。她觉得他能听到,句句进心里,只是
懒得醒来而已。她说父皇你放心,太子一定给你找到,
万一太子有什么不幸,也不会交给三皇叔,我可以一直
等,等到三皇叔也百年,把皇位传给他儿子,天下还是我
们刘家的。

菜汁从嘴角溢出来,一路滴过下巴,她把父皇放平,
接过毛巾给父皇擦于净。头上的绷带也该换新的了,她
让小顺子去唤太医,自己先把他额头残留的血迹清洗
掉。女儿跟你保证,皇位的事情我若没做好,我永不嫁
人。水抹在绷带上渗不进去,她食指拇指捻了捻,将绷
带一层层打开,里面藏着一张羊皮。

“这是谁干的?"她转身问太监。除了太医,别人哪
敢碰。小顺子跑着回来,说太医不见了。五公主展开羊
皮,对着阳光仔细端详。九宫图的事情她也听说过,一
会儿可以去问问方丈,倘若真是一件趋之若鹜的宝贝,
那就让他们去找太子,抓昆仑,怎么说来着,寻龙屠狼,
有谁做到了,这张羊皮赐给谁。
\newline

文思清抱着骨灰盒,头上插着稻草站在集市门口。
集市里人来人往,天上飞的,地上走的,什么都有得卖,
头上有稻草这意味着文思清也是可以卖的,头上稻草的
数目是文思清的价钱,然而没人看这个,每个人都只报
自己想出的价钱。一个窑婆子出价三两银子,文思清身
后的女人笑着直摇头,点着文思清的头顶说,这是两朝
宰相文培源的女儿,三两银子是开玩笑吧。窑婆子忙说
不少啦,上个月买一公主才花二两半。

“这真是文宰相的千金,文家男的抄斩,女的卖奴,
他就这一个女儿。”

窑婆子懒得戳穿她,伸出左手说一口价五两。女人
在犹豫,按理说这身份应该换黄金才对,可是行情就这
样,再拖两个月过了十八,就更出不了手了。文思清转
身跪地求她:“把我留下来吧,我什么都能干,别把我卖
给窑子了。”

“你会干什么啊?你是宰相的女儿啊,我们全家伺
候你还差不多!”

女人冲窑婆子点头,示意出钱交人。一位摇着折扇
的白衣公子叫停了这一笔生意,强人所难卖到窑子里,
可能这位姑娘的一辈子就被你们两个毁了,他出五十两
银子替女孩赎身。窑婆子气得直跺脚,那女人当然乐得
这笔大买卖,文思清看着这位白衣公子,感觉自己终于
熬到头,春天就要来了。

春天也不总是好天气,白衣公子一直想弄清楚一件
事。难得的晴天他把文思清带到花园里,鸟儿成对,蝴
蝶成双,他问文思清是不是真是宰相的千金。文思清点
点头,多谢公子的救命之恩。

“那你以后不要再做那些脏活累活了,那些不是你
该做的,”他拨开她刘海,看到她额头上的犯字,望着她
的眼睛说,“你以后让我一个人舒服就好了。”

说完他扑到她身上,撕开她前襟,嘴里还念念有词,
说宰相的千金被我收了。上衣被撕烂,待要拽她裤子的
时候,文思清从骨灰盒的侧壁抽出一把匕首,抵住自己
的咽喉,警告他不要过来,五十两银子我保证双倍还你,
要是不想这五十两银子打水漂的话,永远不要靠近我。

窑婆子双手叉腰,气鼓鼓地看芙蓉月弹琵琶唱小
曲。这几个月她都在懊悔,五十两又怎么了,管她真千
金还是假千金,倘若把她弄到翡翠楼,肯定比芙蓉月还
要赚钱。曲子快结束的时候,她整理下发髻,上台讲每
天都要说一遍的话,什么春宵一刻值千金,下面哪位想
和芙蓉月共度良宵的,价高者得。

几个纨绔子弟相互抬价.一度到了二十两。奇怪的
是有个长者坐在那儿不说话.一直盯着芙蓉月看。总会
出手的,窑婆子想,一大把年纪了,留着钱还有什么用。
果然价钱还没敲定,长者就上了台把一袋银子扔给窑
婆子,走向芙蓉月。她打开数了数,冲下面的人喊:“三
百两!”

长者没回头,还在看着芙蓉月,口中吐出两个字:
“赎身。”

又是赎身!这次她可不会贪小便宜吃大亏了。窑
婆子掂掂银子,说这可是我们头牌,不赎身,你就是把翡
翠楼买下来,我也要带着芙蓉月走。

长者没理会,路过古筝,指甲在琴弦上滑过一遍,依
次发出由低到高的声音,琴弦随即绷断,大概又过了两
秒,整架古筝断成两半掉在地上。他对芙蓉月说,“跟师
父回去吧。”

芙蓉月含着泪摇头。

“江湖出事了,师父需要你帮忙。”
\newline

京戏很好看,生旦净末丑在台上轮番登场,台下狮
吼帮的弟子时不时地站起来喝彩。可是乔帮主心思不
在这儿,他老是不自觉地瞥一眼乔文君的小腹。其实也
没有,离显形还早着呢,可他就觉得这孩子随时都可能
自己蹦出来。

他和乔文君中间隔着灵牌,是他发妻的,乔文君的
母亲,去世十来年了,活着的时候就爱看京戏。乔帮主
那时开玩笑说,哪天你就算不在了,我看京戏也带着
你。一语成谶,其实他自己以前不喜欢看京戏,打打杀
杀都是假的,照京戏这么一言不合就开打,他乔三在江
湖上早死几百回了。因为要多给发妻看吧,请戏班子来
狮吼帮,虽然只是灵牌,他相信她能听得到,看多了倒会
了一些门道。

今天是最好的戏班子,约了好几年,最近路过重庆
府才演上这么一出。可此时他真看不进去,不止是乔文
君,烦心的事多着呢,皇帝没了,太子不见了,朝廷是五
公主和三王爷并行,武林已经分成了两派,按五公主的
意思,绞杀昆仑公子,但更重要的是太子必须活着找
到。三王爷呢,昆仑公子怎样他不在乎,虽说以寻太子
为名拉拢各门各派,可谁敢把活的太子带到他面前?大
家都在站队,太子是死是活,押上帮运来赌国运,听说丐
帮几乎一分为二,马长老带着人押宝三王爷,而眼瞎的
关长老,则坚持太子是正统的王位继承人。

他狮吼帮虽然算不上百年基业,然而百十号弟子的
身家性命都在他一念之间。观望一下吧,还好守着长江
头,离京城足够沅,请个戏班唱戏都要等三年,反应慢了
点总比站错位置好。其个名角出来的时候他闭眼听了
一会儿,一睁眼又看了一眼女儿的小腹。这一次他气炸
了,他注意到乔文君的双手在揉着自己的小腹,好像担
心胎儿在里面太挤,揉一揉腾些地方给他。他一直盯着
她双手,恨不得双眼射出两把刀,把胎儿扎死在腹中。
她双手停住了,乔帮主抬起头,发现乔文君在瞪着她。

他不怕她,家族之耻,眼睛张得更大回瞪她,低声
说:“别在你娘面前揉,她如果知道这件事,会替你蒙羞,
再死一次的。”

乔文君看了看空位,拿起灵牌贴在小腹上,眼泪打
着转说:“爹,娘,我会死的,你等我把孩子生出来,我以
死谢罪。”
\newline

{\centering\subsection{10}}

好像时辰不多了,常公公抓紧最后的时间,和他并
排赶着夜路。脚下不停,他时不时地看几眼常公公,忽
然右手一探,手持匕首刺向他肚子。常公公弯腰闪开,
试图去抓他手腕,此时匕首已换至左手,向他肩膀劈去。

“两里路你一直在跟踪我,都快贴上我了,你当我瞎啊?”

常公公点着头,做出请的手势,让他先走。之后的
路程是他在前面走,常公公满眼泪水地望着他背影。他
们往东走,太阳就要从前方升起来,附近的农庄已经有
公鸡在打鸣,雨后的清晨道路泥泞。时候要到了,他希
望他能再回来一次,认一认他,和他聊一聊。无论现在
是好是坏,从此以后他们将再也不会回到这种关系。然
而他没有回头,他看见他越走越艰难,磕磕绊绊,终于面
部朝下摔倒在泥潭里。

常公公一时间傻了几秒钟,溅着泥点跑过去扶起
他。泥巴糊住了他的脸,看见是常公公,他紧紧抱住他,
试着去摸他的脸,睁大眼睛说:“我快完了,你别走,求求
你,千万不要带我去百花谷,我害怕,你别走。”

时辰到了,他捂着脑袋,直到昏倒前,他都使劲咬着
嘴唇,好让自己的眼泪不要甩出来。倒是常公公大哭起
来,嘴里不停地念叨:“爹对不起你,爹错了,爹再也不为
难你了,爹现在就带你走,我们去没人知道的地方,我们
再也不去百花谷了。”

他一边说,一边从他身上翻出那些信,坐在泥潭里
一一烧掉,然后哭哭啼啼地把他抱起来,背着他向远方
走去。

\newpage