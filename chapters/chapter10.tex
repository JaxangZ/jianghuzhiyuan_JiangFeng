\section{拾}

{\centering\subsection{1}}

总要有个人死的。

文思清一整天脑子里都在回想着这句话。那天早上她做了个梦,她梦见有个戴面纱的女人,走到她面前,不声不响,就那么哀怨地看了她好半天,临走的时候说了那句没头没脑的话,总要有个人死的。

“谁死,你又是谁?”她追下山去问。可那时她已经快醒了,快追到她,已经见到她背影的时候,她大口喘着粗气,索性睁开了眼睛,那女人反而消失了。

此时文思清身处于南京一家客栈的床上,偶尔有马车从窗外的街上经过。天色未亮,她翻个身想继续入睡,找到那个女人,问清楚,要是她不肯回答,就扯掉她的面纱,记住她的脸。

然而一时却睡不着了,她弯着腿,身子蜷成一团,紧紧地抱住被子,直到卖早点的小贩在街上吆喝起来,她才勉强又睡过去。可惜梦没有续上,这次是一个庆典,一百多个人聚在广场,似乎是过年,放鞭炮、写春联,她在人群中东张西望,看着大家饮酒划拳,乱成一团,居然没有一个是她认识的人。

死一个人,一群人搞庆典,一反一正的两个梦,到底在搞什么?那天吃中午饭时,她还对八光师弟说起了早
上的梦。他们坐在饭馆门口的方桌前,看着往来的行人,她问:“是谁要死呢?那个女人是谁呢?为什么要对我说呢?”

八光看了她一眼,也不知该怎么接话,权且当她是自言自语,继续揉着手心里的鸡蛋,时不时抬手看看,鸡蛋揉得怎么样了。他已经快十年没吃过这东西了,在少林寺吃青菜豆腐,像鸡蛋、牛奶和韭菜这种食物,也算是荤腥。以前吃煮鸡蛋,最重要的是剥皮,像是仪式感,他要先将鸡蛋敲一个小裂缝,然后手心朝下在桌上揉着,把鸡蛋皮揉开,但不至于碎掉,最后再像削苹果一般,一圈一圈地剥开,每一层越窄越好,中间还不许断,一个鸡蛋皮能拉成一尺多长,抻直了摊在桌子上。他两手伸出食指,各摁一头,像刚打出一套好拳一般,满意地点着头。

“到底是谁呢?”文思清问。“啊?”

“总要有个人死的,那是谁要死,又是谁对我说这句话呢?”

八光眨巴着眼睛,一脸为难,梦里的事情,至于讨论小半天吗?他拿起剥好的鸡蛋,问她吃不吃。文思清摇头,看到他把鸡蛋放回到碟子里,反问他:“既然不吃,为什么要剥成这样?”

“因为仪式感,”他说,“因为无聊,时间又长又难熬。”

这已经是他们守在码头的第五天了,昨天说好不来了,要么去百花谷,要么离开南京,向南出发。早上文思清又临时变卦,拿她荒唐的梦说事儿,她说今天有大事,不知道多大的事,有人要死,我们先别动,再去码头坐一天。这次他终于跟文思清明确下来,最后陪你一天,明天要还是坐在这儿,跟个傻子似的守着,他宁愿回少林寺,给沈老前辈扫院子去。

两个人就在桌前那么坐着,看着来往的行人,呼吸着扬起的尘土。下午最困倦的时候,八光玩起了新把戏。他一口将大碗茶喝光,吐掉挂在嘴角上的茶叶,找块抹布将茶碗擦干净,单手抓着空茶碗的碗沿,让文思清看好了。他手腕一抖,茶碗在桌子上转了起来。

文思清看了一会儿,没明白,问他到底要她看什么。“继续看,没完呢。”

那就再看一会儿,除了这个茶碗在原地打转,什么都没发生。到底看什么呢?她看看茶碗,又看看八光。八光叹了口气,跟她解释起来,那种感觉就像是你讲了个笑话,还要一本正经地解释笑点在哪里。

“你看这个碗,它原地转,跟定住了一样,不乱跑。”

“哦。”

“还有,它一直在转,差不多一炷香了,还没有停下来。”

文思清点点头,又“哦”了一声。八光皱眉望着她,问她老“哦”什么。

“没哦什么,你说的我都看到了呀,还有呢?”

“这是手腕上的功夫!”茶碗在桌子上还在转,八光把袖子撸起来给她演示,“手腕一抖,茶碗原地打转,转的时间越长,说明手腕越有准头,功力越深。”

“哦!”这回是真心的了,她明白了,说来说去,原来是功夫。

“没有十几年的内力是下不来的。”八光说,“记得分寸感,力大了它会跑,晃来晃去摔到桌子下面,力小了,转几圈它就停了。”

文思清点着头,表示知道你的意思。知道就好,八光长吐一口气,放心了,解释了一通笑点,但起码听懂了,这笑话为什么好笑,虽然现在笑不出来,没准日后回想起来,还会隐隐觉得蛮好笑的呢。

八光喊店家,再来一个茶碗,不要茶,空碗就行。他把碗推给文思清,说:“该你了。”

“该我什么?”

“练功夫啊,每天坐在这儿无所事事,功夫都荒废了,起码手腕上的功夫要练一练啊。”

“手腕上的功夫?”

八光点着头,告诉她,别以为练功就是练内力,练招
式,手腕同样要练。武林之中高手对决,二流高手杀人,无非就是一剑把你捅死,但有些人,不但杀了你,还能在你身上刺出一个梅花,或是一个八卦的剑花来,这些就是一流的高手,他们练的就是手腕上的功夫。

文思清听懂了,真是开眼界的一课,原来江湖那么大,什么人都有。那就从现在练起,她右手五指向下,握着空茶碗的碗沿,问八光:“这样对吗?”

“没什么对不对的,”他说,“刚开始嘛,就是要试错,错个上百次,自然就知道,对的是什么。”

那就先照错的来吧,文思清倒数三二一,顺时针拧下去,茶碗在桌面上转起来,原地空转,不乱跑,也没有停下来的意思。八光伏在桌前盯了一阵儿,两个转动的茶碗一前一后,文思清的那个比自己的还要稳,看起来一两个时辰都不会停下来。怎么会这样呢,本来要试个一二百次的,她一次就过。下午才刚刚开始,接下来要怎么做,才能打发这漫长而无聊的时光啊?

八光把自己的碗停下来,留文思清一个人的在桌上转,他碗底敲着桌面喊店家续茶,一大口喝下去,又续上一碗。然后他就像个石狮子一样,一动不动地坐在桌前盯着面前的街道。街上的人在动,车在动,他眼珠不转地放空发呆,直到一个稍有姿色的少妇从眼前走过,他不自觉地转头看过去,一直目送她在路口转弯。

少妇消失,还会有少女经过,一个个都盯着看,目送到街角可不行。文思清连咳嗽几声,提醒八光注意,身为佛家弟子,不要忘了淫戒,不要再惹过去的奸淫之乱。八光转回身,满脸通红,他解释只是不小心多看两眼,少林寺清修十年,他早已物我两忘,男也好,女也好,哪怕是猪狗牛羊,对他来说,也不过是一副皮囊。

“真的吗?”文思清反问他。

似乎被问住了,八光一时没说话,低着头,十指交叉在桌下,过了好一会儿他抬头说:“确定无疑,他们只是皮囊。”

文思清看着他眼睛,似笑非笑,抬手随便指了街上一个还不错的女孩,问他这个算什么。八光顺着她手指望过去,只见一个穿紫衣的女孩骑马行进,又盯了有一阵,趁她还没有消失,他转头讲:“皮囊。”

文思清点点头,又指了一对同行的女孩,问他那两个呢。八光看过去,两个女孩起初并行走在路边,直到在前面的岔口分开,一个往左,一个往右,八光左右都看一看,回答道:“皮囊,皮囊。”

于是有了新乐趣,比转茶碗、剥鸡蛋还能打发时间的
新玩法,文思清不断地指着路上的女人,八光一次比一次迅速地回答“皮囊”二字。后来文思清就换个玩法,她不再问这个呢,那个呢,她直接指着女人们问,漂亮吗?八光回答还是不变,依然是皮囊,当然无所谓漂亮不漂亮。

看来真的有进步喔,文思清忽然指着一位赶路的车夫,坐在车前,手握着缰绳,满脸的络腮胡子,她问:“这个呢,漂亮吗?”

八光看过去,这次没能第一时间回答。文思清在旁边捂着嘴笑起来,说:“终于不是皮囊了,对不对?一脸的胡子,可漂亮了呢。”

八光的脸又红了,盯了一会儿车夫,忽然抽了自己两巴掌,沮丧道:“淫心还是未净,漂亮!”

文思清哈哈大笑,看到八光的脸越来越红,她也感觉哪里不对劲了。她看过去,就是一个驾马的车夫啊,就算是漂亮,也只有可能是坐在车里面的娘子啊。

“你能看到车里的女人,是不是?”

“没有,是赶车的人很漂亮。”

文思清站起来,眯着眼睛盯过去,那么密的胡子,脸上皮肤却干净得要命,手也小,脚也小,好像是女扮男装。假如她是女人,她应该在哪见过。那个茶碗还在桌子上转,她伸手压住碗边,不自觉地朝她走过去。没看错,文思清认出这个女人了。
\newline

{\centering\subsection{2}}

头天晚上他们还在扬州的客栈,小五子拿根金条,让苏子瑶替他去赌。苏子瑶赢了不少,换了银票上来,小五子本想抱着银票,美美地睡上一觉。可一个时辰还不到,快天亮时,苏子瑶把小五子叫醒,她说,我们出发吧。小五子揉着眼睛,问她去哪里。她说:“去南京。”

“我知道的,早就说去南京,”小五子说,“我们最终都要去南京。”

小五子坐起来,打了个哈欠,隔着苏子瑶推开窗户看了看,天还没有亮,月牙还嵌在夜色里。他哈欠打了一半,停住了,伸手将半张着的嘴巴慢慢合上,警惕地看着紧闭的房门,低声问:“有人发现我了?”

“没人发现你,”苏子瑶说,“为什么要有人发现你呢?”

“因为你要拉我赶夜路。”

苏子瑶摇头,确定没有危险,之所以摇他起来赶路,是因为她实在睡不着。“如果你还是困,你可以在车上
睡。”她说,“反正我来赶马车就好了。”

苏子瑶说完背过身去,扎起头发,套上黑色的男人长衫,将胡子贴在脸上。小五子从后面看着她,右手无名指剔着眼角上的脏东西。苏子瑶催他准备一下,愣坐在床头干什么。

“你在撒谎。”小五子说。

她头也没回,小指在嘴角抹着胶水,拿起假胡子,将铜镜转到能看见小五子的位置,问道:“怎么看出来的?”

“你的背影告诉我,你在撒谎。”

她把胡子对准,一下子贴在脸上,再按按没粘牢的地方,回身冲小五子笑了笑。这算回眸一笑吧,一脸的胡子露出一口白牙,小五子表情僵住,双手搓着脸,下床洗漱。

但真的有问题,白天坐在车里他还在想,之前还在赌场玩得好好的,回来就急着要走,也许在那里遇到了什么人,追一个人,或是躲一个人。苏子瑶一路扬着马鞭,跑了一清早加一上午都没打算休息。小五子掀开帘子,头伸向车外,只见苏子瑶挥着鞭子,还要留意两侧的路人,他看看前方的路,又看看头顶的日头,路线没有错,的确是往南京去。这是在追人,可昨晚在赌场,她到底碰到谁了呢?

他想,问也没用,以苏子瑶的性格,想说她自己就说了。这么快的马车,轮子在路上颠来颠去,左右都没得扶,后来他干脆躺下来,迷迷糊糊地睡着了。大概在下午,忽然一个急刹车,小五子在车里翻了个圈,醒过来。明显感觉苏子瑶在调头,马车开始往回走。他抓着车桅坐起来,马车速度变缓,慢慢在路边停靠下来。他听见苏子瑶在车外问:“你师姐呢?”

“跟个老头走了。”外面一个男人有些虚弱地回答道。

小五子听过这声音,很熟,应该跟他打过不少交道。他手抓着帘子,先不急着掀开,给自己十秒钟,想想他是谁。苏子瑶还在问话:“什么老头?”

“一个男的,有些年纪,头发都是白的。”

“他跟你师姐走,为什么把你留在这儿?”

“我师姐把我绑起来的,他们俩说是去南京百花谷,不想让我跟着吧,让我回扬州,是我自己好奇,让我走,我不走,非要一路上跟着,逼得师姐又折回来,将我绑在这儿。”

苏子瑶笑了两声,说句“谢了”。小五子在车里感觉马车在调头,重新面朝南京的方向。那男的在车外骂起来,嚷嚷着:“你问的,我都说了,怎么还不放我下来?”苏子瑶笑道:“你还是留在这里的好,我怕后面还有
人要问你。”

“问个屁!本来就是想让我在这儿晒死烤死,能碰着一个你,就算是不错了。求求你,放我下来,我欠你一条命还不行吗?”

一只手!小五子想起来了,不掀帘子就能听出来,自已这儿还欠着三条半的命,还敢去跟别人赊账?小五子喊着:“停车!让我下来,要欠也是欠我的命!”

外面的一只手问道:“五帮主也在?”

小五子打开车门,看见一只手双脚拴着绳子倒挂在在树上。救他下来其实不难,身上还揣着那两把宰过狮子的杀猪刀,随便一把飞出去,砍断绳子,一只手也就下来了。但那是掉下来的,两人多高的半空中,脸朝着路面,何况他还只有一只手,撑不住地面,救和没救也算是一回事了。

他只好下车,过去绕着树根走了一圈,找不到绳头,仰头向上看,绳子收在快两人高的树干处。双手举起来,使劲蹦都够不到。苏子瑶应该没问题,脚尖点两下,就能上去把绳子解开,但他又不想找苏子瑶帮忙。背着手寻思了一会儿,小五子跟他商量道:“这样吧,我要爬树上去救你,多少会比较狼狈,算两条命。”

“可你只救我一次。”

“行,那就救你一次,你自求多福。”

小五子掏出杀猪刀,单眼瞄准绳子,举起刀准备飞出去。一只手脑子慢,先是期待催他,看他举刀晃了一会儿,才明白接下来可能发生的事,绳子割断,他从空中掉下来,然后摔死。他连说别别别,五帮主,您还是上树救我吧。

“上树救你是几条命?”

“两条,”一只手说,想了想,他又补充道,“救一次就够。”

“那是刚才,现在我不高兴上去了,要涨价才能上去。”

“那就三条,四条五条六条,都行!”

“说得这么轻松,你原本就打算赖着不还吧?”

一只手说:“还,一定还,这辈子还不完,下辈子托生到好人家,我继续还,你看我之前也没有躲,没有赖账嘛,江湖之远,人生漫长,总有我还完你五帮主的那一天。”

一只手吊在绳子上,手还在空中跟他比划着解释。小五子听听忽然走神了,叫他等一会儿,跟他确认一下刚才的话:“你刚才说,你师姐跟人走了?”

“啊。”一只手点着头,但在半空中看起来,总感觉怪
怪的,像是使劲勾头。

“是吴思若吗?”

“奇怪了,我那么多师姐,你怎么知道是她?”

“是她吗?”

一只手使劲勾着头。小五子知道那是点头,一刀飞出去,绑着双脚的绳子应声而断,一只手下坠,小五子上前几步,双手逮到什么抓什么。可惜没抓住,最终还是让一只手摔在地上,小五子左手握着一把衣服的碎布,右手攥着一绺头发。一只手在地上哼哼唧唧,他问他,是不是摔得太狠了。

“狠倒是没多狠,”一只手用他的一只手,捂着刚被薅下来一绺头发的头皮说,“扯得我脑袋疼!”

小五子把他拽上车,让苏子瑶继续赶路,约莫要一个时辰,才能到南京。一只手一直想不通,赶车那大胡子怎么是女人?小五子问苏子瑶:“昨天夜里是不是见到了吴思若?”

苏子瑶没回答。 “所以,你在追她?”

苏子瑶还是没声音,但他猜到,她应该在车外点着头。车轮滚滚,小五子问她:“是什么人劫走她,为什么要去百花谷?”

“我真的不知道,”苏子瑶终于说话了,过了好半天,她又加了半句话,“回少谷主。”

“是啊,”小五子自嘲道,“我居然还是百花谷的少谷主。”

苏子瑶没接茬,一只手倒是插话进来,低声对小五子说:“我师姐去百花谷,就是找你。”

小五子愣了一下,看着他。

一只手接着说:“她求那老头给你看病,说你可能在百花谷。”

“什么病?”小五子问他。

“断魂掌,”一只手指着自己的太阳穴说,“脑子不好使的病。”

小五子皱着眉,思索什么人会治我的断魂掌。一只手也识趣,不再多嘴。一时间车里车外三个人,谁也不说话,只听到苏子瑶时不时的“驾”声。外面熙熙攘攘,声音嘈杂起来,车速也慢了下来,估计是到了南京的地界。小五子把帘子拨开一条缝往外看,两三里外横亘一条大江,那就是到长江口了。

苏子瑶说:“先到码头,一会儿我们要下车换船,过了长江就是百花谷了。”

一只手要探头往外看,小五子警告他,小心把命丢在这儿,也不看看跟谁坐在一车上?哦,一只手明白了,旁边可是人人得而诛之的昆仑公子。苏子瑶马车慢慢停下来,看着江面上的每一艘船,和每一个上船下船的赶路人。

扫一眼就知道,茶摊那边有一点不对,门口的一个光头和尚和一个女人,先是那个光头和尚老盯着她看,不一会儿他旁边的女人在桌前起身,朝她走过来。远一点还看不清,她手握着鞭子,随时准备挥出去。走近一些,她看清楚了,那女人是文思清,早在前年冬天,在田独就见过她一次,当时大雪封山,苏子瑶还喝过她一碗羊汤,去年在昆仑山庄还见过一面,凑上吴思若,三个女人在台上,被下面上百个乌合之众审视。

径自走过来,的确是冲她来的,莫非认出她苏子瑶了?她摸摸脸上的胡子,假装不在意,故作轻松地朝左右两侧望去,就这么左看右看,反而有了新发现,目光突然盯在左侧一个白发长者身上,只见他跟一个女人同行,从东边往茶摊过来,准备入座,而那个同行的女人,正是她追了一天的一只手的师姐。

小五子在车里,似乎也拨开帘子看到了他们俩,他失声叫出了她的名字:“吴思若。”
\newline

{\centering\subsection{3}}

吴思若以前见过一次大师伯,不是去南海,大师伯来的罗布泊。八月盛夏,罗布泊最热的时候,阳光底下晒一会儿,头顶都能冒白烟,出了绿洲,往沙漠走几步,那么毒的太阳,沙子上的光都变形了,骑在骆驼上看人看天,看沙漠里的沙蛇、红柳和仙人掌,一切都是影影绰绰,有些恍惚。

那年吴思若十三岁,去杭州紫竹院,还是第二年春天的事情,当时的她,还分不清什么是忧愁,什么是悲伤。所有的关于大师伯的记忆,都还在她无忧无虑的年纪里。大师伯是正午到的,一年最热的一天,一天最热的一刻。随行的还有两个仆人,严重脱水,基本上刚踏进院子,就已瘫倒在地上奄奄一息。即便功力如大师伯,也要连喝几杯水,在阴凉处坐上半个时辰,才能缓过来。

下午她师父大漠仙人吩咐弟子去挑了几十桶井水,将冰泉池填满,请他大师兄坐进去,一直泡到日落。晚上他开了酒席,宴请大师伯和他带来的两个仆人。那两个仆人虽然还没有死,被吴思若和她的师姐们,用一瓢一瓢
的井水灌活,能走路,能说话,可此时面对滋着油花的烤全羊,面对一桌子的葡萄美酒夜光杯,竟一口也咽不下去。反而大师伯胃口出奇的好,大块吃肉,大口喝酒,喝到兴起时,喊他那两个仆人,去把给二师弟带来的礼物搬上来。

两个可怜的人啊,两个月长途跋涉到这里,一下午没吃没喝,已经两腿发软,虚脱到路都走不动了。他们从外面把礼物带进来,三十张从南海带过来的海龟壳。吴思若这时才知道,原来大师伯从南海来。

可海龟壳算什么礼物呢,是磨了做粉吃,还是背在身上防身?她师父拿起一张掂量了一下,看清楚上面的纹路,点了点头,又拿起第二张,比较过后,问道:“大师兄掌力,已精进到如此程度了?这么厚的龟壳,竟可以一掌将海龟击毙?”

南海真人叹了口气,摇头道:“也只是击毙,力气大一些而已,若说让我在这龟壳上使断魂掌,就是连打它三掌,对这千年王八万年龟,也起不到半点作用。”

“怕是再练个十年,你的断魂掌,要远胜于我的仙人掌和三师弟的蓬莱掌。”

“哪里,哪里,十几年不见,二师弟还是那么会捧杀。”

吴思若听不懂,更不明白哪里好笑,两个人能面对面地哈哈大笑。后面的话更加听不懂,大师伯提议,反正十几年,天天只练一掌也无聊,不如咱们两个换掌学学,他日让三师弟见到,羡煞你我二人,如何?她师父沉思片刻,举起酒杯,和大师伯一饮而尽,反复强调,好说好说,何不在我这儿多待几日,咱们来日方长。

大师伯果然在罗布泊呆了好多天,差不多有一个月,那两个丢了半条命的仆人,都已经休养过来,恢复元气,胃口大开,除了吃烤肉、摘葡萄,还能钻到沙漠里捉沙蛇,放在坛子里泡酒喝。师父和大师伯倒是不再进食了,仿佛一顿大餐顶半年,不吃不喝,觉也不睡,每天就是在突厥人留下的石头城里切磋武艺。

吴思若不懂,都是大师姐跟她说的。她说:“天下最厉害的三掌,师父和大师伯占了两掌,师父把仙人掌教给他,再换来大师伯的断魂掌,以后他们就是武林中最厉害的两大高手啦。”

“那之前呢,之前是几大高手?”

大师姐瞪着她,仿佛觉得她笨得不可理喻,手指戳着她脑门说:“我刚刚跟你讲过,天下最厉害的是三掌,之前当然是三大高手了!”

哦,她明白了,搞了半天,原来是三进二的晋级,最终
就是为了淘汰一个。可有必要那么辛苦吗,师父和大师伯没日没夜地在石头城练功,吴思若睡觉的时候,他们在练,吴思若醒来的时候,他们还在练。有时她心疼他们俩,把沙蛇从两个仆人的酒坛子里捞出来,给他们煲汤喝。她小火熬一个时辰,再放一个时辰,等瓦罐凉一凉,捧在怀里给他们送去。可不知道他们练的什么功,刚靠近石头城,她就被一股力道给震了出来,瓦罐碎掉,蛇汤洒了一身,顺着衣角往下滴,还有一条快熬化了的沙蛇挂在肩膀上。

之后她就不敢去了,远远地坐在自家屋顶上看着。终于有一天黄昏,师父和大师伯突然不练了,两人站起来相互瞪着。先是大师伯发问:“原来你在唬我,教我的仙人掌,全都是假的!”

师父冷笑一声,说:“大师兄,你千山万水从南海过来,还以为你有些诚意,又何不是自己编了一套假断魂掌,来罗布泊换我的真本事。”

大师伯重重地“哼”了一声,说:“我看你的真本事,编得也不错!”

话音未落,大师伯先动了手。师父向后退三步,侧身闪出右边的半个圈。大师伯及时收手,向左边攻去。这时师父已经一掌击过来,化守势为攻势。两个人此消彼长,一时分不出胜负,在石头城里周旋起来。

仙人派的弟子都爬到屋顶来观战,想不到这么多人上来,屋顶竟还没有塌。最后上来的是那两个仆人,有些尴尬地坐在一角,盯着二人的交战,有时为南海真人叫好,有时又忍不住地,为大漠仙人喝起彩来。两个仆人那么专注,时不时大喊着“好”,弄得屋顶上的弟子都不看师父和大师伯打架,看他们俩就已经很有趣了。

大家心里千般疑惑,吴思若先问出口,她说:“你们做下人的,也能跟主人学功夫吗?”

“南海真人不收弟子的,”其中一个回答,“他只招仆人,但又教我们功夫,督促我们练习,实际上跟徒弟没两样。”

“但不许我们喊师父,”另外一个补充道,“只能叫他主人。”

哦,原来还有这样的人。大师姐来了兴趣,她提议:“既然我们师父和你们师父在切磋,不,是你们主人,不如我们做晚辈的,也下去比划比划。”

“那为什么要下去呢?”头一个回答的仆人问道,“就在这儿练练手好了。”

“在屋顶?”吴思若惊讶道。

大师姐可不服软,站起来说:“那就在屋顶吧,但先跟你们讲好,谁要是不小心摔下去,有个三长两短,可不能赖上我们。”

“哈,”另外一个起身应战,“听你这口气,好像一定是我们摔下去一般。”

口气都不小,两个人真比起来的时候,可就一般般了。别说是给对方一拳一脚,站在瓦片上就直打晃,俩人除了用眼神盯盯对方,握紧拳头做做样子,全部精力都集中在下盘,一动不动,生怕比对方先掉下去。也是,两个人都不大,大师姐那年也才十八,那个年纪大点的仆人,也不过十六七的样子。吴思若看看就没意思了,继续看石头城里的对决。

里面还是未分高下,但显然大师伯已多了些疲态。大漠仙人劝他不要再打下去。“你,我,加上三师弟,本来就不分伯仲,”她师父说,“如果在中原约个地方,就是打上三天三夜,三年三十年,也决不出胜负,倘若去你的南海,不出三个时辰,潮气和海风上来,我出手迟重,必定不是你的对手,但此时是在我的罗布泊,你早晚会体力不支,完败于我。”

不知大师伯是听进去了,还是没了力气,一掌比一掌缓慢,师父也配合着他,放慢掌势,然而他终归咽不下这口气,忽然发力,连攻出十几掌,不等大漠仙人反击,跳出石头城,三步两步竟上了屋顶,一把抓住刚刚站稳的大师姐,冲石头城里的师父喊话:“我大老远地过来看你,总不能就这么灰溜溜地走,随便送我条人命,让我杀你个徒弟,我自己就走了,我南海真人,以后永不再踏入你罗布泊便是。”

大师姐吓得脸都白了,屋顶上的弟子一个个想逃,又不敢直接跳下去,只能坐在瓦片上,屁股一点点地往旁边蹭。唯有师父最为镇定,没有追出来,站在石头城里冲着他微笑,轻吐一口气,提醒他:“你只能灰溜溜地走。”

南海真人的左手抓得更紧一些,抬起右手,勾起拇指、中指,扣在大师姐的喉咙上,仿佛随时能把她的喉管整根掏出来。

大漠仙人摇着头,脸上保持着微笑,再次提醒他:“你要是动一下我的弟子,我不管你是不是我大师兄,从今天起,你就别想走出罗布泊了。”

南海真人喘着粗气,浑身发抖,似乎胸中一团火急着吐出来,他一把将大师姐甩开,大师姐跟炮弹似的朝石头城飞过去,师父上前几步,抱住大师姐。所有人都望着她,看到大师姐还活着,纷纷松了一口气。而此时,屋顶
上咚咚两声,多了两个窟窿,碎砖碎瓦从窟窿里掉下去,一转眼的功夫,大师伯已经跳下了屋顶,向沙漠中远去。

可那两个仆人呢,教他们武功的主人可没把他们带走。有人趴在窟窿边上,尖叫了起来。吴思若爬过去,透过窟隆往下看,只见那两个仆人仰躺在屋里的地面上,已经死了,脖子上血淋淋的,而从喉咙里抽出来的,是两根还滴着血的喉管。
\newline

{\centering\subsection{4}}

尽管十年没见,但从在赌场看见他的第一眼,吴思若就认出他来了。为了泄愤,能把自己的两个徒弟杀死,而且亲手把他们的喉管拔出来。图什么呢,显得自己本事大吗,告诉对手,自己不是那么好欺负的吗?这种人一辈子也忘不了。

一只手先上去的,后来是那假大胡子,等她离开赌场的时候,南海真人正好进来,头发更白了,但以前也不黑,也不年轻,满头银灰色的那种,吴思若跟他擦肩而过,似乎闻到了他身上的血腥味儿。回到客房,她平躺在床上,思考着应该做点什么。这是个机会,虽然一下子也说不清,这是个干吗用的机会。哄他对付师父?求他把竹林的丧尸坑填了?倒是有好多事可以利用他,可这些对她都不重要,因为都是她吴思若自己的事情。是啊,到底怎么了呢?她自己的事情不重要,那还有什么是重要的?

有那么一件事情,一定要他去做,解开小五子的断魂掌。她不知道有没有这道理,你给对方一掌,深受其毒,再补上一掌,就能把之前那掌消掉?反正仙人掌是没有,就是不小心给了亲爹一下,也只能看着他不吃不喝,熬过一天算一天。可万一能解呢?就算解不了,也要问清楚,小五子从哪来,做什么的,当初为什么给他这么一掌。把这些捋明白,也算是为小五子做一件事吧。

她站在窗前,盯着客栈门口,隔着一堵墙,都能听到隔壁一只手的呼噜声。眼看快天亮了,南海真人没离开,也没上楼,吴思若披上衣服,想下去再看看。刚把门打开,窗外传来声音。她踮脚走过去,从窗口看见南海真人出了客栈,骑马远走。

她去敲一只手的房门,告诉他,现在退房,咱们出去寻一个人。话刚说完,她就后悔了,此行凶吉未卜,何必要拉上他,白搭一条命?她说算了,转身下楼,出客栈牵马。一只手反而追了出来,大老远地冲她喊:“师姐,等一下,我陪你去南京,还不行吗?”

根本没有生他气的意思,她等他一起过来,一路赶到中午,直到在一家饭馆前,才重新看见南海真人。她装作若无其事,和一只手坐在离他不远的一桌前,招呼店小二:“把你们家最贵的菜都上来一遍。”

南海真人侧过头冲她笑,说:“跟了一路,果然很辛苦。”

一只手回头看看真人,又看看吴思若,低声问道:“师姐,咱们一路追的就是他?”

吴思若没理会一只手,起身走到真人面前,鞠躬作揖,说:“弟子十年前曾跟大师伯有过一面之缘,昨日突然遇见,却不敢相认,还请大师伯见谅。”

大概就是这样的开场白,紫竹院几年不是白呆的,跟男人聊天找话,吴思若还是有那么一套的。她先说南海真人去罗布泊的那年夏天,说起他的断魂掌,师父的仙人掌,话锋一转,直接提起九宫图。真人来了兴趣,拐着弯地跟她盘道,试探她对九宫图了解多少。

“九宫图,弟子是一点都不知道,”吴思若说,“只是刚好有那么几张在身上。”

她不等他质疑,直接抽出一张拍桌上。真人拿起那张羊皮,在手里端详了好半天,说道:“这是我三师弟阁老的那张,其余还有哪几张,都在你身上?”

“以大师伯这样的辈份,不会是想强抢我这张吧?”

真人打着哈哈,将九宫图放回到桌上,说:“你就算是有,也不会随身带着。”

“那你就当我只有这一张好了,这张我孝敬给您了。”吴思若招手结账,吩咐店小二:“把这两桌全都算我账上。”

她放下银子,笑笑起身,没有拿九宫图,回到一只手桌前。一只手始终在犹豫着,身后的是大师伯,要不要去拜见一下?照理说,自己早已被师父逐出师门,此时也不该行同门之礼。何况他们师侄俩已经聊起来了,吴思若刚才点的好酒好菜,全都端到他桌前,可他却只是一个人吃。他听到他们聊断魂掌,聊九宫图,好像还聊到了师父,聊到三师叔蓬莱阁老。

吴思若坐回来对他说:“我今天和大师伯去趟南京,到百花谷会一会昆仑公子,此行凶多吉少,你就不要跟着了。”

“昆仑公子不是你意中人吗?怎么会凶多吉少?”吴思若眯眼睛瞪着他,最后冷冰冰地扔下一句:“你走吧。”

嘴上说“回扬州等我”,却要和他一起出饭馆,把大师
伯留下来。一套又一套,把一只手完全绕懵了。刚要走出门,只听到大师伯在身后问着:“我拿你几张九宫图,而你要我做什么?”

吴思若转回身,看着南海真人。他把桌上的九宫图扔过来,又问她一次:“要我做什么?”

“先去趟百花谷,”吴思若接过九宫图说,“会一会昆仑公子。”
\newline

{\centering\subsection{5}}

虽然是叫别的女人,喊的吴思若,文思清听到小五子的声音,心都要化了。她问:“是你吗,小五子?”

小五子沉默几秒,直接从车上下来,望着文思清。这是怎么了,三个人都在。他问她,这段时间都在哪里?文思清回头看一眼,八光从茶摊朝他们走过来。小五子奇怪:“这和尚又是谁?”

不等文思清回答,苏子瑶先抢话说:“那是淫贼田扒光啊。”

一只手听说后,掀开帘子朝外面望过去,感慨道:“原来田扒光就长这个样子,怪不得碰到女人,都是奸淫为主,引诱为辅。可是他怎么出家当和尚了?”

“就是当和尚,也是花和尚吧。”苏子瑶说。

说话间,八光已经走过来,对文思清喊了一声“师姐”。

“这又是怎么回事,”小五子完全绕迷糊了,问道,“你什么时候有门派了,田扒光怎么成了你的师弟,那你们的师父又是谁?”

一时间解释不清,苏子瑶见缝插针,说她是百口莫辩。文思清留意到,她刚才把胡子摘了,甚至盘起的头发都放下来了。她是在嫉妒,不高兴遇见她文思清,可那边还有一个呢?文思清转过去,看看南海真人身旁的吴思若。算上自己,她手指点着,一,二,三,三个女人,是不是太多了?她看着小五子,一下子明白那个梦了。就是那句话,总要死一个的,原来在这里等着她。一妻一妾,齐人之福,才只需两个女人,若是一心一意,一生一世,恐怕死一个还不够呢。

吴思若先看见的假大胡子,坐在马车上四处张望,十有八九是从扬州跟踪她过来的。之后她看到茶摊上的和尚和文思清,这还没有联想到,直到文思清朝假大胡子走过去,两个人并排在同一画面时,吴思若看明白了,那是
苏子瑶啊,那是另一个深爱着小五子的女人呢。

好像是多了一点,感情也变得麻烦起来。反正她会退出,最后为小五子做点什么,体面地离开这里。过去不体面,难以启齿,她不配,要让小五子把爱献给匹配的人。她背过身,不去看她们,这时听到了一个声音:吴思若。她回头望过去,她知道,就在那辆车里,小五子在呼唤她。怎么办?她还计划体面地离开呢。

有人告诉他,吴思若旁边的是南海真人,使断魂掌的那个,小五子就一直盯着他,一直盯到自己点头,转身他爬回车里,将身上的杀猪刀、九宫图都卸下来,把放车里的金条全装上,问一只手有没有匕首一类的东西。一只手没有,但他早就发现座位底下有一把。小五子抬起座位,捡起那把匕首掂量一下,指甲在刀刃上轻轻划一个道,好用就行。他把匕首揣进怀里,下车的时候经过苏子瑶身边,他瞟着南海真人,问她:“他跟我多大的仇?”

“你要干吗?” “跟他聊聊。”

小五子说完朝真人和吴思若的桌前走过去。文思清快步跟上去,小五子转过身,看着她,求她不要动。

“我没那么蠢,不会武功,还赶着过去送死。”小五子说,“原地看着就行。”

文思清看着他走过去,听到他大声打招呼:“阁下是南海真人吧?”

南海真人抬头看他,一时想不起来他是谁。小五子自我介绍:“在下小五子。”

说说而已,没有作揖,没有寒暄,直接坐到真人的旁边,吴思若对面,喊店家上酒。店家过来解释,说我们家是茶摊,只卖茶,不卖酒。

“那别人家卖吗?”

“别人家当然有得卖,只是······”

小五子掏出金条放在桌上,一字一顿地说:“谁家卖酒,你买过来,卖给我。”

店家领会了,赶到街对面抱了两坛酒过来。小五子把茶水倒掉,在茶碗斟满酒,将桌子上的金条递给店家。真人看在眼里,举着茶碗说:“你若有心请我喝酒,移步到对面就是了,何必这么破费?”

小五子也端起茶碗,哈着腰说:“真人千金贵体,怎能随便移步?本来就应该人在哪里,酒到哪里!”

真人哈哈一笑,跟小五子碰了个杯,两个人一口喝下去。文思清和苏子瑶见这边没事,也都陆续过来,站在他
身后,以免有什么突发状况。连干了几碗酒,小五子一句正经话没说过,一直说在田独养猪,怎么赚钱的事情。倒是一坛酒转眼喝光了,小五子抱着空坛子,气不打一处来,一把将酒坛摔碎。一把抽出三根金条,喊店家这次痛快点,一次给我买三坛过来。店家想拿金条,又碍于这么多人看着,说其实不用再出金条,刚才那一根,怕是八十坛、一百坛都够了。小五子要店家收着,请真人喝的酒,一根金条只能喝一坛,要是能喝一百坛,他就拿一百根金条来买。

店家把金条收走,叫人赶快搬三坛好酒过来。除了店家,包括围观看热闹的人,谁看着都心疼这金条花得不值。真人向后一靠,说:“五公子这么破费,看来不只是找我喝酒这么简单吧?”

“酒是小事,本来就是在下想请真人的,他日还想重金请真人帮我个小忙。”

不知是对帮什么忙好奇,还是对重金两个字感兴趣,真人让他说来听听。

“阁下名号南海真人,自然长居南海,对那里了如指掌,不瞒您说,我前两年在南海购置了一个岛,最近发现

······”小五子说到一半,回头看看,除了苏子瑶、文思清和吴思若这三个女人,还有一只手和八光,还有店家和一些围观者等他说出来。小五子跟他商量,能不能借一步说话,近一点,只给他一个人听到。

真人又是哈哈一笑,连声说“好”,身子侧过来,与他离近一点。小五子右手捂着嘴,凑在他耳边,轻轻说:“我买的这座小岛,最近发现了······其实我也不知道。”

南海真人皱了皱眉,突然感觉有把利刃从后背穿过,只听小五子接着说:“现在你已经知道了?一掌之辱,舍生相报!”

说着小五子左手发力,又把匕首使劲往里扎。刀尖都快从胸前穿出来了。南海真人咬着牙,不顾后背涌出来的血,大吼一声,手勾到后背,抓着小五子的手,一把将匕首拔出来。一掌发力把小五子推出去,小五子将茶摊后的墙撞倒,躺在地上,苏子瑶、文思清和吴思若,三个女人,赶到他身边。

看起来南海真人没事,从后背扎到前胸,简直就是怪物,站起来拉伸一下,血似乎就止住了,他走到小五子面前,鞋底踩着他的脸,低头看着他说:“留你条性命,是要问你几句话。你中过断魂掌?”

小五子右脸贴着地面,左脸贴着真人的鞋底,即令如此,还要努力地轻蔑一笑。

真人把脚挪开,蹲下来,摸了摸他的脉,沉思道:“这掌不是我打的。”

所有人都愣住了,小五子睁大眼睛看他。“我杀你轻而易举,用不着骗你。”

小五子强撑着坐起来。

“还记得打你这掌的人,长什么样吗?”话刚问完,真人自己都摇头苦笑,“断魂掌,断魂掌,当然是不记得了。我先不杀你,你给我好好活着,等我查出谁在冒用我的断魂掌,我再取你狗命。”

真人说完,背身走回去,坐在刚才的茶桌前继续喝酒。虽不至于感激,但不能傻到继续跟他搏命。三个女人扶他起来,一步步朝对面的马车走过去。真人在后面喊住他:“就这么走了吗?”

小五子晃了晃神,想起了规矩,他伸出左手,说是这只手捅的,手背朝下放在桌面上,抄起刚才的那把匕首剁下去。忽然真人拉了一下他的左手,匕首剁了个空,刀尖扎在桌面里。

真人举着他左手看着,就像阳光底下看一块劣质的玉石,说道:“捅我一刀,一个爪子就还了?”

“那你要什么?”

真人朝众人看去,目光落在苏子瑶、文思清和吴思若身上,问道:“三个都是你大小老婆?”

小五子想了想,点了点头。

“五公子,你挑一个给我杀吧。”

小五子直摇头,指着自己说:“你还是杀我吧,祸不及妻儿。”

“我说了,不杀你,但你现在要还我一条命。”小五子咬牙瞪着他。

真人继续说:“你来选,虽说都是喜欢的,总有深浅吧,选一个你没那么喜欢的。你最好听我的,别逆着我,你如果不选一个,我三个全杀。”

小五子感觉气都喘不上来了,他回头看着吴思若,看着苏子瑶,看着文思清。

八光抢过来说:“你要是把我师姐杀了,我拿命跟你拼。”

“你弄错了,”南海真人微笑着说,“不是我要杀谁,是这位五公子要杀谁,我全都听他的。”

八光转身催促文思清:“师姐,告诉他,你是谁,你是谁的徒弟?”

“你不要管了,让小五子选吧。”

小五子将每个人看过一遍,转回身闭上眼睛,说:“我
选不了,你把三个都杀了吧,连我也带上。”

“好像玩法有问题,选一个,让她死,是有点残忍。”南海真人想了一会儿,把规则梳理一遍,说,“这样吧,选一个你最爱的,最舍不得她死的那个,保她活下来,剩下两个我来挑,这样好一点吧?”

真人等了一会儿,提醒他是最后一次机会,没得换了,他从五开始倒数,之后是四,三,二,一。小五子喊出苏子瑶。

“留下苏子瑶,”他说,“我小五子这二十多年,活过两辈子,文思清和吴思若是我这辈子的,我薄情也好,深情也好,我总还记得,能还得上。苏子瑶是我上辈子的,我跟她有过什么感情,我全都不知道,我欠她有多少,我也不知道,我要让她活着,就算我以后也没法爱她,我起码不想再亏欠她。”小五子说完,看着文思清和吴思若说:“对不起,文思清,对不起,吴思若。”

吴思若是冲他微笑,文思清早已是哭得稀里哗啦。南海真人指了指苏子瑶说:“这两个人,你再挑一个?让她死,或是让她活,咱们把它玩下去。”

“你杀了我吧,”苏子瑶说,“少谷主,你早该选我的,反正我跟你也没什么了,不如放过她们两个。”

南海真人不耐烦了,说你们好啰唆,剩下的我来吧。他让文思清和吴思若上前一步,站在小五子面前,苏子瑶回到他的安全区。忽然一阵狂风大作,小五子头昏目眩,由不得地闭上眼睛。大概十几秒钟,他睁开眼睛,看见文思清和吴思若都还站在面前,一个都没有死。小五子痛哭起来,从来不服软的他,这次跪了下来,连声说:“感谢南海真人,晚辈这次受教,以后再不会狂妄了。”

南海真人笑了,那神情跟得道高僧一样,云淡风轻点着头,临走时他说:“这算是个小小的教训,不必感激我,你捅我一刀,我杀你个女人,以后你我还是朋友,随时来南海找我。”

小五子越听越皱眉,转回身,又一次地跪了下来。死掉的是苏子瑶,躺在地上,一口气都没留,脖子上血淋淋的一团,南海真人硬生生把她的喉管,一整根地摘了出来。

总要有个人死的。

\newpage