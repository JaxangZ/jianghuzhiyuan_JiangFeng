\section{拾柒}

{\centering\subsection{1}}

小五子拽着文思清说,你跟我去见一个人。他把她拉到太上皇的寝宫,对着昏迷不醒的嘉和皇帝说:“我跟你没血缘关系,打我有记忆这两年,还未能和你说上一句话,但不管怎么说,我叫了你几年的父皇,骗了你几年,更何况,我们沈家的人还杀了你儿子孙天奇。当年,你灭了我们沈家王朝,留下我一个独种,现在我们沈家让你断子绝孙,我们两家也算是扯平了。”

他转身问文思清:“他是怎么受伤的?我又是怎么受伤的?”

“那天晚上,你突然拜见嘉和皇帝,你要把全部实情讲给他,请求嘉和皇帝赐罪,废掉你这个假太子。当时苏子瑶是你的太子妃,她一路跟着你,发现你的事情已经败露,她只能出手杀了嘉和皇帝,若不是你上前阻拦,替嘉和皇帝挡了一掌,你也不会失忆,嘉和皇帝也不至于死不死活不活的,没准你三四年前就当上了皇帝。”

小五子听过之后,对着嘉和皇帝说:“不管怎么讲,我还是你的儿臣,我总得尽孝。”

小五子跪下来,叩了三个头,道:“我给父皇换一次药。”

小五子将纱布一圈圈打开,三四年没换过药,纱布脏得不成样子,凝固的血污结成了黑色的硬块,小五子皱紧了眉头:“这纱布都多久没打开了?”文思清说:“据说是当年那个太医说的,嘉和皇帝醒来之前,纱布不得打开,以免动了真气。”

小五子反倒笑了:“这是哪来的太医,莫非跟我这个太子一样也是假冒的吗?”

纱布一圈圈打开,最后掉出来一块羊皮。小五子转身问道:“这个太医到底是什么人?他怎么会有一张九宫图。”

小五子把羊皮收好,给嘉和皇帝缠上一层新纱布,问文思清:“我当时为什么会来找嘉和皇帝?这可是死罪,就算我再怎么主动请罪,他也不会放我出去的。”

文思清说你着了魔了,就为了一个女人,或者死,或者得到嘉和皇帝的特赦,与她私奔。

小五子摇头不信,说:“我是太子,想和那个女人在一起,我父皇管不了那么多,你别拿这个骗我!”

“你是娶天下女人都行,唯独这个女人不可以。”文思清跟小五子要那块红布,展开了对小五子说,“你拼死拼活,要么死,要么和她在一起的那个女人,就是我在上面没有画叉的那个人,就是五公主。”

小五子要慢慢捋一下,和公主这一年多的种种过往:公主第一次见到吴思若时的那个表情和恨不得挖了她心肝的嫉妒心;公主去昆仑山庄那么熟悉房间的构造;公主在与他分别时,难舍难分的样子。而他呢,却把她嫁给了窝囊废李准驸,去南海打仗的时候把她派到大路当先锋,希望她去送死。

小五子在房间里连走了几圈,外面有太监通报,圣上今天可不要早朝了,三王爷带着重兵将皇宫围了一圈,怎么看都是来者不善。小五子笑道:“去,肯定去,我小五子这辈子半点武功不会,也绝没在任何高手面前过,何况现在做了皇帝,还怕他一个三王爷?”
\newline

早朝,文武百官进殿,除了城外的大军,小五子发现今天进殿的也多了几个人,他看到六公子带着太医跟在三王爷的后面,三王爷手上还牵着一个女孩。小五子先问六公子,最近和乔姑娘可好?转而问太医,离开宫中也有四五年了吧,是不是很挂念嘉和皇帝的伤势?上次你替嘉和皇帝包扎后,我们还都没敢动呢。

太医结结巴巴说:“臣医术低微,只是怕其他人不小心伤了龙体,没有特别的意思。”

小五子转而问三王爷:“你这么一大把年纪了,还贵为王爷,这么早起来上朝,是为了何事啊?还是在家里遛遛鸟,逗逗蛐蛐,哄哄孙子孙女,才是天伦之乐呢。”说着他朝小女孩努努嘴:“三皇叔,那是你的孙女吧,那我得叫她一声侄女。”

六公子接话道:“侄女倒不用叫了,你喊她一声女儿才对。”

小五子坐直了,大喝一声:“放肆!”

文武百官们连忙劝道:“陛下息怒。”

六公子道:“这的的确确是陛下您的女儿,难道陛下自己都不认得了吗?这是属下在汴梁的一处农户家里找到的,这个女孩小名为甜甜,是五年前陛下与一位女子在昆仑山庄所生,当时陛下您还是太子,怕嘉和皇帝知道此事,只好将她寄养在那户农家。“

三王爷像排练话剧一般,转身问六公子:“那孩子的母亲是谁呢?甜甜已经四岁了,她不知道自己的身世吗?”

有人问甜甜,小女孩回答道:“我娘是苗翠花,我爹是刘大柱。不是,不是上面的那个皇帝。”

六公子继续道:“那就对了,陛下您当时和那个女子知道事态严重,给了些银子,让苗翠花和刘大柱两位夫妇,代为抚养,还约定了必须瞒着甜甜她的亲生爹娘究竟是谁。”

“好啊,你说父亲是我,”小五子道,“我相信全天下也都知道,朕四年前中过断魂掌,当然南海真人这个老贼,已经被朕除掉。但是这几年,朕发现一件怪事,朕做过的不记得的事情,有人找我,但是朕没有做过的事情,还是有人找到我,或者捞点银子,或者强塞我一个儿子。你们都别笑,朕经历过,六公子相信你更清楚,曾经的小太子闹闹不出两个月,就被你夫人乔文君给要了回去,你这次又要强塞我一个女儿?”

众人一阵哄笑。六公子恭敬道:“这个女孩确确实实是皇上的骨肉,微臣辛苦找到,不图有功,但求无过,好弥补了上次的过错。”

“好啊,那你讲出来,孩子的母亲是谁,她娘在不在我后宫,在的话就让她娘把她带走,朕也要重重赏你。”

“臣不敢讲,怕陛下怪我妖言惑众,当场斩了我。”

“你尽管讲,文武百官都在,”小五子说,“只要你说的有理有据,孩子的母亲也认,朕有什么好杀的,朕怎么可能杀你?”

六公子道:“孩子的母亲是······”

此时太监喊道:“五公主驾到!”
\newline

公主缓步走上大殿,经过孩子身边时似乎顿了顿,却始终昂着头正视前方,一点也没往别处瞟。跟陛下请过安,小五子让她坐到自己边上来。

五公主道:“听说三王爷又把兵带来宫外救驾,三王爷的这番好意,打我替父皇代理朝政的时候就已经领过几次了。皇兄就让我在旁边听着,或许我有些经验可以教给你。”

“我正在问六公子那孩子是谁的呢,那皇妹也一起来听听,你见过这个孩子吗?听说叫甜甜。”

五公主盯着孩子看了半天,才吐出三个字:“不认识。”

六公子轻蔑一笑,转身问甜甜:“她不认得你,你可认得她?”

五公主悄悄对她使眼色。甜甜先说了个“认”字,又改口说不认识。小五子问:“刚才不是让你讲出来,孩子是谁的,你现在让孩子指认五公主,做什么?就算是我的孩子,五公主是我亲妹妹,也要叫她一声侄女,你这是唱的哪出戏?你要是再讲下去,我可真要怪你妖言惑众,当场斩了你!”

六公子赔着笑,说:“微臣可能真的是弄错了,山野村夫的孩子,被人抱过来,冒充陛下的骨肉,我太轻信那些小人了,微臣罪该万死。”六公子说着拽住孩子的手腕,“噗通”一下跪了下去。

小五子虽然武功粗浅,但也知道六公子这一下是对孩子使了内力。最心痛的是五公主,她正要起身跑过去,被小五子拉住手,低声讲:“先忍一下,我大概知道了。”

小五子对六公子道:“平身吧,朕赦你无罪!”

六公子拉着孩子站了起来。小五子再看那孩子,浑身颤抖不已,显然经受着难以忍受的疼痛,却硬是忍住没有喊过一声。五公主在他耳边几乎是哭着低声说,“我们的女儿,救她。”

小五子点头道:“我知道,这孩子太像你了。”又对六公子说,“我看你也是护主心切才犯下此错,朕不怪你。将这孩子留在殿上,你退下吧。”

六公子眨眨眼睛,问:“陛下真的不怪我?”

小五子咬着牙道:“君无戏言。”

六公子哈哈大笑,说:“几日前,有奸人将这孩子送到我府上,说是陛下你的骨肉,从我这骗走不少银两。依我看,这孩子和那些奸人是一伙的,陛下宅心仁厚,就让我来替陛下动手,以儆效尤。”说完一掌朝着甜甜的天灵盖拍下去。

小五子没想到六公子真敢在大殿上痛下杀手,眼睁睁看着孩子小小的身躯软了下去。五公主疯了似的扑上去,哭着说:“娘对不起你!”

六公子在旁边讥讽道:“五公主可不要胡言乱语,您冰清玉洁,贵为公主,怎么可能生下这山野孩子。”

五公主要与六公子拼命,无奈武功相差太远,回头对小五子哭道:“沈辟朝,他杀了我们的女儿!是你不让我救她的!”

小五子也坐不住了,站起来大吼一声:“侍卫听令!”

大殿里的侍卫听到小五子的命令,都齐刷刷向前迈一步,“噌”的一下拔刀出鞘。

小五子又喊:“给我将西北六公子拿下!”

众侍卫一拥而上,却是将五公主按住,拱卫在六公子身边。

小五子气得大喊:“李准附!李准附!”

三王爷气定神闲地摸摸下巴,道:“别喊了,早就全换成我们的人啦。”

六公子问小五子:“陛下,我只问你,你是姓沈,还是姓孙?前朝遗腹子沈志基又是你什么人?他挥刀自宫,混进皇宫内,做了二十年的太监,相信在场的文武百官都与此人略有交情,他就是常公公!而你,就是沈志基,常公公的儿子,沈辟朝。”说着他转向惊疑未定的文武百官:“我相信各位臣工还一时无法想明白这其中的蹊跷,一个前朝的太子跑到我们孙家的朝廷,潜伏了二十年,并且将自己的儿子送进宫当了太子,我今天带了一个人,以前的大内御医,可以让他给大家讲讲,这二十年来都发生了什么事情。”
\newline

{\centering\subsection{2}}

二十多年前,太医还相对年轻,也没几个人叫他太医,好一点的叫他医生,遇见不礼貌的,就叫他“喂,治病的。”有一天他好好走在路上,忽然被人拿布袋套住头,拐到一个偏僻的小黑屋。头罩取下来,眼前人是沈志
基,也就是后来的常公公。沈志基跟他说:“我知道皇上在这边逍遥快活完了,留下一个孩子叫孙天奇。如今为了弥补他的愧疚之情,却把你留下来照顾他们母子俩。他没带你回宫,你是不是很失落?我给你指条明路。”

太医问他要怎么做。沈志基继续道:“将这孙天奇杀了,我给你一个男孩换上,依然住在你的府里。还有,下药毒死那个余姓女子,写折子称她暴毙。你若配合我,二十年内保你荣华富贵,为本朝第一太医。”

太医没敢问如果不配合会怎么样,他看着沈志基的表情,心里明白,脸都露给自己了,达不到目的的话,恐怕自己是别想活着走出小黑屋了。

那只好干了,太医将男婴掉包,毒死余姓女子,回去向沈志基覆命。手里沾上两条人命,沈志基对太医也放了心,他吩咐太医,以后再见到我,要称我为常公公。开局不错,接下来我要去宫里跟他们磨上二十年。
\newline

小五子一边听太医说着,眼前好像冒出一个人,年轻人的身材,却长着一张常公公的中年人的脸。这人站在太监招募处门前,纠结了许久,一咬牙一跺脚,真的把自己的半辈子搭进去当了太监。

在宫里稍微站住脚,沈志基以常公公的身份对皇上说:“太医这些年照顾皇子孙天奇有功,可否调回宫中?”

嘉和皇帝一拍脑门,说:“朕差点都把他给忘了,赶快招他回来,朕要重用。记着,将皇子安排好,不许带回宫中。”

每当有贵妃临产,常公公就带着太医,去做诊断。太医对贵妃讲:“恭喜,是一个公主。”出门后再对常公公说:“找一个出生不到十日的女婴,这胎是皇子。”

这么多年不断的报喜声音在嘉和皇帝的记忆中都是一样的:“恭喜陛下又添了一位公主。”嘉和皇帝与常公公述说苦闷,说自己尚无子嗣,三弟又对他的皇位觊觎已久,该如何是好?常公公提醒:“陛下在太原不是还有一位孙天奇吗?”

皇上又是拍拍脑门道:“瞧我这记性,睡死得了!”

常公公于是去了太原,在赌场待了有两个时辰,看着小五子输个精光。他说:“公子要是还没有尽兴的话,我这还有十两银子,拿给公子耍耍。”

小五子面对这么善心的陌生人,贫嘴说道:“我跟你说啊,真输光了我也还不起,你也别惦记着拉我去皇宫里当太监。”

“这十两银子赢了尽管拿走,输了我一分不要,只是
买你两个时辰,听我给你讲几句话。”

小五子伸手说:“你再给我十两银子,我听你讲四个时辰。”

十两银子输光,小五子听他讲了四个时辰。天快亮时,小五子坐在窗前一时缓不过神。常公公道:“这由不得你做决定,我们忍了二十多年,就等这一天了,走吧,跟我上路,去京城做太子。”
\newline

{\centering\subsection{3}}

故事讲完,三王爷下令拿下本朝第一逆贼沈辟朝、皇后吴思若、皇妃文思清、太医,以及五公主统统拿下。三王爷宣布,将文思清打入地牢,要细细审问谁是真公主,谁是假公主。沈辟朝、吴思若、五公主推出午门立即斩首。六公子建议道:“太医坦白有功,先不要动大刑,五公主很有可能是嘉和皇帝唯一的子嗣,是你的侄女,可否留她一命,先押入地牢。”

“也好,免得天下说我六亲不认。但是我的登基大典今日就要办。”

上次三王爷登基的时候,念的是嘉和皇帝这么多年的政绩,这次三王爷取代的是小五子的位置,文官改读小五子的劣迹。三王爷已没有了竞争对手,想慢慢享用这一刻,找出昨夜备好的长篇累牍,让文官一个字一个字地读,读它一天一夜才好。

而此时,小五子和吴思若被押上囚车,奔向午门,小五子对吴思若愧疚道:“当太子妃时让你死了一回,当上皇后又要让你死一回,朕对不住你啊。”

吴思若笑道:“你还朕朕朕的呢,你要不要点脸?你早知道你是个假皇上,你逼我成什么亲啊?为你死两回,好像我有多爱你似的,去年你还骗我,说慢慢培养感情,这才一年就要死了,有点快了吧?”

到了午门,刽子手让小五子和吴思若并排跪在断头台前。吴思若忽然说:“这一幕我见过,我当时说的好像是要一个镜子。”

小五子问:“你以为化化妆,漂亮一点,阎王爷能让你投个好人家?”

吴思若若有所思,说:“不是,我记得好像是要照镜子,说要看着自己死。”

小五子叹道:“上次是公主的手下小顺子杀你,还能让你死得痛快,这次他们就不会让你死得那么顺心了。”

吴思若道:“你真行,跑过来装太子,还把人公主给
勾到手了。你这辈子还能不能干成点儿正事?”

小五子道:“你觉着我不行,有一个人对我佩服得可是五体投地,那就是你师弟一只手。我去哪儿他去哪儿,我进宫他也进宫,我赌大他也赌大,就连我去丐帮要饭,他也跟着。”

吴思若又想了一阵,问:“碗呢?”

“什么碗?你还没死呢,你再坚持着清醒一小会儿行不行?”

“我醒着呢,五帮主,我碗呢?现在丐帮可是我吴思若做主!”

小五子眼泪哗就下来了,问吴思若:“你回来了?我是小五子呀!”

“我知道,你哭什么呀?哎?谁把咱俩绑这儿的啊?你又干什么坏事儿连累着我了?”

小五子望着她,刽子手的刀向吴思若脖子挥去,小五子“哇”的一声痛哭出来。
\newline

三王爷趁文官宣读小五子恶行这一阵儿,慢慢换上龙袍,自言自语道:“那个逆贼比我瘦,回头得把咱府上早做好的那一套拿过来。”

几个亲信跪下道:“王爷英明,王爷有先见之明。”

三王爷冲他们一人踹了一脚,道:“王爷?谁他妈是王爷呢?!”

几个人修正道:“陛下英明,陛下有先见之明。”

龙袍穿好后,三王爷也等得不耐烦了,让文官别读了。文官总结道:“于是我们将本朝最大的逆贼绳之以法,午门问斩。”

传令官宣布,新皇登基。文武百官跪地,三呼:“万岁万岁万万岁!”

三王爷特意等了十几秒,伸展双臂道:“平身。”众人未起,三王爷再说一次:“众爱卿平身。”文武百官还是跪地不起。三王爷感觉事情有些不对,只听背后一个声音道:“众爱卿平身。”

三王爷回过头,皱眉道:“你怎么坐在这儿?”

坐在九龙宝座上的竟然是西北六公子。
\newline

眼看刽子手的刀就要落下来,小五子闭上眼,等着脑袋落地。只听耳边一声脆响,脑袋没事,刽子手的刀却飞了。睁眼一看,地上躺着一颗骰子,刽子手的刀竟然是被它震飞的。“冲啊!”一只手带着几个人冲上来,他还特意去看了眼落地的骰子,欢呼道:“果然是六!这宝
刀我要了。”说着一掌将刽子手劈死,夺下宝刀。

蓬莱阁老对一只手吼:“骰子是我打的,你有个屁本事,我是来救我女儿的!”

一只手纳闷了,那我的骰子去哪了呢?

另外一个刽子手也被百花谷主干掉,她身后跟着大漠仙人,笑嘻嘻的,却不动手,看着众人给小五子和吴思若松绑。

阁老骂道:“我就知道你不安好心,一路上都跟着我。”

仙人道:“你这一年都没离开过皇宫三里,天天在树上窜来窜去地看你女儿,我还以为你当了国丈,跟着你能让我这个师弟也享受点荣华富贵呢,没想到碰上的全是这种苦差事啊。”

官兵围了上来,向问和吩咐几个丐帮弟子保护好小五子,小五子问他:“那个瞎子关长老呢?”

向问和回答:“这个逆贼不知怎么回事,看出来我的无为神掌真的是无所作为,要把我这个前任帮主给废掉,我只好带着几个弟子跑出来了。”

“我不是告诉你那八个字吗,怎么露馅了呢?”

“我记着呢,就是关长老逼着我,让我露两手给兄弟看看,我连个蚂蚁都拍不死,实在是撑不住了。”

小五子也骂:“我知道这货,就知道让帮主露两手。”
\newline

来的人虽然都是高手,但是官兵人多势众,缠斗半个时辰,众人才得以脱身,大家跑出去十几里路,进入一片荒林稍作休息。阁老气喘吁吁,瘫坐地上。向问和嘲笑道:“你这蓬莱掌,练就出来之后,和我这无为掌内力差得太多了,你看我一招都还没发,仍然泰然自若,怎么把你累成这样?”

蓬莱阁老站起来,神情严肃,对众人道:“容我和我女儿讲两句话。”

阁老走过去,对吴思若说:“半年没见到,爹年纪也大了,可能见你的机会越来越少了,让爹再好好看看你,看着女儿的脸。”

阁老心中五味杂陈,忍不住把她抱在怀里,这时候听见吴思若说:“蓬莱阁老,你不是说,再见到我一次就杀了我吗,现在倒把这些都忘了?”

阁老一惊,后退两步重新打量她,自语道:“大师兄故意害我。”

吴思若道:“阁老,您还是好自为之吧。”

阁老老泪纵横,道:“爹真没几天活头了,你就原谅
我吧,过去所有的事,都是我的错,你一点错也都没有,永远不要怪你自己。爹欠你的太多太多了。”

“蓬莱阁老,你帮我办一件事情,事成之后,你便不再欠我的。”

阁老怕忘记,咬破手指,撕下一块布:“你说,爹记下来。”

吴思若道:“离你蓬莱阁向北不到十里处,有一个大坑,每天有一个聋哑老人,负责给坑里面的人送饭,帮我把他们全杀了,我认你作父亲。”

阁老在布上写下几个字:蓬莱,北,十里,坑,杀。

一只手看得稀奇,对身边的小五子道:“年纪一大忘性那么厉害?就这么几句话还得拿布记下来?”

阁老听到了,转过头,指着小五子道,“你······照顾我······照顾······照顾好她。”

向问和奇怪道:“师哥你怎么了?”

阁老自言自语道:“谁?师哥?”然后猛然一醒,指着所有人,问:“你们刚才谁对我下了断魂掌、仙人掌两掌?”

向问和也明白过来,大喊道:“三师哥!”

还未等众人反应过来,大漠仙人一掌震飞两个挡路的丐帮弟子,飞奔而走。

只有蓬莱阁老仿佛又坠入梦中:“三师哥又是谁?”他看看布上的字,看看日头,独自向南而去。
\newline

六公子坐在龙椅上,让人给三王爷赐坐,请出太医。三王爷指着六公子的鼻子骂:“好你个叛徒,凭你也敢背着我觊觎大位!”六公子对三王爷笑道:“刚才太医的故事讲得不全,还有那么一点没来得及说。所以三王爷,你先不要急嘛,等太医讲完了,我们再慢慢商议。”
\newline

太医把故事接着往下讲,又回到二十多年前,太医将常公公指定要杀的孩子放进摇篮,对着摇篮凝思了许久,又把孩子在怀里抱紧,冲进雨中。

他去西北教找教主,六公子后来的养父。太医把宫中的铭牌给教主看,跟他讲:“我是宫中来的太医,其他种种不便多说,这个孩子请你当他是亲生儿子般将他养大,你只要照办,当今圣上必定保你一生荣华富贵。”

教主问:“难道他是皇子?”

太医道:“我不能说,你也莫要多问,如果走漏了消息,不要说荣华富贵,就是你们西北教只怕也从此在武林中消失得无影无踪。”

教主点点头:“老夫明白了,太医放心。”

倏忽二十年,这个皇子已成为西北教的第六个公子。他和五个哥哥,远行离开西北,投奔三王爷。一日骑马射猎,六公子百发百中,回来的路上几位哥哥抱怨,父亲对你如此偏爱,一身的武艺悉数教给你。一辆马车在他们旁边停下,车中人掀开车帘问道:“阁下可是六公子?在下是宫中太医,有些事情要和六公子商量,可否上车同行?”
\newline

太医说不下去了,看见三王爷一直拿眼睛瞪他,愈发结结巴巴的。三王爷咬牙问六公子:“太医都跟你说了些什么?”

“我,才是真正的太子孙天奇!”六公子朗声道,殿上众臣一时议论纷纷,“不过常公公把持内宫已久,太医不敢向父皇说出实情,他叫我唯有暂时拉拢你三王爷,为我铺平登基之路。”

“不过,”六公子接着往下说,“虽然我登基之路上你也出了不少力,但你私制龙袍,密谋篡位这事,咱们现在是不是也得来算一算呀?”

三王爷骂道:“沈辟朝一个外姓人,登基之后也没有杀我,你一个孙家的太子,登基之后反而要赶尽杀绝!”

一个胖嘟嘟的御史出来说话,三王爷虽然罪不可恕,但毕竟是皇族血脉,杀之不祥,恐民心生变。六公子点点头,吩咐左右将御史拖出去斩了。至于三王爷嘛,也不多追究了,把兵权交出来,再买些上好的金丝雀和京城最好斗的蛐蛐送给皇叔,让他颐享天伦之乐,不要再关心朝政了。
\newline

{\centering\subsection{4}}

小五子一行人无处可躲,本想回昆仑山庄,半道上看见通缉的告示贴了一路,昆仑山庄也被大批官兵层层守住,幸好那通缉告示大概也是以前小五子当逃犯时同一人画的,根本不像,路上百姓才没认出他们来。但昆仑山庄是去不成了,只好在附近的农庄暂时落脚。小五子想去牢中救五公主和文思清,其他人都劝他,倘若几大高手都在,也许还有希望,现在比如向老前辈,除了无为什么都没有,我们去了地牢只是送死。

唯一的好消息是吴思若真的恢复记忆了,看来南海真人那时候也没忍心真的全力下手。只是吴思若虽然心里想和小五子在一起,但又总是有些自卑,她对小五子说:“你别靠近我。我配不上你对我好。”小五子极力
开导她,前段时间皇后你都当了,现在让你做个农妇怎么反倒不好意思了?吴思若沉默不语,过了一会儿她轻声自语道:“也不知道我爹爹事办得怎么样了。”
\newline

阁老站在坑前发呆,下面的几十个男人一个个伸着手等着发食物。阁老看着手中布上的血字,想不明白自己来这干吗。大漠仙人站在他旁边,说:“二师兄,你来这儿几天了?”

阁老一脸茫然看着他,问:“二师兄?你是谁啊?”

仙人没回答,接着问他:“你在这儿干吗呢?”

阁老拿出手上的布给他看,说:“这是我的字,我站在这儿已经十天了,都没想明白,我为什么要杀死他们?”

仙人道:“你十天没吃没喝吗?”

“是啊,怎么一点都不饿呢?”

“因为你中了一掌仙人掌,中了一掌断魂掌,仙人掌我倒是很精通,至于断魂掌,小弟可就不懂了。”

阁老很感激,说:“谢谢,你分析得很有道理。那我再问你,我为什么要杀这些人?我替谁在办事?”

仙人又一通大笑,道:“你在替你女儿办事,你有个女儿,小的时候被我偷走了,我把她养大,养得可漂亮了,后来我把她卖到紫竹院,她花名在外,全杭州的人都知道你女儿。而这些人呢,都是以前点过你女儿的,都是你女儿的常客,我后来想想,不能让他们把这个事传出去啊,毁了你女儿的名声,我就帮你把他们全都抓到这来了。”

阁老似乎回光返照,转过身来问他:“我女儿是吴思若?你要毁她一辈子?”说着向大漠仙人一掌打过去。

仙人轻轻松松躲开这一掌,笑道:“你清醒了,说明你就要完了,再妄自催动内力,只会死得更快。我劝你还是看看你那个布上写的是什么,先把要办的事办了吧。”

阁老又看看布,大喊着:“吴思若,爹替你报仇了。”

说罢,他跳到坑里的人群中,对着每个人,连拍几掌。随着他身上的伤势越来越重,掌力也越来越弱,那些人逐渐围住他,将他淹没。

大漠仙人一通大笑,用脚将旁边的土石拨到坑里,后来干脆拿着铁铲,将土包上的土铲下来,一锹锹地抛下去,连同阁老一起,将这些人活埋了。

接下来大漠仙人在土堆边立了根小棍,上面刻几个字:蓬莱阁老之位,想了想又弄了一个小棍,插在旁边刻
上“南海真人之位”,他跪下来给他们俩一人磕一个头,说:“不管怎么讲,你们俩一个大师兄,一个二师兄,咱们三个斗了三四十年,一晃你们俩都没了,我留在世上也没什么意思,说来也好玩,有人一生享荣华富贵,有人一生享纸醉金迷,有人一生享声色犬马,我大漠仙人一生不图这些,我就想看你俩过得不好。结果你俩说没就没了,那我活着都没什么乐趣了。”

然后他又找了根木棍,上面刻了大漠仙人之位,插在他们俩后面,跪地道:“我大漠仙人也想死,但是你俩死了再也没人打得过我,我胆子又小,又不敢自杀,弄个小棍,陪你们俩得了。”站起来之后又不忘回头补一句:“别老来找我,我长年不在家。”

刚一转回来,有人在他胸口拍了一掌,大漠仙人目瞪口呆,“不可能是你,不可能!”他还想说话,可张大了嘴却发不出声音,两眼一黑倒下来,最后的时刻他还在想:糟糕,这下要把牌位压倒了。
\newline

小五子那日正在午睡,吴思若披头散发地闯了进来,坐下来开始大哭。小五子起身问她怎么了,她也不回答。小五子端了一杯茶给她,吴思若手一挥把茶杯打翻在地。她只是哭,哭到上气不接下气的时候,她抽泣说:“我梦见我爹死了。”

小五子这才闹明白,他花了一下午,总算把吴思若哄好了。最后嗔怪她说,做梦而已,干嘛\footnote{原文“干吗”}当成真的。
\newline

{\centering\subsection{5}}

六公子在皇宫待了几天,突然想起来一件事,他叫来太医,问他:“我父皇当年留给我母亲的那张九宫图,你一直说在你身上,现在该给我了吧?”

太医道:“我没什么武功,生怕被人家抓住搜出来,我藏在了一个特别稳妥的地方,你随我来。”

进了寝宫,把嘉和皇帝扶起来斜倚着床头,小心翼翼地把老皇帝头上的纱布打开,太医动作慢,六公子就在一边眼巴巴地看着。看了半天发现里面什么都没有。

“图呢?”

“我明明把它包在纱布里的呀?”太医解释不清,然后仔细看着那块纱布,惊道:“有人换药了,把这块羊皮取走了。”

“换药了?换药了·.....”六公子想了想,问太医,“我娘当年是被你杀的吧?”

太医慌张下跪,那真是身不由己,我已经尽力保陛下到今天了。六公子叹息道:“我也是身不由己啊,杀母之仇怎能不报?”

隔天,六公子将太医处死。又叫人把乔姑娘和闹闹接到宫中。六公子对乔姑娘说:“你一直问我要办什么大事,我什么时候才能娶你,你现在也看到了我办的大事是什么了,你就等着当你的皇后吧。”

乔姑娘这一次却不想嫁给他了,她说:“你当你的皇帝,为什么要杀我爹?”

六公子盯了她许久,最后扔了一句话:“你不做皇后也可以,我追封你做皇后。”
\newline

文思清在牢里,有天被两个狱卒的抱怨声吵醒:“你一个和尚跑到大街上奸淫妇女,还要不要点脸?”

文思清冲到门口一看,两个狱卒架着一个和尚正往里走,一看正是八光,说:“师弟,你又犯淫戒了?”

八光叹着气,摇摇头,被狱卒扔进另一间牢房。

本来文思清还在想,下了大牢不是说要严刑拷问吗?怎么好几天了都没人理她?就听到牢口铁门的转动声。过了一会儿,六公子走进来,看看文思清,问她羊皮是不是在小五子那里,还对她说:“我知道你俩没感情,他杀了你全家,你搅得他在宫中坐立不安,也就是杀了我十几个姐妹,你把羊皮的消息告诉我,你这样的人我要重用,不会让你死。”

文思清跟他绕了一圈,知道他要的是皇上脑袋里的那张羊皮,骗他说:“三王爷有一次来看皇上,带着一大捆纱布在里面待了半天,你去问问他吧,嘉和皇帝总不会是自己把那块羊皮消化吸收了吧?”

六公子去找三王爷。三王爷那是有苦说不出,他说:“你看我这天天忙着养鸟、养蛐蛐呢,哪敢搞什么羊皮呀?”

“给你三个时辰,一会儿送到皇宫,不然你也知道,你侄儿是什么性格。”

“你杀了我得了,我真没有,你瞧我这点本事,养几个家丁,还被你造了反,我院里还有一套龙袍,估计这辈子也穿不上了,就送给陛下了,那个能比羊皮值点钱吧?”

六公子冷冷地看着他,不说话,三王爷试了各种办法,他跪下来献龙袍,六公子不接,他让人当场把龙袍烧了,他让人把鸟和蛐蛐退还给六公子,六公子不要,三王爷“噗通”一跪,哀求道:“你还是杀了我吧!”

六公子拂袖离开。回去见文思清,盯着文思清看了半天,一句话不说,文思清不理他。六公子站起来又准备走了。狱卒问:“陛下今天还要动刑吗?“

“今天不用了,给她三个时辰,好吃好喝供着她,要是她还是什么都不讲,也别推到午门了,就在这儿斩了。“

狱卒得令,三个时辰一过,打开牢门,他要进去动手了。文思清手镣脚镣全都铐住,动弹不得。进去的刽子手也不多废话,拔起刀就要往下砍。这时,突然从地里钻出一个人,一掌把刽子手推开。

文思清定睛看了看那人,欣喜地喊:“师弟!”

八光一边跟冲过来的狱卒兵丁打成一团,一边喊:“别谢我,要谢就谢小五子挖的洞吧!”

狱卒们不是八光的对手,脑筋却鬼得很,手里的家伙尽往无法动弹的文思清身上招呼。八光顾此失彼,急起来索性用身体挡在狱卒跟文思清之间,大喊一声,用尽力气把文思清身上的镣铐一一扯断。

文思清施展起沈老前辈教的功夫,不一会儿狱卒们死的死逃的逃,再看八光,身上被几柄武器贯穿了,躺在地上脸色发白。文思清背着八光逃出大牢,一边跑一边哭着说:“师弟,再忍忍,我这就把你抬回少林治伤。”

八光摇摇头说不必了,我永远也无法修炼成佛,就让我死在这吧。说着没一会儿,八光晕倒了。文思清上山摘些野菜,想了想摇醒问他:“你要是想吃荤的,咱们今天就破一次戒吧。”

八光摇摇头陷入昏迷,喃喃自语:“我找你半年了,听说你被关进地牢,我就想尽办法也要进来,我偷人家的钱财,人家见我是和尚,挥挥手让我滚蛋,就当是香火钱了。我去饭店吃白食,老板一见我是和尚,就说放他走吧,反正那些素菜也不值钱。但我得进来救你,没办法,拿出二十多年前最擅长的本事,找一个姑娘把她扑倒,其实我也没干什么,那姑娘尖着嗓子喊,救命啊,强奸啊,那些当兵的一下子就全部来了,我二十多年前奸淫过那么多女子,都没被抓过,这一个还没碰她呢,就乖乖跟官兵走了,我走得比他们还急,就想看看你在牢里好不好,过得怎么样,有没有人欺负你。”

说完他又昏迷不醒。文思清在旁边吃着野菜,自怨自艾地说,苦死了。她越说,吃得越多,最后苦得她泪流不止。

\newpage