\section{拾壹}

{\centering\subsection{1}}

吴思若建议下葬,小五子反问,往哪儿埋,你们要把苏子瑶埋在哪儿?八光问了一句,这姑娘是从哪里来的,哪怕是死了,也要送回那里。

“百花谷,”文思清说,“她是百花谷的人。”

是啊,小五子想,二月二从沉狮谷出发,说好的终点是百花谷的,你回不去,我就带你回去。从码头过去,文思清和八光和尚去打前站。小五子一路抱着苏子瑶的尸体,不上马,也不上车。吴思若和一只手就陪着他一路走,直到傍晚他才答应把苏子瑶抱上车,自己骑马缓缓跟在后面。走到谷口已是深夜。那时文思清二人已经通知了百花谷,苏子瑶的两个丫鬟如琴、如诗,正随着她和八光在谷口等候。

如琴好一些,上车见过尸体,确定是苏子瑶,咬着嘴唇忍住不哭,还不忘对小五子鞠个躬,对五帮主表示感谢。倒是如诗抱着苏子瑶哭个不停。大家在谷口停留片刻,骑马下谷。一行人也没点火把,就借着天上的星
光,七八匹坐骑一点点向谷底走去。夜空里,成片成片的萤火虫在他们身前身后飞舞,隐约还能听见如诗在车里抱着苏子瑶的尸体低声哭泣。

如琴、如诗和文思清、八光,为了接他们,走得比较远。如琴说,还要走一个多时辰才能到百花谷的谷口,而且谷口往下,还要走栈道,过栈桥,进洞穴。小五子骑在马上,不时回头看着车里面的如诗和苏子瑶。有那么一两次,他眼神刚好撞到了吴思若和文思清。两个人反应不同,吴思若是轻轻摇头,文思清是一直望着他,直到小五子眼神躲开。他明白她们的意思,吴思若摇头,是要他别太难过,而文思清望着他则是,但凡你需要我,我始终在这里。

谁也不需要,之前三选一,现在一拖二,小五子反省自己到底怎么了,前脚还在逃亡逃命,活得跟狗一样,后脚就以为自己将享齐人之福,一妻一妾,一妻两妾?他看着面前的萤火虫,你看得到,星星点点,可伸手去抓,却什么也抓不到,甚至连那一点点的光都不见了。

行过快两个时辰,山路已被堵住,一块巨大的岩石
挡在前方,这就是所谓的谷口吧?岩石下面留了一条缝隙,人要侧身才能过得去。七八个人下马弃车,如诗最后一个下来,把苏子瑶的尸体抱下来,搂在怀里。如琴对如诗低声叮咛几句,小五子听不到,但能猜到是鼓励她振作起来,不管怎么说,都要把苏子瑶的全尸平安送到谷里。如琴大声问她听明白了没有。如诗擦干眼泪,点了点头。她要如琴帮忙,将苏子瑶的尸体绑在自己后背上,和其他人一起,从岩石下面的缝隙钻过去。

岩石后面是另一番景象,即使天还未亮,小五子也能看到,一层层白气从下面升上来。低头看过去,白气一直过膝,已看不到脚下的路面。如琴嘱咐大家,先不要动,这白气是从谷底升上来,其实现在我们就踩在悬崖边上,如果乱走的话,说不上哪一步,就从悬崖掉下去了。

每个人站住不动,等如琴的指示。如诗反倒朝东北边走去,很明显是一个上坡,十米二十米的样子,伸手向上抓,这时小五子等人才注意到,那里上面有两根绳索,通往雾气缭绕的谷底。如诗抓起头顶锁钩,后面背着苏
子瑶,起跳前对每个人说:“那我们一会儿百花谷见。”说完她跳起来,抓住锁钩,顺着绳索便向下滑去,片刻间消失在白雾之中。

不是一起走吗?小五子没明白。如琴说,不行的,我们还是要规规矩矩地,先从栈道下去。

小五子问她:“那得走多久?”

如琴看看天色说:“最多两个时辰,大家不熟路,慢一些,三个时辰也到了。”

“那如诗呢,她多久到?”

“她现在应该在谷底啦。”

“那为什么不走索道,是那个绳索太危险?”八光问她。

如琴笑了,说:“那个索道安全得很,别说各位身怀绝技,就一点功底也没有的人,只要牢牢抓住锁钩,闭上眼睛,心里数二十个数,再睁开眼时,两脚就已经踩到下面的地面了。”

“凭什么不让我们走,只有你们百花谷的人才能用?”

“倒也不是,只是各位第一次来百花谷,还是照规矩来好一点。”

“规矩?”八光嘿嘿笑着,“你这规矩邪门,比少林寺和尚的规矩还没道理。”

八光不愿听她的,径自朝东北坡的索道走去。脚下看不到路,走出十几步后,一步踩空掉下去,整个身子消失在白雾中。如琴急着要去救他,八光伸手一撑,脚在悬崖壁上一蹬,又跳了上来。如琴停住脚步,喊他赶紧回来。八光不理会,一路走到索道下。

“八光师父!”如琴在后面喊着他,“你若这么下去,接下来三个月可有罪可受了!”

“怎么?我这么下去,难道百花谷的人,还把我囚禁拷打不成?”八光单手抓着锁钩冲她笑道。之后他招呼文思清:“师姐,要不要一起走?数二十个数的事,总好过在山上走两三个时辰。”

文思清说:“我们客随主便,还是听如琴安排的好。”

倒是一只手跃跃欲试,举着他那剩下的一只手喊:“八光大哥,等等我!把我也捎上!”

吴思若低声警告一只手:“你信不信,你要是敢过去,我把你踢下去!”

一只手愣了一下,看着小五子,等他拿主意。小五子也冲他摇摇头。又扫兴,又折面子。一只手把那一只手放下,喊道:“八光大哥,我得留下来保护他们!”

八光又等了片刻,确定没人跟他一起走,说了句跟刚才如诗一样的话:“那我们一会儿百花谷见?”说完脚下一蹬,顺着索道滑下去,消失在白雾里。

如琴急坏了,八光人都不见了,还往下面望着,急得跺脚说:“但愿他命大,内力够深厚,不至于命丧谷底。”

小五子问她:“到底是什么情况,命都要搭里面?”文思清也担心起来,问道:“下面果真有什么危险吗?”

“一时也解释不清,”如琴转身说,“我们走着看吧。”

说是走着看,其实又看不到,即使天已破晓,众人还是没法透过白雾看到脚下的路。如琴要大家跟她走,大概走出几百米,脚下一颤一颤的,小五子明白,自己已经走在了栈道上。如琴说,右手边是山体,但不要摸,怕是有蛇贴在上面等待觅食。

“铁链在左手边,”如琴说,“尽量离山体远一点。”

小五子伸左手摸出去,在腰间位置摸到了那根链子。吴思若和文思清悄悄商议了一下,文思清在前,她在后,两人想把小五子夹中间保护他。走几步小五子明
白了,他停住不走。这段时间已经活得很丢脸了,还要被两个女人保护,倘若真从这里摔下去,刚好也给丢脸的人生画上一个羞耻的句号。他站着不动,要文思清前面先走。僵持了一阵,文思清和如琴只好先走几步,目测三十米的距离,前面的人来不及拉住他,小五子抓着铁链走起来。

一只手就难过了,他走在最后,前面是吴思若,左边是铁链,右边是趴着蛇的山体,可他缺失的刚好是左手,抓不着链子,只能右手摸山,给毒蛇送过去当早餐。倒着走可以捋着链子,他试了一会儿,那么窄的栈道,说不上哪脚踩空,下面还颤颤悠悠的,倒着走也许死得更快。

倒着走死得快,脑子转得也快,没走几步,他想到一个办法,转回身看着吴思若的背影,跟她商量:“我可以抓你衣服吗?”

“你要干吗?”吴思若头也不回地问。

“我右手闲着,抓你衣服,就好比铁链了。”

“我衣服不结实的,一扯就破。”

“我不扯你衣服。”

“那如果你掉下去呢?你能松手?”

“当然要抓着你衣服······”说着说着,一只手自己就明白了,他如果踩空掉下去,就算是抓住吴思若的衣服,也无非是手里多两个布条,再数着二十个数摔死,不如保师姐衣衫整齐,起码可以美美地进百花谷。

一阵风吹过,栈道有点晃,没处下手的一只手只能蹲下来摸着脚下的木板。吴思若回身扔过来一个红色衣带,让他缠在手里,她在这头攥着,万一有什么闪失,她能把他再拎上来。一只手心头一股感动,感动到还低头闻了闻带子的味道,连贯在一起就有点猥琐了。吴思若眉头一皱,拽了两下带子。一只手好不容易拿到的救命稻草,怎会让她再抽走?他紧紧攥在手里,快步跟上去。

第一条栈道的尽头是一个洞穴,鞋底是湿的,一层的积水。里面没那么大的白雾,水汽在洞里凝结成水珠附在岩壁上,时不时有水滴落在头顶,打在脸上,那些没有落在身上的水滴,就滴在积水上,时不时地“啪嗒”一声,就好像时间发出了声音。小五子仰头看上去,阳光已经上来了,哪怕是在洞里,也能看到光线将水珠照得剔透。

洞穴看不到头,越走里面越黑,踩着水塘,听着水声,小五子等人走了一炷香的功夫,直到深处漆黑一片,完全看不见。

“到头了。”如琴在黑暗中说。

几个人停下来,等她下面怎么安排。如琴蹲下来,在地上划拉一圈,估计东西早放在那里的。不一会儿,她找出一个火折子打火,但没有燃纸,也没有点蜡烛,只是借着一闪而过的光芒,拿出一个药瓶。如琴让大家把手伸出来摊开。

火折子不打,四周又是漆黑一片,如琴在每个人手心上放了一粒药丸,弹珠那么大,小五子放在鼻前闻了一下,无色无味。他问她什么东西,干吗用的。

“就这无色无味,还是谷主研究了十几年,才做到的呢。”如琴笑道,“这是虞美人的根发酵三个月,再捣烂而成,之前可是恶臭无比,后来是加了夹竹桃花和丁香花的花粉,才算是把这臭味遮掉。”

“那要我们做什么呢?”

“当然是吃咯。不然我打开这扇石门,怕你们会挺不过去。”

原来这里是有扇石门的。小五子把药丸放嘴里,没有水顺服,只能咬碎再咽。原来无色无味只是表面上的,药丸崩开的那一刻,哪怕是在嘴里,都能感受到那一股强烈的恶臭。小五子一阵反胃,忍住没吐出来。如琴蹲下来,拾起一个水瓢,在面前的池子打了两瓢水,递给他们,让他们就着水咽下去。小五子连喝了两大口,打了个嗝,责怪如琴,既然有水,为什么不先打给他们。“解药在里面,外面这层药衣很厚的,总要咬开了,才能咽下去。”如琴解释道,“不然一会儿中了毒,怕是毒死了的时候,这解药在胃里还没有化开呢。”

原来是解药,那又会因为什么中毒呢?所有人都强忍着把药吃下去,如琴开始开门。她接过大家递过来的水瓢,在面前的石板上敲了敲,左右滑动,刚好扣住一个螺旋,原来瓢底是有凹槽的。如琴转了两下,石门打开,面前一片水帘洞般的景象,一座栈桥斜着向下伸出去。不远处是瀑布,飞流直下,落到地上却一下子温和起来,汇成溪流从桥下穿过。路也安全许多,前面的栈桥宽阔而平坦,两侧都有绳索链着。

小五子踩上去走了几步,抬头向上看,上面雾蒙蒙的全都是水汽。他明白了,之前的白雾是这瀑布、溪流汇聚上去的。栈桥虽然长,但很好走,几乎小跑着就能过去。走到一半,他闻到一阵芬芳,那种淡淡的清香一丝丝地透过来。小五子忍不住地深吸了两口,问如琴:“这是什么花的味道?”

如琴冲他笑笑,还没有回答,只听见一只手在后面
赞叹“太好闻了,太香了”之类的。说完他就大口呼吸,隔着吴思若,小五子都能感觉后脖颈一股股的热气。喘着喘着,声音越来越急促,忽然“咣当”一下,倒在了后面,从栈桥上掉了下去。吴思若马上拉住带子,趁一只手失去知觉松手前,把他提了上来。如琴从队伍最前面穿到队尾,看着昏迷的一只手,摸了摸他的喉咙两侧,问道:“他没吃解药?”

小五子摇头,不是说没吃,是他也不知道。

如琴在一只手的身上拍了拍,从他衣服里找到那粒解药,用手捏碎,塞进他嘴里。“他把药藏起来了,假装自己吃过。”

“他中的是什么毒?”

“就是你闻的这个花香,虞美人的花,和水汽混在一起,就会产生剧毒。”

如琴说着,小五子还情不自禁地又闻了一下,他看着躺倒在栈桥上的一只手,问:“我们要等多久,他才能醒过来?”

“今天是醒不过来了,还好发现得及时,不然怕是要丧命于此。”

如琴说完将一只手背起来,又走到队伍最前面。这么香的花,却是剧毒,以前总觉得,百花谷这么好的名字,一定是一片芬芳,倒也没错,只是现在看起来,百花谷其实也可以叫百毒谷。小五子问:“你之前说的八光从索道下去,会死在这里,一定是因为没有这虞美人的花香的解药了?”

“八光师父本事那么大,这虞美人对他,应该算不了什么。只是谷里的百花,却需要闻这一路的虞美人,才可以解毒。”

小五子恍然大悟,点着头。

“但是说真的,他也不一定会死,谷主会救他,但总要遭几个月的罪的。”

栈桥走到尽头是一片高草,如琴等人从桥上下来,拨开高草走进去。原来高草是种出来的绿墙,走几米钻出去,面前是一片的花海。各种香气混在一起,仿佛一层一层地扑在脸上。小五子有意抬头看看,找到那两条绳索,他看到绳索一头的下方,八光正躺在地上昏迷不醒。文思清跑过去查看。小五子要跟过去,如琴一把拉住他,说:“少谷主先别走,沈大总管知道你来,早早地在前面守候,等着对你宣读谷主指令呢。”

“谁?”

小五子往前面看过去,只见花丛深处是一幢房子,
门口站着几个人,为首的是个五十岁上下的男人,双手背在身后站在大门前,一副对这边翘首以盼的样子。哈,百花谷的大总管,再熟悉不过了,在田独相处了三年,那是不让他偷吃肉的钱老板嘛,那是说话尖声尖气,只能装哑巴的常公公嘛。哦,原来他还有个身份,百花谷的沈大总管。
\newline

{\centering\subsection{2}}

小五子向钱老板走过去,本来想叙旧,毕竟有三年的情谊。如琴提醒他:“到了百花谷,五帮主别忘了行谷中之礼。”

“那要怎么行?”

如琴给他做示范,先朝钱老板走过去,双手合拳,长揖到地,说道:“卑职拜见沈大总管!”

小五子看着想笑,哪儿和哪儿啊,皇宫那一套怎么搬这儿来了?如琴起身后,一直给小五子递眼神。小五子犹豫要怎么做,以前田独卖肉的时候,也没见过他整这一套啊。他双手抱拳,身子却一直弯不下去。后来钱老板反倒等得不耐烦了,说道:“免礼,快快起身。”

小五子愣了一下,他什么都没干啊。以前钱老板是发不出声,嗓子是哑的,这回嗓子好使了,眼神怎么还不行了呢?虽说是免礼,毕竟当了他三年掌柜的,小五子还是喊了声:“钱老板。”

钱老板没接茬,看着小五子身后的吴思若和一只手,说你们一路过来,辛苦了。小五子以为他没听见,清清嗓子,大点声又说一遍:“钱老板!你好!”

这回听见了,他看了看小五子,生生地不接话,对随从人员一招手,后面人递过来一个黄色布袋,钱老板从里面抽出一个卷轴,大声呼喝着:“谷主令到!”

如琴提醒他:“这回得行礼了。”

“怎么行礼?”

“就当你接圣旨。”

“我又没接过圣旨!”

小五子话没说完,后面“扑通”一声,如琴先跪下了,额头点着地叩首。这都是干吗呀?小五子皱眉看着她,这帮人把百花谷当朝廷了吗?吴思若也识趣,知道留在这儿左右为难,招呼一只手去旁边转转。谷主令是给他小五子的,他还不能走,但他小五子可不跪,跪天跪地跪父母,一个百花谷主令有什么好跪的?

倒是钱老板最会解围,朗声道:“五帮主不必行此大
礼!”然后又低声叮嘱一句:“稍微鞠个躬,我要打开宣读了。”

那就恭敬一下吧,小五子身子不动,只是把头低下来,看起来默哀悼念一般。钱老板把卷轴打开,宣读道:“谷主有令,从即日起,丐帮帮主五帮主昆仑公子,官复原职,依旧为百花谷少谷主!钦此!”

小五子没听错,全都读完了,还加了句“钦此”。他抬头看见钱老板收起谷主令,反倒向他行起谷中之礼,长揖及地,后面的人也有模有样,排练好的一般,跟着钱老板行礼。别人没资格报名姓,只说:“属下拜见少谷主!”只有钱老板,真名终于说出来了,他说:“属下沈志基拜见少谷主!”

啊,原来你叫沈志基。小五子看他作揖鞠躬,有意等了一会儿,说:“免礼,快快起身!”

行礼过后,钱老板才像个人,忽然冲小五子笑道:“以后你是少谷主,官职大我三品,可不再是被我呼来喝去,给我卖肉的小五子了,每次见你,都要我来对你行礼才是。”

小五子眯眼看他,思索钱老板到底有几句实话,先装哑巴,再装太监,这次又是按品算的大总管,到底唱的是哪出啊?倒是有一件事,至少钱老板对他还不错,倘若要坑他害他宰了他,早几年在田独他就下手了。

总还是故知旧交,小五子拉着他到一边说话,他问钱老板:“分别那天,三王爷带着西北六公子和丐帮马帮主去钱记肉铺,我和文思清从田独跑出来,你又是如何脱身的?”

钱老板说,当时他已经被打伤了,不过他们也不是冲自己来的,见小五子跑了,就调头追出去。他在肉铺养了几天,是苏子瑶赶过来,把他接到百花谷。来南京的路上,苏子瑶对钱老板承认道,自从那年冬天,她在田独发现小五子,她并没有走,一直躲在离肉铺不远的房子里,隔三差五地就过去看看。

“就在附近守着,我这次进了狮吼帮,她也是这么把我救出来的。”说完两人沉默了一会儿,小五子说:“苏子瑶被南海真人杀了,你知道吧?”

钱老板点头道:“我也是昨日才得知此事。”

小五子审视着他,忍不住要抬杠,回答的都是什么啊,官话套话也太多了吧?你是昨日才得知,苏子瑶今天才死,早几天得知,苏子瑶还替我在扬州赌钱呢。

可是他说不出口,这不是个开玩笑的日子,这几天
一声叹息,小五子摇头说:“是我连累了她。”

钱老板没说话,小五子也讲不下去了,他想说苏子瑶不能白死,早晚要给她报仇。可他知道,这些话说出来,自己心里一百个没底,打不过南海真人的,到时自己死了,也就算了,怕是连累了文思清和吴思若也跟着丧命。两人一时没说话,已接近中午,太阳正当头顶,小五子低头,几乎看不到自己的影子。沉默有一阵,钱老板主动讲起苏子瑶:“有一年冬天,苏子瑶找你找到田独去了,认出是你,她要把你带走,被我拦住了,我其实当时绝不希望,让你和百花谷再有任何瓜葛的。”

“所以你之前就是百花谷的人?”钱老板点点头。

“跟现在一样,也是大总管?”

“对。”

“那我也是?我是少谷主?”

“是。”

“于是当我中了断魂掌之后,你把我弄到田独,藏起来了,结果你没想到,还是被苏子瑶找到了?”

钱老板没回答,但小五子知道他猜对了。攒了那么多,再见到故人,一连串的新问题想要问他,已经不再是田独最基本的那些,关于“我是谁”的问题了。新的问题更具体,比如,我为什么能当上百花谷的少谷主?昆仑公子到底是谁给我的身份,是你们百花谷给的吗?昆仑公子结了那么多仇家,但我一点武功都不会,我在昆仑山庄见过他们,每一个仇家都认识我,脸一样,连瞎了的人都说声音一样,那就肯定是我,我没武功,又面对面地伤了他们或他们师兄,我想了快一年,只有一种可能,有人在替我,也就是所谓的昆仑公子抓人、伤人,而我,无非就是露个面,告诉他们,这事是我昆仑公子干的,那么替我干这些的,十有八九就是你们百花谷的人。

“我过去跟你们这么干,到底是为什么?”

一连问了七八个,钱老板只是沉吟,一个问题都没回答,但小五子能看出来,这些他全都知情。小五子说:“你过去不讲也就算了,现在就像你说的,我官职比你大三品,现在我命令你讲出来,总行了吧?”

钱老板这时反而大笑起来,他说:“你官职比我大三品,可有比你更大的人,命令我不许讲出来。”

“是百花谷谷主吗?”

“谷主现在在闭关,这几天就会出关,你早晚会见到她老人家。”钱老板说,“等你见到谷主了,还是让谷主给你从头讲起吧。”钱老板说完,动身往大门里走,走到门
口转身看着他,招呼他进来,说:“别愣着了,去看有什么忙的,明天一大早,我们还要把苏子瑶下葬。”
\newline

{\centering\subsection{3}}

灵堂设在百花谷的西北角,门口种了一片的白玉兰,但苏子瑶的最后一夜,却不是在这里过的。如琴、如诗把她的尸体抬到她们俩的卧房,又是沐浴更衣,又是化妆梳头,一直折腾到天亮,才踏着白玉兰的芬芳,把她送回到灵堂。

那天小五子睡得早,没吃晚饭就上床入睡了。以至于次日天还未亮,他就醒过来了,他坐在床头发了一会儿呆,穿好衣服出门,想去灵堂看看。晨光中,他看见如琴、如诗抬着苏子瑶的尸体走进灵堂。她们也看到他了,几个人相互看着,都没有说话。他挥手让她们去忙,先不去打扰。他在外面等了一会儿,也不见二人出来,索性去别处转转。

百花谷不大,同样是谷底,它不像地处西北的沉狮谷那般,大开大合,四外一片苍茫。这里几乎所有的空地都利用上了,要么是种花种树,要么是盖凉亭、建回廊,小半个时辰就可以走完一圈,大概二十几间房,他尽量轻手轻脚,不出声。再回来的时候,灵堂的大门锁上了,他坐在门口的台阶上,看着连成片的白玉兰。莫名其妙地,他居然盯着一只蜜蜂,数起它到底在几朵花上面采过蜜。

下葬在两个时辰之后,除了谷主,百花谷的所有人都参加了。墓碑上还刻了字,估计是请人连夜赶出来的。也不一定,武功好的人,手指点石头,能跟毛笔写字一样轻松。墓碑上写着—苏妃子瑶之墓。小五子想,这可能是百花谷的规矩,有官有品,有叩拜礼,还有圣旨一般的谷主令,一切都是宫廷的序列,加一个“苏妃”也不算过。

抬棺,下葬,入土,最后将墓碑立上面。一切仪式完成后,小五子叫所有人先走,最后请文思清和吴思若也先离开。临走前,文思清对苏子瑶鞠了个躬,低声道:“苏姐姐,本来应该是杀我的。”

吴思若看她一眼,又看看小五子,说:“大家一起死了,总好过现在。”

是啊,一起死了,该有多好。人们离开之后,小五子终于可以在墓前,和苏子瑶单独待一会儿。他以为自己能说很多话,像那些话本故事里讲的那样,活人可以对
死人讲个不停。事到自己身上,一句话都讲不出来。他在墓前站了很久,最后还是说了一句话:“你放心,不可能让你白死,我今天把话放这儿,我小五子早晚替你报仇。”

当然,他心里明白,做不到,但说出来言之凿凿,掷地有声。他怕苏子瑶真能听见,他怕苏子瑶听出他心虚,死不瞑目。

其实心里还有好多话,说不清楚,总结出来就是难过,他难过的是,苏子瑶如此爱他,却这般下场,他更难过的是,苏子瑶如此爱他,自己却没办法爱她,哪怕只有一点点。

那么,他爱谁呢?文思清?吴思若?他不敢去想,但似乎心里早知道,应该是吴思若,大概是她,十有八九是她,百分之百,当然是她!为什么爱她呢,那文思清呢?不能想,没法面对。真像文思清说的,早点死了就好了,把这些难以启齿的秘密,就和他这条贱命一起,挖坑埋了吧。

不能厚此薄彼,也没任何谈情说爱的念头,之后小五子干脆两个人都不理会,偶然碰到也只是客客气气,毕恭毕敬。反倒百花谷的人,他认识了不少。谷中多为女子,而且都是宫女的打扮,为数不多几十位男人,说起话来,也都是太监的腔调。小五子有天想明白了,钱老板那个沈大总管,其实就是太监总管,这些男人应该是真太监。

沈大总管时不时会给他介绍谷中的情况,他先说百花谷的环境,从沈老前辈说起。他说,虽然以前沈老前辈将断魂掌、蓬莱掌、仙人掌,教给了三位师伯,但是百花谷谷主作为小师妹,还是学到了宫中的花卉培育技巧,并以这些花粉花香,制作了毒性成分不一,其解药只有我们百花谷才有的各种毒药。

沈大总管掏出一束植物,说:“这些都是以前宫中才有的奇异花卉,这是天竺曾经进贡的彼岸花,它的花粉含有剧毒,只要吸入,便从口舌开始生疮,直至全身,溃烂而死。”

说着说着,似乎担心小五子不信,他还深深吸了一大口,吐出舌头给小五子看看,什么事都没有,口舌没生疮,没溃烂。

“为什么说只有我们百花谷能解这一味毒呢?”沈大总管又要讲课了,“因为毒药源自于彼岸花,解药自然也要在彼岸花身上找,像它的根部,捣成泥,敷在生疮之处,可以愈合并抑制溃烂蔓延。”

他把根部揪下来,放嘴里嚼起来,咔哧咔哧地说:“倘若像我这样嚼,也相当于捣成泥了,而且我从小吃到大,早可以抵抗彼岸花香,再闻一大口,都没关系。”

钱老板果然又吸了一口,这口更大更长,仿佛在显摆内力,让小五子见识一下,他一口能吸多少气。然后他又吐出舌头,除了彼岸花的根在嘴里嚼烂,红彤彤一片,倒确实没疮。真羡慕你,能抵抗剧毒之物,但不管怎么说,你鸡鸡还是被切了。

小五子装作若有所思,心里笑得很开心,他假模假式地问道:“宫中的奇花异草源自于沈老前辈,前朝也曾是沈家天下,那这个沈老前辈应该就是,遗留的皇帝或太子吧?”

钱老板顿了一下,装作彼岸花的根太难咽,他盯着小五子,嚼了有十几下,承认道:“实不相瞒,沈老前辈真是做了三十年的皇帝。”

小五子回想着,问道:“我以前在田独,听说书的讲,近百年来,能做到三十年的皇帝只有沈成浩一人,而史书记载,他却实实在在的是驾崩发丧,怎么会是沈老前辈?”

钱老板意识到自己讲太多了,他好为人师,但小五子不是好学生,跟他讲点知识,总能找各种证据来拾杠。他忽然换了个嘴脸,满脸赔笑道:“祖师爷的事情,我怎么敢乱说,这里面是真是假,还是由少谷主你日后慢慢探寻吧。”

他想假惺惺地结束对话,但小五子不干,揪住这一话题追问:“钱老板,不,常公公,不不,沈大总管,你本名叫沈志基,同样姓沈,不会是沈老前辈的后代吧?”

钱老板笑了,说:“我这是沈老前辈赐的国姓,我要真是个太子皇帝什么的,怎么还在这里当个太监?哎呀,我的事情,少谷主也可以日后慢慢探寻。”

整不了,官腔打得贼好,他那种好,是假得恰到好处,每回话题聊尬了,钱老板被逼问到死角,他就玩这一套,摆明了告诉你,我不想跟你聊了,别再烦我了。但小五子还挺喜欢跟他聊的,他知道的多,说话还有漏洞,每聊一次,都能推出一两个真相。如果能聊个一年半载,小五子一定可以把自己的过去全推导出来。

误,冒出了“母后身体有恙,要迟些时日相见”的话。话等。小五子装作没听见,但心里对谷主和面前这个太监
对,他们是前朝余孽。

有天联想到沈大总管沈志基这个名字,小五子打趣问道:“志基志基,这名字起得好,志在登基吗?”

沈总管连忙打哈哈,说:“我一个太监,连后都没有,还想着什么皇帝啊?”

小五子能听出来,钱老板这话,算是有意无意地默认了,他和谷主很有可能是皇后和太子的身份,把百花谷搞成这样,宫女,太监,行大礼,心里盼的肯定就是复辟。

小五子问他在哪儿生的,其实他心里想测的是,钱老板生在皇宫,还是平常人家庭。

钱老板环顾一圈百花谷,说:“我就生在这儿,生在南京,生在这百花谷。”

小五子点点头,看得出来,他没撒谎。他跟着钱老板的视线,一起巡视百花谷的二十多栋房子,直截了当地问:“那我生在哪幢楼?”

钱老板皱眉看他,他明白了,这么问是在诈他呢。他摇摇头,又打起官腔来,笑道:“少谷主这样的富贵身份,百花谷怎么能容得下你这条龙?”

“那我到底是在哪里长大的?”

“你长大的地方,可比这里好太多了。”钱老板伸出食指,指了一个方向。小五子看看日头,知道他指的是西北方向。钱老板说:“你是在山西太原长大的。”

“太原,哪里就比百花谷好了?”嘴上这么问,小五子脑子里想着这问题,这不是他预设的答案,说皇宫,说百花谷,怎么又跑出一个山西太原?他问:“江湖上人人叫我昆仑公子,我本家姓什么?”

“我听说你姓孙,叫孙天奇。”

“我叫孙天奇?”小五子慢慢说出自己的名字,似乎想在这三个字里找到和自己有关的什么东西,“听说?你听谁说的?”

钱老板似笑非笑,含糊其辞地说:“我听天下人说的。”

“谁又是天下人?”愈发接近真相,小五子愈发急了起来,“我父亲是谁,母亲又是谁?”

“我实在不能讲,”钱老板想了想,长叹一口气,说,“我只知道,你母亲早不在人世,因为她被你父亲杀死了。”

钱老板说完离开,小五子追出去。他背对着小五子,摇了摇手,要他别跟过来。小五子看着他背影,如鲠在喉。

“我爹为什么会杀了我娘?他究竟是什么人?”
\newline

{\centering\subsection{4}}

孙天奇。

原来他叫这个。没有钱老板,不好接近吴思若和文思清,八光还躺在床上治疗花毒,还有一只手能陪他解闷。

春日午后,小五子拉他玩一个无聊游戏,让一只手反复问他:“你叫什么名字?”小五子依次回答,小五子、少谷主、五帮主、昆仑公子,这回又有了一个新名字,孙天奇,可能这才是他的真实名字,可能不会再变了。

三月二十五那天,小五子终于见到了百花谷谷主,也不算见到,隔着一道纱帘,隐约能看到对方的轮廓,头五分钟,小五子一句话也没听进去,一直有种冲动,想把纱帘拽下来,好好看看。如果换别人,纱帘早就被他扯下来了,可是谷主有一种说不出的威严,让小五子不敢轻举妄动,头一回这么安静。

谷主说着说着,忽然问出一句:“少谷主,我刚才说的话,你都听进去了吗?”小五子想都不想,直接回答:“都听进去了。”

“好,听进去就好,”谷主说,“我年纪大了,忘记我刚才说什么了,你能不能把我说的话,再重复一遍?”

一句话也没听着,但这种长辈前辈的叮咛,总是有标准答案的。小五子挺直身子,朗声回答道:“谷主刚才说,希望我孙天奇,好好练武用功,日后带着百花谷的人涉足武林,救各路英雄为危难之际,重振百花谷的门威!”

谷主叹了一口气,颇为赞许地点了点头,说:“你能从我的话里,领悟到这么多,也真是难得。我刚才只是说,我这几天身体不好,没能早点见到你,我问你在谷里住得可否舒适?”

没法辩解了,再耍小聪明,就成泼皮无赖了。小五子干脆承认,鞠了个躬,说:“谷主果然洞察我心思,说实话,我刚才脑子里一直在好奇,谷主在纱帘后面,是何种样子。”

“我自然明白你的心事,难得你今天能这么稳重,倘若你真是扯开纱帘,你的双手可能就不在了。”

小五子摊开自己的双手看,两只都被剁,那岂不是连一只手都看不起我?忽然起了一阵风,纱帘在床下面露出一个口子,谷主在纱帘后面说:“那你就把手伸过来
吧。”

小五子愣住,下意识把手缩了回去。谷主又说了一遍:“我现在命令你,把手伸过来。”

似乎是难以拒绝的威严,小五子往前坐一点,身子前倾,双手从纱帘下面刚吹开的口子伸过去。

“苏子瑶在世时,跟我说过,你中了断魂掌,让我看一看。”

原来只是号脉,她双手摸他的双腕。谷主手指冰冷,刚碰到他手腕时,小五子还打了个冷战。不同于郎中的号脉,谷主的手指压在手腕上,便一动不动,一炷香都要烧完了,她不说话,也不抬手,手指还是那么冷。小五子大概能感觉到,她的指甲很长,估计有半根手指那么长。

差不多都要睡着了,谷主在里面说话了。她说:“你是四年前中掌的,大概是八月中,三日后,你完全失去记忆,有长达半月的昏迷期,你现在所能想起的,最早到那一年的秋天。”

“对,我能记起的就是在钱记肉铺醒过来,睁眼第一眼看到的就是钱老板,他当时是哑巴,写字告诉我,说是山上采草药时发现我昏迷不醒,把我带回来。”小五子说,“当时也是傻,这都能信,他一养猪的,采什么草药?”

谷主笑了,说:“也难为他了,你只昏迷半个月,他不单要把你带到田独,还得抓紧时间,在你醒来之前,把钱记肉铺开起来。”

“我当时应该发现的,牌匾,杀猪刀,案板都是新的,连猪都是小猪崽儿,生生被我养肥的。”

谷主放下小五子手腕,说:“我没有能力医好你的这一掌,还好沈大总管在这三年里,哪怕是为了强身健体,也没有让你习练任何武功,引发你走火入魔。”

“那我以前有没有武功?”

“你没有武功。不过,你刚出生的十二个时辰内,有人给你输入的一股真气,成为你体内的内力。”

“我也想到过,自己应该没半点功底,只是,我作为昆仑公子,结了那么多仇家,得罪了那么多人,到底是怎么回事?”

谷主回答他:“你能得罪那么多人,是因为你带着兵,你手下有高手,他们帮你抓了这些武林人士,带到昆仑山庄,再由你慢慢审讯折磨的。”

“我哪来的兵?”小五子问,“我为什么要审他们呢?”

“那是因为你要别人为你所用,达成你的目的。”

“我为百花谷做事?”

“不为百花谷,”谷主停顿一下,说,“是为你自己做事。”

“我自己要做什么?是为了九宫图吗?”小五子跟她讲了临走的时候,钱老板给过他一张九宫图,上面什么也没有,就是一张破羊皮,几个月下来,无意中已经有了三张,已经被保管到了一个安全的地方,“谷主若需要,我把这些全拿给你。”

“不用了,”谷主说,“九宫图的事情,我早听说过,怕只是以讹传讹。我几十年前见到过其中的一张两张,知道上面毫无内容,只是担心武林人士为争夺它,大家拼个你死我活,死伤无数。”谷主希望小五子能尽快把这些集全,当着全武林的面,将九宫图销毁,以免以后武林人士再动干戈,互相残杀。

一席话把小五子讲燃了,满腔热血,拍胸脯立誓,要把这些图集全,献给谷主,届时由谷主当众销毁。

谷主又叹一口气:“谁来销毁倒不重要,我只是希望,以后武林少些祸事。”

“已经有些祸事了。”

小五子说了田独何员外的灭门惨案,他说当时就是因为有人怀疑,他在碗中藏了九宫图,结果何府上百个人,惨遭灭门。他讲了那天在后厨目睹的这一切,躲在猪肚子里,才逃过一条命;他讲了何员外提剑,把那些中掌的人全部砍死,之后是怎么把碗托付给他,求他如何去救向老帮主。最后一段讲起来有些难过,何员外越来越疯,以至于到后来没办法,小五子在河边请他吃最后一顿烤肉,亲手把他捅死了。

“我没本事,”小五子说,“杀了两回,才把何员外杀死。”

“这不怪你,我三个师兄,南海真人,大漠仙人,蓬菜阁老,不知是哪一个下的毒手。”

“反正我看,他是三掌都练成了,以后谁也打不过他了。”

“那倒未必,我向师弟练成无为神掌,自然不会怕他,可他现在不知身在何处,是死是活都不清楚。”

“何员外说,他在京城,让我拿这个碗去救他。我后来还真去了,可京城那么大,我去哪找一个人啊。”小五子掏出碗递给谷主,“就是这个碗,怎么看也不像是这碗里能塞下一张羊皮啊?就是一个奇怪的说不上是铁,还是铜的碗,谷主,你看看这里面有什么玄机?”

百花谷谷主接过来查看一番,还给他,说:“真是妖言惑众,这里面什么也没有,你赶快拿回向师弟的镇帮
之宝,我不想睹物思人。”

小五子把碗接过来。

谷主问道:“何员外当时有没有跟你说过,向师弟出关这一天,要顶住他的百会穴?再去点膻中穴什么的?”

小五子恍然道:“的确说过,你不提,我都差点忘了。”

“他有没有告诉你,先从哪个穴位开始?”

小五子回想半天,其实他确实忘了,但嘴上说:“没说过。”

“也不知是你忘了,还是何员外忘说了,这次你一定要记住从右至左,也就是从肺到心,千万不要弄错顺序了。否则心血倒流,经脉尽绝,向师弟一世英雄,可不能被你害死了。”

小五子点点头,表示记住了,这次牢牢记住,绝不会忘记。他问谷主:“那您这边,有没有向老前辈的消息?”

“向师弟神出鬼没,他的心思我可猜不透。你接下来,尽可以四处查访向师弟的下落,倘若他真的是被奸人虏获,百花谷会全力营救。”

小五子有点晃神,不知道谷主什么意思。谷主补充道:“一方面,何员外委托过你,找到向问和,带他出关;另一方面,我在百花谷也担心他的安危,现在你既然是百花谷的少谷主,总要为谷里做点事情,当前百花谷最重要的事情,就是救出向问和。”

这是要出发的意思了,离开百花谷,在这里呆了十几天,没想到跟谷主的第一次见面,竟是告别。忽然还有点感伤,小五子深鞠一躬,对谷主说请多保重,转身出
了门。

他出门右转,大步朝苏子瑶墓走去,一只手一路在后面跟着,小五子让一只手离他远点,他去跟苏子瑶说两句话。一只手停住脚步,看着小五子上山走到墓前,徒手去挖墓碑下的泥土。

一直挖到天黑,苏子瑶的棺材露出来,小五子抬起来棺材盖,看着躺在里面的苏子瑶。百花谷的人事先放了些防腐的奇花,使得苏子瑶的尸体比常人的还光泽亮丽。小五子俯身贴近,对她的尸体说:“我手太脏了,就不碰你了。你果真还是死了,下葬那天,我还以为能像田独说书人讲的那样,有些高手可以闭气十几天,没心跳,没呼吸,看起来死了,半个月之后睁开眼,就像刚睡醒一样,活蹦乱跳的。你下葬那天我守着你,让所有人先走,我下来把棺材的钉子拔掉,因为那时候我害怕,万一你醒过来,漆黑一片,出不来怎么办?今天已经是第十七天了,还是没有奇迹发生。我今天就要走了,从这里出去,离开南京,往北去京城。在这里你要,你要······”

小五子说着哽住了,连说了两遍“你要”,一着急眼泪掉了下来,最后带一点哭腔地说了句:“谢谢你。”

小五子将棺材盖合上,抓起一把土埋进去,站起来,对远处一直在观望的一只手喊道:“叫上所有人,我们出发了!”

\newpage