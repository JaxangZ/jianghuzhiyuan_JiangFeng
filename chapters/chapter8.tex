\section{捌}

{\centering\subsection{1}}

遇见小五子那天吴思若做了个梦,梦见自己在池子里洗澡,倒上牛奶,撒上花瓣,泡在水里吃,泡在水里睡,一直没出来过,可她总是感觉洗不干净,都泡出褶子了,还一遍又一遍用手搓,洗到第三年她终于扛不住了,从池子里走出来,赤身裸体,水淋了一地,她站在铜镜前,双臂环抱着胸,哭道:“洗不掉了,怎么办啊,我真的洗不干净了!”

之后她在夜里醒过来,二楼的客房,头天晚上她到的扬州,睡到现在天还是黑的。她睁着眼平躺在床上,不知道是几点,楼下的赌场依然喧哗。赌场掌柜的怕不热闹,不知从哪请来一位老先生,没日没夜在那儿唱评弹,一口苏州话也听不懂他唱的是什么。就当是背景音乐,赢钱的时候没人注意到他,只有输钱的人,一文不剩还舍不得离开赌场,耷拉着脑袋,听老先生唱那些英雄好汉出门就造反的故事。

赌鬼她见多了,给他们俩胆儿都不敢造反。以前在杭州紫竹院,对面就开一赌场,吴思若就没见过他们打烊。紫竹院是青楼,按理说就够热闹的了,可这边再怎么春色荡漾,总有累了睡觉的时候。感觉对面赌场开的是接力流
水席,有赢有输,有去有回,赌桌上油灯都不带断捻儿的。

吴思若十四岁进紫竹楼,被老鸨练两年,十六岁开始挂牌子,一直待到二十一岁才被她师父大漠仙人赎出去。五年里她见过最多的就是读书人和赌鬼。读书人最麻烦,吟诗作赋还得让吴思若唱出来,清唱不过瘾,得弹琵琶古筝唱。赌鬼干脆多了,隔三差五就有赢钱的过来,大把撒银子,说把你们头牌叫出来,赢来的钱,出手也大方,非要挑缺点,就是这些人有点急,进来就脱衣服上床,完事就想走,气儿还没喘匀呢,裤子就已经穿上了,满口大话说今天手气这么好,过去再押几把,我能把这紫竹院都赢下来。

吴思若干那么多年,也没见哪个能赢下紫竹院的,反倒是回头客都没有,连本带利地又输出去了。那时她还不叫吴思若,紫竹院的时候叫芙蓉月,再往前叫小月,也没个姓。没爹没娘,打记事起就跟着师父,有一搭没一搭地练功,反正师父独宠她,把天捅个窟窿也不会怪她。十几年来,基本上师哥师姐负责受罚,她负责恃宠而骄。人家练到掌掌致命了,她这仙人掌打出之前,还得捧着仙人球拍几下。

后来师父终于着急了,跟她谈,掌法还可以苦练,但是你内力几乎都没有,再练已经来不及了。

“那就不练呗,”吴思若反过来跟师父讲道理,“上个月来的那个道士,说自己什么什么功练了三十多年,还说什么冬练三九,夏练三伏,一分耕耘方有一分收获,结果剑还没拔出来呢,就被师父你一掌拍晕了,一个多月不吃不喝,守着绿洲饿死了。早死晚死都是命,早知道这样,吃那三十年的辛苦干吗?”

大漠仙人想了想,差点让这小姑娘把习武之道给扭曲
了。他说碰到他是例外,如果是江湖上的芸芸众生,多练一分总是好的。

“那我不离开你就好了,”吴思若说,“反正怎么练也打不过你。”

大漠仙人摇头说:“我大你几十岁,总要比你先死的,我死了你怎么办?”

“不还有师哥师姐吗?”

仙人沉默,看着她,最后看得吴思若有些发毛了,提醒她:“他们替你受了这么多年的罚,我死后,他们第一个捅刀子的就是你。”

师父说的没错,吴思若知道,有时候师哥师姐看她的眼神就是一副早晚弄死你的样子。但现在练不是来不及了吗,她想,不行到时候我找个地方躲起来就好了。大漠仙人点点头,他说你的事师父也想了很久,总算有个两全的办法,我要送你去朋友那里学武,以后你练出来了最好,就算没练出来,也没人知道你的下落。

师父的朋友在杭州,大漠仙人送她过去,他们从罗布泊出发,两个多月才到江南,吴思若第一次出大漠,一切都是新鲜的,何况还是杭州,除了天堂就是这儿了,不要说市集、饭馆和水乡,她甚至都没见过这么多人。师父陪她连逛了三天,带她去金银店买首饰,去丝绸店挑料子,七八个颜色,选不出哪一个,吴思若数着泥锅泥碗泥滚蛋,一个个淘汰。后来把师父看心疼了,拿出银子,跟伙计说一种颜色一匹,全扛到紫竹院。那是吴思若头一回听到紫竹院这个名字,放下布料问师父:“他们是紫竹派的吗?”

紫竹院比大漠好多了,灯红酒绿,八仙桌上宴席不断,里面的师姐也好看,而且有几个跟立了大功似的,那些婆子和龟奴都围着她一人伺候。第四天一大早,师父要走了,嘱咐她好好练功,别老想着玩,给你备了八匹丝绸,他打听过了,一匹布能做二十件上衣,三十条裤子,想穿新衣服了就找人定做,过十年师父再来看你,到时候试试你功夫,就知道你有没有偷懒。

十年?吴思若想跟师父一起回去,杭州虽好,可也不用逛十年啊。仙人提醒她又忘了,你是来学艺的,脑子里还想着玩?吴思若不说话了,看着师父把银子装进两个大箱搬下楼。走的时候也不让她送,吓唬她,学艺不精,就不要出紫竹院了。吴思若要过很久才明白,这句话不是吓唬,此后七年多,她真的一步都没能走出紫竹院。

原来这里不叫师父,外人叫窑婆子,本门弟子喊妈妈,而且妈妈不止一个,每个妈妈带十来个弟子。吴思若的妈妈姓王,一把年纪还伶牙俐齿的,跟她说了紫竹院的各种
好。王妈问她叫什么。吴思若说小月。王妈还在等她说。没了,就叫小月,她也不知道姓什么,从小就这么叫。名字没特点,王妈端盆花过来,说以后你房里就养这盆芙蓉,叫芙蓉月吧。

“那我姓芙吗?” “姓芙蓉!”

吴思若没听出她在抬杠,还挺高兴自己的月字留住了,加的姓也不错。王妈问她多大了。吴思若左手比划一,右手比划四,说自己十四岁。

“还早,”王妈说,“你可以再练两年。”

紫竹派都练什么呢,房间里放一顶缸,里面没有水,一只脚迈进去,然后就在缸沿上坐着,什么都不干,手不许扶,脚不许着地,一坐就是一天。到晚上也不让你安生,浑身酸痛刚躺到床上,王妈提了一篮鸡蛋进来,叫她起来,等会儿睡。鸡蛋不是给你补的,她捡十个鸡蛋放在床中央,让吴思若平躺上去,把鸡蛋枕在腰下面。

“明早鸡蛋碎掉一个,一鞭子,碎两个两鞭子,十个全碎,加五鞭,我要打你十五鞭。”

没准真会打,吴思若侧过身,小心翼翼把鸡蛋搂在怀里。睡到一半被一鞭子惊醒了。鞭子抽在后背上,吴思若缩在床头,瞪大眼睛看着黑暗房间里的阴影。

“我让你平躺在上面,可不是侧着睡。”

王妈手拿着鞭子,说完就推门出去了。吴思若从床头慢慢平滑下来,向上挺着腰,“咔嚓咔嚓”地做了一晚上潮湿的梦。醒来时鸡蛋都碎了,十五鞭打了她快一上午,后背开裂,血滴从衣服里渗出来。吃过中饭,吴思若要继续坐缸沿,后背疼得都直不起来。晚饭睡觉前,王妈又拿了十个鸡蛋过来。

“我教你一招,”王妈说,“屁股使劲往下翘,你要借肩膀的力量,挺住胸才能挺住腰,之后绷紧不动,起码保住一个,保一个少打六鞭。”

吴思若不说话,也绝不会哭,瞪着王妈,目送她出门。王妈走到门口,转回身扶着门框说:“芙蓉月,你给我记住了,你有多大委屈,多大仇,都给我咽回去,练不出来,你死在这儿我都不掉一滴眼泪,要是你练出来了,做了紫竹院的头牌,总有我王妈巴着你那天,到时候你有多大仇,多大恨,随便你怎么折磨!”

吴思若想了想,吹灭蜡烛躺进被窝,照着王妈的方法做。屁股还没夹紧,就碎掉一个。无所谓了,反正明早也是全碎。入睡之前她尽量想些好事,早上一床的鸡蛋汤,从上到下十几层被褥,都被人换成新的了。就这一点挺好
的,在紫竹院,王妈不要求她干一丁点的家务。

第二天醒来奇迹发生了,居然有三个鸡蛋没有碎,吴思若恨不得把这三个生吃掉。但七鞭子还是要打的,不知是王妈下手轻了,还是已经打了,好像没那么疼了。第三天早上又退回去,十个只留住一个,赏九鞭。反反复复,八个月过去基本不会碎鸡蛋了,偶尔碎一两个,王妈也舍不得下手打她了。坐缸沿更是轻车熟路,她现在荡着腿在缸沿上吃饭、背词、弹琵琶,干什么都稳稳的。

王妈已经开始训练她拨弦唱曲了,这些她理解,会点才艺总比种地的农妇好一些。可是练八个月的缸沿和鸡蛋到底有什么用呢?王妈轻轻一笑,你只管练就是,以后你就明白了。

那估计是练内力的吧?吴思若跟王妈说了来紫竹院学武的目的,不能光练内力,还得学点一招制敌的本事:“师哥师姐都等着师父一死,就掐死我呢,尤其是大师姐。”

王妈眨巴着眼睛,大概明白她师父是怎么把她骗过来的了,也明白这孩子在大漠长大,没见过世面,什么都不懂。怎么回答,她得好好想一想,最好有个答案能让她不怀疑,自己日后不必再解释,大家都省心,一劳永逸。

王妈想了三天,向杭州城里走江湖的要把事的都打听一遍,晚上让厨房炒两个菜,和她好好喝了一顿酒。吴思若第一次喝酒,一口下去辣得直往外哈气。王妈又给她斟了一杯,说你师说的没错,把你送到紫竹院,就是让你练功来了。然后她讲了男和女,男人是阳,功夫要一天一天练,十年二十年才有所成,而女人是阴,学武练功有先天优势,不用像男人那么辛苦,一拳一脚地练,在紫竹院,你可以把男人练好的内力一点一点吸到你身体里来。

“采阳补阴你听说过吗?”

吴思若点头:“好像听说过,但不是采阴补阳吗?”

“那是男人吓唬我们的,自欺欺人。”

王妈喝口酒,想了想,也许可以把自己推翻,换个更巧妙的表述,她说有些男人会的,采阴补阳,不但不把内力给你,还要把你的内力吸过来。这时候就像斗法,谁法力强,就能把对方吸垮。

“那怎么办,怎么办?”王妈讲了两遍,直到吴思若盯着她时,她说,“你要更加努力练功,一旦到床上,绝不能给对方喘息的机会!”

到了第二年吴思若要学习琴棋书画。王妈解释这是要诱敌深入,功夫练好,在战场上打仗是一回事,把敌人勾引到战场,又是一门技能。到十六岁终于要挂牌接客了,价高者得,最终夺标的是一个三百多斤的老员外,看起来
身体还行,只是太胖了,走两步整幢楼都跟着颤,要四个人把他架到二楼,才不至于把楼梯的木阶全部踩折。

吴思若迎来第一个对手。那天晚上王妈去房间,跟她说了几句话。她把油灯换成红蜡烛,在床上铺一条白绫,将酒菜摆在桌上,把客人请了进去。她边陪笑边后退,面对着两个人关门出去。然后她还是不放心,叫龟奴搬两个小凳守在门前。万一有意外,真的,退双倍钱也不能让芙蓉月被这胖员外压死在房间里。

开始她还能笑出来,她清楚地听见员外坐下来吃东西,吴思若在他身后呵斥,开始吧,你还等什么!可没多久她也笑不出来了,吴思若没有了呵斥,反而不断地哀求,求他不要这样,求他放过她。似乎客人没听她的,到后来吴思若声嘶力竭地哭,大喊:“王妈救我,王妈救我!”

两个龟奴实在听不下去了,从板凳上站起来要冲进去。王妈张臂拦在门前,指着他们说:“谁也不许动,今晚挺不过去,以后也就是个短命姑娘。”

再往后吴思若没声音了,不知是活是死,只是脚下在震动,感觉整个紫竹院都被带得一起颤。房子都要碎了,两个龟奴缓缓坐下来,摇着脑袋低着看着地面。晃动越来越剧烈,忽然一声轰响,其他房间的窑姐儿和客人都出来往这里看。王妈摆手让他们回去,一只手捂着嘴屏息等待。要等好久,时间慢得仿佛每个人都死了几分钟,房门轻轻拉开,老员外整理好衣服走出来,塞给王妈两个元宝,慢悠悠地下了楼。

王妈看着手里的银子,迈进房门。似乎狂风刮过,染了血的白绫被吹到墙边,地上全都是碎了的瓷碗酒杯和饭菜,四条腿的床断了两根,床面像一个山坡面对着房门,吴思若衣服被扯烂,散乱的头发挡住半张脸,裹在红纱帐里,缩在斜下来的床尾处。王妈走过去,银子放在床边,将纱帐一层层打开。她没有死,还有温度,脸上的泪还没干。王妈把她头发捋到后面,露出她的眼睛。吴思若瞪着她,已经没眼泪可哭了,过了好半天才哑着嗓子说出一句话:“我赢了。”

此后她一直赢,第一年,第二年,第三年,一百个,二百个,五百个对手,从来没有输给过他们。每个对手她都会记下来,叫什么名字,家住哪里,当什么官,发什么财。一半是因为成就感,得意,另一半居然是愧疚,她觉得自己成长的道路上有这么多人在帮她,一旦有机会,功成名就那天,照着花名册的住址,欠人家那一份,总要还回去的。

每三天一次,王妈要她喝一种凉药,麝香和水银混在一起,特别香,又特别硬。有时候跟客人在房间,打一个
嗝,满屋子都是怪异香味,这还不是她最怕的,她最怕水银从嗓子眼里蹦出来,像小铁球在地上乱滚。她问王妈喝这个做什么。王妈从任督二脉到急火攻心解释了半天,后来自己也编不下去了,直截了当地问她:“每个女人都会生孩子的,对吧?”

吴思若点点头,她说她明白了。然而她没明白,她要到很久以后,一辈子都生不出来孩子的时候才明白,王妈和她师父毁她一生。

你看不到真相,所有人都对你说谎时,你会以为那是真的,你就应该那么做。紫竹院的规则是什么,姑娘都是来练功的,互相抢生意,就是抢能送你内力的那个男人。姑娘们背地里骂别人是婊子,就是这个意思吧,定期给她送功的男人,被别的姑娘抢走了。

第三年秋天她也抢了其他姑娘的男人。一个读书的公子,吴思若见过他好几次了,每次都是找丹姐。这天晚上还是搂着丹姐的肩膀上楼,吴思若在二楼一直盯着他,照王妈教的办法,两个肩膀各露出一半,冲周公子媚笑。果然在夜里,公子敲了她的门,两人摸着黑对弈斗法。完事后他不紧不慢地穿衣服,长吁一口气,跟她说:“虽然传功这事挺扯的,但你是我见过最卖力气的。”

这算夸她吗,吴思若问他怎么称呼,去哪里能找到他。公子随便说了个名字住址,吴思若心里默念几遍,他刚一出门,就点灯记在花名册上。

第二天紫竹院炸锅了,丹姐站在天井往上骂,她在紫竹院做了五年的头牌,是谁家的婊子,这么没大没小?吴思若推门出去,低头看着丹姐,犹豫何时动手,师姐师妹今天是不是要切磋一场。丹姐完全不怕她,她知道今天若是输了,守了五年的头牌就要让给芙蓉月这个小丫头。她双臂抱胸,越骂越难听,说她就是一贱货,人尽可夫的荡妇,干吗呆在紫竹院啊,就该送她到边塞的军队里去。

吴思若抓着围栏,一点一点泄下来,也许在说谎,她师父,王妈,大漠里的师哥师姐,紫竹院的所有姑娘,他们都在说谎,唯有丹姐在讲真话,她活了一十九年,信任她师父,信任她师父的朋友王妈,相信他们都在为她好,而他们让她做的,鼓励她做的,她想持之以恒努力去做的,居然全是羞耻,一生之耻。
\newline

{\centering\subsection{2}}

吴思若是在那天晚上再次见到小五子的,快十个时辰之后,她在扬州客栈,夜里醒来,看着头顶的一片漆黑,想
熬到天亮去吃点东西。公鸡打鸣的时候她反而有些懒了,侧身对着窗外,看着太阳一点点上来。扛到中午她昏昏欲睡,梦到小五子,梦到自己离他越来越远,再睁开眼睛天又黑了。并不是真的安静,楼下赌场的喊叫押注声时不时传进来,只是她能分得清,那是外面的声音,房间里很静,静得她能听到一只蜘蛛在墙角筑网。

她想她怎么还不死,睡了那么久,应该睡死掉才是,为什么还能睁开眼睛,看着这到不了头的黑暗。而且还饿,肚子叫得已经让她听不到蜘蛛在编网。好比睡不死,她知道她饿不死,只会死去活来。她憋足力气,给自己一掌仙人掌。除了疼什么用都没有,毒蛇是不会被自己咬死的。

她把头发扎起来,扮成男装下了楼。要穿过赌场才是饭堂,如果吃饱了,想回房间休息,也要穿回赌场才能上楼,而且不是正对着,要拐三四个弯才能把饭堂和客房连接,掌柜的在这一点上费尽心机。那些押中了的欢呼尖叫声,一惊一乍地刺激着往来的客人们。

她没兴趣停留,在老先生的苏州评弹中找到通往饭堂的出口,人来人往,好容易挤到门口,吴思若停下来往回看,她看到了小五子,他坐在桌前,半睡半醒地硬撑着头,赢下银子还要庄家拨给他,他就要睡着了,头沉下去,又猛地醒过来,抓着银子要押注,庄家告诉他下把吧,刚叫你押你睡觉。

老先生唱着,那毕娘听,她是羞不胜,但听她句句言辞触奴心。吴思若走过去,离得越近,心越慌,什么都看不到,眼里只有小五子,直到一对骰子飞过来,吴思若才意识到,她看小五子,而有人在看着她。

那人没想伤她性命,来势不快,吴思若伸手即握住一个骰子,另一个骰子打中她眉心,弹到地上。她朝人群看去,或庄或闲,大家各忙各的,唯有一个人和她对视。同样也是女扮男装,一身黑衣,吴思若想在哪里见过她。就在昆仑山庄,也上了高台,头一个叫文思清,说是小五子的老婆,她是救她的那个。叫什么名字呢,挂在嘴边想不起来了。进山海关那阵,小五子跟吴思若说过,要是有前生今世,她就是他第一个对不起的。哦,她叫苏子瑶。

人群中她俩互相望着,小五子硬撑几次,终于收起银子,躺在长椅上睡着了。吴思若确定,就像不知道她在这里,小五子也不知道苏子瑶一直跟着他。她看见苏子瑶将面纱放下来,走到小五子身旁,抓一把碎银子,冲吴思若扭了扭头,意思是我们饭堂见。

就两个人,苏子瑶点了十六个菜,一壶女儿红,回身看到吴思若的眼神,问她怎么了,他有的是钱,他在这儿吃饱
了玩儿,玩儿好了睡,咱们俩就不能吃点好的了?吴思若笑了,眨着眼睛说:“当然要吃好的,妹妹只是想再加两个汤。”

菜都上齐了,反而吃不了几口,彼此印象都不错,两个女人聊个不停。但还是不一样,吴思若叫他小五子,苏子瑶叫他昆仑公子。吴思若问她什么时候找到小五子的,苏子瑶说昆仑公子根本就没丢过。她说南京江里翻船的时候,她就知道昆仑公子在藏金条的隔层里,但她不能说,三个百花谷主都不是大漠仙人和蓬莱阁老的对手,她失声尖叫,装作小五子淹死了,找机会她就溜出来,在岸边随江面的棺材一起走,直到夜里小五子爬出来了,上了秦淮河的花船,不知道和那帮船上贱货都干了什么,反正等了快一个时辰,小五子才换身新衣服被送下船,然后就跟他到了扬州,赖着不走了。

吴思若一阵难受,杭州的花船她也曾坐过,还以为西湖水上是练功的好地方。她尽量不想这些,夹两口菜,问苏子瑶既然都到扬州了,为什么不索性现身,陪他一起走。苏子瑶凑前一点,低声跟她说,因为还有人跟着他,两个人,一黑一白,都在赌场里守着他,你一会儿进去就能看见,正坤桌一个,后乾桌一个,装作来赌钱的,身上又没钱,庄家催了,就押个一文两文的。

吴思若好奇,这两个是什么人,要么抓人,要么放人,一直跟着算怎么回事?苏子瑶说听师父讲过,他们叫黑白镖人,江湖上专门帮忙寻人的,不管你要谁,只要出够了银子,管他是活人还是死人,早晚送到你面前。

“哥俩手头这么紧,看来小五子这单给的钱不多。”

“其实给的不少,但他们找了三年,花光了。”

“啊?”

苏子瑶说:“前天他们俩还吵架来着,我在窗下听到的,白衣服想抓上昆仑公子就走,黑衣服说不急,看他出扬州往哪走,顺路就跟着,省点麻烦,不顺路再抓他也不迟。两人吵一晚上,后来白衣服就开始翻旧账,说当初不让你接这单,你非要接,昆仑公子是谁都能找着的吗,这一找三年多,还把别的事都推了,风里来雨里去,一文钱不进账,你让我喝西北风啊?可黑衣服讲原则,他说行走江湖就是诚信,一件事没做完,怎能急着揽另一件事?”

吴思若也没见过黑白镖人,就觉得苏子瑶学得挺像。她倒杯酒,说一会儿我去对付黑衣服,你对付白衣服,然后就拉上小五子上路吧。酒被她一口喝干,苏子瑶握着酒杯不动,提醒她:“我们打不过黑白镖人的。”

吴思若说你都打不过,我就更不行了。苏子瑶点头,只能静观其变,说完还是不喝酒,筷子也放下了。吴思若让她
多吃点菜,这可是扬州的蟹黄狮子头。苏子瑶摇头,笑了笑,表情忽然凝住,认真跟她说:“你不要再跟着我们了。”

吴思若吓到了,不清楚她怎么回事,反应了一阵,说:“我没想跟着,我也是碰到,还有什么叫跟着你们?你们是谁,你和小五子就是你们了,对吗?”

吴思若问了一连串问题,苏子瑶一句话都不说。她给自己又倒一杯酒,第二杯下去,她拍桌子,把小二叫过来,问他是什么酒。小二说女儿红,十八年的绍兴女儿红。吴思若苦笑,我杭州紫竹院喝大的,你说是绍兴女儿红,一滴兑一缸吗?小二为难,说水多少兑了一点,但肯定没兑一缸。吴思若不想和他争,让他上原浆,价钱不是问题。小二站原地不动,说这是扬州,又不是绍兴,哪来的原浆女儿红?吴思若指着墙上的木板,那么大的字,原浆绍兴女儿红,是我瞎了,还是你瞎了?她站起来要打小二,掌柜的过来解围,承认字是他写的,但字是字,酒是酒,写字就是辅助你喝酒的心情,我写绍兴女儿红,就为了让你喝这酒的时候,能体会到喝女儿红的心情。

“够了!”

苏子瑶喊停,她一直靠在椅背上,看着他们吵。她让掌柜的去忙,至于小二,女儿红也好,状元郎也好,只请你现在走开。反而对吴思若,她一眼都不看,重拾起筷子夹菜吃,貌似也吃不动了,所做的一切,注意力集中在菜上,就是为了忽视对面的吴思若。她夹起狮子头,盯着里面的肉馅和蟹黄,漫不经心地说:“我刚才已经很客气了,还假装跟你亲近,和你一起吃饭,就是希望你能听明白我的话,离昆仑公子远一点。”

“为什么?”吴思若有点懵,“你到底是个什么样的女人,说变脸就变脸。”

“因为你不配他,”语气之冷漠,即使旁观者都会寒心。苏子瑶把狮子头放碗里,用筷子挑碎,低头闻了一下,“我喜欢吃狮子头,但这个我不碰,因为这个坏了,肉馅其实不错,五个月黑猪的前腿肉,可惜这蟹黄不行,不知哪个臭水沟里捞上来的,和这么好的肉馅搅在一起,把整个狮子头都毁了。”
\newline

{\centering\subsection{3}}

她是离开紫竹院才改叫吴思若的,二十一岁,在杭州呆了七年,她被师父带回大漠。回去以后,多余的话她不问,每天只睡两个时辰,醒来就去练功。既然没有勇气去死,就得拼了命地好好活着。

又过了两年,大漠仙人放她出去了。行走江湖的第一年,她去了杭州紫竹院,她想去灭门,从王妈到龟奴,到丹姐,到紫竹院的每一个姑娘们,谁也逃不掉,满门抄斩。然而她做不到,雷峰塔,断桥下,她看着紫竹院前门庭若市,赌场里赢了钱的那些人,还是揣了银子就往对面跑,她居然情不自禁地笑了。嫖客,妓女,拉皮条,都是些寡廉鲜耻的人,凭什么他们就该死,凭什么她吴思若就能寡廉鲜耻地活着?

寡廉鲜耻会怎样呢,不会缺块肉,又不会少条胳膊,甚至能让你更有风情更妩媚,遇见小五子她知道了,羞耻的人生,会让你没有资格去爱别人。资格这个词有多可怕,我能,但我没有资格。

回到房间连睡两天,她感觉自己病了,裹在被子里一夜一夜地咳。第四天中午她下楼吃东西,穿过赌场时忍不住地多看小五子两眼,然后她看见了苏子瑶在远处盯着她。

她不想再讨嫌,没资格做的事情,就不要再去做了。但师父要她在扬州等他,说有大事商议,而且她病得越来越重,没办法离开扬州。她忍住不下楼,反正连下床的力气都没有。第六天夜里身体好些了,她走下楼梯,穿过赌场,发现小五子已经不在了,而那些人,苏子瑶和黑白镖人都不见了。

黑白镖人动手了吗,她问小二是怎么回事,赌场里应该叫赶羊的,羊是羊牯,生手菜鸟进来,先会被他们削一遍。吴思若问他,之前坐这里的那位公子哪去了?赶羊的正埋头数赏钱,一遍数不对,又来一遍,两遍都数完才抬头说:“你要找他翻本儿?人都走啦。”

自己走的就好,吴思若问他去哪里了。赶羊的不理她,低头数第三遍银子。这意思再明显不过了,吴思若拿两贯钱给他。赶羊的连同这两贯钱一起数,数完之后说:“我帮他买的箱子,帮他雇的船,我说明儿白天再走,非要今晚走。你赶紧去江边码头,兴许还来得及。”

她问清楚怎么走,黑夜里追过去。路不算远,小半个时辰追到了江边,离老远就看见两个船夫帮他搬行李。吴思若记得他没行李,一个月不见,哪来的这几个箱子。看了一会儿她明白了,箱子是做样子,空手上船反倒是令人起疑。

小五子独占一艘大船,不远处还有几艘小船,整夜停在江边,只等客满起锚。她看着黑白镖人赶到江边,上了后面的小船。吴思若犹豫要不要也上那艘船,这时有人跳下来,在后面点了她的穴。

偷袭的人是苏子瑶,她绕到前面笑着说,送到这里就
可以了,接下来就不用麻烦吴妹妹了。吴思若看着她,想解释自己并没想跟小五子。可为什么要跟她解释呢,她干脆不说,只说黑白镖人上了后面的船。

吴思若一客气,苏子瑶反倒不好意思了,纠结片刻,还是没给她解穴。她说以吴妹妹功力,一个时辰之后穴位自然就能冲开,劳烦你欣赏一会儿江景。然后她朝江面走出几步,似乎想到什么,回身对她说:“我再警告你一次,下一次就是杀了你,然后扒光你的衣服,哪儿高挂哪儿。”

她愈发觉得苏子瑶恐怖,不是做事狠,而是一点征兆都没有的变脸。她看着苏子瑶跑过去,离江边不远时停下来换成女装。船夫大老远就冲她喊,不上啦不上啦,这两位公子包船了!苏子瑶冲黑白镖人求情,说小女子命苦,赶着去奔丧,请二位公子通融一下。黑白镖人低声商量一番,白衣服的挥手让她上来。而小五子呢,开船之前他左顾右盼,看有没有人跟着他。真够可以的,吴思若看着笑出声来,四个人跟着他,还以为自己聪明绝顶,神出鬼没。

大船先开走,又过一炷香的功夫,小船也跟着起航。江面又恢复宁静,起码两个时辰,吴思若还是站在原地动不了。苏子瑶的功力胜她三五倍不止,一个时辰冲开的话,是高估吴思若了。

直到东方既白,江水涨潮,吴思若才能活动。刚开始浑身发麻,她瘫坐在地上。后来下雨了,她还是站不起来,努力让自己转过去,不去看江面,浑身湿透地看着来时的路。三年多以前也是这样,那时被大漠仙人从紫竹院接走,连骑了三天三夜的骆驼,紫竹院练了七年的采阳补阴,现在连赶路都要浑身酸痛,还没到绿洲便已从骆驼摔下来,动也动不了。她坐在砂砾上喘着粗气,师父提醒她早点起来赶路,不然等起风就跑不出去了。

她摇头,哑着嗓子一句话说不出来,于是再一次地摇头。她想死在这里。后来果真刮风了,那些碎沙会像海水一样汹涌,一层层地翻滚起来。两只野骆驼受惊发毛,嘶吼着逃窜。仙人没法驯服两只,一怒之下将它们全都击毙。仿佛掉到海水里,骆驼刚倒下来,淹没在流沙里全然不见。吴思若也差不多,沙石淹到她腰间,淹到她胸口,最后脖子以下全都埋进去了。寻死成功的一刻,她反而挣扎起来,她喊师父救我,她求师父别让她死在这里。

可流沙已淹到了鼻子,抓着她的头往上拔自然是身首异处。除了绝望地喊,吴思若没有哭,但第一次看到师父哭了出来。他说你别着急,你要是死这儿,师父就陪你一起埋进来!后来她看见了,流沙已到头顶。他徒手挖她四周的沙子,流沙每罗预淹三尺,他拼了命也要保持住每罗
预挖三尺的速度,别让吴思若被淹掉。

大概一个时辰,风终于停了,大漠仙人一点点把她从沙子里掏出来。他背着她回去,吴思若在他背上睡了大半宿。就是这个时候,东方既白,她在师父背上醒来,透过肩膀看着师父又走了两里路,她说没必要救她,自己像一张写错的宣纸,就该揉成一团扔掉,难道还可以接着往下写吗?师父背着她,又走出几十步,说:“也许我是故意写错的。”

“所以我恨你。”

仙人没说话,也没放下她,慢步前行。吴思若仰头看到前面大朵大朵的云,她知道快到了。她说师父,万一我还活着,往下活,活二十二岁,二十三岁,你别叫我小月了,也不叫芙蓉月,我想好自己叫什么了。

“吴思若,”她说,“从此我叫吴思若,吴是吾,是我,思是想念,若是你,不知道那个你是谁,但既然活着,就得盼点什么,哪怕盼不到。”

“以后就叫你吴思若。”

她伏在背上笑了笑,即使盯着看也不易察觉的微笑。前面下雨了,沙漠里的雨只下一片云,远远望去就像一个断点的水柱。那些零零散散的七色光芒已经准备就绪,一等雨停,就彼此相连,成为一道彩虹。

雨没有停,吴思若可以走路了,告诫自己一百次,能动的时候还是去了趟岸边,站在小五子上船的地方,对着江水又哭又笑。可能没有哭,只是雨太大了,一颗颗地打在脸上。她冲江面大喊,仿佛要让埋在沙漠里的芙蓉月也听到,她遇见这个若了,虽然她没资格去爱,但她可以活下去,她可以用尽余生去想念他。
\newline

{\centering\subsection{4}}

她师父加急送信给吴思若,要和三师叔绕道去趟少林,让她在扬州多等几天。她也不知道他们去少林干什么,仙人和阁老两个人,想想都能猜出,他们要把少林天翻地覆成什么样。那就多留几天吧,赌场楼上她是不想住了,她留下记号,找个清静地方,推开门就是一小片银杏林。扬州盛产银杏树,有庭院的地方都会种上几棵银杏树,到了入秋时节,满眼都是金灿灿的银杏花。

仙人和阁老造访少林的时候,文思清那时已经走了,和田扒光一起下了山。仙人和阁老开始还算礼貌,没往山上闯,叫知客僧传话,说大漠仙人携蓬莱阁老拜访少林方丈。阁老怕他没听明白,又重复一遍,蓬莱阁老携大漠仙
人拜访少林方丈。等了两刻钟,知客僧下来了,说方丈最近闭关修炼,还请二位晚些再来。仙人问他要闭关多久。知客僧说大概要两个甲子。

阁老还要算一算,仙人早就反应过来,反问他:“两个甲子是一百二十年?”

“啊,方丈悟性这么高,兴许一个半甲子就能出关了。”

阁老问他什么功夫,要练这么久。仙人白他一眼,这都不算敷衍,简直在羞辱,你还听不出来。阁老不理会这些,他深知一些精妙功夫,练个百年也是可能的。他只是在追问:“你刚才是怎么传话的?”

“我说,大漠仙人携蓬莱阁老拜访少林方丈。”

“怪不得不见我们,你传错了。你想想,我是怎么跟你说的?”

知客僧看看面前两位,明白他的点在哪了,他改口说刚传了两遍,第二遍是蓬莱阁老携大漠仙人。

“你在撒谎,”阁老盯着他,推开知客僧就往山上走,回头指着他说,“不是我要闯你们少林,我很懂礼貌,我师父在这儿,我绝不会失礼,是因为你在撒谎,我要跟你们方丈讲。”

仙人没那么多托辞,陪他一起上了山。两人赖在少林寺不走,阁老找了各种借口要见方丈。吃过晚饭有好消息传过来,罗汉堂的一位高僧春光满面地告知二位,本来要闭关两个甲子的,借二位佛光,方丈师兄两个时辰就顺利出关了。

方丈坐正位,仙人和阁老坐在一侧。方丈问他们所来何事。阁老说,路上接到大师兄消息,得知师父在贵寺藏经阁。方丈问他大师兄为何人。阁老说,大师兄是南海真人,我是三弟蓬莱阁老,这是我二师兄大漠仙人。方丈点点头,问道:“所来何事?”

问几句话,两个人明白了,这是中了断魂掌,跟小五子那掌不一样,大师兄的功力更深了。不单是忘掉过去,此时此刻还会持续健忘,说三句话就想不起来第一句说的是什么。方丈还在打听,阁老干脆不回答,四周看着,反正告诉了也是忘,你甚至都会忘记我曾对你失礼,拒绝回答你问题。

刚才满面春风那和尚像是掌事的,阁老扭头问他,能不能让我师兄弟二人见一面师父。高僧迟疑不决,方丈怒视道:“我连你二人所来何事都不知道,如何带你们见师父!”

高僧还是决定带他们去,方丈的罗圈话比车轱辘还圆,八光又不在山上,不遂了他俩心愿,两个老头能把少林寺拆了,改成武当紫霄殿。

过去的时候天黑了,要不是跟着一群和尚,脑袋反光,两个人都看不清脚下的路。藏经阁的院门没有关,仙人、
阁老往里走,藏经阁的大门也开着,两个十几岁小和尚分列两扇门前。自从八光和文思清下山以后,照顾沈老前辈的担子落在兄弟俩身上。站在门左侧的哥哥扬起右手,说沈老前辈吩咐了,二位就在这儿止步吧。

阁里没有点蜡烛,外面望进去一片漆黑,阁老低声确认,我师父在里面?这次哥哥弟弟差不多一起抢话,说在的,有什么话就在这里说吧。阁老抬起头,疑惑地看着黑洞洞的大门,这时里面传来沈老前辈的声音:“是宁肃告诉你们的吧?”

阁老愣了一会儿,跪地行礼。仙人站在后面,也跟着喊了声师父,没阁老那么激动,双臂撑地,弯膝下跪。沈老前辈失望叹息道:“谁让你们一起来的?”

当年师徒四人在藏南学武,三个弟子都是自幼被师父收养,自大到小起的三个名字是宁肃、静肃、霄肃,至于真人、仙人和阁老,都是三人后来自立的门号。二十四年前沈老前辈的《三藏经》被偷,一时查不出来是谁,索性将三个弟子全都逐出师门。宁肃往南,做了南海真人,静肃进西域做了大漠仙人,霄肃一直往东走到尽头,成了蓬莱阁老,从此以后都没有见过师父。分别那么多年,此时竟相顾无言。

仙人问,师父近来身体可好?沈老前辈说一百多岁的人了,身体好又如何,坏又如何。然而中气很足,听声音仙人还不敢在师父面前造次。阁老一直叩首在地不说话,反倒要师父主动询问。他说霄肃,你不是到处放话,一直想让我出来,问我几句话?阁老支吾了半天,说知道师父很好,我那些话不问也罢。

沈老前辈嗯了一声,转而夸仙人内力长进不少。仙人说这么多年,一直不敢忘记师父教诲。沈老前辈笑了笑,说你假意行礼,双膝没沾地面,一直靠臂力支撑,居然气息还能那么平匀,真是难得。既然被识破,仙人干脆站起来,承认师父果然洞察得仔细。阁老皱眉看着仙人,他反正是不会起来。

“静肃,你还在生师父的气?”

“不敢,”仙人说,“我既然是师父养大教大,师父自然也有权力将我逐出师门。”

“不是我赶你出去,为师当年忧虑的是《三藏经》,你们也知道,经书里面讲的是断魂掌、仙人掌、蓬莱掌,这三掌只要学会其中一掌,便足以称霸武林,为了不令某一个人胡作非为,你们师兄弟三人我各教了一掌,能让你们相互制衡,不至于有谁一家独大,成为武林的祸害。只是不知你们谁起的歹心,偷走了《三藏经》,要知道,谁学会这三
掌,随时就可以杀另外两位师兄弟。留你们就是在害你们,不要说江湖如何如何,我师门就先遭这不肖子弟灭门,所以我让你们出藏,往南往西往东,一直走到头才可以停,互相离得远一点,老死不相往来。”

仙人深鞠一躬,说弟子明白。师徒又说几句话,沈老前辈下了逐客令,说什么不早了,这是少林寺的地界,就不要打扰出家人清修了。阁老又磕了一个头,说师父保重,弟子明年再来看你。师父要他不许再来,不要逼得他再换地方。阁老面朝着大门退出去,就要走出藏经院时,沈老前辈叫住阁老,霄肃,我知道你要问我什么,你不必再找灵儿了,当年她的的确确是摔死了,灵儿的事是师父的错,师父这二十多年早该跟你说,师父对不住你了。

阁老站在门口,一下子就绷不住了,几乎是哭出来的声调喊师父,扑通一下就跪了下来。
\newline

{\centering\subsection{5}}

仙人带阁老来到扬州,说要给他看样宝贝。头几天住在那家有赌场的客栈,阁老输得精光,顺便帮仙人也输了不少。好像仙人有输不完的钱,前五十两银子刚砸进去,又有二百两齐整整地装在盒里,摆在阁老面前。三番五次,这回阁老不拿了,扭头把银子推回去,说你带我来扬州看宝贝,你要是想用千八百两银子,就把我九宫图换走,我还是趁早给你打个借条。

“银子而已,”仙人说,“就是三万五万两,也不叫宝贝。”

仙人让他放心去玩,先把赌瘾过足了,咱们再聊。可阁老不是小五子,武功那么高,看人摇骰子就跟瞎了眼似的,那种最次的羊牯都能把他赢得晕头转向。连玩几天,阁老自己放弃了,天生不是赌博的料。他给仙人写欠条,从你这借一万两有没有,算利息还你一万五!

仙人看着他笑,说其实借了你三万五,不过无所谓,数目你随便写。阁老倒抽一口气,仰头看大厅棚顶,三万五,买这赌场都够了。他咬牙写了个五万两的欠条,按过手印推过去。仙人拿过来,一字一句地读一遍,一扬手把欠条揉成了粉末。

说让你过足瘾,肯定是不要你的钱。仙人问他玩好了没有,我们换个地方去玩。他去门口看柱子上的仙人刺,阁老知道那是记号,以前师父教过,阁老一直没用着,师兄弟都是老死不相往来,自己又不愿收徒弟,不像仙人,二十多年收了百十来个弟子。好像仙人改了一些,左改成右,南改成北。约莫一时辰,两人到了一个客栈,门口一片银
杏林,吹起风来哗啦哗啦的。

仙人那个女弟子也在这里,二师兄一百多个弟子,阁老就记住这么一个女娃娃。他记得脸,不知道叫什么名字,好像是昆仑公子相好的。客栈环境不错,仙人干脆把整个客栈包下来,其他客人赶走,进扬州城抓来两个厨子,换着样地给阁老做菜吃。光有好菜也不下酒,仙人隔天又请来一个戏班子,生旦净末丑,一个不少地在台上翻跟头。

仙人问阁老怎么样,这里住得还舒心吗。阁老说好是挺好,但你能不能把那个女娃娃支走,她在我紧张,酒都喝不下去。仙人看着他桌前,菜没吃几口,筷子都被他扎到桌子里去了。

不是年纪大,早二三十年前,阁老就这样,见着漂亮姑娘,满脸通红,浑身不自在,不是掰筷子,就是怼汤匙。那时候师兄弟三人关系还不错,他大师兄,后来的南海真人老笑话他,说你就一直这样吧,没姑娘能看上你。没外人的时候,阁老还能据理力争,不行就跟大师兄打一架,可要是有外人在,比如大师兄后来娶进门的嫂子,阁老就结结巴巴地说不明白,干脆再捏碎两个勺。还好嫂子如月对他不错,给他台阶下,跟南海真人说,以后会有好多姑娘喜欢你三师弟的,他那个叫少年感,哪像你,看字画比看我还亲,不关己的事情,一点好奇心都没有。

几年后他真遇到一个能跟他两情相悦的女人,他克服紧张,每天都提醒自己说话别结巴,看她别脸红。后来有了灵儿,再后来一拍两散,那么多人,那么多事,像一拳打在镜子上,一下子全都碎掉了。

想起往事,阁老一时难过,一声不吭地连喝了几坛酒。喝到深夜,仙人让那个女娃娃扶他回房间。他指着夜路警告她,你别碰我,我能走回去。然后也不知道,走的是不是直线,只知道女娃娃一直跟在他后面,怕他摔倒,随时去扶他。走到分岔口,他停步左右看,哪一个更像回房间的路。他问后面的娃娃叫什么名字。女娃娃说她叫吴思若,三师叔,您往右边走就行。他点点头说好名字,嗯,好名字,然后摔倒在左边的路上。

第二天他状态好多了,再见着吴思若也不再浪费餐具,心情不错还上去唱了两段。他给女娃娃灌酒,盼着她能喝多,送她回房间,把昨天的人情还上。吴思若千杯不倒,结果他又喝多了,他记得走右边,记得走直线,结果上楼梯时还是绊了个跟头,被吴思若扶到床上。

天天这么喝,确实比在蓬莱阁好多了,但天下没有不散的筵席,吃喝饮用都是二师兄开付,也该识趣点告辞了。这天晚上他主攻仙人,一次又一次地跟他碰杯,他说
吃你的,住你的,赌的还是你银子,这些都是我欠你的,反正九宫图在我这儿也没用,所以你也不用拿什么宝贝跟我换了,我送你了。

“说一不二,宝贝还是要给的。”

仙人让吴思若去后厨加两个菜。阁老明白他是在支走吴思若,他看着仙人,仙人看着吴思若,仿佛在等她走远,才把宝贝亮出来。可是,好奇心啊,阁老还是忍不住催他快点,让他看看是什么宝贝。见仙人没反应,阁老干脆要伸手到他怀里掏。

“不在我身上。”

“那你放哪了?给师弟看看。”仙人笑了笑,转头又盯着吴思若。“她去拿了?”

“差不多,我带你来扬州看宝贝,已经看了几天了,居然还问我,宝贝是什么。”仙人指着吴思若的背影说,“她就是宝贝啊。”
\newline

{\centering\subsection{6}}

不可能,一切都太糟糕了,虽然你是我师父,虽然你养过我,救过我,教了我武功,可你也恶心了我七年,甚至恶心了我下半辈子。阁老离开后,仙人和吴思若谈了这件事。她明确跟师父讲,我是吴思若,不是芙蓉月,跟阁老生活一年,陪他吃一年,跟他睡一年,怎么可能提出这种要求?吴思若觉得不用再谈了。她收拾包裹,准备离开,她让仙人想清楚,不甘心的话,你就在这客栈把我杀了,一旦我走出房门,你我师徒恩断义绝。

吴思若说完转身继续收东西,她听见仙人在身后叹息,此时听来愈发假情假意。她背对着他,等他一掌打死她,哪怕是仙人掌,滴水不进,干涸而死,也胜过永远活在他的阴影下。可仙人舍不得杀她,包裹收好时,他站在门口说:“你走吧。”

出门的时候,仙人已经备好马车和车夫,他从车里拿出一个箱子,说为师一直想送给你,也算是师徒一场。她问他是什么,他示意她看看。她把箱子打开,莫名奇妙的一些小玩意儿,一些首饰,一些衣帽,但都是用过的。

“到底是什么?” “你仔细想想。”

肯定有点意义,她一件一件翻,没一样是她的,别人的,甚至都不算二手货,耳环只有一只,项链还是断的,但当她把耳环和项链摆在一起的时候,她想起来了,这是丹
姐戴过的。她拿起金镶玉的戒指,王妈一直戴在中指上的,一顶深青色帽子,那是紫竹院龟奴的。她把所有东西放回去,想了想,抬头问:“所有人你都杀了?”

“我知道你下不了手。”

不是下不了手,而是他们没有罪。说不出感激,也怪不着他,她合上箱子,原封不动地还给仙人,说既然过去了,就像流沙一样埋起来吧,她不用马车,给她一匹马就行。仙人牵出两匹,说为师送一送你。吴思若皱眉看他,摇着头。

“之所以让你来扬州,我带阁老来扬州,本来是要带你去个地方。”

她也不知道去哪里,仙人骑马走在前面,说不着急,扬州城往东三十里就到。吴思若在后面慢慢跟着,走出客栈的银杏林是一片稻田,秋收季节,田里的农民也多了起来。仙人感慨,要是当初没跟师父走,不学武,老老实实种一辈子田也挺好的,至少不像现在这样提心吊胆,机关算尽,拿你去换九宫图,这就是个幌子,我就是希望你在蓬莱呆两年,好知道他在练什么功,偷《三藏经》的人是他,还是你大师伯。

“我以为偷《三藏经》的人是你。”

“是我就好了,我是想偷来着,谁让我害怕师父,下手晚了。我们被师父赶下山,互相提防,离得越远越好,南海、大漠、蓬莱,三个人画了那么大的三角形,但那个人总有三掌练成,找到我的时候。一掌打死还好,拿断魂掌和蓬莱掌羞辱我,可是生不如死。所以早二十年,我就在准备礼物,投其所好,念在旧情能让我死个痛快。我爱钱,你大师伯爱字画,这二十年我帮他收集了好字好画,你三师叔是好色,还不是普普通通的淫魔,很腼腆的那一种,总要下点功夫,找到那么一个女人,美貌,聪明,年轻,甚至还精通房中术,彻底把他拴住。你以前问过我,为什么送你去紫竹院,因为打从你还是婴儿的时候,我收养你那天,你就是我想送给他的礼物。”

吴思若停下来,盯着他。貌似快到了,前面又是一片银杏林,仙人招呼她跟上来,看完你就走,相忘江湖,各不相欠。已是斜阳落日,阳光洒在银杏上,感觉全身都在被金光笼罩。林子深处有一个宅子,四进的大院。仙人下马,带着她穿过每一个院子。

差不多十几个人住在山庄,看到仙人鞠躬致意。他们身手都不错,仙人说,守得住我要给你看的东西。吴思若跟着他,左右张望,宅子走到头了,还没见到是什么东西。仙人在后门停下来,敲了敲上面的铁索。不一会儿,两个
中年人端着两个大盆,小跑着过来。一个是满盆的菜,另一盆是几百个馒头。端馒头的人打开锁,推开大门。

吴思若以为有什么,门外只是一片金灿灿的银杏林。脚下是往来多了,踩出的一条土路,仙人走出宅子,端馒头的回身端起铁盆跟上来。仙人问他,这两个月怎么样。他说都挺好的,没有死的。

“尽量都活着,我要用他们。”

几人在一片草坪前不走了,吴思若走近才发现,那是染了绿色的一张席子。仙人对她招呼,就是这里了。说完他把席子掀开,吴思若完全傻掉了。席子下面是一个百尺深的大坑,底下全都是人,上千人之多,衣衫褴褛,脸上混着血,混着泥,甚至还混着尿液粪便,一个个伸出手臂往上看。仙人冲装馒头的点点头,他端起铁盆把馒头倒下去,之后是菜,泼水一样撒进去,菜叶子浇在头上脸上,他们捡起来塞进嘴里,再去抓别人脸上的菜。

“这是什么?”吴思若问。

“我去紫竹院,帮你杀掉那些窑婆子窑姐儿,清理衣物的时候发现,你还有个本子忘在那儿,从第一个到第一千多个,多大年纪,哪的人,叫什么名字,你都认认真真地记下来了。我心疼你啊,这些都是从我们小月床上爬下来的,怎么能放到江湖上,任由他们瞎说呢?于是师父破费十万两,花了半年多的功夫,雇人把他们一个个都找回来,一千多个人,除了几十个死了的,都给你找齐了。”

吴思若看到了那个公子,她从丹姐那儿抢来的,半夜来敲她的房门,好像临走时还说她卖力气。有个老头很眼熟,人虽然不算胖,可是皮肤一层一层跟沙皮狗一样窝在一起,刚抢了半个馒头坐在地上吃。她想起来了,她的第一次,三百斤的老员外。

吴思若跑几步去银杏树下呕吐,吐过一次还是恶心,她手指抠着嗓子眼,胃里吐不出来,倒是眼泪止不住地往外涌。仙人走过来,轻拍她后背,让她注意身体,千万别死了。

“当然,更不能寻死。你死了,你那个小五子看见这些人,会很伤心的,帮师父做点事,只要你去和蓬莱阁老住上几个月,这些人我帮你埋了,就是了。”

实在吐不出来了,吴思若站起来,满眼泪水看着大漠仙人问:“为什么,仙人派那么多师姐师妹,长得好看的,比我聪明的,那么多女孩,为什么你偏偏选中我?”

“命吧,”大漠仙人又叹一口气,望着斜阳说,“有些人就是生来命苦。”

\newpage