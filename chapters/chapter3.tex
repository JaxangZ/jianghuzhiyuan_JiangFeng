\section{叁}

{\centering\subsection{1}}

他想找丐帮,把碗还回去,跟他们说三件事:一、你
们何帮主死了;二、向老帮主在皇宫大狱等着你们去救;
三、马长老是坏人,差点把碗抢去。可是哪有乞丐啊,人
生无常,丐帮还没找到,他先成了臭要饭的。

有钱的时候,南行八百里也不见一个要饭的,等盘
缠没了,终于硬着脸皮讨口饭,刚坐下来吃两口,丐帮的
人就出现了。七八个乞丐把他围成一圈,用竹竿敲着岸
边的石头,问他是哪堂哪会的。小五子吃着剩饭直摇
头。没堂没会,那你就是个臭要饭的,这里可是丐帮的
地盘。丐帮不就是臭要饭的吗?小五子听得直搓脸,仰
头问他们领头的是谁。一个二百多斤的胖子站了出来,
一脸的横肉,让人觉得真的是,跟着丐帮有肉吃。小五
子终于把那番温习无数次的话讲了出来,你们何帮主死
了,向老帮主在大牢里,马长老是坏人。后两句还没说
出口,领头的就摆手打断他,回头问他的几个弟兄:“谁
是何帮主?”

估计是新来的,别的地儿还不算,单在田独何员外
就隐居了五年。看他胖得不成体统,最多也就要过两个
月的饭。小五子换个问法:“丐帮现在谁说了算?”

“我们长老。”

“你们长老是谁,”

七八个人不是瞎子就是瘸子,在这个问题上,跟一
二三预备起似的异口同声:“马长老。”

小五子倒抽一口气,那完了,后面的话不用说了。
人家在田独没杀成你,你现在倒是追着人家往南跑。他
起身拍拍屁股,低头说那没事了。这时刚才的合唱团里
发出了不和谐的声音,一个瞎子弱弱地说:“我们关长老
说了算。”

小五子转回身,却看不到那个瞎子,以胖子为首的
一帮瘸子把他围住了,揪着他衣领,问他敢不敢再说一
遍。此时旁边几个瞎子仿佛被感染,说出了心里话:“我
们听关长老的。”

“我没问你们!”领头的胖子有点镇不住了,又揪揪
那个瞎子的衣领,“我让你再说一遍。”

瘸子围着他,几个瞎子又把瘸子围住了,大肠包小
肠,瞎子想了想,说:“关长老管事,不过在这儿,我们听
你的。”

死胖子算是满意了,转回来,瞪着小五子,看他再问
点什么。小五子也听明白了,无非是群龙无首,两个长
老相互夺权那点事,碗总要送到,希望关长老比马长老
好点吧。他长吐口气,先走一步说:“那就带我去见关长
老吧。”

这下他们不干了,大步跨过去。小五子愣了一下,
看来要饭也要凭本事,这些人根本不是瞎子、瘸子,一着
急全都跑他前面来了。领头的胖子这回腰板也硬了,刚
才几句话,差点让这小子带到沟里去。他带着头用竹竿
敲地面,威武升堂似的阿斥他:“你个臭要饭的,哪配见
我们长老!”

那就跟着他们,走在队尾吧\footnote{原文“走在队尾巴”},看先碰见哪个长老,随
时准备跑。走了几天小五子打听明白了,关长老在锦州
呢,这帮人就是往南走,要入关跟其他兄弟们汇合,一起
去昆仑山庄参加寻龙屠狼会。杀龙宰狼的事他不关心.
重要的是和其他弟兄们汇合,他相信那些弟兄里总有些
老前辈,哪怕不是关长老,但是瘦骨嶙峋的,一看就是要
了十年二十年的饭,受过何帮主的恩泽,到那时小五子
再去跟他们讲,何帮主死了,快去救向老帮主,老前辈们
抱头痛哭,跪下来接碗,感谢他是丐帮的大恩人。他要
的就是这个结果,不像这几个新来的,都腆\footnote{原字[典见]}着脸要饭了,
还要挑荤素。

一脸横肉的胖子姓胡,在队伍里面负责炊事,丐帮
居然还配厨子。白天他们边赶路边要饭,晚上一人端一
盆剩菜回来,堆在胡胖子面前,他把铁锅烧热,十来盆剩
菜一股脑倒进铁锅里,烩成一大锅。一人一大碗,菜品
齐全,营养丰富,里面有菜有肉,多吃两口还能发现面条
和饭粒。

小五子既不是瞎党,也不是瘸党,每天要得比谁都
少,可是饿得比谁都快,锅还没烧热呢,他先偷摸把何员
外那铜碗推到第一个。胡胖子把碗踢开,跟他说两条规
矩:第一条是,要不到饭就卖点力气,以后赶路他背锅
灶;第二条是,吃饭别再用碗了,你空着碗来,盛满了回
去,让弟兄们看见不像话。

那就像话一点吧,第二天小五子早早起来,把铁锅
绑后背上,等着队伍出发,胡胖子皱眉瞅了他老半天,指
着地上烧黑了的砖说:“锅是带了,灶呢?”

灶?灶太沉了,几十块砖捆结实了,扛肩上比一头
母猪还重。那也强过拉下脸要饭,最难受的是吃饭没
碗,还得排队尾,一人一碗零一勺,轮到他剩多少抓多
少,双手捧在脸上。几乎顿顿都是锅底的锅巴,硬得要
死,嚼起来青筋暴起,太阳穴跟着牙床疼。有一天他忽
然意识到,这似曾相识,他吃过这个,他以前过的绝不是
什么富贵人生,他是苦出身,说不定是吃着百家饭长
大的。

天天吃锅巴,又要干力气活儿,九月初五,他扛着锅
灶在山路上晕倒了。午后烈日,他躺在泥地上喘着粗
气,黑砖掉了一地,身后的大铁锅滚来滚去,一直落到山
下的溪水里。胡胖子下马过来,低头看着他,一大坨肥
肉在小五子脸上投下重重的一道阴影。他问他怎么样,
死了没有。小五子使劲摇摇头,可别把我活埋了。胡胖
子点点头,意思是很好,要死也别死在我们丐帮。他叫
人下去把锅捡上来,砖收好,捆在马背上继续赶路。临
走时还过来关心一下小五子,说我们先下山了,你在山
顶不要急,慢慢死。

对啊,他们有马载货。小五子仰躺着,听马蹄声踏
踏远去,丛林里的知了依然吱吱乱叫。想死也不容易,
太阳照得他睁不开眼睛,也不知道是晕死是睡着,合上
眼晴都是一片光芒。

傍晚下雨了,醒来的时候身上浇透了。小五子张开
嘴灌了几口雨水,他还不能死,文思清在田独等着他。
他大喊一声给自己打气,撑着双臂爬起来,迎着大雨往
山下走。饥肠辘辘,找到半山腰才看到几个果子挂在树
上,他抱着树干往上爬,下过雨太滑了,一次次地从树上
溜下来。后来他就靠树下,等雨停,把衣服脱下来拧干,
那时他已没半点力气,死活也爬不上去,他大喊两声继
续下山。

跌跌撞撞他下到山脚,远处有篝火,丐帮的人已经
安营扎寨,貌似又来了一队乞丐,合在一起百十来号
人。他不管这些,先坐在角落里候着,到开饭的时候混
到了队伍里。打饭的还是胡胖子,离老远见着他就乐
了,过去伸手示意他上前。小五子走过去,看到两个大
桶里分别装着鸡腿和馒头。胡胖子稍显浮夸地钦佩道,
你命够硬,来吃点热乎的,养胃。他竖起拇指,把两个铁
锅搬下去,从灶上端一口新锅上来,隔着锅盖都能看出
来里面热腾腾的。开盖之前他提醒小五子,老规矩,你
空着碗来,只能用手。锅盖都快被热气项开,小五子先
咽口水再咬牙,双手伸出去。掀开锅盖,一锅白粥刚煮
好,胡胖子用铁勺在里面舀了几下,一层层白气扑上來,
然后他抬头看小五子。

小五子手没有撤,双手拢成碗状等着他,一勺稀粥
浇上去,疼得他双手打战,把米汤沥掉,指间还剩个十几
粒米。胡胖子一脸笑意,说你先喝着,不够我再给你打
一碗。小五子盯着他,舔掉手上的十几粒米,掏出自己
身上的那只碗,说再来一碗。胡胖子对他摇着手指,提
醒他空碗来的不准用碗。小五子依然盯着他,将碗伸进
锅里,不紧不慢地舀了一碗粥,送到嘴前。

第一口还没喝到,胡胖子掀翻他的碗,整碗粥都扬
到他脸上。小五子左手抹着脸上的米汤,右手掐着碗
边,瞪着眼睛呼了胡胖子一个铁巴掌。血从耳根顺着
脸颊淌下来,胡胖子抄起铁勺朝小五子的太阳穴凿
去。小五子向后躲,铁勺还是一下子将他鼻梁抽折
了。鼻血进出来,混着脸上的米汤,一时间满脸是血。
从来都是这样,屁大点武功没有,但是贱命一条,小五
子不要命地往胡胖子身上扑,揪着他头发把头往盛粥
的铁锅里按。

场面乱了起来,近一些的弟兄们跑过来帮架,其实
是趁乱偷锅里的鸡腿。那些还在等鸡腿的乞丐看出端
倪,一个个从远处跑过来,将小五子、胡胖子和一大锅鸡
腿结结实实地围成了一个圈。架不能打得大快,一刀捅
死小五子,谁都别想吃鸡腿。在里圈吃饱了还出不来的
乞丐,一边拉着胡胖子要杀人的手,一边扒小五子的衣
服,一件件衣服从圈里扔出来,小五子始终死攥着铜
碗。大家使劲掰开他的手,将碗夺下来,扔出人圈,在空
地上转了几个圈,扣在地面。

先跟丐帮汇合的是关长老,晚饭都没吃他就找地方
休息了。不知道是乞丐做惯了,还是修身养性,这几年
关长老都是过午不食。他二徒弟帮他在两里之外找到
一间破庙,前后无门,四处透风,他让二徒弟早点回去,
摸黑找到一个避风的凹槽睡了一觉。听见打架他也没
急着赶过来,丐帮嘛,天天都有冲突,伙食不好抢粥,伙
食好了抢肉。只是醒来后,他发现这凹槽原来是菩萨盘
起的两腿间,他向上摸去,摸到菩萨的脸。这让他吓出
一身冷汗,连滚带爬地跪在地上磕了三个响头,求菩萨
恕罪,我关震有眼无珠,有失体统。然后他冲着菩萨,举
起右手的食指中指,抠进自己的眼睛给菩萨看,自己的
确是有眼无珠啊。

他离开寺庙,循着打架声,跌跌绊绊走回营地。也
许管不了这么多了,只是别让途经此地的武林中人看笑
话罢了。这次去关外,听说何府灭门,他围着田独方圆
百里都找遍了,也不见何帮主及向老帮主的踪影。倘若
他们死了,最担心的还不是丐帮散了,百年江山也有易
主的时候。他在忧虑谁来继任帮主,换几年前,哪怕那
时他已经眼瞎,但总还年轻几岁,也不会将帮主之位让
给马长老,任由他一人独大。而现在不行了,两位帮主
既死,他关震被马长老杀了倒也不足惜,可丐帮在他手
里,早晚会发展为邪教。

里面人群在扎堆,不断地往外扔东西。他喊了两声
住手,无人应声。外圈几个正往里拱的弟子回头看看
他,欺负他眼瞎,悄没声都绕到人圈的另一侧,继续往里
挤。他清楚自己就是个废物,那些弟子之所以还喊他一
声师父,其实只是学学怎么装瞎子讨饭吃,除了一点内
力,已经没有本事再教弟子们了。他心中有个盘算,在
与马长老会师前,将既有的丐帮解散。那就让他们打
吧,这种想法大逆不道,但打散伙了,总比跟着马长老作
恶强。

他仰着头,蹚着碎步,用拐杖扒拉地上的东西。被
扯烂了的长衫摊在地面,几根鸡骨头杵在上面,不知道
谁吃的,还要发着内力往地上扎。关长老用拐杖拔了
两下,鸡腿骨横空断开,下半截依然插在地下。往前走
几步,好像是靴子,挑起来却不见鞋底,靴筒将拐杖套
个半截。任凭他们打吧,别让外帮外派看见就好。他
缓慢转头,辦别风向,迎着晚风打算回寺庙。大概走出
三步,踢到了一个铜碗。丐帮沦落如此,他心中苦闷,
抬起拐杖朝碗底戳下去,铜碗纹丝不动。他皱了皱眉,
难不成这双眼瞎了,功力竟退步如此?他用拐杖尖在
碗边滑了一圈,深吸一口气,发力向碗中砸去。一声闷
响,似乎地面都已经凿出一道细纹,而这只碗却全然没
有破裂的清脆。

他蹲下来捡起碗,在碗底摸到嵌在里面的玉,这一
下去就没再站起来,单膝下跪,将碗举在头顶,朗声道:
“丐帮关长老叩拜帮主!”

声音在山洞回荡几次,乞丐们全部停住,仰头张望
一图后一个个开始往外挤,人圈变大变疏,里圈的胡胖
子也松开手,捂着已经血凝的耳朵看二当家的,最后是
小五子,几乎都快被扒光了,从盛满鸡腿的铁锅里站了
出来。
\newline

{\centering\subsection{2}}

这是个真瞎子,小五子见过,他是好人。关长老要
和他单独谈谈,丐帮弟子原地驻扎。他们往山里走一
点,站在瀑布脚下。小五子觉得该说了,我见过你,以前
在赌场救过我一命。关长老点点头,原来是你啊,他问
他碗从哪里来的。小五子说,何员外给他的,他只是田
独一个卖肉的,何员外办丧事他送头猪过去,,正好赶上
何府灭门,何员外临死前把碗给他,要丐帮在腊八之前
去京城大牢送给向老帮主。说了几句小五子停住了,一
路上想了那么多遍,重要的一点居然没想过。何府灭
门,碗是何员外给的,那何员外是怎么死的呢?

他得换个说法,不是说两个长老争权吗,顺水推舟
做个人情。他说:“碗差点被马长老掠走,我拼老命抢回
来的,现在我给你,你来做帮主,丐帮千万别落到马长老
手里,你就说是何帮主传给你的,需要的话,我去给你在
马长老面前做个见证。”

这样就好了,首先你们何帮主是求我杀的他,再就
是,你万一查出来是我干的,还有用得着我的时候,不至
于杀了我。全是要饭的,也不图你封我个一官半职,留
我条小命就好。

关长老不说话,也不知道想什么呢,老眼昏花的,没
准就这么站着睡着了。小五子额头上直冒油,拽起袖子
抹一下,感觉更油了,他扯一绺头发到前面闻了闻,桶里
呆了那么久,哪哪都是鸡屎味。小五子跳到瀑布下面的
池子里,在里面把衣服脱掉。冷水从悬崖扑下来,把他
浇个通透。

上来时关长老还站在原地,瞎就算了,话还那么
少。小五子把衣服尽可能拧干,一件件穿上。碗就在他
脚边,小五子把碗捡起来递给他,说消息我也传到了,这
里先恭喜关帮主,以后要是马长老不信你,尽管去田独
找我做证。

关长老不接碗,直接抓住他手腕,面冲着瀑布说道:
〝我现在不行了,就算我做了帮主,马长老一样会杀了
我,篡夺帮主之位。”

小五子使了半天劲,手腕挣脱不开,说:“他本事这
么大,就让他做帮主啊。

关长老摇头,松开他的手,指着小五子的方向说:
“找到向老帮主之前,你来做帮主。”

小五子连往后闪,连摆着手说不行,我被他打过,他
弄死我更容易,帮主还是他的。

“你死了再想办法,不能让他那么快得逞。”

小五子倒抽一口冷气,对关长老吼了起来:“杀死我
也就三秒的事,能争取多少时间让你再想办法?要不你
现在就弄死我得了。”

关长老不杀他,也不让小五子了,提着小五子的肩
膀往营地走。转过一个山坡他安慰道,你不会白白送
命,你是丐帮帮主,他若把你杀了,我自可以号令天下来
讨伐逆贼。小五子脑袋嗡嗡的,他没想这些,他还在想,
刚才那么大的瀑布声,怎么转个山坡,就什么都听不
见了。
\newline

{\centering\subsection{3}}

关长老眼睛虽然瞎,吹牛的功夫倒是一等一。大清
早把关外的丐帮弟子集结起来,让小五子站旁边,就听
他往大了吹。他先说咱们何帮主仙逝了,向老帮主生死
未卜。说完他连叹三声,这帮弟子也不解风情,没一个
掉眼泪的,瞪大眼睛等关长老往下说。也难怪,都是新
来的,个个肥得流油,吃得比盐商富贾还好,只知魏晋,
不知有汉。关长老又叹三声,哀其不争,那就使劲地吹
呗。他先说,诸位也不必太难过,我旁边的这位吾先生
是丐帮的新任帮主,向老帮主的关门弟子,武功可比他
师哥,我们何帮主不知强到哪里去了,向老帮主叱咤江
湖数十年,到老了可将一身的武学尽数教给了他的关门
弟子,即使是老夫,我二人昨夜在瀑布下切磋一番,也全
然不是吾先生的对手。关长老捅捅他,要他跟大家打个
招呼。

小五子有点走神,他一直在跟自己新换的衣服较
劲。以前听说过,丐帮没什么好行头,地位越高穿得越
寒碜。可是这帮主的衣服实在是大邋遢了,打几个补了
也就算了,可这颜色搭配得他一时灵魂出窍,灰衣服上
打个绿补丁,绿补丁上又嵌了两个红补丁,前襟还甩出
两道彩虹色的百褶,往合上一站,就像掉了毛的孔雀。
关长老让打个招呼,他好半天才把手从翻毛袖口里伸出
来,勉强作了个揖,低声说:“诸位弟兄好,我是你们帮
主。”

下面的弟兄直摇头,胡胖子那几个带头嚷嚷起来,
怎么这臭要饭的成我们帮主了,昨儿连个鸡腿都抢不
着,今天就什么关门弟子?要选帮主,也是马长老在场
一起商议!说着他提着菜刀就往上冲,后面几个也都举
着枪棍跟着他上。

小五子想往后躲,什么事啊,马长老没见着,先让这
几个死胖子给宰了。大步刚跨出去,被关长老提住后衣
襟,跑都跑不掉。他听见关长老在后面说:“几个逆徒,
任由他们胡闹,你先扶我下去吧。”

可是往哪下啊,人家从两头堵过来,左边的胡胖子
上得快一些,第一刀就下死手,直奔他面门砍去。慌乱
之中小五子只能伸手挡脸,关长老在后面轻推一下,小
五子脚步一乱,右手正好抓住胡胖子的手腕,左手上去
一抄,夺下他的砍刀。他把刀换到右手,朝胡胖子腰上
横劈,劈到一半被关长老收力,感觉左腿腾空,一脚踹过
去,上来的又一个弟兄被他蹬到台下。后面几个举枪举
棒的,跑到一半见情况不妙,当场就跪下了。

小五子这下明白了,老家伙自己不出头,拿他遥控
着打可利索了。有人撑腰就好办多了,小五子拍拍前襟
的彩虹百褶,让那些人跪成一排,提着菜刀走过去。关
长老在后面“扑通”一声跪下了,求吾帮主手下留情。小
五子回头看看,这也大浮夸了吧,再转回来,前面一百多
个弟兄全都跪了。那还挺好的,他把菜刀扎地上,拍拍
手说,自己不是不出手,被自家弟兄打两下又怎么了,他
跟向师父学了一身的功夫,出手即杀招,伤了自己兄弟,
我心里难过,当务之急不是谁强谁弱,而是应该万众一
心,保我师父练成无为神掌,待他腊月初八大功告成,我
就是把我这天下第一的位子,还给我师父又何妨?

好像有点过,小五子转身看关长老,想怎么把天下
第一的大话收回来。奇怪的是,下面也没人质疑,等了
片刻响起的是一片欢呼声。也是,丐帮势弱这么多年,
不管是真假,起码有五十年没听过“天下第一”这几个字
了吧,就当他是真的,就让大家高兴这么一回。
\newline

{\centering\subsection{4}}

头几天还蛮威风,快入关的那天他忽然有点难过
了,可能入了山海关,就彻底离开田独了吧?他问自己
当初为什么南下来着,给丐帮送碗是一个,再就是看看
百花谷是个什么地方。那个苏子瑶,不是喊他少谷主
吗?他现在不想去百花谷了,那个少谷主不做也罢,百
花谷,听起来就是妻妾成群的地方,没谁他这个少谷主,
以前就是江湖第一大淫贼。百花又怎么了,哪怕个个貌
美如苏子瑶,也不及文思清的一根头发丝。

人是会变的吗,倘若过去他真的万恶淫为首,如今
怎会毫无兴趣?也许是因为遇见了文思清吧。他开始
想念文思清了。仿佛被蚊虫咬的包,一阵一阵地想她,他
想她做的精致饭菜,想她同房的另一张床,想她一旦有
姑娘和他多说两向话,跳出来骂街的样子。连带着常公
公他都想,跟老太监死了似的,他说的每句话都记得真
真切切,有一句说什么来着?要时刻小心,江湖起码有
一半想杀你的人。应该是吓唬他,怕他大胆跑出田独,
他小五子哪有这本事。但多少得注意,用不着一半,哪
怕仇家就一个,他又没记性,还当是新交的仗义朋友,聊
着聊着突然来一刀,含着眼泪骂,我平生娶了五个老婆,
六个小妾,全他妈被你戴了绿帽子,快拿命来!

没准,百花谷少谷主嘛。那得化化装,小五子对着
铜镜盯了好半天,哪里会易容术,真要是世仇,不都是说
化成灰我也认识你吗。干脆就弄点炉灰,和水糊在脸
上,已然都是要饭头子了,破罐破摔吧。关长老眼瞎,看
不出来,下面的弟子开始一愣,以为这是防晒的土法,再
走两天,他们的吾帮主长什么样都不记得了。

帮主他也不想当了,尤其给一帮要饭的当头儿。浩
荡大军沿着山路前行,回头一看全是破衣烂衫的乞丐,
锅碗瓢盆,残羹剩饭,比他妈士匪还寒酸。关长老说,到
了关里人就多了,到时候我一家一家给你介绍,让整个
江湖都知道,丐帮的新任帮主是你小五子,而不是马
长老。

那我死得更快了。入关头一天晚上他睡不着,在山
海关客栈想着如果明天他就死了,今晚要和文思清说点
什么。别等我,把我烧了,骨灰就放你娘那盒里面,好让
你成天抱着,但别忘找个板隔开,跟你娘水乳交融的算
怎么回事。想着想着下雨了,马棚里传来扬蹄嘶叫的声
音。丐帮就这一匹马,有时候他骑,有时候关长老用,大
部分时间都是他的,走两个时辰腰酸背痛,关长老不想
别人看出来,这届帮主内力不行。

他从窗户跳下去,还好雨够大,岗哨的乞丐都撤了,
跳下去那么大声都没人听到。他踩着泥水把马牵出來,
上马的一瞬间,他觉得他经历过这场景,当时也是夜里,
大雨,偷偷把马牵出来,急着去见一个人。见到那个人
了吗,是谁呢,苏子瑶?他晃神几秒,这一次不能再错
过,跳上马背扬鞭而去。

他往北跑,都是来时的路,越往北雨越大,每一脚都
踩出一个泥坑,跑出半个时辰,一个趔趄把马跑翻了。
他全身趴在泥汤子里,侧起头喘了两口气,灌进一大口
雨水,努力爬起来。马早就不见了,他抓着树枝自己走,
每一脚都要好大劲把靴子从泥里拔出来。真够讽刺的,
丐帮衣服没法看,倒是配双上好的靴子。后来他把靴子
系腰上,光着脚走。

大路积水更多,他拣小路走。前方一片影影绰绰的
光点朝他这边来,侧身退到树后,趴地上等着都是什么
东西。

有队伍在前行,穿得花花绿绿的,那些光点都是小
伞,每人打着一把小伞,伞那么点大,不撑在头顶,全都
打在胸前给手上的什么东西挡雨。人人手上都捧一个,
走过五六排,小五子才看清楚,手上捧着的都是仙人
球。他大概知道,这些人是谁了。

“他们有多少人?”

小五子吓一跳,关长老就在他一米多远的地方,跟
他一样,也趴在泥浆里。苍蝇飞我兜里了吧,你个瞎子
能跟我这么远?小五子不想理他,脑袋耷拉在肩膀上。
红男绿女陆续从小路走过,也没多少人,队伍虽然挺长,
但是道窄,每排也就三五人。

“帮主大老远顶着雨过来,是仙人教里的哪位朋友
要见吧?”

哟,帮我找台阶呐。我谁也不见,就是不想在你们
丐帮呆了,不想给你当挡箭牌,跑不掉我认裁,弄死我,
就地把我埋了得了。他盯着队伍,想过冲出去会怎样,
反正都是死,仙人掌他中过,不吃不喝,滋味更不好受。
队伍走了一大半,后面的断断续续,有的隔百十米才又
上来几个,倒是有要见的,黑灯瞎火的也看不出谁是谁。

“有个叫吴思若的在仙人派,你把她弄出来见我。”

“吴思若是谁?”

小五子抹抹脸上的雨水,说:“我朋友的女儿。”

“哪位朋友?”

“何帮主的女儿。”

“何帮主姓何。”

他编不下去了,你不是会找台阶吗,再给我找一个,
何帮主的女儿为什么叫吴思若,找好了我还是你帮主,
给你当傀儡。人走得差不多了,最后几个人过去十几分
钟,也不见有人经过。关长老从泥浆站起来,走近把小
五子提了起来,恳求道:“先回去吧,帮主,属下明日帮你
办成此事。”
\newline

{\centering\subsection{5}}

小五子最初听到的是江湖三大高手,大漠仙人、蓬
莱阁老和南海真人,到关长老这儿变四大高手了,他们
的向老帮主也算一个。估计少林的方丈,武当的道长,
在他们版本里面都有自己的四大高手。三个老怪物是
毋庸置疑的厉害,仙人掌、蓬莱掌和断魂掌,他亲眼见过
这令人发指的阴毒。

天一亮就可以进关了,守在山海关口关长老让人回
去打探,仙人教行进到哪里。回来的消息说,他们还有
两个时辰就可以到关口。关长老说再等一等,他让胡胖
子准备午饭,说大家吃饱了再入关,别让人以为,咱们这
些关外的是饿死鬼投胎。

可是早饭还没打完,锅还没腾出来呢。胡胖子一肚
子气去劈柴烧火,丐帮三堂十六会,大小也是个会长,吾
帮主上任,他彻底论为一个厨子。雨下一夜,木头都是
湿的,胡胖子趴在炉灶前扇风吹气,熏得直流眼泪。他
爬起来揉眼睛,想着跟他们拼了,这时关长老却大哭起
来,双手拍地号啕,说向老帮主,我对不起您,丐帮对不
起您,女儿托付给我们,却由她入了邪教。

那就先不动手,把耳朵留着听,身子趴下去继续往
炉灶里吹气。小五子开始也是一愣,听了几句知道他唱
哪出了,过去低声提醒他,不是向老帮主,是何帮主的女
儿。说完他站起来嘴上大声问:“关长老,为何如此难
过?”

这回他改过来了,说何帮主年轻的时候风流不羁,
情事不顾,跟心爱的女人,黑苗五毒教主的女儿吴玲,在
外面留下一个私生女,几经辗转,女孩已随母姓改为吴
思若,谁知她长大却误入歧途,加入了仙人教,何帮主临
死之前曾嘱咐他,一定要找到这个女儿,留在他师弟吾
帮主身边,代他严加管教,去一去她身上的邪气。

“那向老帮主呢?”众人还在等,他到底是怎么对不
起向老帮主的。

关长老停住不语,他编不下去了,一个劲地摇头,说
至于各种情由,我们吾帮主是再了解不过了。有帮主的
指示最好不过了,他们又扭头看着吾帮主。小五子后退
两步,上了个台阶,面对丐帮众人,清清嗓子说:“我师父
同为天下四大高手,当然不惧怕什么大漠仙人,只是他
们仙人掌的刺着实令人讨厌,况且两派交好,能不正面
交锋,就不要拼个两败俱伤,还请各位出些计策,让这个
姑娘落单,把她救出来。”

能有什么计策呢,这些人天天残羹剩饭,脑子都不
大好用了。大家闭眼冥思,只等着开午饭。有个进来没
多久的,脑子还在的,举手说他倒是有个办法。胡胖子
大老远咳嗽一声,意思说你是我的人,好主意要留给我
来出。新来的过去跟他耳语几句,胡胖子放下铁锅大
勺,过来说:“帮主,仙人教是实实在在的邪教,你想要哪
个人,准备好银子,问问她身价,过去买就是了。”

这也大邪了吧?小五子左右看看,除了闭眼睡着
的,没一个有他表情这么惊讶。那就是真的了,他指指
胡胖子,让他接着往下说。胡胖子说,江湖本来就不好
混,他们又不能像我们这些丐帮,肯低下头要饭乞讨,这
么多教徒,多大一笔支出,不赚点银子能撑得百年大派
吗?大多数门派维持生计的办法,就是收些年轻弟子,
把他们练出一些功夫,明码标价地卖给镖局,或者是达
官贵人做侍从。

如果只是钱,那就好办了,小五子让人把银子都兑
出来。关长老死活不干,这点银票都是十文二十文攒下
来的,就怕是哪地广人稀,要不到饭,好拿来买干粮。前
任盖个员外府没问题,我买个人都不行,真当我是傀儡
帮主。两任帮主的嘱托,小五子口气不禁凛然,为了丐
帮的复兴大业,从即日起开始一日一餐运动,不吃最好,
直到把这笔亏空填补上。

跟着马长老最高能混到哪呢,堂主,副长老?可眼
前的这位是帮主啊,胡胖子终于想明白这一道理,鞍前
马后地伺候小五子。一直等到黄昏,仙人教才稀稀拉拉
地过来,五十多人能分成四十多排。估计大漠仙人不
在,远望过去,打头的是吴思若那师姐。小五子问,要不
要躲起来。胡胖子摇头道:“躲什么,咱们丐帮光明磊
落,我去给您打头阵。”

胡胖子迎上去,挡住仙人教的来路,和师姐一番交
涉,回身喊着有请吾帮主。小五子踩蹬上马,确定自己
脸上的炉灰还在。师姐己然认不出这是店铺的伙计,双
手作揖说:“仙人教见过丐帮帮主。”

两队人马停在路上,后面的教徒陆陆续续眼上来。
小五子看到了队伍里面的吴思若,还是那身白衣。这一
点小五子在田独就想不明白,老是那些红衣少女、绿衣
公子,行走江湖,他们衣服都不洗的吗? 胡胖子问他是
哪位,还不等小五子指给他,就挥舞着手臂说:“今日两
帮偶遇,我们帮主一时仁心大发,想买你们一个教徒收
为弟子,不知哪位有这个福分?”

大富人家,达官显贵,卖哪不好,偏要卖丐帮。生怕
被挑中,仙人教的人都低头不敢对视,这时人群里一只
没有手的手臂举起来,高喊着,买我买我!师姐回头呵
斥他:“早说你这一只手卖不上价钱,还是乖乖地跟我
去,看是被师父处死吧,还是罚你下半辈子伺候我!”

抢猪爪这小子等会儿再说,小五子指指吴思若,问
这位姑娘是什么价钱。师姐满脸笑意,奉承他,吾帮主
好眼力,一眼就能看出这是个姑娘。小五子接不上话,
长女相,穿女装,当然是姑娘。

“那可未必呢,”她指着关长老说,“像这位长老,就
不一定看出她是个姑娘。”

他眼瞎,还看不出这是个人呢。小五子懒得理她,
想他们长期在大漠,乏于交际,一两句寒暄都让人尴
尬。胡胖子去跟他们讨价还价,一张口就二百两,比买
栋楼还贵,好说歹说杀到了三十六两。师姐收过钱,将
吴思若拱手奉上,问小五子要不要买个保险。这是什么
东西?师姐解释,买了我们教的人,如果她跑了,或是伤
了你,我们负责把她抓回来,惩罚过后奉还给你。小五
子问,保险要多少钱?

“一百六十四两。”师姐回答他。

那不还是二百两吗?小五子差点笑出来,摆手说不
必了,我的人我自己会管好。师姐把银子揣好,忽然喝
令一句,吴思若,归队!吴思若一跃,又回到了他们队伍
里。小五子带人正要上抢时,仙人教已个个扬起右手,
左手捧仙人掌防御。

那算了,小五子叫胡胖子数出一百六十四两。这
时师姐又谈条件了,她说一百多两不够,要二百两,你
花三十六两买人,没买保险,人给你了,她又回来了,刚
才的买卖清了。小五子瞪着她,倒抽一口冷气。关长
老在旁边说,我早提醒过你,他们是邪教。

“拿二百两,”他让胡胖子数钱,“不过这一只手我也
要,连人带保险,这两个人都是我的。”

一只手乐了,欢天喜地自己往这边走,要饭怎么了,
不会死在师父掌下,也用不着伺候这母夜叉几十年。

听起来可以,谁让他出去搞七捻三的,真到师父那
里也不一定保他活命。银货两讫后,师姐还是好奇,要
他这个废人做什么?小五子双手合十转着手碗,说自己
在练五脏俱裂掌,苦于找不到活体做实验,既然你这有
个半残,早晚要被你师父处死,就拿来给我练一练。

师姐愣了半晌,作揖别过,行至尽头还回头看了两
眼。一只手用仅存的一只手抓向师姐的方向,痛哭流涕
地不让他们走。小五子让一只手放心,好容易找到你这
么个活人,怎舍得一掌把你打死了,要慢慢来,青蛙用温
水煮,味道才最美。说完他拍拍一只手的胸脯,问他这
两掌感觉怎样。一只手瞪大着眼睛感受心肺,说胸口
闷,心慌。

那就对了,小五子点点头,吩咐关长老收拾行李入
关,这两人虽是买了保险,但也要看住了。“至于你,”他
冲吴思若笑了笑,“晚点到我房间里来,我细细跟你讲,
你的身世。”
\newline

{\centering\subsection{6}}

进到关里就热起来了,田独三年,他都不知道出汗
是什么滋味。一切妥当,到晚上他纠结起来了,是先调
戏一只手还是吴思若呢。银钱掷了三次都是一只手,他
还是想见吴思若。关长老比他还犹豫,这么晚把何帮主
的女儿送到你房间里,不合适吧?

“她是我师哥的女儿,关长老何出此言?”

关长老想了想,皱眉道:“可何帮主不是你师哥啊。”

“她也不是何帮主的女儿啊,不都是你编的吗?”

入戏大深,关长老想半天才反应过来,退身出去,说
我这就送吴姑娘与帮主同房。

“把她仙人球收掉,既入了丐帮,就得学我们丐帮自
己的功夫。”小五子还在戏里呢,“先把她捆起来,若是她
反抗逃跑,我武功这么高,怕是一出手杀了她。”

小五子想想先不洗脸,一直等吴思若进来。她是绑
好了蹦着进来的,关长老说声,帮主请慢用,在外面把门
关好。老家伙够不正经的,小五子拽张桌子从里面项住
门,走回来打量吴思若。还是那身白衣,这不行,入我丐
帮,就得穿我丐帮的衣服。小五子说着去解她衣服,可
是五花大鄉,袖子都拽不下来。吴思若看他忙乎半天,
说我这衣服穿几天脏了,跟你们的不就一样了?

说得有道理,那就进行第二步,小五子拉开桌子,将
门露出一条缝,让人把仙人球送进来。巴掌大的花盆,
他捧着仙人球在她脸上比画半天。吴思若始终盯着仙
人球,就快扎到时尖声叫道:“吾帮主,要是你对我有什
么非分之想,我宁可咬舌自尽!”

小五子退后一步,鼓励她咬吧,长这么大老听说咬
舌自尽,还没见过谁咬呢,咬吧!吴思若使了半天劲,脸
憋得通红,终于口齿不清地说:“我咬到舌头了。”

小五子有点失望,天真,还真以为她能吐出半截舌
头呢。小五子又拿起仙人球,问她是划左脸,还是划右
脸。吴思若说容我想想。等得小五子都犯困了,她还没
想好。小五子打个哈久,说我惯用右手,就先划左脸
吧。眼看刺就要贴到脸上,吴思若高喊:“且慢!”

又怎么了,小五子看她还能玩出点什么花样来。

〝帮主白天不是说,要细细给我讲,我的身世,我父
亲是谁,我母亲是谁,我堂兄堂弟是谁,我表姐表妹是
谁,一定要细细地讲。”

你的身世,要我告诉你?小五子翻眼皮回忆了半
天,虽然是编的,也得把话说圆了:“你父亲是丐帮帮主
何振生,情事所扰,爱上了你的母亲,黑苗教主的女儿吴
玲,但是他有家室,出于大义不能陪在你和母亲身边,后
来你就随了母姓吴。”

“原来是这样!”吴思若眨着眼睛问,“我父亲有老
婆,为什么我母亲还会爱上他?”

小五子左右看看,实在编不下去了,端起仙人球凑
到她面前。

“等等!”

这都第几回了,他放下仙人球,再等最后一回。

“为什么呀?花二百两银子,就为了扎我两下?你
就是脱我衣服,我也不至于这么蒙啊?”

那我就告诉你,他喊人打盆水进来,把脸上的炉灰
洗掉,露出真面目,反问她:“你说呢?”

吴思若看了他半天,摇摇头:“不知道。”

你他妈不记得我!鼻子一酸,一时间想哭的心情都
有了。他继续暗示她,往北边到过哪儿,田独去过没有,
在哪儿抓的一只手。吴思若一直在摇头,小五子原地打
着转,我就这么不起眼?他去照照铜镜,脸上一点泥都
没有了。又捧回仙人球,他问有双胞胎姐姐妹妹吗,还
是中过断魂掌。吴思若依然不解,摇头。小五子长叹一
声,将仙人球对准她颧骨,说:“希望你以后记得我吧。”

“啊啊啊!我想起来了!你是卖猪肉那伙计!我还
以为你早死了。”她兴奋起来,蹦了两下说,“快快快,快
把我解开,快给我讲讲,你个杀猪的,怎么就当上丐帮帮
主了?”

小五子还站在原地,吴思若冲他耸耸肩,让他快点
儿,你是丐帮帮主,你怕什么呀,好好给我讲讲,杀猪卖
肉的,怎么就成吾大帮主了?
\newline

{\centering\subsection{7}}

醒来时他发现自己在地上,隔着裤子掐一下自己大
腿,还疼,还活着,然后他问自己三个问题,我是谁?今
年多大?我最爱的女人是谁?第一个答案他不知道,继
续当他的小五子吧,第二个他不确定,是常公公告诉他
的,嘉和三年生人,五月初七的生日,今年二十五岁,唯
有第三个千真万确,他很想念文思清。

那就没有中断魂掌,浑身酸疼,此时正躺在地上。
床上还有一个,好像还有点细细的鼾声。他努力爬起
来,站在床边,抓起枕头旁边的仙人球。床上的姑娘早
已松绑,嘴唇微张,仰面熟睡。小五子举起仙人球,想着
数三个数就把这玩意儿落下去。数到第十六时,昊思若
睁开了眼睛,怒视他:“你敢!”

不敢。我要是敢,你今早都不用洗脸。一整天他都
难妥,行进时小五子一直霸着那匹马,关长老倒也理解,
只是提醒他,虽是洞房花烛夜,但一定要注意身体。可
能是那么好的事吗,小五子回头看几眼吴思若。都知道
这是帮主房里人,也没人敢看着她,就剩一只手鞍前马
后地求她说说情,五脏俱裂掌害人害己。他们背着阳光
行进,一时间看得他眯起眼睛,到底是怎么了呢,感觉被
醋泡了一夜,从头顶到脚尖都酸得要死。

丐帮今天也不行,没走多少路,就一个个嚷着饿。
安营扎寨,胡胖子给帮主开小灶,把吴思若拉过来,俩人
面前摆了四个菜。小五子说,你先吃吧,我还不饿。他
打两个饱嗝,起身看帮里的弟兄,都在狼吞虎咽,太阳从
西边正缓缓下落,原来他们走了一天。他明白了,走回
来夺下吴思若手中的碗,低声质问道:“你他妈又给我一
掌仙人掌?”
\newline

{\centering\subsection{8}}

一整天都食不下咽,早早地进了帐篷休息。睡又睡
不着,脑子空空地看着帐篷里的两只蚊子。关长老过来
通报,何帮主的女儿求见。小五子腾地一下坐了起来,
问她又来干什么。

“她说,还要跟你打听她的身世。”

小五子赶紧起来穿衣提鞋,嘱咐关长老:“跟她说我
不在,一会儿把一只手给我带过来。”

排不了毒,总得把一肚子气排出去。他走出帐篷,
丐帮的弟子们就地靠在树下,或纳凉,或酣睡。小五子
穿过人群,于不远处找到一个山洞,外面没有老虎,里面
没有文思清。他在黑暗中静坐了一会儿,让人把一只手
带进来。

行走了一天,一只手也没动逃跑的心思,跟得比谁
都紧。他坚信,既然这世上有五脏俱裂掌,没准就有五
脏俱愈掌。一进门他便跟小五子讲:“帮主如果想拿我
练五脏俱裂掌,我建议啊,还是等到了京城,找个郎中先
络我检查一遍身体,到时候帮主再拿我练掌,才知道这
掌法有没有练到家。”

有道理啊,可是这五脏俱裂掌都是没有的事。小五
子摸着黑给他号脉,一分钟里东想西想,最后放下他的
手腕,说:“你五脏没问题,咱们开始吧。”

小五子看不到,听声音应该是扑通一下跪下了。就
先聊聊吧,他问一只手,那只手哪去了?赌场里被一个
杂种给剁了。哎哟,也就背后那点能耐,手都被剁了,敢
当面骂杂种吗?可惜见不着啦,那杂种早就被我给宰
啦。用不着可惜,我送你去见他,咱们开始吧。

小五子堵在山洞门口活动筋骨,一只手看着月光下
伸腰拉背的黑影,失声哭了出来。那就等等呗,一边热
身一边听他哭,后来他也哭不出来了,杀猪似的干嚎。
小五子听得直烦,喊着“五脏俱裂掌”就往上扑。第一下
没打着,一只手也不敢往外跑,昨天打那两掌还指望吾
帮主解呢。于是两人就在山洞里绕着圈。五脏俱裂掌,
五脏俱裂掌!小五子一遍比一遍大声,最后一下全身的
力气扑上去,一只手大叫一声,倒在了山洞里。

瞅你这小胆儿。小五子让人拿些蜡烛,围着自己点
了一圈,再把脸上的炉灰洗掉,盘着腿坐到蜡烛中央。
大概有两刻钟,一只手睁开眼睛,看见被烛光笼罩的小
五子,认出是肉铺的伙计,眨着眼想了半天,说:“原来你
真死了。”

“死了也不会放过你!”小五子跳出蜡烛圈,继续喊
口号:“五脏俱裂掌!”

一只手哭都哭不动了,“咚咚咚”地在磕头。同样的
露出真面日,吴思若的反应是,你个杀猪的,怎么当上帮
主的?一只手想的则是,你都是帮主了,怎么还去杀
猪?他想起关长老也见过,当时陪同帮主赌钱的乞丐,
丐帮不是要饭的吗,他们还有些什么秘密组织啊?

一只手服得五体投地,这回是真磕头。小五子找出
一张纸,说这欠条是你摁了手印的,欠我半条命,刚你说
宰了那杂种,是说我吗?一只手直摇头。小五子表示先
不找你,记着,现在是欠我一条半的命,我从你师父手里
救出来,又是一条,我今天没给你用五脏俱裂掌,第三条
命,一共三条半,先不能让你死,杀了你,你还欠我两条
半的命,你死后投了胎,我还不一定能找得着你。

一只手听得频频点头,真是集功夫与智慧于一身的
大人物。他爬过去求小五子收了他,下半辈子给师姐做
牛做马他可不甘心,要是给你吾帮主,三生三世都嫌不
够,还要再加上半辈子呢。
\newline

{\centering\subsection{9}}

过了关就慢下来了,走走停停,第三天才到北京。
蒸羊羔、蒸熊掌、蒸鹿尾儿,烧花鸭、烧维鸡儿、烧子鹅,
京城那么多好吃的,小五子什么都咽不下去,勉强能喝
点开水,加两片茶叶又忍不住地反胃。吴思若有点不好
意思了,中午吃饭时坐他对面。本来想安慰他两句,但
是实在太俄了,菜刚一上桌就被她席卷一空。小五子远
离桌面,向后靠在椅背上,眼神涣散,他连瞪吴思若的力
气都没有了。

第一碗吃完她放下碗,把嘴角上的饭粒塞到嘴里嚼
掉,跟小五子说,不然喝点稀粥,吃点面条,这么干耗下
去可不是办法。你在劝我吗,说得好像是我没心情吃东
西,小五子他想说要不是你给我这一掌,这一桌子饭菜
根本就没你的事。话到嘴边却说不动,嗓子干得说不出
话来。他指了指吴思若,叹了口气,转半个身对着门
口坐。

先跟你借一碗,回头好了再还你,吴思若拿起小五
子的铜碗,开始第二碗饭。有个手捧鲜花的小姑娘从外
面走进来,到小五子跟前说:“大侠,买朵花吧,送给这位
女侠。”

小五子苦笑摆手,吴思若却是笑出声来,敲着铜碗
说:“小姑娘你也不看看,这人自己就是要饭的,哪儿来
的钱买花?〞

小姑娘看到吴思若手里的碗,扔下怀里的花就跑出
去。过一会儿领了一大帮人进来,也是破衣烂衫的,不
过每人手里都有点东西,有的拿着二胡,有的举着刀枪
棍棒,还有个年纪大点的,手里牵了只猴。这些人看到
他们两个也不知道朝谁跪,先是一边一下鞠个躬,然后
双手作揖,素性单膝朝中间跪下,牵猴的老头朗声道:
“北京堂堂主陈少卿,拜见帮主!”

这也是丐帮的?小五子将他们扶起来,清清嗓子,
悄悄话似的声音跟他们问话。陈堂主没听清,小五子又
问一遍,声音比刚才更小了。陈堂主看着小五子,还是
没听见。之前他们还怀疑来着,怎么年纪轻轻就做了丐
帮帮主,现在看这病恹恹的样子,一定是少年英雄,练了
某种神功,大功告成之日没准都能发出女声来。小五子
又问一遍,只见张嘴不出声,陈堂主带着人干脆又跪一
次,北京堂堂主陈少卿,拜见帮主!

小五子叹了口气,靠回到椅背上。吴思若抓紧时间
吃完第二碗,替小五子翻译:“快起来吧,帮主问你们,堂
堂丐帮的人,怎么又是卖花,又是要猴的?”

不问则已,提起伤心事,他们几个抱头痛哭。陈堂
主说,京城人家虽非富即贵,对乞丐却是十二分瞧不起,
不与施舍也就算了,碰上蛮横的还要棍棒驱赶,所以北
京堂的弟子光靠要饭难以为生,为了活命只能自谋生
路,卖花、要猴都已经算好了,天桥下面的几个弟兄,天
天都要表演胸口碎大石。

小五子听不下去了,嘶哑着声音就喊起来:“还胸口
碎大石?你们这么干,对得起丐帮的称号吗?”

陈堂主愣了两秒,慌忙跪了第三次,抹着眼泪说:
“北京弟子何以为继,还请帮主指条明路!”

哪里是明路呢,烧杀掳掠肯定不行,既然是丐帮,还
得是伸出双手看人脸色。小五子要等两天,待大功告成
他来给弟兄们做个样子。第二天醒来突然就有胃口了,
一个人吃了半个满汉全席,警告吴思若,要是再敢碰他
一下,丐帮这么多要饭的,我拿你当福利发下去。之后
他带着队伍出发,浩浩荡荡,走最繁华的大街,敲最有钱
的人家。张府、李府、王府,把响当当的名号报出来,丐
帮吾帮主请张大人、李大人、王大人赏口饭吃。个个都
一样,话没说完就被管家关到了大门外。

小五子对着大门思考了一阵儿,转身对陈堂主说:
“这就是我给你指的明路,你就像我这样,一家一家地敲
门,持之以恒,不言放弃,一定能要到钱。”

陈堂主把猴儿抱在怀里,盯了一会儿小五子,既没
有哭,也没有发火,转身对众弟子说:“都散了吧,该干嘛
干嘛\footnote{原文好像有此通病“该干吗干吗”},别耽搁吾帮主时间了。”

还好花还在,大石也没扔,猴子还活蹦乱跳的,没毁
了这些人的生路。小五子没脸看他们,低着头等他们散
场。吴思若凑到他耳边说,不然我帮你再指一条明路
吧,随后冲他们大喊:“帮主有令,全体弟子过来,跪在张
府门口,府内老老少少不得出进,直至给钱为止!”

有人把大石搬过来,请小五子坐上面,众人跪在他
身后。几百人乌泱泱地糊在张府门口,府上的人报了
官,巡捕过来一看,都是下跪请愿的,奇怪为什么不拿点
银子打发了呢。到了午时,张大人终于想通了,让管家
出来商量。

“五十两行不行?”

“当然不行!”小五子指着他脸质问,“你打发要饭的
呢?”

管家蒙了,挠着头看这满地的乞丐,你们不是要饭
的吗?管家进去跟老爷商量,出来重新报价,二百两怎
么样?小五子摇摇头,男儿膝下有黄金,我们还是继续
给您把门吧。这次回去久一些,小五子听到里面的几个
奶奶跟张大人吵起来了。瓷器摔碎满院,管家开门送出
银票,一千两,再不走我家老爷就要出兵了。

小五子起身对他鞠躬,目送他进院关门。大门合上
的一刻,小五子举起银票跳到大石上,跟着丐帮弟子欢
呼,挥舞着手臂带人去下一家。他看着陈堂主,总觉得
哪里不对劲,命令道:“陈堂主,你还是把你牵的那猴儿
放了吧。”
\newline

{\centering\subsection{10}}

要是以后小五子有子孙,有机会对子孙后代描述这
段经历,他该怎么定义江湖呢?行走的江湖,天天都是在
赶路。偏安一隅守在一方不是挺安逸吗,一大帮人候鸟
迁徙似的南行北进,也不知道图点什么,到了京城还要往
南走,日行几十里,说要去昆仑山庄,说要参加寻龙屠狼
会。小五子说,你们去吧,我在京城晃晃,到了腊八救向老
前辈出来。可你是帮主,关长老提醒他,丐帮二十年才等
来一个肯露脸的帮主,怎么能留你在京城当小混混?

行进队伍浩浩荡荡,这回是真的,沿途不时有丐帮
分队汇进来,在关长老的引见下,一一拜过新帮主。仿
佛一个雪球越滚越大,小五子成了大要饭头子,但他知
道总会有一天蹿出一个人,一脚将雪球踹碎,将队伍打
散,他知道这个人就是马长老。

关长老命令他不得走远,不管你是收了吴思若还是
一只手,都不许离开他的视线,超出他伸出手臂就能给
你助力的范围。小五子听着叮嘱,手掌在关长老面前画
圈,还不能逃出你的视线,其不知道多远才算远啊。

马长老迟迟未到,已经有人感觉新上任的帮主是个
草包。刚过行唐的那个上午,不知谁拽了一下马尾巴,
马蹄前扬,小五子从马上摔了下来。他躺在地上,听着
千八百人憋不住地笑,他在回想刚才都谁在马后,谁成
心让他出洋相? 他是领头的,千百号人在身后。关长老
伸手过来,说山路崎岖,请帮主搀扶一下。

小五子没理会他,自己站起来,看陆续从他身边经
过的人群。他们还在笑,那种即便是捂住了嘴,鼻孔却
憋不住的笑声。吴思若也冲他似笑非笑,唯有一只手满
险费解,帮主智慧与武功齐飞,怎么就被一匹马给甩下
来了呢?关长老又问一遍,小五子摇摇头,也不管他能
否看见自己拒绝,他重新上马,就走在最前面,看看谁这
么大胆子,能摔他第二遍。

余下行程总算平安,行至高邑他们安营扎寨,胡胖
子还是一脸谄媚的备好小灶。小五子没胃口,这次是真
的没脸吃,他说出去转转,关长老摸到身边的拐杖,说陪
帮主一同前往。

“不必陪同!”连同吴思若、一只手在内,小五子怒视
丐帮所有人,打从当帮主以来,他第一次用这样的口气
命令众人,“容我一人独处!”

关长老还是摸起拐杖,朝他这边走。小五子迎着过
去,低声对他说:“你既当我是帮主,就别这么盯着我,你
放心,我不会那么快就死了。”

他没往远走,前面转一个弯,握着杀猪刀守在关卡,
看有谁跟上来。直到日头西下,天色渐暗,,连只野兔子
都没守到,他收起刀,走远一些,坐到河边。不然就回去
吧,他对着河水想,马长老在前面等着宰了他,在那之前
还得被这一千多号人耻笑个够。他又想念文思清了,他
根本就不属于江湖,一点武功不会,走二里路都喘个不
行,还冒充什么少年英雄?他捡起石子冲河边打两个水
漂,回去吧,回田独杀猪卖肉,把文思清八拾大轿娶过
门,今晚就走。他起身,拍拍屁股,刚一转身肚子上挨了
一脚,面前一片黑,有布袋套在了他头上,他伸手去扯,
腰上又被踹了一脚,黑暗中他三晃两晃,倒在了河边。

他喘着粗气,哈气被布袋挡回来糊在脸上。他们一
共三人,伸手在吾帮主身上捋一遍,抄出一把杀猪刀,然
后把他背过身,脸朝下趴在岸边。不时涌上来的河水搅
着稀泥,透过黑布渗到他嘴里。他听他们在笑,这回不
用憋了,畅怀大笑,真是一草包,好意思当咱们帮主,用
不着马长老,我动动手指头就能捏死他。说着他还掐了
一下他肋骨,小五子也不叫,咬着牙回想,这些声音都是
谁,以前有没有听到过。

掐不够过瘾,另一个都上脚了,单脚踩在他后脖颈,
平时看你趾高气扬,怎么一落单就那么废物啊,知道我
们是谁吗,知道吗,你不知道,你吾大帮主怎么能记得我
们呢?说着脚跟向上,前脚掌在他后脖颈上又蹍了两
下,追着问他,你倒是说话啊,吾帮主,跟我们聊两句啊。

小五子嘴里咬着浸湿的布袋,后脖颈顶着他的脚往
上撑,顶出点空隙把泥水吐出来,咳了几声说:“你们三
个别让我认出来,不然我让你们活不到明天。”

说完就没了力气,脖子一松,又趴到泥水里。最刻
薄那个乐了:“那倒认识认识啊,来来来,我把你翻过来,
认认我们,看谁活不到明天。”

小五子死猪一样被提起来,又重重地仰摔在地上。
后脑还在震荡,脸上又挨了耳光。他们隔着布袋抽他
脸,说你醒醒,认认我们是谁。六七个耳光吧,小五子感
觉头顶的布袋在往上提。有个同伴觉得过分了,拉住
他,说要是掀开,咱就真得杀他了,不管怎么说,他也算
咱们帮主。另一个同伴也劝他,是不是帮主倒无所谓,
主要是今天把他杀了,明天咱玩什么呀,留着他,隔三岔
五玩玩帮主,不是挺好吗?

貌似有道理,他说那就收工吧,改天再来找你玩。
小五子仰躺着,耳边“嗖”的一声,他们把杀猪刀扎在泥
地上,听不到他们走远,也听不到他们说话。现在跟赌
场那次不一样了,他多了文思清,还多了何员外的遗愿,
什么时候他的命这么值钱,哪怕被凌辱之后。

等了几分钟,他又吐口河水,在布袋里睁开眼睛,打
算站起来。一阵急促的脚步,布袋忽然被拉起来。只回
来一个人,他把杀猪刀提起来,说咱俩没玩够呢,你倒是
睁眼看看我啊,看着我了,我就不留着你了。小五子紧
闭双眼,他说你们走吧,我今天认栽。

“你不栽,好东西我还给吾帮主您留着呢。”

小五子不说话,只闭眼。那人也不说话。好半天都
没动静,忽然有水喷下来。一边喷,他一边大笑,让小五
子睁眼看看,是什么好东西。小五子闭着眼感受,水是
温的,一柱冲下来,喷在鼻孔,喷在嘴角,很重的味道。
他指甲向地里抠,将污泥捏在拳头里。

最后几滴落下,那人打了个哆嗦,弯腰把杀猪刀塞
到小五子手里,让他握紧了,说:“你这他妈是什么玩意
儿,菜刀还是杀猪刀啊?装腔作势也得选个差不多的
啊。你歇着吧,我先回去了,我给俩选择,要么追过来砍
我,要么数一百个数,可别让我回头看见你睁眼。”他拍
拍小五子的脸,想起来很脏,手又在他裤子抹两下。“记
住了,没了关长老,你屁都不是。”

他还真在数数,闭着眼睛,想一个人数一个数,钱老
板、文思清、苏子瑶、何员外……他只有三年生命,认不
到一百个人,后来他就想事情,想自己都做了哪些事,想
自己还需要做哪些事,趁自己还活着。差不名一百了,
他在河边滚两个圈,头朝下浸在河水里,这一次他终于
把眼睛睁开了。

事情还没有想清楚,也许都不知道自己要想什么,
他从河里坐起来,露出半个身子。营地里在喊帮主,已
经有人发现他不见了。倒是吴思若先找到了小五子,他
不肯上来,让她先回去。吴思若就抱腿坐在岸边,两人
一个水上一个水下对视。吴思若问谁干的,用什么兵
器,穿什么鞋,说哪儿的口音,我错杀一百个,也不会让
他们就这么了了。

小五子不说话,一头扎进河水里,他怕再望一会儿
吴思若就要哭出来了。大概有一分钟,两分钟,他从河
里拔出来,一步步上岸,他说:“以前在田独,我们钱老
板,你见过他,喂猪的那个常公公,他希望我一直留在他
身边,怕我不听话跑了,骗我说江湖凶险,说我是武林第
一大通缉犯,甚至还扣我工钱。我当时气得要死,现在
想想这算什么呀,常公公是太监,留我的手段也都婆婆
妈妈的。但江湖不是,没那么多情分道义,想让我听话
很容易。”他蹚着水上来,经过吴思若往营地走,走出十
几步他停下来,背对着她说,“这事是关长老干的。”
\newline

{\centering\subsection{11}}

武林大会在汴梁,到了封丘县,意味着离目的地只
剩半日的里程。各门各派都会早到几天,提前通通气,
小帮小派会达成结盟,万一会上有人起腻子,彼此还有
个照应。丐帮用不着拉拢关系,少林武当丐帮,三大帮
派几百年来就是坚定盟友。可是关长老这次要摆宴,破
天荒第一次,要饭的请客吃饭。小五子明白,这是要显
摆他,告诉整个武林,丐帮有新帮主了,以后他要是被害
了,那一定是马长老干的。

小五子倒盼着马长老来,带着一队人马,跟关长老
拼个你死我活。起码还有一次幹旋的机会,这半个月他
确定了一件事,在关长老身边他逃不掉的,必死无疑。

关长老备了十几份请贴,陪着小五子一家家送过
去。大多数掌门人和他是老相识,双方作揖寒暄,都快
聊完作别了,关长老拉着小五子对他们说,对了,这就是
我们新帮主。掌门人打量这个满脸炉灰的小伙子,心里
都明白,丐帮看来是要姓关了。

请帖里没有百花谷,收贴人也不见常公公,小五子
最后的希望也破灭了。送的最后一家叫狮吼帮,靠嗓门
取胜的门派,高老远就听见屋里面吵架,一男一女,好像
是父女俩,问题是还有孩子哭。小五子和关长老在门外
候了一阵,他们越吵越凶,后来直接变成了羞辱,小五子
还没听哪个当爹的这么说女儿,说她生的野种、小杂种,
说一把摔死了都洗不净我这张老脸。

小五子听不下去了,隔着窗子咳嗽一声,关长老双
手抱拳自报家门:“丐帮关长老、吾帮主见过乔帮主!”

屋里也没处躲,乔姑娘鞠躬问候过,就抱着孩子鞠
躬背对着他们。小五子刚才也听明白了,大概就是未婚
生子那种事,问题是那孩子又不是刚生的,看样子有两
三岁了,就算没有爹,难不成天天被乔帮主奚落?

关长老介绍,这是我们新任帮主。乔帮主打量小五
子,小五子却一直打量乔姑娘背影。乔帮主冷笑一声,
继续和关长老寒暄。都是孩儿他娘了,小五子盯着人家
后身看,可能是心疼,自己过得不好,就特别心疼一样苦
命的人。乔姑娘左前方有面铜镜,小五子装作不经意走
几步,手倚在房柱上,刚好可以从镜子里看到乔姑娘的
脸。他看到她在哭,一点声音都没有,就是眼泪充盈在
眼睛里,每次眨眼刚好有滴眼泪挤出来。

小五子看得都失礼了,关长老眼瞎看不到。乔帮主
气鼓鼓的又不好明说,一再请小五子这边坐,老抱着那
根房柱干什么。反正早晚死在关长老手里,那就活得撒
野一点吧,小五子说,我不过去,这边好看。关长老好
奇,什么字画这么好看。乔帮主铁青着脸不说话。小五
子说,可是胜过好字好画呢。这时乔姑娘也注意到了,
瞥了一眼镜子,把镜面倒扣在桌上。

小五子坐回去,听着他们寒暄,可总忍不住偷看几
眼她背影。乔帮主已经很生气了,生怕关长老还听不出
来,故意很大声地拂下袖子,说时候不早,乔某晚点定去
赴宴。其实关长老也明白了,问题在小五子和乔姑娘。
他拉着小五子起身告辞。虽然很嫌弃,乔姑娘还是转回
身冲他们微微点头。就那两秒钟,小五子又看愣神了,
为人妻,为人母,竟如此美人出画。是真的美。文思清
也很美,可是小五子跟她同屋一年没起过色心;吴思若
也好看,五花大绑到面前,小五子只想拿个仙人球闹她;
苏子瑶是那种端庄的美,觉得有个这样的妻子会是个很
体面的人生;唯有眼前的乔姑娘,美得让他心生淫意,只
想天天跟她腻一起。

晚宴她不会来,乔帮主绝不会让女儿及外孙成为酒
桌的话题。挺不了几天,早晚要死在关长老手里,可能
是最后一次见到她了。一生一期,一期一会,小五子想
说点什么留给她一辈子,都走到门口了,他回头对她说:
“你生得这么美,其实用不着害怕的。”

在说什么呢,他也不知道,挺好的意思讲出来是乱
的。乔姑娘这次没有躲,皱着眉盯他。他做了见谅的手
势,想再解释一两句,手腕一吃力,被关长老拉出了
房间。

晚上他喝多了,来了十三个掌门人,他一个也没记
住,倒是各派拎来的酒记得一清二楚。又不为他,全是
冲关长老的面子,小五子就铛铛铛地喝闷酒。最气他的
乔帮主,后来都有点看不过去了,陪他连喝了三杯酒。

戌时开席,不出三刻便开始晕了,但他一句话也不
说,言多必失,他怕一张嘴会求助,来的都是前辈,帮我
主持公道,我不求帮主之位,只求活着回家。他不能说
这些,说出来也许活不过今晚。他想好了,武林大会也
许是他最后的机会,趁乱逃走,或者干脆跑台上去伸冤,
没准马长老会跳出来保他这条小命呢。那就继续喝,兰
亭派的女儿红,桂党的三花酒,峨眉派的五粮醇,黄山的
宣酒,他以为他会倒下,有两个帮主比他醉得还快,三五
句不和,隔着桌子就要比画比画。后来他们扯到外面,
找一片空场。小五子眯眼看了几十招,胃里一阵阵地恶
心,拐到后门弯腰吐了起来。

连吐了两三次,感觉肠子都要吐出来了,咽喉还是
止不住地干呕,双手扶墙看着口水往下坠。有人在轻拍
他后背,墙上一个女人的影子,这让他些许感动,手背抹
掉口水说:“还好你来了,我差点死在这儿。”

“你居然没死。”

不是吴思若,他挺身站起來。是乔姑娘,她拿出湿
手帕,伸手擦小五子的脸,从领头到脸颊,从鼻尖到嘴
唇,当炉灰擦尽,他的样子一点一点呈现在她面前时,她
说:“真的是你,你还活着。”

他已然酒醒,左右看看,显然乔姑娘今晚一直在外
面等着他。他拉过她肩膀,再往里躲一躲,问道:“你是
百花谷的?”

“什么?”

小五子说没事,凝眉回想她到底是谁,过去怎么会
遇上这么美的姑娘?他想多问几句,比如我和你是否好
过,比如难道那孩子是我的?这些都是亵渎,他问不出
口,近在咫尺望着她的眼睛,等她说下一句话。

她看了看他,摸了摸擦干净的脸,低声对他说:“你
今晚就走,千万别去昆仑山庄。”

“为什么?”

有人朝这边走,自言自语,都好好喝酒,怎么见面就
打?她没时间了,把手帕塞给他,又叮嘱一次:“千万别
去,那种地方你有去无回。”

\newpage